%!  pour pdfLatex
\documentclass[a4paper]{article}
%\usepackage[hmargin={1.5cm,1.5cm},vmargin={2.4cm,2.4cm},headheight=13.1pt]{geometry}
\usepackage[a4paper,landscape,twocolumn,
            hmargin=1.8cm,vmargin=2.2cm,headheight=13.1pt]{geometry}

\usepackage[pdftex]{graphicx,color}
\usepackage[pdftex,colorlinks={true},urlcolor={blue},pdfauthor={remy Nicolai}]{hyperref}

\usepackage[T1]{fontenc}
\usepackage[utf8]{inputenc}

\usepackage{lmodern}
\usepackage[frenchb]{babel}

\usepackage{fancyhdr}
\pagestyle{fancy}

\usepackage{floatflt}
\usepackage{maths}

\usepackage{parcolumns}
\setlength{\parindent}{0pt}

\usepackage{caption}
\usepackage{subcaption}

\usepackage{makeidx}

\usepackage[french,ruled,vlined]{algorithm2e}
\SetKwComment{Comment}{\#}{}
\SetKwFor{Tq}{tant que}{}{}
\SetKwFor{Pour}{pour}{}{}
\DontPrintSemicolon
\SetAlgoLined

\usepackage{listings}
\lstset{language=Python,frame=single}
\lstset{literate=
  {á}{{\'a}}1 {é}{{\'e}}1 {í}{{\'i}}1 {ó}{{\'o}}1 {ú}{{\'u}}1
  {Á}{{\'A}}1 {É}{{\'E}}1 {Í}{{\'I}}1 {Ó}{{\'O}}1 {Ú}{{\'U}}1
  {à}{{\`a}}1 {è}{{\`e}}1 {ì}{{\`i}}1 {ò}{{\`o}}1 {ù}{{\`u}}1
  {À}{{\`A}}1 {È}{{\'E}}1 {Ì}{{\`I}}1 {Ò}{{\`O}}1 {Ù}{{\`U}}1
  {ä}{{\"a}}1 {ë}{{\"e}}1 {ï}{{\"i}}1 {ö}{{\"o}}1 {ü}{{\"u}}1
  {Ä}{{\"A}}1 {Ë}{{\"E}}1 {Ï}{{\"I}}1 {Ö}{{\"O}}1 {Ü}{{\"U}}1
  {â}{{\^a}}1 {ê}{{\^e}}1 {î}{{\^i}}1 {ô}{{\^o}}1 {û}{{\^u}}1
  {Â}{{\^A}}1 {Ê}{{\^E}}1 {Î}{{\^I}}1 {Ô}{{\^O}}1 {Û}{{\^U}}1
  {œ}{{\oe}}1 {Œ}{{\OE}}1 {æ}{{\ae}}1 {Æ}{{\AE}}1 {ß}{{\ss}}1
  {ű}{{\H{u}}}1 {Ű}{{\H{U}}}1 {ő}{{\H{o}}}1 {Ő}{{\H{O}}}1
  {ç}{{\c c}}1 {Ç}{{\c C}}1 {ø}{{\o}}1 {å}{{\r a}}1 {Å}{{\r A}}1
  {€}{{\euro}}1 {£}{{\pounds}}1 {«}{{\guillemotleft}}1
  {»}{{\guillemotright}}1 {ñ}{{\~n}}1 {Ñ}{{\~N}}1 {¿}{{?`}}1
}

%pr{\'e}sentation des compteurs de section, ...
\makeatletter
\renewcommand{\thesection}{\Roman{section}.}
\renewcommand{\thesubsection}{\arabic{subsection}.}
\renewcommand{\thesubsubsection}{\arabic{subsubsection}.}
\renewcommand{\labelenumii}{\theenumii.}
\makeatother


\newtheorem*{thm}{Théorème}
\newtheorem{thmn}{Théorème}
\newtheorem*{prop}{Proposition}
\newtheorem{propn}{Proposition}
\newtheorem*{pa}{Présentation axiomatique}
\newtheorem*{propdef}{Proposition - Définition}
\newtheorem*{lem}{Lemme}
\newtheorem{lemn}{Lemme}

\theoremstyle{definition}
\newtheorem*{defi}{Définition}
\newtheorem*{nota}{Notation}
\newtheorem*{exple}{Exemple}
\newtheorem*{exples}{Exemples}


\newenvironment{demo}{\renewcommand{\proofname}{Preuve}\begin{proof}}{\end{proof}}
%\renewcommand{\proofname}{Preuve} doit etre après le begin{document} pour fonctionner

\theoremstyle{remark}
\newtheorem*{rem}{Remarque}
\newtheorem*{rems}{Remarques}

%\usepackage{maths}
%\newcommand{\dbf}{\leftrightarrows}

%En tete et pied de page
\lhead{Informatique}
%\chead{Introduction aux systèmes informatiques}
\rhead{MPSI B Hoche}
\lfoot{\tiny{Cette création est mise à disposition selon le Contrat\\ Paternité-Partage des Conditions Initiales à l'Identique 2.0 France\\ disponible en ligne http://creativecommons.org/licenses/by-sa/2.0/fr/  
} 
\rfoot{\tiny{Rémy Nicolai \jobname \; \today } }
}
\makeindex

%En tete et pied de page
\lhead{IPT}
\chead{Corrigé DS 1 le 10/12/14}
\begin{document}
\section{Questions de cours}
\begin{enumerate}
  \item Quels entiers relatifs sont représentables en 64 bits sur les machines actuelles?\newline
Les machines actuelles utilisent la représentation des entiers par complément à $2$. Les entiers relatifs de
\begin{displaymath} 
  \llbracket -2^{63} , 2^{63} -1 \rrbracket
\end{displaymath}
sont représentables.
\begin{itemize}
  \item Les entiers de $\llbracket -2^{63} , 2^{63} -1 \rrbracket$ sont représentés par leur décomposition binaire sur 64 bits. Le bit de poids fort est $0$.
  \item Les entiers $x$ de $\llbracket -2^{63} , 2^{63} -1 \rrbracket$ sont représentés par la décomposition binaire de $2^{64} -x$ sur 64 bits. Le bit de poids fort est $1$.
\end{itemize}

  \item Quelle est le représentation par complément à 2 sur 64 bits de $-2^{62}$ ?\newline
En suivant les principes de la question précédente, la représentation en binaire sur 64 bits de 
\begin{displaymath}
2^{64}-2^{62} = (4-1)2^{62} = 3\times 2^{62} = 2^{63} + 2^{62} 
\end{displaymath}
est
\begin{displaymath}
  (1, 1, \underset{62 \text{ zéros}}{\underbrace{0, \cdots,0}})
\end{displaymath}

  \item Représentation normalisée de $\frac{8}{7}$? de $\frac{128}{15}$?\newline
Les nombres normalisés sont ceux de la forme
\begin{displaymath}
  s\times m \times 2^{n}\text{ avec }
\left\lbrace 
\begin{aligned}
  &s\in \{-1,1\} &:&\; 1\text{ bit}\\
  &m\in[1,2[ \text{ binaire } &:&\; 52 \text{ bits}\\
  &n \in \llbracket -1022, 1023 \rrbracket &:&\; 11 \text{ bits} \\ 
\end{aligned}
\right. 
\end{displaymath}
Comme $1\leq \frac{8}{7} < 2$, l'exposant est 0. Ce nombre est égal à sa mantisse. De plus, on fait facilement apparaitre la partie entière
\begin{displaymath}
  \frac{8}{7} = 1 + \frac{1}{7} \hspace{0.5cm} dec \rightarrow [1]
\end{displaymath}
On utilise ensuite l'algorithme de décomposition dans une base d'un nombre entre 0 et $b-1$ (ici $b=2$) en multipliant par $2$ jusq'à dépasser 1.
\begin{displaymath}
\left\lbrace \
\begin{aligned}
\frac{2}{7} <1\\
x \rightarrow\frac{2}{7}\\  
dec \rightarrow [1,0]  
\end{aligned}
\right. 
\rightsquigarrow
\left\lbrace \
\begin{aligned}
\frac{4}{7} <1\\
x \rightarrow\frac{4}{7}\\  
dec \rightarrow [1,0,0]  
\end{aligned}
\right. 
\rightsquigarrow
\left\lbrace \
\begin{aligned}
\frac{8}{7} >1\\
x \rightarrow\frac{1}{7} <1\\  
dec \rightarrow [1,0,0,1]  
\end{aligned}
\right. 
\rightsquigarrow \cdots
\end{displaymath}
et l'exécution est ensuite périodique. On pouvait aussi remarquer que :
\begin{displaymath}
  \frac{8}{7} = \frac{8}{8 -1} = \frac{1}{1-2^{-3}} = 1 + \frac{1}{2^{-3}}+ \frac{1}{2^{-6}}+ \frac{1}{2^{-8}}+\cdots+ \text{ reste adéquat}
\end{displaymath}
La représentation binaire de la mantisse ne tient compte que de la partie fractionnaire; c'est donc :
\begin{displaymath}
  (\underset{17 \text{ fois } 0,0,1 = 51 \text{ bits}}{\underbrace{0,0,1,0,0,1,\cdots,0,0,1}},0)
\end{displaymath}
Pour $\frac{128}{15}$, l'exposant est $3$ car
\begin{displaymath}
\frac{128}{15} = \frac{8\times 16}{15} = 2^3 \times \frac{1}{1-2^{-4}}  
= 1 + \frac{1}{2^{-4}}+ \frac{1}{2^{-8}}+ \frac{1}{2^{-12}}+\cdots+ \text{ reste adéquat}
\end{displaymath}
La représentation binaire de la mantisse est :
\begin{displaymath}
  (\underset{13 \text{ fois } 0,0,0,1 = 52 \text{ bits}}{\underbrace{0,0,0,1,0,0,0,1,\cdots,0,0,0,1}})
\end{displaymath}
L'algorithme du cours permet aussi d'obtenir cette décomposition périodique.
\end{enumerate}

\section{Séparation d'une liste}
\begin{enumerate}
  \item Pour chacune des trois boucles $d-g$ est une fonction de terminaison. Elle reste à valeurs entières à cause des conditions $g\leq d$. Elle diminue strictement dans les deux boucles internes par incrémentation de $g$ ou décrémentation de $d$. En ce qui concerne la boucle externe, si $d-g$ n'a pas diminué lors des boucles internes $g$ et $d$ n'ont pas changé de valeur donc $g$ reste inférieur à $d$ et $d-g$ est diminué de 2 lors du contrôle \texttt{if}.
  \item Les propriétés 
\begin{displaymath}
  \left( k < g \Rightarrow L[k] \leq v\right)  \text{ et } \left( d < k \Rightarrow L[k] > v\right) 
\end{displaymath}
sont des invariants de la boucle externe. Elles sont donc encore valables à la fin. Comme $d = g - 1$ à la sortie de la boucle externe, on en déduit que $L$ vérifie
\[
 \forall k \in \llbracket 0, l-1 \rrbracket, \;
 L[k]
 \left\lbrace 
 \begin{aligned}
  \leq v &\text{ si } k < g \\
  > v &\text{ si } g \leq k
 \end{aligned}
\right. .
\]

  \item Implémentation en Python de l'algorithme de séparation.
\lstinputlisting[firstline=8, lastline=20]{Csep.py}
Comme une liste est un objet modifiable, elle est modifiée à lors de l'appel à l'intérieur de la procédure. 
\end{enumerate}


\section{Calculs en binaire}
\subsection{Calcul de carrés}
\begin{enumerate}
  \item Dans une expression \texttt{ a // b}, les noms \texttt{a} et \texttt{b} doivent désigner des objets du type entier. L'expression s'évalue alors au quotient entier de la division de \texttt{a} par \texttt{b}. 
  \item Tous les nombres (y compris $a$ et $b$) représentés par des listes de longueur $l+1$ sont plus petits que
\begin{displaymath}
  1 + \frac{1}{2} + \frac{1}{2^2} + \cdots + \frac{1}{2^l} = \frac{1-\frac{1}{2^{l+1}}}{1-\frac{1}{2}}= 2 -\frac{1}{2^l} < 2
\end{displaymath}
Il est commode de considérer une troisième propriété: \og \texttt{t} désigne 0 ou 1 ou 2 ou 3\fg. Les trois propriétés sont des invariants car:
\begin{multline*}
\left. 
\begin{aligned}
a_i \in \{0,1\}\\ b_i \in \{0,1\} \\r_i \in\{0,1\}  
\end{aligned}
\right\rbrace 
\Rightarrow t_{i+1} \in \{0,1,2,3\}\\
\Rightarrow
\left\lbrace 
\begin{aligned}
&l_{i+1} \in\{0,1\}\text{ (reste de $t_{i+1}$ modulo 2)}\\
&r_{i+1} \in\{0,1\}\text{ (quotient de la div. de $t_{i+1}$ par 2)}  
\end{aligned}
\right. 
\end{multline*}
\`A la fin, la \og retenue\fg~ \texttt{ret} peut prendre la valeur 0 ou 1. La liste $S$ représente $a+b$ lorsque $a+b<2$ c'est à dire lorsque \texttt{ret} désigne $0$. Sinon $s+2 = a+b$.
  \item 
\begin{enumerate}
  \item Codage Python
\lstinputlisting[firstline=28, lastline=36]{Cbrigglog.py}
\item Désignons par $a_l$ la dernière valeur de la liste \texttt{A} avant le décalage.
\begin{displaymath}
  a' =
\left\lbrace 
\begin{aligned}
&\frac{a}{2}&\text{ si } a_l = 0 \\
&\frac{a}{2} - \frac{1}{2^{l+1}}&\text{ si } a_l = 1 
\end{aligned}
\right. 
\end{displaymath}
\end{enumerate}

\item
\begin{enumerate}
  \item Si on remplace l'instruction \og\texttt{B = [a for a in A]}\fg~ par \og\texttt{B = A}\fg, on change profondément l'algorithme. En effet les deux noms désignent alors la même liste et cette liste sera modifiée par l'appel \texttt{shift(B,d)}. En effet cette fonction modifie l'objet (modifiable) désignée par \texttt{B}. Cela va pertuber la boucle d'énumération des objets de \texttt{A} :\og \texttt{for a in A}\fg~ puisqu'ils sont modifiés à l'intérieur même de l`énumération.
  
  \item Le produit de deux nombres binaires de longueur $l$ n'est pas forcément un nombre binaire de longueur $l$. Par exemple, si $a$ et $b$ sont dans $\llbracket 0,l \rrbracket$ avec $a+b >l$, les nombres $2^{-a}$ et $2^{-b}$ sont binaires de longueur $l$ mais pas leur produit.
  
  \item L'appel de la fonction correspond à une suite de décalages et d'additions
\begin{align*}
\begin{array}{clllll}
                &0 & 0 & 0 & 0 & 0\\
\text{shift 0}: &1 & 1 & 0 & 1 & 0
\end{array}
\Rightarrow 
\left\lbrace
\begin{aligned}
&ret \rightarrow 0 \\ &C \rightarrow (1,1,0,1,0)  
\end{aligned}
\right. \\
\begin{array}{clllll}
                &1 & 1 & 0 & 1 & 0\\
\text{shift 1}: &0 & 1 & 1 & 0 & 1
\end{array}
\Rightarrow 
\left\lbrace
\begin{aligned}
&ret \rightarrow 1 \\ &C \rightarrow (0,0,1,1,1)  
\end{aligned}
\right. \\
\begin{array}{clllll}
                &0 & 0 & 1 & 1 & 1\\
\text{shift 2}: &0 & 0 & 0 & 1 & 1
\end{array}
\Rightarrow 
\left\lbrace
\begin{aligned}
&ret \rightarrow 1 \\ &C \rightarrow (0,1,0,1,0)  
\end{aligned}
\right. 
\end{align*}

  \item Cette fonction renvoie une valeur approchée par défaut du carré d'un nombre binaire de $[0,1[$. Un tel nombre est dans $[0,4[$, la retenue permet de savoir s'il est plus grand que $2$. Il est bien clair que le nombre renvoyé est plus petit que le carré. Le principal reproche à faire à cette fonction c'est que l'erreur n'est pas facile à majorer. En particulier, pour certains $x$, le nombre renvoyé \emph{ne sera pas} l'approximation binaire par défaut de $x^2$.
\end{enumerate}

\end{enumerate}

\subsection{Calculs de logarithmes particuliers}
\begin{enumerate}
  \item Les suites $\left( x_n\right)_{n\in \N}$, $\left( b_n\right)_{n\in \N}$ vérifient les relations de récurrence:
\begin{displaymath}
b_{k+1} =
\left\lbrace 
\begin{aligned}
  &0 &\text{ si } x_k^2 < 10 \\
  &1 &\text{ si } x_k^2 \geq 10
\end{aligned}
\right., \\
\hspace{1cm} x_{k+1} = \frac{x_k^2}{10^{b_{k+1}}}
\end{displaymath}

\item Le nom $x$ désigne au départ un nombre entre $1$ et $10$. Après chaque assignation $x \leftarrow x *x$, il désigne un nombre entre $0$ et $100$. On teste s'il est plus grand que 10 et on le divise par 10 si c'est le cas (ce qui redonne un nombre plus petit que $10$). On en déduit que \og $x\in [1,10[$\fg est un invariant de boucle.\newline
La liste $B$ est augmentée par la valeur de $b$ qui n'est assigné qu'à 0 ou 1. On en tire que \og$B$ est une liste de 0 et de 1\fg est aussi un invariant de boucle. 

\item On montre par récurrence la relation demandée en l'élevant au carré et en utilisant $x_{k+1}10^{b_{k+1}}=x_k^{2}$.
\begin{multline*}
  x_{k+1} = 
\left( 
\frac{x_{ini}^{2^k}}{10^{2^{k}\left(\frac{b_1}{2}+\frac{b_2}{2^2}+\cdots+ \frac{b_k}{2^k}\right) }}
\right)^2 \frac{1}{10^{b_{k+1}}}
=
\frac{x_{0}^{2^{k+1}}}{10^{2^{k+1}\left(\frac{b_1}{2}+\frac{b_2}{2^2}+\cdots+ \frac{b_k}{2^k}\right) +b_{k+1}}}\\
=
\frac{x_{0}^{2^{k+1}}}{10^{2^{k+1}\left(\frac{b_1}{2}+\frac{b_2}{2^2}+\cdots+ \frac{b_k}{2^k} + \frac{b_{k+1}}{2^{k+1}}\right) }}
\end{multline*}

\item Par définition, comme la liste $B$ commence par un $0$, le nombre binaire représenté est
\begin{displaymath}
  b = \frac{b_1}{2}+\frac{b_2}{2^2}+\cdots+ \frac{b_n}{2^n}
\end{displaymath}
La relation de la question 3 s'écrit alors
\begin{multline*}
x_n 10^{2^nb} = x_0^{2^{n}}
\Rightarrow
\ln x_n + 2^n\,b\,\ln 10 = 2^n \ln x_0
\Rightarrow
\frac{\ln x_0}{\ln 10} = b + 2^{-n}\frac{\ln x_n}{\ln 10}\\
\Rightarrow
b \leq \frac{\ln x_0}{\ln 10} < b + 2^{-n}
\end{multline*}
car on a montré que $0<x_n<10$. Cet algorithme permet donc de calculer le développement binaire du logarithme décimal d'un nombre entre $1$ et $10$.
\end{enumerate}

\end{document}