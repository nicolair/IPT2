%!  pour pdfLatex
\documentclass{beamer}

%\usepackage[pdftex]{graphicx,color}
%\usepackage[pdftex,colorlinks={true},urlcolor={blue},pdfauthor={remy Nicolai}]{hyperref}

\usepackage[utf8]{inputenc}
\usepackage[T1]{fontenc}
\usepackage{lmodern}
\usepackage[frenchb]{babel}

\usetheme{Warsaw}

%\usepackage{fancyhdr}
%\pagestyle{fancy}

%\usepackage{floatflt}
\usepackage{maths}

\usepackage{parcolumns}
\setlength{\parindent}{0pt}

%\usepackage{caption}
%\usepackage{subcaption}

\usepackage[french,ruled,vlined]{algorithm2e}
\SetKwComment{Comment}{\#}{}
\SetKwFor{Tq}{tant que}{}{}
\SetKwFor{Pour}{pour}{}{}
\DontPrintSemicolon
\SetAlgoLined

\usepackage{listings}
\lstset{language=Python,frame=single}

%pr{\'e}sentation des compteurs de section, ...
\makeatletter
\renewcommand{\thesection}{\Roman{section}.}
\renewcommand{\thesubsection}{\arabic{subsection}.}
\renewcommand{\thesubsubsection}{\arabic{subsubsection}.}
%\renewcommand{\labelenumii}{\theenumii.}
\makeatother


\newtheorem*{thm}{Théorème}
\newtheorem{thmn}{Théorème}
\newtheorem*{prop}{Proposition}
\newtheorem{propn}{Proposition}
\newtheorem*{pa}{Présentation axiomatique}
\newtheorem*{propdef}{Proposition - Définition}
\newtheorem*{lem}{Lemme}
\newtheorem{lemn}{Lemme}

\theoremstyle{definition}
\newtheorem*{defi}{Définition}
\newtheorem*{nota}{Notation}
\newtheorem*{exple}{Exemple}
\newtheorem*{exples}{Exemples}


\newenvironment{demo}{\renewcommand{\proofname}{Preuve}\begin{proof}}{\end{proof}}
%\renewcommand{\proofname}{Preuve} doit etre après le begin{document} pour fonctionner

\theoremstyle{remark}
\newtheorem*{rem}{Remarque}
\newtheorem*{rems}{Remarques}

%\usepackage{maths}
%\newcommand{\dbf}{\leftrightarrows}

%En tete et pied de page
%\lhead{Informatique}
%\chead{Introduction aux systèmes informatiques}
%\rhead{MPSI B Hoche}
%\lfoot{\tiny{Cette création est mise à disposition selon le Contrat\\ Paternité-Partage des Conditions Initiales à l'Identique 2.0 France\\ disponible en ligne http://creativecommons.org/licenses/by-sa/2.0/fr/
%} }
%\rfoot{\tiny{Rémy Nicolai \jobname}}

\nonstopmode

\begin{document}

\begin{frame}
  \frametitle{Partie entière - Partie décimale}
Tout nombre réel se décompose de manière unique sous la forme
\begin{displaymath}
  \text{un nombre réel} = \text{un nombre entier} + \text{un nombre réel dans $[0,1[$}
\end{displaymath}
Notation
\begin{displaymath}
  x = \underset{\text{partie entière de } x\in \Z }{\underbrace{\lfloor x \rfloor}} + \underset{\text{partie décimale de } x\in [0,1[ }{\underbrace{\{ x \}}}
\end{displaymath}
\end{frame}

\begin{frame}
  \frametitle{Partie entière et division}
Dans la division euclidienne de $m\in \Z$ par $n\in \N^*$ 
\begin{displaymath}
  m = \underset{\in \Z}{\underbrace{n}}q + \underset{\in \llbracket 0, n\llbracket}{\underbrace{r}}
  \Leftrightarrow \frac{m}{n} = \underset{\in \N}{\underbrace{q}} + \underset{\in [0,1[}{\underbrace{\frac{r}{n}}} 
\end{displaymath}
Le quotient de la division de $m$ par $n$ est $\lfloor \frac{m}{n} \rfloor$.
\end{frame}

\begin{frame}
  \frametitle{Algorithme \og les petits d'abord\fg}
\begin{algorithm}[H]
  $y\longleftarrow $ un entier \;
  \Tq{ $y > 0$}{
      enregistrer le reste de la division de $y$ par $b$\;
      $y\longleftarrow $ le quotient de la division de $y$ par $b$\;
  }
  \caption{les petits d'abord.}
  \label{nbbin_1}
\end{algorithm}
\end{frame}

\begin{frame}
  \frametitle{Algorithme \og les grands d'abord\fg}
\begin{algorithm}[H]
  $y\longleftarrow $ un nombre dans $[0,1[$\;
  \Tq{ $y > 0$}{
      enregistrer $\lfloor by \rfloor$\;
      $y\longleftarrow \{ by\}$\;
  }
  \caption{ les grands d'abord.}
  \label{nbbin_2}
\end{algorithm}
\end{frame}

\begin{frame}
  \frametitle{Développement d'un réel}
Base $b$ : entier naturel strictement plus grand que $1$.\newline
Nombre  $x>0$.\newline
Développement de $x$ dans la base $b$:
\begin{itemize}
  \item vers la gauche de la virgule :\newline \og les petits d'abord \fg initialisé à $\lfloor x \rfloor$.
  \item vers la droite de la virgule :\newline \og les grands d'abord \fg initialisé à  $\{ x \}$.
\end{itemize}
\end{frame}

\begin{frame}
  \frametitle{Exemple : développement de $7$ en base $2$}
Vers la gauche de la virgule.
\begin{align*}
  & & 7 & & . \\
7 = 3\times 2 + 1 & & 3 & & 1. \\
3 = 1\times 2 + 1 & & 1 & & 11. \\
1 = 0\times 2 + 1 & & 0 & & 111.
\end{align*}
\end{frame}

\begin{frame}
  \frametitle{Exemple : développement de $.3$ en base $2$}
Vers la droite de la virgule.
\begin{align*}
    & & .3 & & .   \\
.6  & & .6 & & .0  \\
1.2 & & .2 & & .01 \\
.4  & & .4 & & .010 \\
.8  & & .8 & & .0100 \\
1.6 & & .6 & & .01001
\end{align*}
Le développement se répète ensuite vers la droite.
\end{frame}

\begin{frame}
  \frametitle{Exemple : développement de 7.3 en base 2}
Le développement binaire du réel $7.3$ est
\begin{displaymath}
  111.01001 01001 01001 \cdots
\end{displaymath}  
\end{frame}

\end{document}
