\subsection*{I. Chiffrement de César}
\begin{enumerate}
\item 
\lstinputlisting[firstline=9, lastline=13]{Cchiffr1.py}

\item On présente le résultat dans un tableau:
\begin{center}
\begin{tabular}{|c|c|c|c|c|c|c|c|c|c|c|c|c|c|c|c|c|c|c|c|c|c|c|c|c|c|}
\hline
txt              & i  & n  & f  & o  & r  & m  & a & t  & i  & q  & u  & e \\ \hline
nbs              & 8  & 13 & 5  & 14 & 17 & 12 & 0 & 19 & 8  & 16 & 20 & 4 \\ \hline
nbs décalés      & 14 & 19 & 11 & 20 & 23 & 18 & 6 & 25 & 14 & 22 & 0  & 10 \\ \hline
txt     décalées & o  & t  & l  & u  & x  & s  & g & z  & o  & w  & a  & k \\ \hline
\end{tabular}
\end{center}
Pour la lettre 'u', le décalage fait obtenir $26$, que l'on ramène à $0$.\newline
La liste obtenue est :
\[
 [14,19,11,20,23,18,6,25,14,22,0,10]. 
\]


\item
\begin{enumerate}
\item \lstinputlisting[firstline=15, lastline=19]{Cchiffr1.py}

\item On choisit un mot et un décalage qui font intervenir le reste, par exemple \og informatique\fg~ et $6$. Pour aller plus vite, on choisit plutôt \og xyz\fg~ et $2$.\\
On teste ensuite le code proposé à la question précédente à l'aide d'un tableau (ici pour \og xyz \fg~ et $2$):
\begin{center}
\begin{tabular}{|c|c|c|c|c|c|c|c|c|c|c|c|c|c|c|c|c|c|c|c|c|c|c|c|c|c|} \hline
k       &     & 23   & 24     & 25       \\ \hline
k+d     &     & 25   & 26     & 27       \\ \hline
chiffre & [ ] & [25] & [25,0] & [25,0,1] \\ \hline
\end{tabular}
\end{center}

\item Pour déchiffrer, il suffit de chiffrer avec la clé opposée.
\lstinputlisting[firstline=21, lastline=22]{Cchiffr1.py}
On teste la fonction avec la liste $[25,0,1]$ et $d=2$, et on retrouve bien [23,24,25].
\end{enumerate}

\end{enumerate}

\subsection*{II. Cryptanalyse du chiffrement de César}
\begin{enumerate}
\item Le tableau des fréquences est indexé de 0 à 26. Les valeurs de $L$ sont des indices du tableau des fréquences. 
\lstinputlisting[firstline=24, lastline=28]{Cchiffr1.py}

\item La recherche d'un indice de valeur maximale est un algorithme usuel.
\lstinputlisting[firstline=30, lastline=35]{Cchiffr1.py}

\item On déchiffre de selon l'algorithme suivant.
\begin{itemize}
 \item On cherche le nombre $f$ le plus fréquent.\newline (en appelant \texttt{indice$\_$max(frequences(L))}).
 \item En principe, ce nombre correspond au 'e', c'est-à-dire à $4$. La clé est donc $f-4$ si $f-4$ est positif, et $f-4+26$ sinon. Dans tous les cas, c'est $(f-4) \% 26$.
 \item On déchiffre à l'aide de la fonction \texttt{dechiffrement$\_$cesar}.
\end{itemize}
L'implémentation en Python peut s'écrire comme la fonction qui suit.
\lstinputlisting[firstline=37, lastline=41]{Cchiffr1.py}

\item Le nombre de plus grande fréquence de la première liste est $6$. La clé est donc probablement $2$. Le message déchiffré est $[11,24,2,4,4,7,14,2,7,4]$ qui correspond à \og lyceehoche\fg. Comme ce texte a un sens, c'est vraisemblablement le texte original.\\
Le nombre de plus grande fréquence dans la seconde liste est $15$. La clé doit donc être $15-4=11$. \newline
Le message déchiffré est alors $[19,17,4,10,25,4,21]$, qui correspond à \og trekzev\fg. Comme ce texte n'a pas de sens, il ne doit pas être le texte original.\\
Dans le premier cas, la lettre la plus fréquente du message chiffré est bien un 'e', alors que dans le second elle ne l'est probablement pas.\newline
En fait, le texte en clair était \og cantine\fg chiffré avec la clé 2.\newline
Le cryptanalyste doit donc disposer de messages codés assez longs et \og génériques\fg (par exemple pas le texte du roman \og\emph{ La disparition}\fg~ de Perec écrit sans 'e') pour que l'analyse des fréquences de lettre soit pertinente.
\end{enumerate}

\subsection*{III. Chiffrement de Vigénère}
\begin{enumerate}
\item On procède comme dans l'énoncé:
\begin{center}
\begin{tabular}{|c|c|c|c|c|c|c|c|c|c|c|c|c|c|c|c|c|c|c|c|c|c|c|c|c|c|} \hline
texte en clair & l  & y  & c  & e  & e  & h  & o  & c  & h  & e\\ \hline
liste en clair & 11 & 24 & 2  & 4  & 4  & 7  & 14 & 2  & 7  & 4\\ \hline 
clé            & m  & p  & s  & i  & m  & p  & s  & i  & m  & p  \\ \hline
liste clé      & 12 & 15 & 18 & 8  & 12 & 15 & 18 & 8  & 12 & 15  \\ \hline
liste chiffrée & 23 & 13 & 20 & 12 & 16 & 22 & 6  & 10 & 19 & 19 \\ \hline
texte chiffré  & x  & n  & u  & m  & q  & w  & g  & k  & t  & t  \\ \hline
\end{tabular}
\end{center}

\item Implémentation du chiffrement de Vigénère:
\lstinputlisting[firstline=43, lastline=48]{Cchiffr1.py}
\end{enumerate}