\begin{enumerate}
  \item Le compteur nommé $i$ est initialisé à $5$. Une boucle le décrémente tant qu'il est désigne un nombre positif ou nul. \`A la sortie de la boucle, $i$ désigne le premier nombre entier qui n'est pas positif ou nul c'est à dire $-1$. Bien remarquer que le bloc de code dans la boucle ne fait que décrémenter. \`A cause de l'indentation en début de ligne, l'instruction d'affichage ne se trouve pas dans le bloc. Elle se produit donc après la sortie du bloc. Le programme affichera seulement $-1$.
  \item 
\begin{enumerate}
  \item \lstinputlisting[firstline=3, lastline=8]{exo_intro0_C.py}
\lstinputlisting[firstline=10, lastline=14]{exo_intro0_C.py}

  \item \lstinputlisting[firstline=16, lastline=21]{exo_intro0_C.py}
\lstinputlisting[firstline=23, lastline=27]{exo_intro0_C.py}
\end{enumerate}

  \item 
\lstinputlisting[firstline=81, lastline=93]{exo_intro0_C.py}
\lstinputlisting[firstline=41, lastline=53]{exo_intro0_C.py}
Pour n'afficher que le dernier terme, il suffit de faire sortir l'instruction d'affichage du bloc. Avec la syntaxe de Python, il faut le désindenter c'est à dire l'écrire en début de ligne.

  \item Ici on garde invariante la phrase : \og $u$ désigne la valeur de la suite dont l'indice est désigné par $i$\fg.
\lstinputlisting[firstline=55, lastline=63]{exo_intro0_C.py}

  \item Dans le programme suivant, l'affichage des triplets dans le coeur des boucles imbriquées ne sert qu'à expliquer la méthode.
\lstinputlisting[firstline=65, lastline=79]{exo_intro0_C.py}
\end{enumerate}
