%<dscrpt>Exponentiation rapide.</dscrpt>
% <rel_id_elt_parent>6469</rel_id_elt_parent> id de ``Algorithmique''
% <rel_id_type_rel>14</rel_id_type_rel> id de ``contient sans ordre l'élément''
Que fait ce morceau de code ?
\begin{verbatim}
n = 25
mot = "."
while n > 0:
  mot = str(n % 2) + mot
  n = n // 2
print(mot) 
\end{verbatim}
et celui là?
\begin{verbatim}
n = 25
e = 0
while n > 0:
  if n % 2 == 1:
    print(e)
  n = n // 2
  e += 1
\end{verbatim}
Dans le morceau de code suivant,
\begin{verbatim}
n = 250 ; a = 7
p = a ; x = 1
while n > 0:
  if n % 2 == 1:
    x = x*p
  n = n // 2
  p = p*p
print(x) 
\end{verbatim}
que désigne le nom \verb|p| lors du déroulement de l'exécution? Montrer que la valeur de \verb|x| affichée à la fin est $7^{250}$.\newline
Modifier le code précédent en introduisant un compteur \verb|c| permettant de connaître et d'afficher le nombre de multiplications réellement exécutées.

