%!  pour pdfLatex
\documentclass[a4paper]{article}
%\usepackage[hmargin={1.5cm,1.5cm},vmargin={2.4cm,2.4cm},headheight=13.1pt]{geometry}
\usepackage[a4paper,landscape,twocolumn,
            hmargin=1.8cm,vmargin=2.2cm,headheight=13.1pt]{geometry}

\usepackage[pdftex]{graphicx,color}
\usepackage[pdftex,colorlinks={true},urlcolor={blue},pdfauthor={remy Nicolai}]{hyperref}

\usepackage[T1]{fontenc}
\usepackage[utf8]{inputenc}

\usepackage{lmodern}
\usepackage[frenchb]{babel}

\usepackage{fancyhdr}
\pagestyle{fancy}

\usepackage{floatflt}
\usepackage{maths}

\usepackage{parcolumns}
\setlength{\parindent}{0pt}

\usepackage{caption}
\usepackage{subcaption}

\usepackage{makeidx}

\usepackage[french,ruled,vlined]{algorithm2e}
\SetKwComment{Comment}{\#}{}
\SetKwFor{Tq}{tant que}{}{}
\SetKwFor{Pour}{pour}{}{}
\DontPrintSemicolon
\SetAlgoLined

\usepackage{listings}
\lstset{language=Python,frame=single}
\lstset{literate=
  {á}{{\'a}}1 {é}{{\'e}}1 {í}{{\'i}}1 {ó}{{\'o}}1 {ú}{{\'u}}1
  {Á}{{\'A}}1 {É}{{\'E}}1 {Í}{{\'I}}1 {Ó}{{\'O}}1 {Ú}{{\'U}}1
  {à}{{\`a}}1 {è}{{\`e}}1 {ì}{{\`i}}1 {ò}{{\`o}}1 {ù}{{\`u}}1
  {À}{{\`A}}1 {È}{{\'E}}1 {Ì}{{\`I}}1 {Ò}{{\`O}}1 {Ù}{{\`U}}1
  {ä}{{\"a}}1 {ë}{{\"e}}1 {ï}{{\"i}}1 {ö}{{\"o}}1 {ü}{{\"u}}1
  {Ä}{{\"A}}1 {Ë}{{\"E}}1 {Ï}{{\"I}}1 {Ö}{{\"O}}1 {Ü}{{\"U}}1
  {â}{{\^a}}1 {ê}{{\^e}}1 {î}{{\^i}}1 {ô}{{\^o}}1 {û}{{\^u}}1
  {Â}{{\^A}}1 {Ê}{{\^E}}1 {Î}{{\^I}}1 {Ô}{{\^O}}1 {Û}{{\^U}}1
  {œ}{{\oe}}1 {Œ}{{\OE}}1 {æ}{{\ae}}1 {Æ}{{\AE}}1 {ß}{{\ss}}1
  {ű}{{\H{u}}}1 {Ű}{{\H{U}}}1 {ő}{{\H{o}}}1 {Ő}{{\H{O}}}1
  {ç}{{\c c}}1 {Ç}{{\c C}}1 {ø}{{\o}}1 {å}{{\r a}}1 {Å}{{\r A}}1
  {€}{{\euro}}1 {£}{{\pounds}}1 {«}{{\guillemotleft}}1
  {»}{{\guillemotright}}1 {ñ}{{\~n}}1 {Ñ}{{\~N}}1 {¿}{{?`}}1
}

%pr{\'e}sentation des compteurs de section, ...
\makeatletter
\renewcommand{\thesection}{\Roman{section}.}
\renewcommand{\thesubsection}{\arabic{subsection}.}
\renewcommand{\thesubsubsection}{\arabic{subsubsection}.}
\renewcommand{\labelenumii}{\theenumii.}
\makeatother


\newtheorem*{thm}{Théorème}
\newtheorem{thmn}{Théorème}
\newtheorem*{prop}{Proposition}
\newtheorem{propn}{Proposition}
\newtheorem*{pa}{Présentation axiomatique}
\newtheorem*{propdef}{Proposition - Définition}
\newtheorem*{lem}{Lemme}
\newtheorem{lemn}{Lemme}

\theoremstyle{definition}
\newtheorem*{defi}{Définition}
\newtheorem*{nota}{Notation}
\newtheorem*{exple}{Exemple}
\newtheorem*{exples}{Exemples}


\newenvironment{demo}{\renewcommand{\proofname}{Preuve}\begin{proof}}{\end{proof}}
%\renewcommand{\proofname}{Preuve} doit etre après le begin{document} pour fonctionner

\theoremstyle{remark}
\newtheorem*{rem}{Remarque}
\newtheorem*{rems}{Remarques}

%\usepackage{maths}
%\newcommand{\dbf}{\leftrightarrows}

%En tete et pied de page
\lhead{Informatique}
%\chead{Introduction aux systèmes informatiques}
\rhead{MPSI B Hoche}
\lfoot{\tiny{Cette création est mise à disposition selon le Contrat\\ Paternité-Partage des Conditions Initiales à l'Identique 2.0 France\\ disponible en ligne http://creativecommons.org/licenses/by-sa/2.0/fr/  
} 
\rfoot{\tiny{Rémy Nicolai \jobname \; \today } }
}
\makeindex


\usepackage{parcolumns}
\setlength{\parindent}{0pt}

 \begin{document}
\lhead{TP IPT}
\chead{TP 1 (19/09 et 26/09 2016)}
Les objectifs de ce TP sont
\begin{itemize}
  \item  savoir éditer, enregistrer (dans une arborescence de dossiers), exécuter du code python dans un fichier;
  \item se familiariser avec les premiers éléments du langage: indendation, boucle \texttt{while}, structure de contrôle \texttt{if};
  \item se familiariser avec les bonnes pratiques relatives aux algorithmes:
  \begin{itemize}
    \item \og phrases invariantes\fg~ pour la conception,
    \item conventions de codage pour l'implémentation (rédaction).
  \end{itemize}
\end{itemize}
En Python, l'élément de langage \texttt{for} n'est pas vraiment à sa place dans un contexte de boucle, sa portée est beaucoup plus grande. Penser en terme de \og boucle for\fg~ risque de vous priver de la pleine puissance de \texttt{for}; c'est pour cela que je ne l'introduit pas ici, il le sera plus tard dans le contexte des objets \og énumérables\fg.


\section{Enregistrement et accès aux fichiers}
Le premier exercice est de maitriser l'arborescence de fichiers dans laquelle vous travaillez. Créer un dossier dans lequel vous allez placer les fichiers python, il comportera éventuellement des sous-dossiers. \newline
Si vous utilisez votre machine personnelle, vous le placez où vous voulez. Si vous utilisez les machines du lycée, ce dossier doit être à votre nom.\newline
Le réseau entre les stations de travail des salles de TP est mal commode. Pour communiquer entre elles ou entre votre station de travail etune autre ou une machine perso externe, il est plus facile de passer par un hébergeur web du type Google Drive ou Dropbox. Il peut être utile de me donner ponctuellement un accès en lecture seule sur ce dossier de TP Python.

\section{Conventions de codage}
Pour que votre code écrit en Python soit plus facile à comprendre et à relire par vous ou par d'autres, il est utile d'adopter des conventions dans la présentation des fichiers considérées comme de bonnes pratiques par les développeurs. 

\begin{itemize}
  \item taille des indentations: 4 espaces,
  \item  taille maximale d'une ligne: 79 caractères,
  \item toujours placer un espace après une virgule, un point-virgule ou un deux-points (sauf pour la syntaxe des tranches),
  \item ne jamais placer d'espace avant une virgule, un point-virgule ou un deux-points,
  \item toujours placer un espace de chaque côté d'un opérateur,
  \item ne pas placer d'espace entre le nom d'une fonction et la première parenthèse délimitant sa liste d'arguments.
\end{itemize}


\section{Phrase invariante}
Le concept de \og phrase invariante\fg sera détaillé plus tard dans le cours d'algorithmique. Il est introduit ici comme bonne pratique dans la recherche d'une méthode pour résoudre un problème.\newline
Considérons le problème du calcul de la factorielle d'un nombre et la phrase
\begin{quote}
  \og $f$ désigne la factorielle du nombre désigné par $n$.\fg
\end{quote}
Pour résoudre le problème du calcul de $5!$, on va assigner à $f$ et $n$ des valeurs particulières telles que la phrase soit vraie puis modifier les valeurs de telle sorte que la phrase reste vraie jusqu'à ce que $n$ désigne 5.
\lstinputlisting[firstline=3, lastline=16]{phraseinvariante.py}
Remarquer que l'instruction \texttt{n = n + 1} pourrait aussi s'écrire \texttt{n += 1}. De même dans l'autre sens, \texttt{f *= n} pourrait aussi s'écrire \texttt{f = f * n}.

\section{Premiers exercices}
\begin{enumerate}
  \item En utilisant les documents de cours distribués, expliquer sans l'exécuter ce que va faire et va afficher le code suivant:
\lstinputlisting[firstline=1, lastline=4]{exo_intro0_E.py}  
Vérifier qu'il se passe bien ce que vous aviez prévu en l'exécutant après l'avoir enregistré dans un fichier. Que se passe-t-il si on indente le \texttt{print(i)}?

  \item \'Ecrire deux programmes dont l'un exécutera 100 fois un bloc de code contenant un test et l'autre exécutera 49 fois un bloc contenant une incrémentation plus pertinente pour les problèmes suivants.
\begin{enumerate}
  \item Afficher tous les nombres impairs entre $0$ et $100$ par ordre croissant.
  \item Afficher tous les nombres pairs entre $0$ et $100$ par ordre décroissant.
\end{enumerate}

  \item La suite de Fibonacci est définie par
\begin{displaymath}
  f_0 = f_1 = 1, \hspace{1cm}\forall n \in \N,\; f_{n+2} = f_{n+1} + f_n
\end{displaymath}
Le calcul de ses termes doit se faire en utilisant la phrase invariante:
\begin{quote}
  \og $n$ désignant un entier, $f$ désigne $f_n$ et $ff$ désigne $f_{n+1}$\fg
\end{quote}
\'Ecrire deux programmes qui affichent les 20 premiers termes. Le premier programme utilisera les assignations multiples et pas le second. Modifier les programmes pour qu'ils n'affichent que le 20ème terme (c'est à dire $f_{19}$)

  \item En précisant la phrase invariante utilisée, écrire un programme qui affiche les 10 premiers termes de la suite récurrente définie par
\begin{displaymath}
  u_0 = 3, \hspace{0.5cm} \forall n \in \N, \;u_{n+1} = n + 2u_n
\end{displaymath}

  \item (à faire en dernier) \`A l'aide d'un programme utilisant 3 boucles "while" imbriquées, calculer le nombre de triplets $(x,y,z)$ d'entiers de $\llbracket 1, 10 \rrbracket$ tels que
  \begin{displaymath}
  1  \leq x < y < z \leq 10
  \end{displaymath}
Vérifiez en comparant avec une formule pour cette somme obtenue avec les techniques de calcul de somme.
\end{enumerate}


\section{Exemple de séquence d'instructions}
On imagine une pile\footnote{Le terme \emph{pile} désigne en informatique une structure précise. La structure décrite ici est bien une pile particuliere} de nombres avec la contrainte que, de bas en haut, les nombres doivent strictement diminuer 
\begin{displaymath}
\begin{bmatrix}
1 \\ 2  \\ 5                                                                                                          
\end{bmatrix} 
\end{displaymath}
On peut enlever un nombre d'un tas: mais seulement celui du haut. On peut ajouter un nombre à un tas mais uniquement en haut du tas et seulement lorsqu'il est plus petit que celui qui est déjà en haut. Dans ces conditions, si on dispose de plusieurs tas, on peut déplacer un nombre d'un tas à un autre.
\begin{displaymath}
\begin{bmatrix}
2 \\ 5  \\ 7                                                                                                          
\end{bmatrix} 
\begin{bmatrix}
 3  \\ 4                                                                                                          
\end{bmatrix} 
\hspace{0.5cm} \longrightarrow \hspace{0.5cm}
\begin{bmatrix}
 5  \\ 7                                                                                                          
\end{bmatrix} 
\begin{bmatrix}
2 \\ 3  \\ 4                                                                                                          
\end{bmatrix} 
\end{displaymath}
En numérotant les tas "1" et "2" de gauche à droite, l'opération précédente est effectuée par l'instruction \verb|deplacer(1,2)|.
\begin{enumerate}
 \item On part de trois tas dont deux sont vides et on veut déplacer les nombres pour passer d'une configuration à une autre
 \begin{displaymath}
\begin{bmatrix}
1 \\ 2  \\ 3                                                                                                          
\end{bmatrix} 
\begin{bmatrix}
  \\   \\                                                                                                           
\end{bmatrix} 
\begin{bmatrix}
 \\   \\                                                                                                           
\end{bmatrix} 
  \hspace{0.5cm} \longrightarrow \hspace{0.5cm}
\begin{bmatrix}
  \\   \\                                                                                                           
\end{bmatrix} 
\begin{bmatrix}
 \\   \\                                                                                                           
\end{bmatrix} 
\begin{bmatrix}
1 \\ 2  \\ 3                                                                                                          
\end{bmatrix}   
 \end{displaymath}
 
Compléter les listes suivantes d'instructions pour exécuter la tâche.
\begin{itemize}
 \item liste A : \verb|deplacer(1,2)| \verb|deplacer(1,3)| ... 
 \item liste B : \verb|deplacer(1,3)| \verb|deplacer(1,2)| ...
\end{itemize}
\item Démontrer par récurrence que, pour tout entier $n\geq 1$, il existe une séquence d'instructions permettant d'exécuter
 \begin{displaymath}
\begin{bmatrix}
1 \\ 2  \\ \vdots \\ n                                                                                                          
\end{bmatrix} 
\begin{bmatrix}
  \\   \\                                                                                                           
\end{bmatrix} 
\begin{bmatrix}
 \\   \\                                                                                                           
\end{bmatrix} 
  \hspace{0.5cm} \longrightarrow \hspace{0.5cm}
\begin{bmatrix}
  \\   \\                                                                                                           
\end{bmatrix} 
\begin{bmatrix}
 \\   \\                                                                                                           
\end{bmatrix} 
\begin{bmatrix}
1 \\ 2  \\ \vdots \\ n                                                                                                          
\end{bmatrix}   
 \end{displaymath}
(Vous ne devrez pas chercher à expliciter une telle séquence)\newline
Préciser en fonction de $n$, le nombre d'instructions \texttt{deplacer} figurant dans une telle séquence. 
\end{enumerate}



\end{document}