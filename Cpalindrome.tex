\subsubsection{Retournement d'une liste}
Dans l'algorithme de retournement, il faut faire attention à ne pas permuter deux fois.
\begin{itemize}
  \item \verb|i| $\longleftarrow 0$ (index du début)
  \item \verb|n| $\longleftarrow$ (longueur de \verb|L|)$-1$ (index de la fin)
  \item tant que \verb| i < j|
  \begin{itemize}
    \item échanger les valeurs de \verb|L[i]| et \verb|L[j]|
    \item incrémenter \verb|i|
    \item décrémenter \verb|j|
  \end{itemize}
\end{itemize}
Le code Python et un petit test
\begin{verbatim}
def retournL(L):
    i = 0
    j = len(L) - 1
    while i < j:
        L[i] , L[j] = L[j] , L[i]
        i += 1
        j -= 1

L = [1,2,3,4,5,6]
print(L)
retournL(L)
print(L)
>>>>
[1, 2, 3, 4, 5, 6]
[6, 5, 4, 3, 2, 1]
\end{verbatim}

\subsubsection{Décomposition et composition}
\begin{enumerate}
  \item Il s'agit de l'algorithme de numération
\begin{itemize}
  \item \verb|p| $\longleftarrow$ le plus grand des $10^k$ tels que $ 10^k \leq n$
  \item \verb|L| $\longleftarrow$ une liste vide
  \item tant que \verb|p >= 1|
  \begin{itemize}
    \item ajouter le quotient de la division de $n$ par $p$ à \verb|L|
    \item \verb|n| $\longleftarrow$ le reste de la division de $n$ par $p$ à \verb|L|
    \item \verb|p| $\longleftarrow$ $p// 10$
  \end{itemize}
  \item renvoyer \verb|L|
\end{itemize}
L'implémentation en Python est
\begin{verbatim}
def decomp(n):
    p = 1
    while p <= n:
        p *= 10
    p = p // 10
    L = []
    while p >= 1:
        L.append( n // p)
        n = n % p
        p = p // 10
    return L
\end{verbatim}
\item
\begin{enumerate}
  \item Les deux algorithmes renvoient le nombre dont la liste \verb|L| est la représentation décimale. On suppose que cette liste est de longueur $n$.\newline
  Dans le premier algorithme, l'opération \verb|v| $\longleftarrow$ \verb|v + c*10**i| est exécutée $n$ fois. Or cette opération comprend une addition et $i$ multiplication. Le premier algorithme effectue donc $n$ additions et $\frac{n(n+1}{2}$ multiplications.\newline
  Dans le deuxième algorithme, l'opération \verb|v| $\longleftarrow$ \verb|10*v + L[i]| est aussi exécutée $n$ fois mais elle comprend une seule addition et une seule multiplication. Le deuxième algorithme effectue donc $n$ additions et multiplications.  
  \item Le deuxième algorithme est implémenté par le code suivant
\begin{verbatim}
def comp(L):
    v = 0
    i = 0
    while i < len(L):
        v = 10*v + L[i]
        i += 1
    return v
\end{verbatim}
\end{enumerate}

\end{enumerate}

\subsubsection{Retournement d'un nombre}
\begin{enumerate}
  \item \begin{enumerate}
  \item Les fonctions demandées sont immédiates
\begin{verbatim}
def retournN(n):
    L = decomp(n)
    L.reverse()
    return comp(L)
    
def sym(n):
    return n + retournN(n)
\end{verbatim}
\item Le pseudo-code de \verb|delta| est immédiat
\begin{itemize}
  \item \verb|i| $\longleftarrow 0$ (un compteur)
  \item \verb|s| $\longleftarrow$ \verb|retournN(n)|
  \item tant que \verb|s != n| et que \verb|i <= p|
  \begin{itemize}
    \item \verb|n| $\longleftarrow$ \verb| n + s|
    \item \verb|s| $\longleftarrow$ \verb|retournN(n)|
    \item incrémenter \verb|i|
  \end{itemize}
  \item afficher \verb|n|
  \item renvoyer \verb|i|
\end{itemize}
Avec un tel code, \verb|delta(n,p)| est un nombre inférieur ou égal à $p+1$. S'il est inférieur ou égal à $p$ c'est bien $\delta_n$ mais s'il est égal à $p+1$, on sait seulement que $\delta_n$ (s'il existe) est strictement supérieur à $p$.
Code Python
\begin{verbatim}
def delta(n,p):
    i = 0
    s = retournN(n)
    while s != n and i <= p:
        n = n + s
        s = retournN(n)
        i += 1
    print(n)
    return i
\end{verbatim}
\item En utilisant le code
\begin{verbatim}
p = 5000
n = 197
d = 0
nmax = 1
dmax = 0
while d <= p and n <= 300 :
    if d > dmax:
        dmax = d
        nmax = n
    n += 1
    d = delta(n,p)

print(n,nmax,dmax)
\end{verbatim}
d'abord avec $n=1$, on trouve que le premier nombre suspect est 196. Le suivant que j'ai trouvé (qui n'est pas palindromisé par 5000 itérations) est 295. Le nombre 196 a fait l'objet de nombreux calculs. Quelques millions d'itérations ne conduisent pas à un palindrome sans que l'on sache pour autant montrer qu'il est impossible d'obtenir un palindrome par itérations.
\end{enumerate}

\end{enumerate}
