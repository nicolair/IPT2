Pour $n \in \N$, on note $D_n = \llbracket 0, 2^n -1 \rrbracket$, on appelle \og point\fg tout couple d'entiers $(x,y)\in D_n\times D_n$. On cherche \`a calculer efficacement\footnote{d'après X 2017 MP (hors spé. info) PC PSI} l'intersection de deux parties de $D_n\times D_n$.

{\bf Rappels de Python.}\newline
Si {\tt a} est une liste alors {\tt a[i]} d\'esigne le {\tt i}-i\`eme \'el\'ement de cette liste o\`u l'entier {\tt i} est sup\'erieur ou \'egal \`a $0$
et strictement plus petit que la longueur {\tt len(a)} de la liste.\newline
La commande {\tt a[i]=e} affecte la valeur de l'expression {\tt e} au {\tt i}-i\`eme \'el\'ement de {\tt a}.\newline
L'expression {\tt []} construit une liste vide.\newline
L'expression {\tt n*[k]} construit une liste de longueur {\tt n} contenant {\tt n} occurrences de {\tt k}.\newline
La commande {\tt a=list(b)} construit une copie de la liste {\tt b} et l'affecte \`a la variable {\tt a}.\newline
La commande {\tt a.append(x)} modifie la liste {\tt a} en lui rajoutant un nouvel \'el\'ement final contenant la valeur de {\tt x}.\newline
{\bf Important} : seules les op\'erations sur les listes apparaissant dans ce paragraphe sont autoris\'ees dans les r\'eponses. Si une fonction Python standard est n\'ecessaire, elle devra \^etre r\'e\'ecrite.

\subsection{Une solution na\"\i ve.}
Un point de coordonn\'ees $(x,y)\in D_n\times D_n$ est repr\'esent\'e en Python par une liste de deux entiers naturels {\tt [x,y]}.\newline
Un ensemble de points est repr\'esent\'e par une liste de points sans r\'ep\'etition, donc comme une liste de listes d'entiers naturels de longueur $2$.
\begin{enumerate}
\item \'Ecrire une fonction {\tt membre(p,q)} qui renvoie {\tt True} si le point {\tt p} est dans l'ensemble repr\'esent\'e par la liste {\tt q} et qui renvoie {\tt False} dans le cas contraire.
\item \'Ecrire une fonction {\tt intersection(p,q)} qui renvoie une liste repr\'esentant l'intersection des ensembles repr\'esent\'es par {\tt p} et {\tt q}. On impl\'ementera l'algorithme qui consiste \`a it\'erer sur tous les points de {\tt p} et \`a ins\'erer dans le r\'esultat ceux qui sont aussi dans {\tt q}
\item Si la comparaison entre deux entiers naturels est prise comme op\'eration \'el\'ementaire, quelle est la complexit\'e de l'algorithme de la question pr\'ec\'edente exprim\'ee en fonction de la longueur de {\tt p} et {\tt q}~?
\end{enumerate}

\subsection{Codage de Lebesgue.}
Le codage de Lebesgue d'un point de coordonn\'ees $(x,y)\in D_n\times D_n$ s'obtient à partir de l'entrelacement des bits des repr\'esentations binaires de $x$ et $y$ en commençant par les bits de $x$.
On suppose que les bits de poids fort sont situ\'es \`a gauche dans les repr\'esentations binaires des entiers naturels.

Par exemple, dans le cas où $n=3$, si $x = 6$ (c'est à dire $\overline{\mathbf{110}}^2$ en binaire) et $y = 3$ (donc $\overline{011}^2$ en binaire) alors l'entrelacement associé au point de coordonn\'ees $(x,y)$ est $\mathbf{1}0\, \mathbf{1}1\, \mathbf{0}1$. \newline
Chaque paire de bits est la représentation binaire d'un chiffre parmi 0, 1, 2, 3 :
\[
 0 = \overline{00}^2, \; 1 = \overline{01}^2, \; 2 = \overline{10}^2, \;  3 = \overline{11}^2.
\]
Le codage de Lebesgue d'un point consiste à remplacer les paires de l'entrelacement par le chiffre correspondant. On le note avec une barre au dessus et un $l$.\newline
Ainsi, pour $n=3$, le codage de Lebesgue du point $(6,3)$ est $\overline{231}^\ell$.

En Python, un codage de Lebesgue d'un point de $D_n\times D_n$ est stock\'ee dans une liste de longueur $n$ écrite dans le même ordre. Ainsi, si $n=3$, le codage de Lebesgue du point $(6,3)$ est repr\'esent\'e par la liste {\tt [2,3,1]}.

\begin{enumerate}[resume]
\item Soit $n=3$, quelle liste Python repr\'esente le codage du point $(1,6)$~?
\item On suppose que l'on dispose d'une fonctions {\tt bits(x,k)}, qui prend en arguments deux entiers naturels {\tt x} et {\tt k}, et qui renvoie la valeur du bit de coefficient $2^k$ dans la repr\'esentation binaire de {\tt x}.\newline
\'Ecrire une fonction {\tt code(n,p)} qui prend en arguments un entier strictement positif {\tt n} et un point {\tt p} repr\'esent\'e par une liste de longueur 2 dont les deux coordonn\'ees sont prises dans $D_n$.
Cette fonction renvoie le codage de Lebesgue de {\tt p} repr\'esent\'e sous la forme d'une liste Python.
\end{enumerate}

\subsection{Codage dichotomique d'un quadrant.}
Considérons $D_n \times D_n$ comme un carré de côté $2^n$. On peut le couper par son milieu en quatre carrés de côté $2^{n-1}$ et recommencer cette opération (au plus $n$ fois). On appelle \emph{quadrant} un carré qui est une partie  de $D_n \times D_n$ obtenue ainsi. Le plus grand des quadrants est $D_n \times D_n$ lui même (obtenu par $0$ découpage) les plus petits sont les singletons (obtenus par $n$ découpages).\newline
Quelle que soit la taille du carré que l'on découpe, on numérote conventionnellement les quadrants formés.
\begin{figure}[ht]
 \centering
\renewcommand{\arraystretch}{1.5}
\begin{tabular}{|c|c|}\hline 1&3\\ \hline 0&2\\\hline\end{tabular} 
 \caption{Numérotation des quadrants}
 \label{fig: numquad}
\end{figure}

On code un quadrant par la liste des numéros conventionnels des quadrants qui le contiennent dans la séquence de découpages.\newline
Par exemple, la figure \ref{fig: quadrant} représente le carré $D_3\times D_3$ et les points d'un quadrant. L'origine est en bas \`a gauche, les abscisses croissent de gauche \`a droite et les ordonn\'ees du bas vers le haut. Ce quadrant particulier est codé par $(2,1)$.\newline
Un quadrant de $D_n \times D_n$ qui se réduit à un singleton est donc codé par une liste de longueur $n$ à valeurs dans $\left\lbrace 0,1,2,3\right\rbrace$. Un quadrant contenant $2^{n-k}$ points est codé par une liste de longueur $k$. On convient de la compléter par $n-k$ chiffres $4$ pour former une liste de longueur $n$.\newline
Ainsi, pour $n=3$, le quadrant $D_3\times D_3$ (tout entier) est codé en Python par {\tt [4,4,4]} et celui de la figure \ref{fig: quadrant} par {\tt [2,1,4]}.
\begin{figure}[ht]
 \centering
 \setlength{\unitlength}{5mm}
\begin{picture}(8,8)
\multiput(0,0)(1,0){9}{\line(0,1){8}}
\multiput(0,0)(0,1){9}{\line(1,0){8}}
\put(5.5,2.5){\circle*{0.5}}
\put(5.5,3.5){\circle*{0.5}}
\put(4.5,2.5){\circle*{0.5}}
\put(4.5,3.5){\circle*{0.5}}
\end{picture}
 \caption{Exemple de quadrant}
 \label{fig: quadrant}
\end{figure}
\begin{enumerate}[resume]
 \item Quel est le quadrant codé par {\tt [1,1,2]}?
 \item Soit $P\in D_n\times D_n$. Dans la partie 2, on a codé $P$ comme une certaine liste et dans cette partie, on a codé $\left\lbrace P \right\rbrace$ par une certaine liste. En fait ces listes sont identiques. Présenter le principe de démonstration de cette propriété. 
\end{enumerate}
Dans la suite on désignera par codage de Lebesgue le codage d'un quadrant en étendant la notation (avec barre et $l$) utilisée pour le codage d'un point. Lorsque'un quadrant se réduit à un singleton, on identifera ce quadrant avec son unique point. Bien noter que ce qui caractérise un quadrant par rapport à un point est la présence d'au moins un 4 à droite du codage. 

\subsection{Ordre lexicographique sur les quadrants.}
On utilise l'ordre lexicographique (autrement dit, l'ordre du dictionnaire) pour comparer les quadrants à l'aide de leurs codages.\newline
Soient $c=\overline{c_{n-1}\dots c_0}^\ell$ et $d=\overline{d_{n-1}\dots d_0}^\ell$ deux codages de quadrants de $D_n\times D_n$.
On note $c \prec d$ pour \og$c$ est strictement plus petit que $d$\fg si et seulement si
\[
\exists i,\ 0\leq i< n\ \text{ tel que }
\left\lbrace 
\begin{aligned}
 &\forall j,\ j>i\Rightarrow c_j=d_j \\ &c_i < d_i
\end{aligned}
\right. 
\]
Si $c, d$ représentent les quadrants $C,D$, on notera $C \prec D$ si et seulement si $c \prec d$. 
\begin{enumerate}[resume]
\item Trier les codages suivants par ordre croissant pour l'ordre lexicographique~:
\[\{\overline{311}^\ell,\overline{000}^\ell,\overline{012}^\ell,\overline{101}^\ell,\overline{233}^\ell\}\]
\item \'Ecrire une fonction {\tt compare\_pcodes(n,c1,c2)}, qui prend en arguments les codages de deux quadrants de $D_n\times D_n$ et renvoie $0$ s'il sont \'egaux, $1$ si {\tt c1} est strictement plus petit par l'ordre lexicographique que {\tt c2} et $-1$ sinon.
\end{enumerate}
Toute partie $\mathcal{P}$ de $D_n\times D_n$ se décompose de manière unique comme une union de quadrants disjoints. On appelle \og AQL de la partie $\mathcal{P}$\fg la liste ordonnée (lexicographiquement) des codages des quadrants qui la composent.

\begin{enumerate}[resume]
 \item Les figures \ref{fig:S0} et \ref{fig:S1} définissent des parties $S_0$ et $S_1$. Décomposer ces parties en quadrants et former leur AQL.
\end{enumerate}

\begin{figure}[ht]
 \centering
 \begin{subfigure}[t]{3.5cm}
  \centering
 \setlength{\unitlength}{5mm}
\begin{picture}(4,4)
\multiput(0,0)(1,0){5}{\line(0,1){4}}
\multiput(0,0)(0,1){5}{\line(1,0){4}}
\put(0.5,0.5){\circle*{0.5}}
\put(1.5,0.5){\circle*{0.5}}
\put(0.5,1.5){\circle*{0.5}}
\put(1.5,1.5){\circle*{0.5}}
\put(2.5,2.5){\circle*{0.5}}
\put(3.5,0.5){\circle*{0.5}}
\end{picture}
 \caption{$S_0 \subset D_2 \times D_2$.}
 \label{fig:S0}
 \end{subfigure}
 ~
 \begin{subfigure}[t]{5.5cm}
\centering
 \setlength{\unitlength}{5mm}
\begin{picture}(8,8)
\multiput(0,0)(1,0){9}{\line(0,1){8}}
\multiput(0,0)(0,1){9}{\line(1,0){8}}
\put(0.5,6.5){\circle*{0.5}}
\put(1.5,6.5){\circle*{0.5}}
\put(0.5,7.5){\circle*{0.5}}
\put(1.5,7.5){\circle*{0.5}}
\put(2.5,5.5){\circle*{0.5}}
\put(3.5,5.5){\circle*{0.5}}
\put(2.5,4.5){\circle*{0.5}}
\put(3.5,4.5){\circle*{0.5}}
\put(4.5,0.5){\circle*{0.5}}
\put(5.5,0.5){\circle*{0.5}}
\put(6.5,0.5){\circle*{0.5}}
\put(7.5,0.5){\circle*{0.5}}
\put(4.5,1.5){\circle*{0.5}}
\put(5.5,1.5){\circle*{0.5}}
\put(6.5,1.5){\circle*{0.5}}
\put(7.5,1.5){\circle*{0.5}}
\put(4.5,2.5){\circle*{0.5}}
\put(5.5,2.5){\circle*{0.5}}
\put(6.5,2.5){\circle*{0.5}}
\put(7.5,2.5){\circle*{0.5}}
\put(4.5,3.5){\circle*{0.5}}
\put(5.5,3.5){\circle*{0.5}}
\put(6.5,3.5){\circle*{0.5}}
\put(7.5,3.5){\circle*{0.5}}
\put(6.5,6.5){\circle*{0.5}}
\put(7.5,6.5){\circle*{0.5}}
\put(6.5,7.5){\circle*{0.5}}
\put(7.5,7.5){\circle*{0.5}}
\put(0.5,0.5){\circle*{0.5}}
\put(1.5,1.5){\circle*{0.5}}
\put(2.5,2.5){\circle*{0.5}}
\put(3.5,3.5){\circle*{0.5}}
\end{picture}
 \caption{$S_1 \subset D_3 \times D_3$.}
 \label{fig:S1}  
 \end{subfigure}
 \caption{Parties $S_0$ et $S_1$.}
\end{figure}


\subsection{Calcul efficace de l'intersection.}
On souhaite utiliser les AQL pour calculer l'intersection de deux parties de $D_n\times D_n$. Le premier problème est de former l'AQL d'une partie. La question 11 est une instance de ce problème. L'approche visuelle utilisée par un humain est difficile à généraliser. On procède de la manière suivante.\newline
\`A partir d'une liste de points connus par leurs coordonnées, on forme la liste des codage de Lebesgue de ces points puis on ordonne lexicographiquement cette liste.\newline
L'algorithme de compaction doit rassembler les points d'un même quadrant par le codage de ce quadrant (question 12).

\begin{enumerate}[resume]
\item \'Ecrire une fonction {\tt ksuffixe(n,k,q)} qui prend en arguments un entier {\tt n} strictement positif, une liste {\tt q} repr\'esentant le codage d'un quadrant de $D_n\times D_n$ et un entier naturel {\tt k} inf\'erieur strictement \`a {\tt n}. Si les {\tt k} derniers chiffres de la liste {\tt q} ont pour valeur $4$, cette fonction renvoie une nouvelle liste semblable \`a la liste {\tt q} mais dont les {\tt k+1} derniers chiffres valent 4. Sinon, cette fonction renvoie {\tt q} inchang\'ee. Ainsi,\newline 
{\tt ksuffixe(4,2,[0,1,4,4])}  renvoie {\tt [0,4,4,4]}\newline
{\tt ksuffixe(4,2,[0,1,2,4])} renvoie {\tt [0,1,2,4]}.

\item L'algorithme de compaction consiste \`a parcourir $n$ fois la liste ordonnée des codages des points considérés comme des quadrants à un seul élément. L'it\'eration $k$ vise \`a remplacer quatre codages successifs de quadrants formant un quadrant complet de c\^ot\'e $2^{k+1}$ par la repr\'esentation de ce quadrant.\newline
\'Ecrire une fonction {\tt compacte(n,s)} qui prend en arguments un entier strictement positif {\tt n} et une liste tri\'ee {\tt s} représentant les codages de Lebesgue des points d'une partie $\mathcal{P}$ de $D_n\times D_n$. Cette fonction doit renvoyer l'AQL de la partie $\mathcal{P}$.

\item On souhaite comparer les codages de deux quadrants en terme d'inclusion et d'exclusion des ensembles de points qu'ils repr\'esentent.

\'Ecrire une fonction {\tt compare\_ccodes(n,p,q)} qui prend en arguments un entier strictement positif {\tt n} et deux listes {\tt p}, {\tt q} contenant les codages des quadrants $P$, $Q$ de $D_n\times D_n$. Cinq valeurs de retour sont possibles~:
\begin{itemize}
\item[-] l'entier $0$ si les quadrants sont \'egaux~;
\item[-] l'entier $1$ si les quadrants $P$ et $Q$ sont disjoints et $P \prec Q$~;
\item[-] l'entier $-1$ si les quadrants $P$ et $Q$ sont disjoints et $Q \prec P$~;
\item[-] l'entier $2$ si $P\subset Q$~;
\item[-] l'entier $-2$ si $Q\subset P$
\end{itemize}
Par exemple, {\tt compare\_ccodes(3,[1,4,4],[2,4,4])} renvoie 1 et\\
{\tt compare\_ccodes(3,[1,2,4],[1,4,4])} renvoie 2.
\item Pour calculer efficacement l'intersection de deux ensembles de points repr\'esent\'es par leur AQL respectif, on {\it fusionne} les deux listes tri\'ees qui leur correspondent.

En utilisant {\tt compare\_ccodes}, \'ecrire une fonction {\tt intersection(n,p,q)} qui prend en argument un entier strictement positif {\tt n} ainsi que deux AQL {\tt p}, {\tt q} repr\'esentant respectivement deux parties $P$, $Q$ de $D_n\times D_n$. Cette fonction doit renvoyer un AQL repr\'esentant $P\cap Q$.\newline
Le nombre d'appels \`a {\tt compare\_ccodes} effectu\'es par {\tt intersection(n,p,q)} doit \^etre $O({\tt len(p)+len(q)})$.
\end{enumerate}

