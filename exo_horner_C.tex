Il est bien évident que $i$ est un variant pour la boucle.\newline
Pour montrer que $\Phi$ est un invariant, introduisons des notations.  Après l'exécution de $k\in \N$ fois le corps de la boucle, notons $i_p$ l'entier désigné par $i$, $y_p$ l'objet désigné par $y$ et $\Phi_p$ l'évaluation de $\Phi$.\newline
Dans cette convention, $y_0$ et $i_0$ représentent les valeurs désignées \emph{avant} la première exécution. Soit $y_0=a_n$ et $i_0=n$ ce qui montre que $\Phi$ s'évalue à \verb|vrai| à cette étape.\newline
Montrons maintenant que $\Phi_p = \verb|vrai|$ entraîne $\Phi_{p+1} = \verb|vrai|$.
\begin{displaymath}
\left. 
\begin{aligned}
i_{p+1} = i_{p} -1\\
  y_{p+1} = a_{i_{p+1}} +x*y_p\\ 
  \Phi_p = \verb|vrai|\Leftrightarrow y_p=a_{i_p}+\cdots+a_nx^{n-i_p}
\end{aligned}
\right\rbrace 
\Rightarrow
y_{p+1} = a_{i_{p+1}} +x*y_p
\end{displaymath}
d'où $y=a_{i_{p+1}} + a_{i_p}x+\cdots a_nx^{n-i_p+1}$ avec $n-i_p +1 = n-i_{p+1}$.