\subsection*{Partie I.}
\begin{enumerate}
  \item On classe les couples décroissants $(i,j)$ de $a$ selon le premier terme $i$. Le nombre de couples décroissant est donc le nombre de couples décroissants commençant par $0$ plus le nombre couples décroissants commençant par $1$ plus ... Cela s`'écrit
\begin{displaymath}
  d(a) = \sum_{i=0}^{n-1}k(a,i)
\end{displaymath}
On pourrait sommer jusqu'à $n-2$ seulement car il n'y a pas de couples commençant par $n-1$, on considère $k(a,n-1)=0$. 

  \item On suppose que $(i_0, i_0 + 1)$ est un couple décroissant contigu de $a$. Cela se traduit par:
  \begin{displaymath}
    a_{i_0 + 1} < a_{i_0}
  \end{displaymath}
Notons $p=k(a,i_0+1)$ le nombre de couples décroissants commençant par $i_0+1$ et 
\begin{displaymath}
  (i_0+1,j_1),\;(i_0+1,j_2),\;\cdots (i_0+1,j_p)
\end{displaymath}
ces couples. On veut montrer tous les couples 
\begin{displaymath}
  (i_0,i_0+1)\; (i_0,j_1),\;(i_0,j_2),\;\cdots (i_0,j_p)
\end{displaymath}
sont décroissants.
En effet, on sait déjà que $(i_0,i_0+1)$ est un couple décroissant. De plus, pour tout $j>i_0+1$,
\begin{displaymath}
  (i_0,j)\text{ couple décroissant }
  \Leftrightarrow a_{i_0+1} > a_j
  \Rightarrow a_{i_0 } > a_j \text{ car } a_{i_0} > a_{i_0+1}
\end{displaymath}
On en déduit
\begin{displaymath}
  k(a,i_0) \geq k(a,i_0+1) + 1
\end{displaymath}

  \item Ici $a_{i_0}>a_{i_0+1}$ car $(i_0, i_0 + 1)$ est un couple décroissant contigu de $a$. Par définition de $b$, pour tout $j>i_0+1$,
\begin{displaymath}
  b_j < b_{i_0} \Leftrightarrow a_j < a_{i_0 + 1}
\end{displaymath}
De plus, $(i_0, i_0 + 1)$ n'est plus un couple décroissant contigu de $b$ à cause de l'échange effectué. Ceci prouve
\begin{displaymath}
  k(i_0,b) = k(i_0+1 ,a)
\end{displaymath}
De même,
\begin{displaymath}
\forall j>i_0+1,\;   b_j < b_{i_0 +1} \Leftrightarrow a_j < a_{i_0}
\end{displaymath}
On obtient ainsi tous les couples décroissants commençant à $i_0$ sauf $i_0,i_0+1$ donc $k(i_0+1,b) = k(i_0 ,a) - 1$.\newline
Si $i < i_0$, comme $i_0$ et $i_0+1$ sont tous les deux strictement à droite de $i$ la permutation pour passer de $a$ à $b$ change l'ordre mais pas les valeurs, le nombre de couples décroissants sera le même. Lorsque $i_0+1<i$, le changement de $a$ à $b$ n'a rien changé à droite de $i$ donc le nombre de couples décroissants au dela de $i$ sera conservé.\newline
Deux valeurs sont \og presque\fg~ échangées (attention au $-1$) les autres valeurs sont conservées. On en déduit
\begin{displaymath}
  d(b) = d(a) - 1
\end{displaymath}


\end{enumerate}

\subsection*{Partie II.}
\begin{algorithm}
  \Donnees{\;
     $A$ une liste de $l$ objets indexée entre 0 et $l-1$\;
     $i$ un entier entre $0$ et $l-2$
     }
  \Comment{initialisation}
  $j\leftarrow i$\;
  $k\leftarrow$ 0 nb de couples décroissants entre $i$ et $j$\;
  \Tq{ j < l }{
      \Si{A[j]<A[i]}{
        incrémenter $k$\;
      }
      incrémenter $j$\;
  }
  \caption{Pseudo code pour II.1.}
  \label{Ccpledec_1}
\end{algorithm}
\begin{enumerate}
  \item Le pseudo code est présenté comme Algorithme \ref{Ccpledec_1}.
  \item Dans l'implémentation suivante, on a remplacé le \texttt{k} du pseudo code par \texttt{nb} pour éviter un conflit avec le nom de la fonction.
\lstinputlisting[firstline=1, lastline=9]{Ccpledec.py}
  \item On implémente une fonction \texttt{remplirK}:
\lstinputlisting[firstline=11, lastline=16]{Ccpledec.py}

\end{enumerate}

\subsection*{Partie III.}
\begin{enumerate}
  \item Pour trouver le premier couple contigu décroissant après $i$, on imcrémente une variable $j$ initialisée à $i$ tant que les valeurs augmentent.
\lstinputlisting[firstline=18, lastline=23]{Ccpledec.py}
  \item Le pseudo code est présenté comme Algorithme \ref{Ccpledec_2}.\newline
\begin{algorithm}
  \Donnees{\;
     $A$ une liste de $l$ objets indexée entre 0 et $l-1$\;
     }
  \Tq{ PCDC(A) < l-1 }{
      échanger les valeurs du premier couple décroissant contigu\;
  }
  \caption{Pseudo code pour III.3.}
  \label{Ccpledec_2}
\end{algorithm}
On doit prouver que l'on sort de la boucle (terminaison) et que les valeurs de la liste \texttt{A} sont alors rangées par ordre croissant (correction).\newline
La correction est facile à prouver. La proposition \og la suite est croissante jusqu'à la valeur renvoyée par \texttt{PCDC(A)}\fg est invariante. Si on est sorti de la boucle, c'est que \texttt{PCDC(A)} a renvoyé la longueur de la liste c'est à dire que la suite associée complète est croissante. \newline
La preuve de la terminaison est plus difficile. Elle repose sur l'existence d'un variant qui est le nombre de couples décroissants de $A$. Il s'agit bien d'un nombre entier. D'après la partie I, ce nombre décroît strictement de 1 chaque fois que s'exécute l'échange à l'intérieur de la boucle. 

  \item On implémente l'algorithme présenté dans la question précédente dans la fonction \texttt{tribulle}.
\lstinputlisting[firstline=25, lastline=30]{Ccpledec.py}
\end{enumerate}

