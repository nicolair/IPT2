\begin{enumerate}
 \item Soit $x\in ]0,1[$. Montrer qu'il existe un unique couple $(a,y)\in \N^* \times [0,1[$ tel que
 \begin{displaymath}
  x =  \frac{1}{a+y}
 \end{displaymath}
On note $h(x)$ l'unique $y$ ainsi formé à partir de $x$. Ceci définit une fonction $h$ de $]0,1[$ dans $[0,1[$.

\item Former le code Python implementant le pseudo-code
\begin{verbatim}
x <-- un float > 0
tant que h(x) > 0
  x <-- h(x)
\end{verbatim}
Vous ne devez pas écrire de fonction ou de procedure  (la syntaxe de ces objets n'a pas encore été introduite). Que peut-on dire de $x$ lorsque la boucle se termine ?
\item Comme rien ne justifie que la boucle de l'algorithme précédent se termine, ajouter ce qu'il faut pour former un code qui n'exécute le bloc que $n$ fois au plus ($n$ entier fixé à l'avance). Expérimentez l'exécution de ce code qui se termine toujours avec quelques valeurs de $x$.

\item Existe-t-il un (ou plusieurs) $x>0$ tels que
\begin{displaymath}
 x = 1 + \frac{1}{2+x}
\end{displaymath}
Que peut-on en conclure quant à la terminaison de la boucle?
\end{enumerate}
