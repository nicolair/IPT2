\begin{enumerate}
  \item Calcul des coefficients du binôme avec des listes à partir de la définition par le triangle de Pascal.\newline
On se donne un entier naturel \texttt{nmax}. On doit fabriquer une liste de listes nommée \texttt{CoeffBin}. Pour chaque $n$ entre 0 et la valeur de \texttt{nmax}, la liste \texttt{CoeffBin[n]} doit contenir les $n+1$ coefficients du binôme $\binom{n}{k}$ avec $k\in \llbracket 0,n\rrbracket$.

 \item Qu'affiche le code suivant ?
\begin{verbatim}
n = 6
p = 4
num = 1
den = 1
for i in range(p):
    num *= n - i
    den *= p - i
print(num // den)\end{verbatim}
Pourquoi utiliser un double slash plutôt qu'un simple?

\item Former un code qui retourne une liste donnée en utilisant une boucle \texttt{while  i < j} avec deux indices \texttt{i} (à incrémenter) et \texttt{j} (à décrémenter).\newline
La même méthode peut-elle être utilisée avec une chaîne de caractères? Coder le retournement d'une chaîne de caractères.\newline
Les listes possèdent une méthode \texttt{reverse} qui réalise ce retournement; la tester.
 \end{enumerate}
