%!  pour pdfLatex
\documentclass[a4paper]{article}
%\usepackage[hmargin={1.5cm,1.5cm},vmargin={2.4cm,2.4cm},headheight=13.1pt]{geometry}
\usepackage[a4paper,landscape,twocolumn,
            hmargin=1.8cm,vmargin=2.2cm,headheight=13.1pt]{geometry}

\usepackage[pdftex]{graphicx,color}
\usepackage[pdftex,colorlinks={true},urlcolor={blue},pdfauthor={remy Nicolai}]{hyperref}

\usepackage[T1]{fontenc}
\usepackage[utf8]{inputenc}

\usepackage{lmodern}
\usepackage[frenchb]{babel}

\usepackage{fancyhdr}
\pagestyle{fancy}

\usepackage{floatflt}
\usepackage{maths}

\usepackage{parcolumns}
\setlength{\parindent}{0pt}

\usepackage{caption}
\usepackage{subcaption}

\usepackage{makeidx}

\usepackage[french,ruled,vlined]{algorithm2e}
\SetKwComment{Comment}{\#}{}
\SetKwFor{Tq}{tant que}{}{}
\SetKwFor{Pour}{pour}{}{}
\DontPrintSemicolon
\SetAlgoLined

\usepackage{listings}
\lstset{language=Python,frame=single}
\lstset{literate=
  {á}{{\'a}}1 {é}{{\'e}}1 {í}{{\'i}}1 {ó}{{\'o}}1 {ú}{{\'u}}1
  {Á}{{\'A}}1 {É}{{\'E}}1 {Í}{{\'I}}1 {Ó}{{\'O}}1 {Ú}{{\'U}}1
  {à}{{\`a}}1 {è}{{\`e}}1 {ì}{{\`i}}1 {ò}{{\`o}}1 {ù}{{\`u}}1
  {À}{{\`A}}1 {È}{{\'E}}1 {Ì}{{\`I}}1 {Ò}{{\`O}}1 {Ù}{{\`U}}1
  {ä}{{\"a}}1 {ë}{{\"e}}1 {ï}{{\"i}}1 {ö}{{\"o}}1 {ü}{{\"u}}1
  {Ä}{{\"A}}1 {Ë}{{\"E}}1 {Ï}{{\"I}}1 {Ö}{{\"O}}1 {Ü}{{\"U}}1
  {â}{{\^a}}1 {ê}{{\^e}}1 {î}{{\^i}}1 {ô}{{\^o}}1 {û}{{\^u}}1
  {Â}{{\^A}}1 {Ê}{{\^E}}1 {Î}{{\^I}}1 {Ô}{{\^O}}1 {Û}{{\^U}}1
  {œ}{{\oe}}1 {Œ}{{\OE}}1 {æ}{{\ae}}1 {Æ}{{\AE}}1 {ß}{{\ss}}1
  {ű}{{\H{u}}}1 {Ű}{{\H{U}}}1 {ő}{{\H{o}}}1 {Ő}{{\H{O}}}1
  {ç}{{\c c}}1 {Ç}{{\c C}}1 {ø}{{\o}}1 {å}{{\r a}}1 {Å}{{\r A}}1
  {€}{{\euro}}1 {£}{{\pounds}}1 {«}{{\guillemotleft}}1
  {»}{{\guillemotright}}1 {ñ}{{\~n}}1 {Ñ}{{\~N}}1 {¿}{{?`}}1
}

%pr{\'e}sentation des compteurs de section, ...
\makeatletter
\renewcommand{\thesection}{\Roman{section}.}
\renewcommand{\thesubsection}{\arabic{subsection}.}
\renewcommand{\thesubsubsection}{\arabic{subsubsection}.}
\renewcommand{\labelenumii}{\theenumii.}
\makeatother


\newtheorem*{thm}{Théorème}
\newtheorem{thmn}{Théorème}
\newtheorem*{prop}{Proposition}
\newtheorem{propn}{Proposition}
\newtheorem*{pa}{Présentation axiomatique}
\newtheorem*{propdef}{Proposition - Définition}
\newtheorem*{lem}{Lemme}
\newtheorem{lemn}{Lemme}

\theoremstyle{definition}
\newtheorem*{defi}{Définition}
\newtheorem*{nota}{Notation}
\newtheorem*{exple}{Exemple}
\newtheorem*{exples}{Exemples}


\newenvironment{demo}{\renewcommand{\proofname}{Preuve}\begin{proof}}{\end{proof}}
%\renewcommand{\proofname}{Preuve} doit etre après le begin{document} pour fonctionner

\theoremstyle{remark}
\newtheorem*{rem}{Remarque}
\newtheorem*{rems}{Remarques}

%\usepackage{maths}
%\newcommand{\dbf}{\leftrightarrows}

%En tete et pied de page
\lhead{Informatique}
%\chead{Introduction aux systèmes informatiques}
\rhead{MPSI B Hoche}
\lfoot{\tiny{Cette création est mise à disposition selon le Contrat\\ Paternité-Partage des Conditions Initiales à l'Identique 2.0 France\\ disponible en ligne http://creativecommons.org/licenses/by-sa/2.0/fr/  
} 
\rfoot{\tiny{Rémy Nicolai \jobname \; \today } }
}
\makeindex

%En tete et pied de page
\lhead{IPT}
\chead{Corrigé DS 1 le 9/12/15}
\begin{document}
\section{Questions de cours.}
\begin{enumerate}
  \item Un \og segment itératif\fg~ est une boucle de type \og tant que \fg. On doit montrer deux points :
\begin{itemize}
  \item que l'on sort de la boucle,
  \item que la boucle résoud le problème proposé.
\end{itemize}
Pratiquement, la preuve de terminaison se fait en justifiant l'existence d'une fonction à valeurs dans $\N$ et qui décroit strictement à chaque itération. Pour la deuxième propriété, on caractérise le problème par une proposition qui doit être vérifiée à la fin de la boucle.
  
  \item Quels entiers relatifs sont représentables en 64 bits sur les machines actuelles?
Les machines actuelles utilisent la représentation des entiers par complément à $2$. Les entiers relatifs de
\begin{displaymath} 
  \llbracket -2^{63} , 2^{63} -1 \rrbracket
\end{displaymath}
sont représentables.
\begin{itemize}
  \item Les entiers de $\llbracket -2^{63} , 2^{63} -1 \rrbracket$ sont représentés par leur décomposition binaire sur 64 bits. Le bit de poids fort est $0$.
  \item Les entiers $x$ de $\llbracket -2^{63} , 2^{63} -1 \rrbracket$ sont représentés par la décomposition binaire de $2^{64} -x$ sur 64 bits. Le bit de poids fort est $1$.
\end{itemize}

  \item Quelle est le représentant par complément à 2 sur 8 bits de $-8$ ? \newline
On représente $-8$ par la décomposition binaire de $-8 + 2^{8}$
\begin{displaymath}
  2^{8} - 2^{3} = (2-1)(2^{7} +2^{6} +2^{5} +2^{4} +2^{3})
\end{displaymath}
soit
\begin{center}
\begin{tabular}{|l|l|l|l|l|l|l|l|} \hline
bit de poids fort &   &   &   &   &   &   & bit de poids faible \\ \hline
1 & 1 & 1 & 1 & 1 & 0 & 0 & 0 \\ \hline
\end{tabular}
\end{center}

\end{enumerate}

\section{Listes. Fermeture transitive.}

\begin{enumerate}
  \item 
\begin{enumerate}
  \item L'initialisation de \texttt{R0} est présenté dans les lignes 1 et 2 du code présenté à la fin. Le code est sur 2 lignes uniquement pour faciliter la présentation. Il faut reformer une seule ligne pour l'exécution.
  \item L'initialisation se fait en deux temps. On assigne d'abord la valeur \texttt{False} partout. Puis, en parcourant le tableau \texttt{R0}, on assigne les valeurs \texttt{True} aux couples en relation. Voir les lignes 5, 6, 7 du code.\newline
Pour la première assignation, il faut faire attention à ne pas utiliser\newline
\texttt{r = [[False]*n]*n}
Car on multiplie les références à un \emph{même} tableau élémentaire. 

  \item On utilise trois boucles imbriquées (lignes 10 à 14).
\end{enumerate}

  \item 
\begin{enumerate}
  \item La relation associée à $\mathcal{R}$ n'est pas transitive si et seulement si
\begin{displaymath}
 \exists (i,j,k) \in \Omega^3 \text{ tel que } a \mathcal{R} b \text{ et } b \mathcal{R} c \text{ et } \left( a \mathcal{R} c \text{ FAUX}\right) 
\end{displaymath}

  \item Avec le tableau à valeurs booléennes, la condition s'écrit
\begin{multline*}
 \exists (i,j,k) \in \Omega^3 \text{ tel que } \texttt{r[a][b] == TRUE} \text{ et } \texttt{r[b][c] == TRUE}\\ \text{ et } \texttt{r[a][c] == FALSE} 
\end{multline*}
\end{enumerate}
Dans ce code, les lignes ont été coupées pour permettre l'insertion dans le texte. Il convient de les recoller pour exécuter.
\lstinputlisting[firstline=1, lastline=25]{Cfermtrans.py}
  \item La condition booléenne \texttt{pasTrans} caractérise la non transitivité du tableau. Elle est vraie si et seulement si il existe un triplet comme dans les question précédentes. La boucle s'effectue tant que le tableau n'est pas transitif. On ne demandait pas de fonction de terminaison. Le nombre de \texttt{False} dans le tableau en est une. 
  
\end{enumerate}

\section{Plus petit commun multiple.}
\begin{enumerate}
  \item La fonction \texttt{continuer(lili)} renvoie une valeur booléenne. La boucle se termine avec $i=p$ ou (exclusif) $lili[i] \neq lili[i-1]$. Dans le premier cas toutes les valeurs sont égales. Onpeut donc conclure que \texttt{continuer(lili)} renvoie 
\begin{itemize}
  \item \texttt{TRUE} si toutes les valeurs de \texttt{lili} ne sont pas égales entre elles,
  \item \texttt{FALSE} si toutes les valeurs de \texttt{lili} sont égales entre elles.
\end{itemize}

  
  \item
\begin{enumerate}
  \item Il existe des multiples communs à $a_0, \cdots , a_{p-1}$, par exemple le produit de tous. L'ensemble des multiples communs est donc une partie non vide de $\N^*$. Elle admet un plus petit élément (propriété de $\N$). Il existe des entiers $q_0,\cdots q_{p-1}$ tels que $\mu = q_ia_i$ pour tous les $i \in \llbracket 0, p-1\rrbracket$.
  
  \item La condition \texttt{continuer(M)} est vérifiée si et seulement si les valeurs du tableau \texttt{M} ne sont pas toutes égales entre elles c'est à dire si et seulement si la plus petite des valeurs (notons la $v$) est strictement plus petite que la plus grande (notons la $V$).\newline
  Toutes les valeurs de \texttt{M} sont inférieures ou égales à $\mu$ si et seulement si $V \leq \mu$. Cette condition est réalisée à l'initialisation car $\mu$ est un multiple de $V$. Montrons qu'elle est invariante.\newline
  Après $k$ itérations notons $V_k$ la plus grande valeur et $v_k$ la plus petite. Supposons $V_k\leq \mu$ et que l'on entre dans l'itération suivante c'est à dire $v_k < V_k$. Une seule valeur change, celle de l'indice $i$ tel que $M[i]=v_k$. On en déduit que
\begin{displaymath}
  V_{k+1}=
\left\lbrace 
\begin{aligned}
  V_k        &\text{ si } v_k + a_i \leq V_k \\
  v_k + a_i &\text{ si } v_k + a_i > V_k
\end{aligned}
\right. 
\end{displaymath}
Si $V_{k+1}=V_k$, on a toujours $V_{k+1}\leq \mu$. Dans l'autre cas, $V_{k+1}=v_k+a_i$. Remarquons que l'on ne fait qu'ajouter dans le tableau les valeurs initiales. Il existe donc un \emph{entier} $u_i$ tel que 
\begin{displaymath}
v_k = u_i a_i< V_k \leq \mu = q_i a_i \Rightarrow u_i < q_i  \Rightarrow u_i + 1 \leq q_i \Rightarrow V_{k+1} \leq \mu 
\end{displaymath}

  \item Lors du déroulement de la boucle, la somme $S$ des valeurs de $M$ est strictement croissante. De plus $S \leq p \mu$ d'après l'invariant de la question précédente. On en déduit que $p \mu - S$ est une fonction de terminaison.\newline
Après la sortie de la boucle, toutes les valeurs de $M$ sont égales entre elles. Cette valeur est un multiple commun à tous les $a_i$ par construction de l'algorithme elleest donc supérieure ou égale à $\mu$ . Elle est forcément égale à $\mu$ car lors de l'itération précédente toutes les valeurs étaient inférieures à $\mu$ et une seule a changé.   
\end{enumerate}

  \item
\begin{enumerate}
  \item La recherche du plus petit indice est usuelle.
\lstinputlisting[firstline=8, lastline=16]{Cppcm.py}

  \item Il faut recopier les \emph{valeurs} de la liste initiale dans la variable locale \texttt{M} car ce sont elles qui sont ajoutées et non assigner deux alias à la même liste.
\lstinputlisting[firstline=18, lastline=25]{Cppcm.py}
\end{enumerate}

\end{enumerate}

\section{Calculs en binaire.}
\begin{enumerate}
  \item Division euclidienne : $136 = 10 \times 13 + 6$ donc $\frac{136}{13} = 10 + \frac{6}{13}$ avec $10 = 2^3 + 2^1$.\newline
Algorithme \og les grands d'abord\fg
\begin{multline*}
\frac{6}{13}: 0. ;\hspace{0.5cm} \frac{12}{13}: 0.0 ;\hspace{0.5cm} \frac{11}{13}: 0.01 ;\hspace{0.5cm} \frac{9}{13}: 0.011 ;\hspace{0.5cm} \frac{5}{13}: 0.0111;\\
\frac{10}{13}: 0.01110 ;\hspace{0.5cm} \frac{7}{13}: 0.011101 ;\hspace{0.5cm}\frac{1}{13}: 0.0111011 ;\hspace{0.5cm}\frac{2}{13}: 0.01110110 ;\\
\frac{4}{13}: 0.011101100 ;\hspace{0.5cm}\frac{8}{13}: 0.0111011000;\hspace{0.5cm} \frac{3}{13}: 0.01110110001;\\
\frac{6}{13}: 0.011101100010;\hspace{0.5cm} \cdots  \text{ périodicité }
\end{multline*}
La décomposition cherchée est $ 1010.01110110001\,01110110001\, \cdots$.

  \item On présente les résultats dans un tableau. Les décompositions sont obtenues pour le premier et le troisième nombre à partir des puissance de $2$ fournies
\begin{center}
\renewcommand{\arraystretch}{1.6}
\begin{tabular}{|c|c|c|c|} \hline
nombre          & décomposition             & exposant & mantisse                 \\ \hline
$257$           & $2^8 + 2^0$               & $8$      & $1.000000010\cdots0$     \\ \hline
$\frac{1}{31}$  & $2^{-5} + 2^{-10}+\cdots$ & $-5$     & $1.00001\,00001\,\cdots$ \\ \hline
$512.001953125$ & $2^9 + 2^{-9}$            & $9$      & $1. \underset{17 \text{ zéros}}{\underbrace{0\cdots0}} 10\cdots 0$\\ \hline
\end{tabular}
\end{center}
Pour la fraction:
\begin{displaymath}
  \frac{1}{31} = \frac{1}{2^5 - 1} = \frac{2^{-5}}{1-2^{-5}} = 2^{-5}\left(1 + 2^{-5} + 2^{-10} + 2^{-15} + \cdots  \right) 
\end{displaymath}

  \item La boucle permet de calculer les valeurs de suites définies par récurrence. Les noms \texttt{x} et \texttt{xx} désignent les valeurs de la suite définie par
\begin{displaymath}
x_0 = 2, \; x_1 = 2.5,\; x_{n+2} = \frac{5}{2} x_{n+1} - x_n   
\end{displaymath}
Le nom \texttt{y} désigne les valeurs de la suites géométrique $y_n = 2^n$.\newline
On peut remarquer que lors de l'évaluation du test : \texttt{xx} désigne $x_{i+1}$ et \texttt{y} désigne $2^{i+1}$.\newline
Les données sont initialisées comme des nombres en virgule flottante.\newline
L'étude mathématique de la suite des $x_n$ conduit à
\begin{displaymath}
  \forall n \in \N,\; x_n = 2^{n} - 2^{-n}
\end{displaymath}
Mathématiquement, la condition $x^{i+1} \neq 2^{i+1}$ sera toujours vérifiée. Ce n'est pas le cas numériquement. Un erreur \og d'underflow\fg~ va se produire lorsque la mantisse du nombre normalisé ne peut plus représenter (sur 52 bits) la puissance négative.
\begin{displaymath}
  2^{i+1} + 2^{-i-1} = 2^{i+1}\left( 1 + 2^{-2i-2}\right) 
\end{displaymath}
Dès que $2i+2>52$ c'est à dire pour $i=26$, on aura \texttt{xx == y} et le programme s'achevera en affichant 26.
\end{enumerate}
On reproduit les affichages produits par l'exécution du code
\begin{figure}[htp]
  \centering
\begin{verbatim}
(4.25, 4.0, 0.25)
(8.125, 8.0, 0.125)
(16.0625, 16.0, 0.0625)
(32.03125, 32.0, 0.03125)
(64.015625, 64.0, 0.015625)
(128.0078125, 128.0, 0.0078125)
(256.00390625, 256.0, 0.00390625)
(512.001953125, 512.0, 0.001953125)
(1024.0009765625, 1024.0, 0.0009765625)
(2048.00048828125, 2048.0, 0.00048828125)
(4096.000244140625, 4096.0, 0.000244140625)
(8192.000122070312, 8192.0, 0.0001220703125)
(16384.000061035156, 16384.0, 6.103515625e-05)
(32768.00003051758, 32768.0, 3.0517578125e-05)
(65536.00001525879, 65536.0, 1.52587890625e-05)
(131072.0000076294, 131072.0, 7.62939453125e-06)
(262144.0000038147, 262144.0, 3.814697265625e-06)
(524288.0000019073, 524288.0, 1.9073486328125e-06)
(1048576.0000009537, 1048576.0, 9.5367431640625e-07)
(2097152.000000477, 2097152.0, 4.76837158203125e-07)
(4194304.000000238, 4194304.0, 2.384185791015625e-07)
(8388608.00000012, 8388608.0, 1.1920928955078125e-07)
(16777216.00000006, 16777216.0, 5.960464477539063e-08)
(33554432.00000003, 33554432.0, 2.9802322387695312e-08)
(67108864.00000001, 67108864.0, 1.4901161193847656e-08)
(134217728.0, 134217728.0, 0.0)
26
\end{verbatim}
\caption{Exécution et terminaison du code encadré en IV 3.}
\end{figure}


\end{document}