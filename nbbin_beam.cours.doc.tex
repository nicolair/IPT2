%!  pour pdfLatex
\documentclass{beamer}

%\usepackage[pdftex]{graphicx,color}
%\usepackage[pdftex,colorlinks={true},urlcolor={blue},pdfauthor={remy Nicolai}]{hyperref}

\usepackage[utf8]{inputenc}
\usepackage[T1]{fontenc}
\usepackage{lmodern}
\usepackage[frenchb]{babel}

\usetheme{Warsaw}

%\usepackage{fancyhdr}
%\pagestyle{fancy}

%\usepackage{floatflt}
\usepackage{maths}

\usepackage{parcolumns}
\setlength{\parindent}{0pt}

%\usepackage{caption}
%\usepackage{subcaption}

\usepackage[french,ruled,vlined]{algorithm2e}
\SetKwComment{Comment}{\#}{}
\SetKwFor{Tq}{tant que}{}{}
\SetKwFor{Pour}{pour}{}{}
\DontPrintSemicolon
\SetAlgoLined

\usepackage{listings}
\lstset{language=Python,frame=single}

%pr{\'e}sentation des compteurs de section, ...
\makeatletter
\renewcommand{\thesection}{\Roman{section}.}
\renewcommand{\thesubsection}{\arabic{subsection}.}
\renewcommand{\thesubsubsection}{\arabic{subsubsection}.}
%\renewcommand{\labelenumii}{\theenumii.}
\makeatother


\newtheorem*{thm}{Théorème}
\newtheorem{thmn}{Théorème}
\newtheorem*{prop}{Proposition}
\newtheorem{propn}{Proposition}
\newtheorem*{pa}{Présentation axiomatique}
\newtheorem*{propdef}{Proposition - Définition}
\newtheorem*{lem}{Lemme}
\newtheorem{lemn}{Lemme}

\theoremstyle{definition}
\newtheorem*{defi}{Définition}
\newtheorem*{nota}{Notation}
\newtheorem*{exple}{Exemple}
\newtheorem*{exples}{Exemples}


\newenvironment{demo}{\renewcommand{\proofname}{Preuve}\begin{proof}}{\end{proof}}
%\renewcommand{\proofname}{Preuve} doit etre après le begin{document} pour fonctionner

\theoremstyle{remark}
\newtheorem*{rem}{Remarque}
\newtheorem*{rems}{Remarques}

%\usepackage{maths}
%\newcommand{\dbf}{\leftrightarrows}

%En tete et pied de page
%\lhead{Informatique}
%\chead{Introduction aux systèmes informatiques}
%\rhead{MPSI B Hoche}
%\lfoot{\tiny{Cette création est mise à disposition selon le Contrat\\ Paternité-Partage des Conditions Initiales à l'Identique 2.0 France\\ disponible en ligne http://creativecommons.org/licenses/by-sa/2.0/fr/
%} }
%\rfoot{\tiny{Rémy Nicolai \jobname}}

\nonstopmode

\begin{document}

\begin{frame}
  \frametitle{Opérations sur les mots binaires: poids fort ou faible}
\begin{itemize}
  \item Mots ( ou listes) de $n$ bits avec $n$ entier naturel fixé.
  \item Un tel mot peut prendre $2^n$ valeurs.
  \item Le \emph{bit de poids fort} est son bit le plus à gauche.
  \item Le \emph{bit de poids faible} est celui le plus à droite.
\end{itemize}
\end{frame}

\begin{frame}
  \frametitle{Opérations sur les mots binaires: décalage}
Décalage vers la droite $\delta$ ou vers la gauche $\gamma$.
  \begin{align*}
    (b_0,b_1,\cdots,b_{n-1}) &\xrightarrow{\delta} (0, b_0,b_1,\cdots,b_{n-2}) \\
    (b_0,b_1,\cdots,b_{n-1}) &\xrightarrow{\gamma} (b_1,b_2,\cdots,b_{n-1}, 0)
  \end{align*}
\begin{itemize}
  \item Ces opérations font perdre chaque fois un bit d'information.
  \item Pour $p<n$, on peut décaler $p$ fois vers la droite ou vers la gauche en répétant l'opération.
\end{itemize}
\end{frame}

\begin{frame}
  \frametitle{Opérations sur les mots binaires: addition bit à bit} 
\begin{itemize}
  \item On procède de droite à gauche en introduisant une \emph{retenue} de $1$ (dans le cas $1+1$) exactement comme pour l'addition en base 10.
  \item L'éventuelle retenue au delà du bit de poids fort est ignorée.
\end{itemize}
  Exemple avec un octet
  \begin{center}
  \begin{tabular}{cccccccc}
    1 & 1 & 0 & 0 & 1 & 1 & 0 & 1 \\ 
    1 & 0 & 1 & 0 & 1 & 0 & 0 & 0 \\ \hline
    0 & 1 & 1 & 1 & 0 & 1 & 0 & 1 \\ 
  \end{tabular}    
  \end{center}
\end{frame}

\begin{frame}
  \frametitle{Interprétations}
Soit $m$ mot représentant binaire d'un entier $\overline{m}\in \llbracket 0, 2^{n}-1 \rrbracket$.
\begin{itemize}
  \item Bit de poids faible = reste de la division de $\overline{m}$ par $2$.
  \item $\delta(m)$ = représentant binaire du quotient de la division de $\overline{m}$ par $2$.
  \item L'entier représenté par $\gamma(m)$ est congru à $2\times\overline{m}$ modulo $2^{n}$. Il y a égalité lorsque le bit de poids fort de $m$ est nul.
\end{itemize}
\end{frame}

\begin{frame}
  \frametitle{Interprétation de l'addition bit à bit}
Pour deux mots $m_1$ et $m_2$ représentant des entiers $\overline{m}_1$ et $\overline{m}_2$ de $\overline{m}\in \llbracket 0, 2^{n}-1 \rrbracket$, et $m$ obtenu par l'addition bit à bit:
\begin{displaymath}
  \overline{m} \equiv \overline{m_1} + \overline{m_2} \mod 2^{n}
\end{displaymath}
\begin{itemize}
  \item \'Egalité si et seulement si l'addition ne conduit pas à ignorer la dernière retenue 
  \item Réalisé lorsque lorsque les deux bits de poids fort sont nuls.
\end{itemize}
\end{frame}

\begin{frame}
  \frametitle{Multiplication}
Combiner addition bit à bit et décalage pour multiplier modulo $2^n$.\newline
Exemple
  \begin{center}
  \begin{tabular}{ccccccccc}
    $m_1$: & 1 & 0 & 1 & 0 & 1 & 0 & 0 & 0 \\
    $m_2$: & 0 & 0 & 1 & 0 & 1 & 1 & 0 & 1 \\ \hline 
           & 1 & 0 & 1 & 0 & 1 & 0 & 0 & 0 \\
           & 1 & 0 & 1 & 0 & 0 & 0 & . & .  \\
           & 0 & 1 & 0 & 0 & 0 & . & . & . \\
           & 0 & 0 & 0 & . & . & . & . & . \\ \hline
    $m$:   & 1 & 0 & 0 & 0 & 1 & 0 & 0 & 0
  \end{tabular}    
\end{center}
  
\begin{displaymath}
\left. 
\begin{aligned}
\overline{m_1} &= 8 + 32 + 128 &= 168 \\
\overline{m_2} &= 1 + 4 + 8 + 32 &= 45 \\
\overline{m} &=  8 + 128 &= 136
\end{aligned}
\right\rbrace \Rightarrow
\overline{m_1}\times \overline{m_2} = 7560 =136 + 29\times 2^8 
\end{displaymath}
\end{frame}

\begin{frame}
  \frametitle{Représentation des entiers}
Avec des mots binaires de longueur $n$, on représente les $2^n$ entiers de 
\begin{displaymath}
  \llbracket -2^{n-1}, 2^{n-1}-1 \rrbracket
\end{displaymath}
\begin{itemize}
  \item Les nombres positifs sont représentés normalement.
  \item le bit de poids fort est toujours nul. Par exemple
\begin{multline*}
  \overline{00\cdots00} = 0, \hspace{.5cm}\overline{00\cdots01} = 1, \hspace{.5cm}\overline{00\cdots10} = 2,\\
  \cdots,\hspace{.5cm}  \overline{011\cdots11} = 2^{n-1}-1
\end{multline*}
\end{itemize}
\end{frame}


\begin{frame}
  \frametitle{Représentation par complément à 2}
\begin{itemize}
  \item Un nombre négatif $x\in \llbracket -2^{n-1}, -1 \rrbracket$ est représenté par la décomposition binaire du nombre congru à $x$ modulo $2^{n}$ dans l'intervalle c'est à dire $x + 2^n$.
  \item Le bit de poids fort de la représentation d'un entier négatif est toujours égal à 1. 
\begin{displaymath}
  -2^{n-1} \leq x \Rightarrow x + 2^n \geq 2^n - 2^{n-1} = 2^{n-1}
\end{displaymath}

\end{itemize}
\end{frame}

\begin{frame}
  \frametitle{Représentation de $-x$.}
Soit $x$ entier tel que $0<x<2^{n-1}$ et $(0,b_{n-2},\cdots,b_1,b_0)$ sa représentation binaire sur $n$ bits.\newline
Comment obtenir la représentation de $-x$? 
\begin{itemize}
  \item on inverse le développement: $(1,1-b_{n-2},\cdots,1-b_1,1-b_0)$
  \item on ajoute $1$ (attention aux retenues)
\end{itemize}
\end{frame}

\begin{frame}
  \frametitle{Exemple $-1000$ sur 2 octets.}
\begin{itemize}
  \item représentation de $1000$: $0000001111101000$
  \item inversion $1111110000010111$
  \item on ajoute 1, représentation de $-1000$ : $1111110000011000$ 
\end{itemize}
On le retrouve en développant $2^{16}-1000$ en base $2$.\newline
commande \texttt{bin(2**16 - 1000)}  
\end{frame}


\begin{frame}
  \frametitle{Autre méthode pour le codage de $-x$.}
\begin{itemize}
  \item représentation de $1000$: $0000001111101000$
  \item on cherche le mot qui ajouté au précédent donne $0000000000000000$ 
\end{itemize}
\begin{tabular}{cccccccccccccccc}
  0 & 0 & 0 & 0 & 0 & 0 & 1 & 1 & 1 & 1 & 1 & 0 & 1 & 0 & 0 & 0 \\
\end{tabular}
\uncover<2->{
\begin{tabular}{cccccccccccccccc}
\phantom{0}& \phantom{0} & \phantom{0} & \phantom{0} & \phantom{0} & \phantom{0} & \phantom{0} & \phantom{0} & \phantom{0} & \phantom{0} & \phantom{0} & \phantom{0} & \phantom{0} & 0 & 0 & 0 
\end{tabular} 
}
\uncover<3->{
\begin{tabular}{cccccccccccccccc}
\phantom{0}& \phantom{0} & \phantom{0} & \phantom{0} & \phantom{0} & \phantom{0} & \phantom{0} & \phantom{0} & \phantom{0} & \phantom{0} & \phantom{0} & \phantom{0} & 1 & 0 & 0 & 0 
\end{tabular}
attention à la retenue
}
\uncover<4->{
\begin{tabular}{cccccccccccccccc}
\phantom{0}& \phantom{0} & \phantom{0} & \phantom{0} & \phantom{0} & \phantom{0} & \phantom{0} & \phantom{0} & \phantom{0} & \phantom{0} & 0 & 1 & 1 & 0 & 0 & 0 
\end{tabular}
}

\uncover<5->{
\begin{tabular}{cccccccccccccccc}
  1 & 1 & 1 & 1 & 1 & 1 & 0 & 0 & 0 & 0 & 0 & 1 & 1 & 0 & 0 & 0 
\end{tabular}
}
\begin{tabular}{cccccccccccccccc} \hline
  0 & 0 & 0 & 0 & 0 & 0 & 0 & 0 & 0 & 0 & 0 & 0 & 0 & 0 & 0 & 0 \\
\end{tabular}

\end{frame}


\begin{frame}
  \frametitle{Exercice}
\begin{itemize}
  \item Quel est l'entier $x$ dont la représentation par complément à 2 sur $n$ bits ne contient que des $1$?
\uncover<2->{
\newline C'est un nombre négatif car le poids fort est $1$. Le codage de $-x$ est $1$ donc $x=-1$.
}
  \item Soit $m = b_{n-1}\,b_{n-2}\,\cdots\,b_1\,b_0$ un mot binaire de longueur $n$.
\begin{displaymath}
  \forall i\in \llbracket 0,n-1\rrbracket, \; b'_i = 1-b_i \text{ et } m' = b'_{n-1}\,b'_{n-2}\,\cdots\,b'_1\,b'_0
\end{displaymath}
  Quel est l'entier représenté par complément à 2 par l'addition bit à bit de $m$ et $m'$? 
\uncover<3->{
\newline $-1$ d'après la question précédente. Donc
\begin{displaymath}
  \overline{m'} = -1 - \overline{m}
\end{displaymath}
}
\end{itemize}
\end{frame}

\begin{frame}
  \frametitle{Exercice (en Python 2.7)}
Les entiers de type \texttt{int} sont stockés par complément à 2 sur 8 octets. Quel est l'intervalle des entiers du type \texttt{int}?\newline
Lorsque l'on sort de l'intervalle des entiers \texttt{int}, python 2.7 transtype automatiquement en entier \texttt{long}. On le sait avec \texttt{type(nom)}.\newline
Faire afficher à python le plus grand des \texttt{int}.
\end{frame}

\begin{frame}
  \frametitle{Représentation flottante normalisée d'un binaire}
Un nombre binaire non nul $x$ s'exprime de manière unique sous la forme
\begin{displaymath}
 x = s\times m\times 2^n 
\end{displaymath}
\begin{itemize}
  \item $s\in\{-1,+1\}$ est le signe du nombre
  \item $m\in [1,2[$ est sa mantisse
  \item $n\in \Z$ est son exposant.  
\end{itemize}
\end{frame}

\begin{frame}
  \frametitle{Forme normalisée}
\begin{algorithm}[H]
  $x \leftarrow $ un nombre entier $>0$\;
  $m \leftarrow x ; \hspace{0.5cm} n \leftarrow 0$\;
  \Si{$m \geq 2$}{
    \Tq{$m \geq 2$}{
      $m\leftarrow m/2; \hspace{0.5cm} n \leftarrow n + 1$;
    }
    \Comment{$m$ désigne un nb $<2$ avec $2m \geq 2 \Rightarrow m \geq 1$}
  }
  \Si{$m < 1$}{
    \Tq{$m < 1$}{
      $m\leftarrow 2m ; \hspace{0.5cm} n \leftarrow n - 1$;
    }
    \Comment{$m$ désigne un nb $\geq 1$ avec $m/2 < 1 \Rightarrow m < 2$}
  }
  \caption{Justification algorithmique}
\end{algorithm}
\end{frame}


\begin{frame}
\frametitle{Flottants sur 64 bits}
On utilise 1 bit pour le signe, 11 bits pour l'exposant et 52 bits pour la mantisse.
\begin{itemize}
 \item Le signe $+1$ est représenté par $0$, le signe $-1$ par $1$.
 \item L'exposant $n$ doit être compris entre $-1022$ et $1023$, on le représente par $n + 1023$ qui est compris entre $1$ et $2046$. Les deux valeurs restantes $0$ et $2047$ sont réservées pour des situations exceptionnelles ($+\infty$, $-\infty$, \texttt{NaN} )
 \item La mantisse comprend un seul chiffre avant la virgule qui est toujours $1$ car $1\leq m <2$. On utilise donc les 52 bits pour représenter les chiffres après la virgule. 
\end{itemize}  
\end{frame}

\begin{frame}[fragile]
  \frametitle{L'addition des flottants n'est pas associative}
\begin{verbatim}
G = 1.            # GRAND
p = 2**(-54)      # petit
print(p,G)
print(-G + G + p)
print(-G + (G + p))
\end{verbatim}


\end{frame}


\begin{frame}
  \frametitle{Bizarreries flottantes}
Quels sont les deux plus petits flottants $0<x_1<x_2$ sur 64 bits ?
\begin{multline*}
  \text{zéro :} 2^{-1022},\; x_1 = 2^{-1022}(1 + 2^{-52}) = 2^{-1022} + 2^{-1074},\;\\ x_2 = 2^{-1022} + 2^{-1073}  
\end{multline*}
Quel est le plus grand flottant strictement inférieur à $1$? Quel est le plus petit strictement supérieur à $1$?
\begin{displaymath}
  1.1111\cdots 111 \times 2^{-1} = 1 -2^{-53},\hspace{0.5cm} 1.0000 \cdots 0001 = 1+2^{-52}
\end{displaymath}

\end{frame}

\end{document}
