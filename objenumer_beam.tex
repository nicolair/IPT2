%!  pour pdfLatex
\documentclass{beamer}

%\usepackage[pdftex]{graphicx,color}
%\usepackage[pdftex,colorlinks={true},urlcolor={blue},pdfauthor={remy Nicolai}]{hyperref}

\usepackage[utf8]{inputenc}
\usepackage[T1]{fontenc}
\usepackage{lmodern}
\usepackage[frenchb]{babel}

\usetheme{Warsaw}

%\usepackage{fancyhdr}
%\pagestyle{fancy}

%\usepackage{floatflt}
\usepackage{maths}

\usepackage{parcolumns}
\setlength{\parindent}{0pt}

%\usepackage{caption}
%\usepackage{subcaption}

\usepackage[french,ruled,vlined]{algorithm2e}
\SetKwComment{Comment}{\#}{}
\SetKwFor{Tq}{tant que}{}{}
\SetKwFor{Pour}{pour}{}{}
\DontPrintSemicolon
\SetAlgoLined

\usepackage{listings}
\lstset{language=Python,frame=single}

%pr{\'e}sentation des compteurs de section, ...
\makeatletter
\renewcommand{\thesection}{\Roman{section}.}
\renewcommand{\thesubsection}{\arabic{subsection}.}
\renewcommand{\thesubsubsection}{\arabic{subsubsection}.}
%\renewcommand{\labelenumii}{\theenumii.}
\makeatother


\newtheorem*{thm}{Théorème}
\newtheorem{thmn}{Théorème}
\newtheorem*{prop}{Proposition}
\newtheorem{propn}{Proposition}
\newtheorem*{pa}{Présentation axiomatique}
\newtheorem*{propdef}{Proposition - Définition}
\newtheorem*{lem}{Lemme}
\newtheorem{lemn}{Lemme}

\theoremstyle{definition}
\newtheorem*{defi}{Définition}
\newtheorem*{nota}{Notation}
\newtheorem*{exple}{Exemple}
\newtheorem*{exples}{Exemples}


\newenvironment{demo}{\renewcommand{\proofname}{Preuve}\begin{proof}}{\end{proof}}
%\renewcommand{\proofname}{Preuve} doit etre après le begin{document} pour fonctionner

\theoremstyle{remark}
\newtheorem*{rem}{Remarque}
\newtheorem*{rems}{Remarques}

%\usepackage{maths}
%\newcommand{\dbf}{\leftrightarrows}

%En tete et pied de page
%\lhead{Informatique}
%\chead{Introduction aux systèmes informatiques}
%\rhead{MPSI B Hoche}
%\lfoot{\tiny{Cette création est mise à disposition selon le Contrat\\ Paternité-Partage des Conditions Initiales à l'Identique 2.0 France\\ disponible en ligne http://creativecommons.org/licenses/by-sa/2.0/fr/
%} }
%\rfoot{\tiny{Rémy Nicolai \jobname}}

\nonstopmode

\begin{document}

\begin{frame}
  \frametitle{Boucles d'énumération}
En Python, la notion boucle \texttt{for}
\begin{itemize}
  \item  n'est pas une boucle non conditionnelle. Elle n'a rien à voir avec la boucle \texttt{while},
  \item  est remplacée par une notion de \emph{boucle d'énumération} liée à des types particuliers d'objets dits \emph{énumérables}.
\end{itemize}
Un objet de type \emph{énumérable} est une sorte de conteneur qui rassemble \og dans une même boîte\fg~ divers objets que l'on peut énumérer  à l'aide d'une boucle
\begin{center}
\texttt{for ... in ...}.  
\end{center}
On va présenter plusieurs types d'objets énumérables ...

mais on utilisera essentiellement les chaînes de caractères et les listes.
\end{frame}

\begin{frame}
  \frametitle{Propriétés d'un type énumérable}
Un objet énumérable est
\begin{itemize}
  \item \emph{mutable} lorsque l'on peut modifier un objet contenu,
  \item \emph{indexable} lorsque l'on peut accéder directement (à l'aide d'un \emph{index}) à un objet contenu particulier sans les énumérer tous.
\end{itemize}

Dans l'écriture \emph{littérale} d'un objet énumérable,
\begin{itemize}
  \item le \emph{délimiteur} est le couple de caractères entourant les objets contenus,
  \item le \emph{séparateur} est le caractère entre les désignations des objets contenus.
\end{itemize}
\end{frame}

\begin{frame}
  \frametitle{Principaux types d'objets énumérables}
\begin{description}
 \item [Chaînes de caractères] Strings (non mutable - indexable) 
 \item [tuples] (non mutable - indexable) 
 \item [Listes] (mutable - indexable)
 \item [Ranges] (non mutable - indexable) En python 3, \texttt{range} renvoie un objet de ce type (et non une liste comme en 2) constituée d'entiers consécutifs. Minimise la place mémoire utilisée.
 \item [Ensembles] (Sets) (mutable - non indexable) 
 \item [Dictionnaires] (mutable - indexable) Un dictionnaire est formé par des couples (clé: valeur). 
\end{description}
\end{frame}

\begin{frame}
  \frametitle{index}
  \begin{itemize}
    \item Sauf pour les dictionnaires, les index sont dans $\llbracket 0, n-1 \rrbracket$ où $n$ désigne le nombre d'objets contenus.
    \item On obtient le nombre d'objets contenus avec la fonction \texttt{len} (pour length -longueur).
      \begin{center}
        \texttt{len(NomContenant)}
      \end{center}
    \item Pour un dictionnaire le terme \emph{index} est remplacé par \emph{clé}. Il est raisonnable de se limiter aux types chaîne de caractères ou entier pour les clés.
    \item On utilise un crochet \texttt{[]} pour attendre un objet avec une valeur de l'index
      \begin{center}
        \texttt{NomObjetIndexable[NomIndex]}
      \end{center}
  \end{itemize}
\end{frame}

\begin{frame}
  \frametitle{\'Ecriture littérale des principales classes énumérables}
  \begin{description}
    \item [Chaînes de caractères] délimiteur \texttt{" "} ou \texttt{' '}- pas de séparateur
    \item [tuples] pas de délimiteur - séparateur \texttt{ ,}\newline
    Parfois nécessaire d'utiliser des ( ) autour du tuple
    \item [Listes] délimiteur \texttt{[ ]} - séparateur \texttt{ , }
    \item [Ensembles] délimiteur \texttt{{ }} - séparateur \texttt{ ,}
    \item [Dictionnaires] délimiteur \{ \} - séparateur \texttt{ , } entre les couples.\newline
    Dans un couple, "\texttt{ : }" entre la clé (à gauche) et la valeur (à droite)
  \end{description}
\end{frame}

\begin{frame}
  \frametitle{Exemples : chaînes, listes, ranges}
  \lstinputlisting[firstline=1, lastline=10]{objenumer.py}  
\end{frame}

\begin{frame}
  \frametitle{Exemples : tuples, ensembles, dictionnaires}
  \lstinputlisting[firstline=12, lastline=20]{objenumer.py}  
\end{frame}

\begin{frame}
  \frametitle{Assignation = Référence}
Avec les objets dont le type est mutable, l'assignation est une \emph{référence} et non une copie.
\lstinputlisting[firstline=23, lastline=33]{objenumer.py}
\end{frame}

\begin{frame}[fragile]
  \frametitle{Syntaxe boucle d'énumération}
Exécuter une boucle non conditionnelle, c'est parcourir les objets constituant un objet énumérable en les désignant tour à tour par un même nom et exécuter un bloc d'instructions. 
\begin{verbatim}
for nom in obj:
  instruction1
  instruction2
  ...
instructionapres la boucle\end{verbatim}
\begin{itemize}
  \item \texttt{obj} désigne un objet énumérable
  \item \texttt{nom} : nom désignant successivement chacun des objets le constituant. 
\end{itemize}
\end{frame}

\begin{frame}
  \frametitle{Exemples d'énumération}
\lstinputlisting[firstline=34, lastline=42]{objenumer.py}
\end{frame}

\begin{frame}
  \frametitle{Chaînes: concaténation, délimiteurs, échappement}
\begin{itemize}
  \item L'opérateur de concaténation est \texttt{+}.
  \item Les délimiteurs \texttt{' '} (apostrophe, single quote) ou \text{" "} (guillemets, double quote) sont équivalents.
\end{itemize}
Certains caractères jouent un rôle particulier: \og ils ne se représentent pas\fg. Que faire ?
\begin{itemize}
  \item Utiliser guillemets ou apostrophes pour apostrophes et guillemets
  \item Utiliser le caractère \emph{d'échappement} \\ (anti-slash) qui enlève tout rôle au caractère qui le suit.
\end{itemize}
\end{frame}

\begin{frame}
  \frametitle{Chaînes: exemples}
\lstinputlisting[firstline=46, lastline=51]{objenumer.py}
\end{frame}

\begin{frame}
  \frametitle{Chaînes, sous-chaînes, indexation, slicing}
\begin{itemize}
  \item caractère = chaîne de longueur 0.
  \item sous-chaîne : formée avec des caractères consécutifs.
  \item sous chaînes obtenues avec crochets, index et \texttt{:} (slicing)
\end{itemize}
\lstinputlisting[firstline=54, lastline=62]{objenumer.py}
\end{frame}

\begin{frame}
  \frametitle{Chaînes: opérateur \texttt{in}}
  L'opérateur à valeur booléenne \texttt{in} permet de savoir si une chaîne est dans une chaîne.
  \lstinputlisting[firstline=65, lastline=67]{objenumer.py}
  Ce type d'algorithme  est un attendu du programme. (algorithmes usuels)
  
  Si le deuxième argument de l'opérateur n'est pas du type itérable, on obtient une erreur.
\end{frame}

\begin{frame}
  \frametitle{Listes: constructeurs, concaténation}
  En plus du constructeur littéral avec crochets et virgules
  \begin{itemize}
    \item fonction \texttt{range} directement (P2) ou avec \texttt{list} (P3)
    \item syntaxe de boucle d'itération à l'intérieur de crochets:
      \begin{center}
        \texttt{[i**2 for i in range(5)]}
      \end{center}
    \item concaténer deux listes avec +
    \item concaténer une liste avec elle même un nombre entier de fois avec *.
  \end{itemize}
\end{frame}

\begin{frame}
  \frametitle{Listes de listes}
  On peut emboîter les listes :
  \begin{itemize}
    \item matrice = liste de ses lignes
    \item matrice identité: matrice nulle modifiée en tirant parti du caractère mutable.
  \end{itemize}
  \lstinputlisting[firstline=74, lastline=79]{objenumer.py}
\end{frame}

\begin{frame}
  \frametitle{Listes de listes: pièges des références}
  \lstinputlisting[firstline=81, lastline=86]{objenumer.py}
\end{frame}

\begin{frame}
  \frametitle{Listes: fonctions et méthodes}
  \begin{itemize}
    \item Instruction : \texttt{del} non spécifiques aux listes. Détruit le nom.
    \item Fonctions : \texttt{len} pour tous les objets énumérables.
    \item Méthodes : \texttt{append}, \texttt{pop},   \texttt{extend}, \texttt{insert}, \texttt{remove}, \texttt{index}, \texttt{count}, \texttt{sort}, \texttt{reverse}
  \end{itemize}
  méthode = fonction spécifique à une classe\newline
  appelée en plaçant son nom avec un point après le nom de l'objet. Ne pas oublier les parenthèses pour les éventuels paramètres.
\end{frame}

\end{document}
