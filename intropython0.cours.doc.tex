%!  pour pdfLatex
\documentclass[a4paper]{article}
%\usepackage[hmargin={1.5cm,1.5cm},vmargin={2.4cm,2.4cm},headheight=13.1pt]{geometry}
\usepackage[a4paper,landscape,twocolumn,
            hmargin=1.8cm,vmargin=2.2cm,headheight=13.1pt]{geometry}

\usepackage[pdftex]{graphicx,color}
\usepackage[pdftex,colorlinks={true},urlcolor={blue},pdfauthor={remy Nicolai}]{hyperref}

\usepackage[utf8]{inputenc}
\usepackage[T1]{fontenc}
\usepackage{lmodern}
\usepackage[frenchb]{babel}

\usepackage{fancyhdr}
\pagestyle{fancy}

\usepackage{floatflt}
\usepackage{maths}

\usepackage{parcolumns}
\setlength{\parindent}{0pt}

\usepackage{caption}
\usepackage{subcaption}

\usepackage{makeidx}

\usepackage[french,ruled,vlined]{algorithm2e}
\SetKwComment{Comment}{\#}{}
\SetKwFor{Tq}{tant que}{}{}
\SetKwFor{Pour}{pour}{}{}
\DontPrintSemicolon
\SetAlgoLined

\usepackage{listings}
\lstset{language=Python,frame=single}
\lstset{literate=
  {á}{{\'a}}1 {é}{{\'e}}1 {í}{{\'i}}1 {ó}{{\'o}}1 {ú}{{\'u}}1
  {Á}{{\'A}}1 {É}{{\'E}}1 {Í}{{\'I}}1 {Ó}{{\'O}}1 {Ú}{{\'U}}1
  {à}{{\`a}}1 {è}{{\`e}}1 {ì}{{\`i}}1 {ò}{{\`o}}1 {ù}{{\`u}}1
  {À}{{\`A}}1 {È}{{\'E}}1 {Ì}{{\`I}}1 {Ò}{{\`O}}1 {Ù}{{\`U}}1
  {ä}{{\"a}}1 {ë}{{\"e}}1 {ï}{{\"i}}1 {ö}{{\"o}}1 {ü}{{\"u}}1
  {Ä}{{\"A}}1 {Ë}{{\"E}}1 {Ï}{{\"I}}1 {Ö}{{\"O}}1 {Ü}{{\"U}}1
  {â}{{\^a}}1 {ê}{{\^e}}1 {î}{{\^i}}1 {ô}{{\^o}}1 {û}{{\^u}}1
  {Â}{{\^A}}1 {Ê}{{\^E}}1 {Î}{{\^I}}1 {Ô}{{\^O}}1 {Û}{{\^U}}1
  {œ}{{\oe}}1 {Œ}{{\OE}}1 {æ}{{\ae}}1 {Æ}{{\AE}}1 {ß}{{\ss}}1
  {ű}{{\H{u}}}1 {Ű}{{\H{U}}}1 {ő}{{\H{o}}}1 {Ő}{{\H{O}}}1
  {ç}{{\c c}}1 {Ç}{{\c C}}1 {ø}{{\o}}1 {å}{{\r a}}1 {Å}{{\r A}}1
  {€}{{\euro}}1 {£}{{\pounds}}1 {«}{{\guillemotleft}}1
  {»}{{\guillemotright}}1 {ñ}{{\~n}}1 {Ñ}{{\~N}}1 {¿}{{?`}}1
}

%pr{\'e}sentation des compteurs de section, ...
\makeatletter
\renewcommand{\thesection}{\Roman{section}.}
\renewcommand{\thesubsection}{\arabic{subsection}.}
\renewcommand{\thesubsubsection}{\arabic{subsubsection}.}
\renewcommand{\labelenumii}{\theenumii.}
\makeatother


\newtheorem*{thm}{Théorème}
\newtheorem{thmn}{Théorème}
\newtheorem*{prop}{Proposition}
\newtheorem{propn}{Proposition}
\newtheorem*{pa}{Présentation axiomatique}
\newtheorem*{propdef}{Proposition - Définition}
\newtheorem*{lem}{Lemme}
\newtheorem{lemn}{Lemme}

\theoremstyle{definition}
\newtheorem*{defi}{Définition}
\newtheorem*{nota}{Notation}
\newtheorem*{exple}{Exemple}
\newtheorem*{exples}{Exemples}


\newenvironment{demo}{\renewcommand{\proofname}{Preuve}\begin{proof}}{\end{proof}}
%\renewcommand{\proofname}{Preuve} doit etre après le begin{document} pour fonctionner

\theoremstyle{remark}
\newtheorem*{rem}{Remarque}
\newtheorem*{rems}{Remarques}

%\usepackage{maths}
%\newcommand{\dbf}{\leftrightarrows}

%En tete et pied de page
\lhead{Informatique}
%\chead{Introduction aux systèmes informatiques}
\rhead{MPSI B Hoche}
\lfoot{\tiny{Cette création est mise à disposition selon le Contrat\\ Paternité-Partage des Conditions Initiales à l'Identique 2.0 France\\ disponible en ligne http://creativecommons.org/licenses/by-sa/2.0/fr/  
} 
\rfoot{\tiny{Rémy Nicolai \jobname \; \today } }
}
\makeindex


\usepackage{parcolumns}
\setlength{\parindent}{0pt}

 \begin{document}
\lhead{cours IPT}
\chead{cours 0: Introduction à l'utilisation de Python le 11/09 2017}
Python est un langage de programmation\footnote{c'est un langage interprété et non compilé} qui peut être utilisé de deux manières: 
\emph{interactivement} c'est à dire en écrivant une instruction (ou plusieurs) dans un interpréteur puis en l'exécutant immédiatement ou bien en écrivant (avec un éditeur) un fichier texte enregistré (\emph{extension} .py) ce qui permet de le conserver et de l'exécuter de manière différée.\newline
Sous diverses réserves (en particulier des problèmes de \emph{chemins de dossier} peuvent se produire), cela peut se faire dans n'importe quelle console\footnote{une console est une application dont l'interface avec l'utilisateur est uniquement du texte (ligne de commande).}. Toutefois le programme officiel de la classe nous demande d'utiliser Python par l'intermédiaire d'un \emph{environnement de programation}; les machines des salles de TP du lycée sont équipées de Spyder.\newline
L'objectif de ce cours est de se familiariser avec la notion d'environnement de travail et avec les premiers principes de Python.

\section{Environnement}
Un environnement de travail est constitué de plusieurs fenêtres (Figure \ref{fig:spyder}) qui doivent permettre les deux modes d'utilisation. Les environnements évolués comme Spyder offrent d'autres fonctionalités\footnote{par exemple des outils de vérification (débogage).}.
\begin{itemize}
 \item Dans une fenêtre console/interpréteur, on peut écrire directement une intruction Python et l'exécuter en tapant "Entrée". 
 \item Dans la fenêtre "Editeur de texte" on peut écrire, enregistrer, ... des fichiers d'extension \verb|.py| puis les faire exécuter (bouton Executer du menu ou flèche verte(suivant les versions)) dans une fenêtre console/interpréteur)
\end{itemize}
 
 \begin{figure}[h!]
 \centering
 \includegraphics[width=10cm,keepaspectratio=true]{./spyder.pdf}
 % spyder.pdf: 595x842 pixel, 72dpi, 20.99x29.70 cm, bb=0 0 595 842
 \caption{Environnement de développement spyder}
 \label{fig:spyder}
\end{figure}


Exemple avec les instructions \texttt{1+1} et \texttt{print "coucou"} ou \texttt{print("coucou")} à interpréter directement ou à exécuter par l'intermédiaire d'un fichier. Lors de l'utilisation d'un fichier, bien maitriser sa place dans l'arborescence des disques.

Il arrive souvent que le code soit mal écrit et conduise à un processus qui ne s'arrête jamais (boucle infinie). Il faut savoir comment l'arrêter.
Par exemple
 \begin{verbatim}
i = 0
while 0 < 10:
    print('coucou'+str(i))
    i+=1
 \end{verbatim}
\emph{Bien respecter l'indentation} c'est à dire les espaces en début de ligne. Interpréter dans Spyder avec la flèche verte. Repérer le petit triangle jaune avec un point d'exclamation dans le coin en haut à droite de la fenêtre de l'interpréteur.
On peut citer d'autres environnements, notamment des interpréteurs en ligne: par exemple \href{http://shell.appspot.com}{shell.appspot.com} ou \href{http://live.sympy.org/shellmobile}{live.sympy.org/shellmobile} qui est adapté aux écrans de smartphone et orienté calcul formel  (utilisant le module sympy). 
%\clearpage

\section{Exemples d'instructions}
Diverses instructions sont proposées dans la colonne de gauche. On se propose de les faire interpréter et de comprendre ce qui est renvoyé à l'aide des commentaires fournis dans la colonne de droite.
\begin{parcolumns}[rulebetween,distance=1cm,colwidths={1= .35\linewidth}]{2}
  \colchunk{\texttt{128, id(128), type(128)}}
  \colchunk{\texttt{128} est l'expression littérale \index{littéral} du nombre $128$ et \texttt{id(128)} renvoie l'identifiant unique de cet objet dans la mémoire de l'interpréteur, \texttt{type(128)} renvoie son type.}
  \colplacechunks
  
  \colchunk{
  \begin{verbatim}
   5 ** 5
   5 ** 200
  \end{verbatim}
  }
  \colchunk{Les entiers ne sont pas limités. En Python 2.7, le "L" à la fin signifie que le résultat est un entier "long" c'est à dire trop grand pour être représenté par un mot machine. Python sait gérer ces entiers; en cas d'opération dans laquelle intervient un entier long, le résultat est toujours un entier long. En Python 3, on peut dire pour simplifier que tous les entiers sont longs et le marqueur "L" n'est plus utilisé}
  \colplacechunks
  
  \colchunk{
  \begin{verbatim}
   14 // 3
   14 % 3
   14.1 / 3.0
  \end{verbatim}
  }
  \colchunk{Dans une division euclidienne, le quotient est obtenu par //, le reste par \%.\newline
  En Python 2 le / entre deux entiers calcule le quotient entier de la division euclidienne alors qu'en Python 3 il calcule une valeur approchée (float: nombres en virgule flottante) du quotient rationnel.\newline
  Bonne pratique: réserver le / à une division entre nombres en virgule flottante. Bien noter que les opérations entre entiers en virgule flottante sont toujours approchées.}
  \colplacechunks
  
  \colchunk{
  \begin{verbatim}
   pi
   sin(pi)
  \end{verbatim}
  }
  \colchunk{Python n'est pas spécialement orienté vers les maths. $\pi$ n'est pas un décimal, $\sin$ est inconnu.}
  \colplacechunks
\end{parcolumns}

\begin{parcolumns}[rulebetween,distance=1cm,colwidths={1= .35\linewidth}]{2}
  \colchunk{
  \begin{verbatim}
   from math import sin
   sin(pi)
   cos(3.1)
   from math import *
   cos(3.1)
  \end{verbatim}
  }
  \colchunk{Si on veut utiliser des opérations mathématiques, on doit les importer de la bibliothèque mathématique. Le "*" permet de tout importer ce qui n'est pas une bonne pratique. Il vaut mieux garder l'espace de nommage aussi petit que possible. Noter que le nom du module est \texttt{math} sans \og s\fg.}
  \colplacechunks
  
  \colchunk{
  \begin{verbatim}
   type(42)
   type(4.2)
   type(5 ** 20)
  \end{verbatim}
  }
  \colchunk{Les entiers et les nombres en virgule flottante sont des types élémentaires de valeurs pour le langage Python}
  \colplacechunks
  
  \colchunk{
  \begin{verbatim}
    i = 1.0j
    c = i ** 2
    i, c
    type(i), type(c)
  \end{verbatim}
  }
  \colchunk{Python connait les nombres complexes. La première ligne contient une expression littérale complexe.
  }
  \colplacechunks
 
  \colchunk{
  \begin{verbatim}
   help(floor)
   int(3.2)
   int(-3.2)
   help(int)
   int("abc",16)
  \end{verbatim}
  }
  \colchunk{On peut trouver de l'aide sur une fonction dont on connait le nom en utilisant la fonction \verb|help|. Que font les fonctions \verb|ceil| ou \verb|round| ? Que renvoie la fonction \verb|int|? Les valeurs renvoyées par \verb|int(3.5)| et \verb|round(3.5)| sont-elles les mêmes pour Python?}
  \colplacechunks
\end{parcolumns}

Pour insérer un \emph{commentaire} c'est à dire une ligne qui est ignorée par l'interpréteur et ne sert qu'à aider le programmeur, il suffit de la faire commencer par un dièse \#. Pour insérer des commentaires sur plusieurs lignes, il faut les encadrer par deux lignes contenant seulement  """.


\end{document}