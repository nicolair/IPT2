\begin{enumerate}
 \item 
 \textbf{Schéma décentré à droite.} 
 L'équation devient
\[
 \frac{1}{\delta_t}\left( u_{k+1,l} - u_{k,l}\right) +  \frac{1}{\delta_x}\left( u_{k,l+1} - u_{k,l}\right) = 0
\]
Les relations permettant de calculer les $u_{k,l}$ sont
\[
 \forall k\in\llbracket 0, n_t-1\rrbracket, \, \forall l \in \llbracket 0, n_x \rrbracket, \;
 u_{k+1,l} = u_{k,l} - \frac{\delta_t}{\delta_x}\left( u_{k,l+1} - u_{k,l}\right) 
\]
en convenant que, $u_{k,n_x + 1} = u_{k,1}$ pour tout $k\in\llbracket 0, n_t-1\rrbracket$ car pour tout $t$, la fonction $u_t$ est périodique de période $1$.
 
 \textbf{Schéma centré.}
 L'équation devient
\[
 \frac{1}{\delta_t}\left( u_{k+1,l} - u_{k,l}\right) +  \frac{1}{2\delta_x}\left( u_{k,l+1} - u_{k,l-1}\right) = 0
\]
Les relations permettant de calculer les $u_{k,l}$ sont
\[
 \forall k\in\llbracket 0, n_t-1\rrbracket, \, \forall l \in \llbracket 0, n_x \rrbracket, \;
 u_{k+1,l} = u_{k,l} - \frac{\delta_t}{2\delta_x}\left( u_{k,l+1} - u_{k,l-1}\right) 
\]
en convenant que, $u_{k,n_x + 1} = u_{k,1}$ et $u_{k, -1} = u_{k,n_x - 1}$ pour tout $k\in\llbracket 0, n_t-1\rrbracket$ par périodicité de $u_t$.
 
 \textbf{Schéma décentré à gauche.}
 L'équation devient
\[
 \frac{1}{\delta_t}\left( u_{k+1,l} - u_{k,l}\right) +  \frac{1}{\delta_x}\left( u_{k,l} - u_{k,l-1}\right) = 0
\]
Les relations permettant de calculer les $u_{k,l}$ sont
\[
 \forall k\in\llbracket 0, n_t-1\rrbracket, \, \forall l \in \llbracket 0, n_x \rrbracket, \;
 u_{k+1,l} = u_{k,l} - \frac{\delta_t}{\delta_x}\left( u_{k,l} - u_{k,l-1}\right) 
\]
avec $u_{k, -1} = u_{k,n_x - 1}$ pour tout $k\in\llbracket 0, n_t-1\rrbracket$ par périodicité de $u_t$.

 \item
 Dans chacune des fonctions proposées on calcule une fois pour toute au début 
\[
\frac{\delta_t}{\delta_x}=\frac{t\,n_x}{n_t} \text{ noté } \tau. 
\]

 \begin{enumerate}
  \item Schéma décentré à droite.
     \lstinputlisting[firstline=15, lastline=38]{Cequtransp.py}
  \item Schéma centré.
     \lstinputlisting[firstline=41, lastline=59]{Cequtransp.py}
  \item Schéma décentré à gauche.
     \lstinputlisting[firstline=62, lastline=78]{Cequtransp.py}
 \end{enumerate}

 \item
 \begin{enumerate}
  \item La fonction $u$ définie dans $\R^2$ par:
\[
 \forall (t,x)\in \R^2, \; u((t,x)) = \sin(2\pi(x - t))
\]
est bien solution de l'équation $(1)$ car
\[
 \frac{\partial u}{\partial t} = -2\pi \cos(2\pi(x-t)), \hspace{0.5cm} \frac{\partial u}{\partial x} = 2\pi \cos(2\pi(x-t))
\]
Elle vérifie la condition aux limites car $u_x(t_0) = u((0,x)) = \sin(2\pi x)$.

  \item Les figures présentent les graphes des évaluations numériques des fonctions $u_t$ pour la solution décrite dans la question précédente. Il s'agit de sinusoïdes.\newline
  Dans le schéma décentré à droite (figure 1 gauche), les courbes pour $u_0$ et $u_{0.1}$ ressemblent bien à des sinusïdes mais la courbe se dégrade pour $u_{0.2}$. Diminuer le pas de temps (figure 1 droite) n'améliore pas la situation mais accentue au contraire la divergence. 
  Le cas du schéma centré (figure 2) semble analogue. \`A gauche la courbe se dégrade pour $t=1.0$. Diminuer le pas de temps permet de retrouver une courbe correcte pour $t=1$ mais de nouveau la courbe n'est pas correcte pour $t=1.35$. Il faudrait de nouveau augmenter le nombre de points pour diminuer à nouveau le pas de remps. Il est difficile de conclure quant à la convergence du schéma.\newline
  Dans le cas du schéma décentré à gauche, les courbes semblent correctes même pour des valeurs de $t$ plus grandes. Ce schéma semble être le plus convergent des trois.
 \end{enumerate}

 \item 
 \textbf{Schéma décentré à droite.} 
Les relations permettant de calculer les $u_{k+1,l}$ sont
\[
 u_{k+1,l} = u_{k,l} - \frac{\delta_t}{\delta_x}\left( u_{k,l+1} - u_{k,l}\right) 
 = \left( 1+\frac{\delta_t}{\delta_x}\right)u_{k,l} -  \frac{\delta_t}{\delta_x} u_{k,l+1}.
\]
On en déduit (en majorant en valeur absolue)
\[
 \left( \forall l, \; |u_{k+1,l}|\leq \left( 1+\frac{\delta_t}{\delta_x}\right)M_k +  \frac{\delta_t}{\delta_x} M_k\right) 
 \Rightarrow M_{k+1} \leq \left( 1 + 2\frac{\delta_t}{\delta_x}\right)M_k.
\]
 \textbf{Schéma centré.}
Les relations permettant de calculer les $u_{k+1,l}$ sont
\[
 u_{k+1,l} = u_{k,l} - \frac{\delta_t}{2\delta_x}\left( u_{k,l+1} - u_{k,l-1}\right) 
 = \frac{\delta_t}{2\delta_x}u_{k,l-1} + u_{k,l} - \frac{\delta_t}{2\delta_x}u_{k,l+1}.
\]
On en déduit
\[
 \left( \forall l, \; |u_{k+1,l}|\leq \frac{\delta_t}{2\delta_x} M_k + M_k + \frac{\delta_t}{2\delta_x}M_k\right) 
 \Rightarrow M_{k+1} \leq \left(1 + \frac{\delta_t}{\delta_x}\right)M_k.
\]
 \textbf{Schéma décentré à gauche.}
Les relations permettant de calculer les $u_{k+1,l}$ sont
\[
 u_{k+1,l} = u_{k,l} - \frac{\delta_t}{\delta_x}\left( u_{k,l} - u_{k,l-1}\right) 
 = \underset{\geq 0}{\underbrace{\left( 1 - \frac{\delta_t}{\delta_x}\right)}}u_{k,l} + \frac{\delta_t}{\delta_x}u_{k,l-1}. 
\]
On en déduit 
\[
 \left( \forall l, \; |u_{k+1,l}|\leq \left( 1-\frac{\delta_t}{\delta_x}\right)M_k +  \frac{\delta_t}{\delta_x} M_k\right) 
 \Rightarrow M_{k+1} \leq M_k.
\]

\end{enumerate}
