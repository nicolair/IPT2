Une machine de Turing à 6 états s1, s2, s3, s4, s5 et ! représentant l'arrêt avec un alphabet 0, 1 est donnée par le tableau suivant
\begin{center}
\renewcommand{\arraystretch}{1.5}
\begin{tabular}{|l|l|l|l|} \hline
   & 0  & 1     \\ \hline
s1 & !      & 0 D s2  \\ \hline
s2 & 0 D s3 & 1 D s2  \\ \hline
s3 & 1 G s4 & 1 D s3  \\ \hline
s4 & 0 G s5 & 1 G s4  \\ \hline
s5 & 1 D s1 & 1 G s5  \\ \hline
\end{tabular}
\end{center}
Elle démarre avec le ruban ... 0 0 1 1 0 0 ... , la tête de lecture placée sur le 1 le plus à gauche et dans l'état s1. Quel est le ruban après l'arrêt de la machine ? Avec quels types de ruban peut-on généraliser? \`A quoi sert une telle machine?
