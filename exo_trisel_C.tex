\begin{algorithm}
  $L \leftarrow$ une liste de $n$ objets comparables indexée à partir de $0$.\;
  $i\leftarrow 0$\;
  \Tq{$i < n$}{
    \Comment{recherche de la plus petite valeur pour les indices $\geq i$}
    $imin \leftarrow i$\;
    $vmin \leftarrow L[i]$\;
    $j \leftarrow i+1$\;
    \Tq{$j<n$}{
      \Si{ $L[j] < vmin$}{
        $imin \leftarrow j$\;
        $vmin \leftarrow L[j]$\;
      }
      $j \leftarrow j +1$\;
    }
    \Comment{échanger les valeurs}
    $L[imin]\leftarrow L[i]$\;
    $L[i]\leftarrow vmin$\;
    \Comment{incrémenter $i$}
    $i \leftarrow i+1$\;
  }
  \label{exo_trisel_C}
  \caption{tri par sélection}
\end{algorithm}
L'algorithme \ref{exo_trisel_C} présente en pseudo-code le tri par sélection. Le code suivant propose une implémentation en Python
\begin{verbatim}
L = [4,1,2,5,8,7,9,12,36,45,78,5,12,51]
n = len(L)
i = 0
while i < n:
    imin = i
    vmin = L[i]
    j = i+1
    while j < n:
        if L[j] < vmin:
            imin = j
            vmin = L[j]
        j += 1
    L[imin] = L[i]
    L[i] = vmin
    i += 1
    
print(L)
\end{verbatim}
