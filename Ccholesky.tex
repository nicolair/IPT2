\subsection*{Partie I. Algorithme 1.}
\begin{enumerate}
 \item La matrice $A$ est symétrique car 
 \begin{displaymath}
  \trans \left( \trans U \, U\right) = \trans U \trans \trans U = \trans U U = A 
 \end{displaymath}

 \item La matrice $U$ est triangulaire supérieure si et seulement si
 \begin{displaymath}
 \forall (i,j)\in \llbracket 1,n \rrbracket^2, \; i > j\; \Rightarrow u_{i j} = 0 
\end{displaymath}

 \item Examinons le terme $1, j$ de $\trans U \, U = A$:
\begin{displaymath}
\trans C_1(U)\, C_j(U) =
\begin{pmatrix}
 u_{1 1} & 0 & \cdots & 0
\end{pmatrix}
\begin{pmatrix}
 u_{1 j} \\ u_{2 j} \\ \vdots \\ u_{n j}
\end{pmatrix}
 = u_{1 1} u_{1 j} = a_{1 j}
\end{displaymath}

 \item Pour tout $i\in \llbracket 2,n \rrbracket$ et tout $j\in \llbracket i,n \rrbracket$, examinons le terme $i, j$ de $A$:
\begin{displaymath}
\begin{pmatrix}
 u_{1 i} & \cdots & u_{i i} & 0 & \cdots & 0
\end{pmatrix}
\begin{pmatrix}
 u_{1 j} \\ \vdots \\ u_{j j} \\0 \\ \vdots \\ 0
\end{pmatrix} 
 = \sum_{k=1}^{i}u_{k i} u_{k j} = a_{i j}
\end{displaymath}
La somme est limitée à $i$ car $i\leq j$. La formule suivant s'obtient simplement en isolant $k=i$:
\begin{displaymath}
 \sum_{k=1}^{i-1}u_{k i} u_{k j} + u_{i i} u_{i j} = a_{i j}
\end{displaymath}

 \item On complète le pseudo-code proposé en utilisant la relation de la question 3. (d'abord avec $j=1$ puis pour les autres valeurs de $j$) puis celle de la question 4. (d'abord avec $j=i$ puis pour les autres valeurs de $j$ de $i+1$ à $n$). Le résultat est présenté comme algorithme \ref{Ccholesky_1}.
\end{enumerate}
\begin{algorithm}
  $n\leftarrow $ naturel non  nul\;
  $A\leftarrow $ matrice $n\times n$ définie positive \;
  $U$ désigne la matrice triangulaire supérieure que l'on veut calculer \;
  $i\leftarrow 1$\;
  $u_{1 1} \leftarrow \sqrt{a_{1 1}}$\;
  \Pour{$j$ de $2$ à $n$:}{
    $u_{1 j} \leftarrow \frac{a_{1 j}}{u_{1 1}}$
  }
  \Pour{$i$ de $2$ à $n$}{
      $u_{i i} \leftarrow \sqrt{a_{i i} - \sum_{k=1}^{i-1}u_{k i}^2}$\;
    \Pour{$j$ de $i+1$ à $n$:}{
      $u_{i j} \leftarrow \frac{a_{i j} - \sum_{k=1}^{i-1}u_{k i} u_{k j}}{u_{i i}}$
    }
  }
  \caption{Partie I. Question 5}
  \label{Ccholesky_1}
\end{algorithm}

\subsection*{Partie II. Application aux systèmes.}
\begin{enumerate}
 \item La résolution d'un système $SX =Y$ avec $S$ triangulaire supérieure inversible est \emph{la phase de remontée} du cours sur l'algorithme du pivot. Remarquer que tous les $s_{i i}$ sont non nuls car la matrice $S$ triangulaire supérieure est inversible. Le résultat est présenté comme algorithme \ref{Ccholesky_2}.
 \begin{algorithm}
 $i \leftarrow n$\;
 \Tq{ $i \geq 1$:}{
   $x_i = \frac{1}{s_{i i}}\left( y_i - \sum_{j = i+1}^{n} s_{i j} x_j\right)$\;
   $i \leftarrow i-1$\;
 }
 \caption{Phase de remontée}
 \label{Ccholesky_2}
\end{algorithm}
  
  \item Codes Python.
\begin{enumerate}
\item Le code de la fonction \texttt{resol\_sup} est présenté au dessous.
\lstinputlisting[firstline=5, lastline=17]{Ccholesky_1.py}

\item Celui de la fonction \texttt{resol\_inf} est très proche du précédent et procède dans l'autre sens.
\lstinputlisting[firstline=19, lastline=31]{Ccholesky_1.py}
\end{enumerate}

 \item Code pour la fonction \texttt{transpose}
\lstinputlisting[firstline=33, lastline=39]{Ccholesky_1.py}

 \item Pour résoudre le système $\trans U\, U\, X = T$, on résoud successivement deux systèmes triangulaires: d'abord inférieur puis supérieur.
\begin{displaymath}
\left. 
\begin{aligned}
 \trans U\, X_1 =& Y \\ U\, X = X_1
\end{aligned}
\right\rbrace 
\Rightarrow 
\trans U\, U\, X = Y
\end{displaymath}
On peut donc résoudre le système avec le code
\begin{verbatim}
U = cholesky(A)
L = transpose(U)
X1 = resol_inf(L,Y)
X  = resol_sup(U,X1)
\end{verbatim} 
\end{enumerate}

\subsection*{Partie III. Méthode des moindres carrés.}
\begin{enumerate}
 \item Il suffit d'écrire
\begin{displaymath}
 p(t_i) = a_0 + a_1t_i + a_2 t_{i}^{2}=
\begin{pmatrix}
 1 & t_i & t_i^2
\end{pmatrix}
\begin{pmatrix}
 a_0 \\ a_1 \\ a_2
\end{pmatrix}
\end{displaymath}

 \item Chacune des $14$ relations $p(t_i) = z_i$ correspond à une ligne de l'équation matricielle. Les matrices $V$ et $Z$ ont donc $14$ lignes:
\begin{displaymath}
 V =
\begin{pmatrix}
1 & t_0 & t_0^2 \\ 1 & t_1 & t_1^2 \\ \vdots & \vdots & \vdots \\ 1 & t_{13} & t_{13}^2
\end{pmatrix}
\in \mathcal{M}_{14 , 3}(\R)
\hspace{0.5cm}
Z =
\begin{pmatrix}
z_0 \\ z_1 \\ \vdots \\ z_{13} 
\end{pmatrix}
\in \mathcal{M}_{14 , 1}(\R)
\end{displaymath}

\item Le code complété est présenté au dessous. Il faut faire attention, comme dans la question II.4 à l'ordre dans lequel on résoud les deux systèmes triangulaires. 
\lstinputlisting[firstline=54, lastline=70]{Ccholesky_1.py}
\end{enumerate}
