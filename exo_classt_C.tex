On peut choisir
\begin{align*}
  \mathcal{P}_1 = \left( a \leq b \right) & & 
  \mathcal{P}_2 = \left( a \leq c \right) & &
  \mathcal{P}_3 = \left( b \leq c \right)
\end{align*}
Lors du deuxième branchement, la valeur de $a$ diminue ou ne change pas, celle de $b$ ne change pas, la condition $\mathcal{P}_1$ reste donc vraie. La valeur de $a$ est donc inférieure ou égale aux deux autres valeurs\newline
Lors du troisième branchement, $a$ ne change pas et les valeurs de $b$ et $c$ sont conservées ou échangées donc $a\leq b$ et $a\leq c$ restent vraies.\newline
La troisième propriété s'évalue à \verb|vrai| car la fin coïncide avec la pause 3. On en déduit qu'à la fin $a\leq b \leq c$.  