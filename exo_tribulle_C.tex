L'expression $n-i$ est bien à valeurs dans $\N$ mais elle ne décroît pas strictement. Elle est incrémentée dans le cas de l'échange. Le nombre de montées n'est pas non plus un variant car il est conservé lorsqu'il n'y a pas d'échange. Comment évolue-t-il lors d'un échange?\newline
Supposons que $i$ désigne $i_0$ avant un échange. Alors $(i_0,i_0+1)$ est une montée. classons les autres couples susceptibles d'être des montées : 
\begin{itemize}
  \item les couples $(k,i_0)$ et $(k,i_0+1)$ avec $k<i_0$.
  \item les couples $(i_0,l)$ et $(i_0+1,l)$ avec $i_0+1 < l$.
  \item les couples $(k,l)$ avec $k<i_0$ et $i_0+1<l$.
\end{itemize}
Pour chaque paire indiquée, le nombre de montées parmi ces couples peut être $0$, $1$ ou $2$ avant l'échange. Il reste identique après l'échange des valeurs. On en déduit que le nombre de montées diminue de $1$ par un échange.\newline
Notons $m$ le nombre de montées et considérons $V = n-i +2m$. Il est bien à valeurs entières. S'il n'y a pas d'échange, $i$ augmente de $1$ et $m$ reste identique donc $V$ décroît de $1$. S'il y a échange, $i$ augmente de $1$ et $2m$ diminue de $2$ donc $V$ décroît encore de $1$.  $V$ est donc bien un variant qui prouve que la boucle se termine.\newline
Considérons la propriété
\begin{displaymath}
\Phi : \forall k\in \N, 0 \leq k < i \Rightarrow A[k]\geq A[k+1]   
\end{displaymath}
Elle est vraie après l'initialisation quand $i$ désigne $0$ car il n'existe pas de $k$ vérifiant $0\leq k <i$.\newline
Notons $i_0$ ce que désigne $i$ avant un échange et supposons que $\Phi$ est vraie à ce moment. Après l'échange, $i$ désigne $i_0-1$. Comme $k<i_0-1$ entraîne $k+1<i_0$, les valeurs du tableau intervenant dans $\Phi$ ne sont pas modifiés donc $\Phi$ reste vraie. Si on incrémente $i$ sans échanger, les deux dernières valeurs de $A$ sont dans le bon ordre donc $\Phi$ reste vraie. 
La propriété $\Phi$ est donc un invariant.\newline
\`A la sortie de la boucle, $i$ désigne $n$ car la condition de réalisation est fausse. L'invariant montre alors que, pour tous les $i$ entre $0$ et $n-1$, $A[i+1]\leq A[i]$ c'est à dire que la liste est décroissante.

\begin{verbatim}
def TriBulle(A):
    n = len(A)
    i = 0
    nbPerm = 0
    while i < n-1:
        if A[i]<A[i+1]:
            A[i],A[i+1] = A[i+1],A[i]
            nbPerm += 1
            if i > 0:
                i -= 1
        else:
            i += 1
    return nbPerm

A = [1,2,4,5,1,47,52,35,14,41]
print(TriBulle(A))
print(A)  
\end{verbatim}
