%!  pour pdfLatex
\documentclass{beamer}

%\usepackage[pdftex]{graphicx,color}
%\usepackage[pdftex,colorlinks={true},urlcolor={blue},pdfauthor={remy Nicolai}]{hyperref}

\usepackage[utf8]{inputenc}
\usepackage[T1]{fontenc}
\usepackage{lmodern}
\usepackage[frenchb]{babel}

\usetheme{Warsaw}

%\usepackage{fancyhdr}
%\pagestyle{fancy}

%\usepackage{floatflt}
\usepackage{maths}

\usepackage{parcolumns}
\setlength{\parindent}{0pt}

%\usepackage{caption}
%\usepackage{subcaption}

\usepackage[french,ruled,vlined]{algorithm2e}
\SetKwComment{Comment}{\#}{}
\SetKwFor{Tq}{tant que}{}{}
\SetKwFor{Pour}{pour}{}{}
\DontPrintSemicolon
\SetAlgoLined

\usepackage{listings}
\lstset{language=Python,frame=single}

%pr{\'e}sentation des compteurs de section, ...
\makeatletter
\renewcommand{\thesection}{\Roman{section}.}
\renewcommand{\thesubsection}{\arabic{subsection}.}
\renewcommand{\thesubsubsection}{\arabic{subsubsection}.}
%\renewcommand{\labelenumii}{\theenumii.}
\makeatother


\newtheorem*{thm}{Théorème}
\newtheorem{thmn}{Théorème}
\newtheorem*{prop}{Proposition}
\newtheorem{propn}{Proposition}
\newtheorem*{pa}{Présentation axiomatique}
\newtheorem*{propdef}{Proposition - Définition}
\newtheorem*{lem}{Lemme}
\newtheorem{lemn}{Lemme}

\theoremstyle{definition}
\newtheorem*{defi}{Définition}
\newtheorem*{nota}{Notation}
\newtheorem*{exple}{Exemple}
\newtheorem*{exples}{Exemples}


\newenvironment{demo}{\renewcommand{\proofname}{Preuve}\begin{proof}}{\end{proof}}
%\renewcommand{\proofname}{Preuve} doit etre après le begin{document} pour fonctionner

\theoremstyle{remark}
\newtheorem*{rem}{Remarque}
\newtheorem*{rems}{Remarques}

%\usepackage{maths}
%\newcommand{\dbf}{\leftrightarrows}

%En tete et pied de page
%\lhead{Informatique}
%\chead{Introduction aux systèmes informatiques}
%\rhead{MPSI B Hoche}
%\lfoot{\tiny{Cette création est mise à disposition selon le Contrat\\ Paternité-Partage des Conditions Initiales à l'Identique 2.0 France\\ disponible en ligne http://creativecommons.org/licenses/by-sa/2.0/fr/
%} }
%\rfoot{\tiny{Rémy Nicolai \jobname}}

\nonstopmode

\begin{document}

\begin{frame}
 \frametitle{Codage binaire}
Soit $b$ un entier naturel supérieur ou égal à $2$.\newline
Pour tout $x \in \llbracket 0 , b^n-1 \rrbracket$, il existe un unique $n$-uplet
\begin{displaymath}
 (a_0,a_1,\cdots,a_{n-1})\in \llbracket 0 , b-1\rrbracket^n
\end{displaymath}
tel que 
\begin{displaymath}
 x = a_0 + a_1 b +\cdots +a_{n-1}b^{n-1}
\end{displaymath}
\end{frame}

\begin{frame}
 \frametitle{Bijection}
Considérons la fonction
\begin{displaymath}
 \Phi\left\lbrace 
\begin{aligned}
\llbracket 0 , b-1\rrbracket^n &\rightarrow \llbracket 0 , b^n-1\rrbracket\\
(a_0,a_1,\cdots,a_{n-1}) &\rightarrow a_0 + a_1 b +\cdots +a_{n-1}b^{n-1} 
\end{aligned}
\right. .
\end{displaymath}
On doit montrer qu'elle est bijective.
\end{frame}

\begin{frame}
 \frametitle{Lemme fondamental}
\begin{displaymath}
 a_0 + a_1 b +\cdots +a_{n-1}b^{n-1}\in \llbracket 0,b^n-1\rrbracket
\end{displaymath}
Ceci résulte de l'encadrement
\begin{multline*}
 0\leq a_0 + a_1 b +\cdots +a_{n-1}b^{n-1} 
\leq (b-1) + (b-1) b +\cdots +(b-1)b^{n-1}\\
\leq (b-1)(1+b+\cdots +b^{n-1})=b^n-1
\end{multline*}
\'Evidemment c'est vrai aussi pour tout $m\leq n$:
\begin{displaymath}
 a_0 + a_1 b +\cdots +a_{m-1}b^{m-1}\in \llbracket 0 , b^m-1\rrbracket.
\end{displaymath} 
\end{frame}

\begin{frame}
 \frametitle{Numération par les quotients}
\begin{displaymath}
 x = a_{n-1}b^{n-1} + \underset{\in \llbracket 0 ,b^{n-1}-1\rrbracket}{\underbrace{a_{n-2}b^{n-2}  + \cdots + a_1 b + a_0}}
\end{displaymath}
Donc dans la division de $x$ par $b^{n-1}$
\begin{itemize}
 \item $a_{n-1}$ est le quotient,
 \item le nombre au dessus de l'accolade est le reste
\end{itemize}
Ceci assure l'unicité de $a_{n-1}$ et on peut poursuivre le raisonnement en divisant le reste précédent par $b^{n-2}$.\newline
Inconvénient: pour un $x$ arbitraire, il faut d'abord calculer le $n$.
\end{frame}

\begin{frame}
 \frametitle{Pseudo-code}
\begin{algorithm}[H]
 $b \leftarrow $ la base , $x \leftarrow $ un nombre entier $>0$ \;
 \Comment{recherche plus gde puissance $p$ de $b$ tq $p \leq x$.}
 $p \leftarrow 1$\;
 $q \leftarrow p * b$\;
 \Tq{$q \leq x$}{
     $p\leftarrow q$\;
     $q\leftarrow p*b$}
 \Comment{Formation du développement}
 $resultat \leftarrow$ mot vide \;
 \Tq{$p >= 1$}{
     $q \leftarrow $ quotient de la division de $x$ par $p$ \;
     Concaténer $q$ à droite de $resultat$ \;
     $x \leftarrow $ reste de la division de $x$ par $p$ \;
     $p \leftarrow p // b$ }\;
     \caption{Numération par les quotients}
\end{algorithm}
\end{frame}

\begin{frame}
 \frametitle{Numération par les restes}
\begin{multline*}
 x= a_0 + a_1 b +\cdots +a_{m-1}b^{m-1} \\
 = a_0 +(a_1+a_2b+\cdots+a_{n-1}b^{n-2})b
\end{multline*}
Donc dans la division de $x$ par $b$:
\begin{itemize}
 \item le reste est $a_0$,
 \item le quotient est 
\begin{displaymath}
a_1+a_2b+\cdots+a_{n-1}b^{n-2}. 
\end{displaymath}
\end{itemize}
Ceci assure l'unicité du $a_0$ et le raisonnement se poursuit en divisant par $b$ le quotient précédent.
\end{frame}

\begin{frame}
 \frametitle{Pseudo-code}
\begin{algorithm}[H]
 $b \leftarrow $ la base \;
 $x \leftarrow $ un nombre entier $>0$ \;
 \Comment{Formation du développement}
 $resultat \leftarrow$ liste vide \;
 \Tq{$x > 0$}{
     $r \leftarrow $ reste de la division de $x$ par $b$ \;
     Concaténer $r$ à gauche de $resultat$ \;
     $x \leftarrow $ quotient de la division de $x$ par $b$ \;}
 \caption{Numération par les restes}
\end{algorithm}
\end{frame}

\begin{frame}
 \frametitle{Par les quotients: code Python}
\lstinputlisting[firstline=10, lastline=25]{intronumer.py}
\end{frame}

\begin{frame}
 \frametitle{Par les restes: code Python}
\lstinputlisting[firstline=30, lastline=37]{intronumer.py}
\end{frame}


\end{document}
