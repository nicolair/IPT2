Le concept de \og phrase invariante\fg sera détaillé plus tard dans le cours d'algorithmique. Il est introduit ici pour illustrer l'usage des noms (au sens grammatical) dans un langage (usuel ou de programmation). C'est aussi une bonne pratique dans la recherche d'une méthode pour résoudre un problème.\newline
Considérons le problème du calcul de la factorielle d'un nombre et la phrase
\begin{quote}
  \og $f$ désigne la factorielle du nombre désigné par $n$.\fg
\end{quote}
Pour résoudre le problème du calcul de $5!$, on va assigner à $f$ et $n$ des valeurs particulières telles que la phrase soit vraie puis modifier les valeurs de telle sorte que la phrase reste vraie jusqu'à ce que $n$ désigne 5.
\lstinputlisting[firstline=3, lastline=16]{phraseinvariante.py}
Remarquer que l'instruction \texttt{n = n + 1} pourrait aussi s'écrire \texttt{n += 1}. De même dans l'autre sens, \texttt{f *= n} pourrait aussi s'écrire \texttt{f = f * n}.