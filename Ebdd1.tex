On modélise \footnote{Questions Q26, Q27, Q28 du Concours commun Mines Ponts 2017 (i17mice.pdf)} ici un réseau routier par un ensemble de \emph{croisements} et de \emph{voies} reliant ces croisements. Les voies partent d'un croisement et arrivent à un autre croisement. Ainsi, on modélise une route à double sens par deux voies circulant en sens opposés.\newline
La base de données de ce réseau routier est constituée des relations suivantes:
\begin{itemize}
 \item Croisement (\underline{id}, longitude, latitude)
 \item Voie (\underline{id}, id\_croisement\_debut, id\_croisement\_fin)
\end{itemize}

Dans la suite, $c$ désigne l'identifiant (id) d'un croisement donné.
\begin{enumerate}
 \item \'Ecrire la requête SQL qui renvoie les identifiants des croisements que l'on peut atteindre à partir du croisement identifié par $c$ en utilisant une seule voie.
 \item \'Ecrire la requête SQL qui renvoie les longitudes et latitudes des croisements que l'on peut atteindre à partir du croisement identifié par $c$ en utilisant une seule voie.
 \item Que renvoie la requête SQL suivante ?
\begin{verbatim}
 SELECT V2.id_croisement_fin
 FROM Voie AS V1
 JOIN Voie AS V2
 ON V1.id_croisement_fin = V2.id_croisement_debut
 WHERE V1.id_croisement_debut = c
\end{verbatim}

\end{enumerate}
