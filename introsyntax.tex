%!  pour pdfLatex
\documentclass[a4paper]{article}
%\usepackage[hmargin={1.5cm,1.5cm},vmargin={2.4cm,2.4cm},headheight=13.1pt]{geometry}
\usepackage[a4paper,landscape,twocolumn,
            hmargin=1.8cm,vmargin=2.2cm,headheight=13.1pt]{geometry}

\usepackage[pdftex]{graphicx,color}
\usepackage[pdftex,colorlinks={true},urlcolor={blue},pdfauthor={remy Nicolai}]{hyperref}

\usepackage[T1]{fontenc}
\usepackage[utf8]{inputenc}

\usepackage{lmodern}
\usepackage[frenchb]{babel}

\usepackage{fancyhdr}
\pagestyle{fancy}

\usepackage{floatflt}
\usepackage{maths}

\usepackage{parcolumns}
\setlength{\parindent}{0pt}

\usepackage{caption}
\usepackage{subcaption}

\usepackage{makeidx}

\usepackage[french,ruled,vlined]{algorithm2e}
\SetKwComment{Comment}{\#}{}
\SetKwFor{Tq}{tant que}{}{}
\SetKwFor{Pour}{pour}{}{}
\DontPrintSemicolon
\SetAlgoLined

\usepackage{listings}
\lstset{language=Python,frame=single}
\lstset{literate=
  {á}{{\'a}}1 {é}{{\'e}}1 {í}{{\'i}}1 {ó}{{\'o}}1 {ú}{{\'u}}1
  {Á}{{\'A}}1 {É}{{\'E}}1 {Í}{{\'I}}1 {Ó}{{\'O}}1 {Ú}{{\'U}}1
  {à}{{\`a}}1 {è}{{\`e}}1 {ì}{{\`i}}1 {ò}{{\`o}}1 {ù}{{\`u}}1
  {À}{{\`A}}1 {È}{{\'E}}1 {Ì}{{\`I}}1 {Ò}{{\`O}}1 {Ù}{{\`U}}1
  {ä}{{\"a}}1 {ë}{{\"e}}1 {ï}{{\"i}}1 {ö}{{\"o}}1 {ü}{{\"u}}1
  {Ä}{{\"A}}1 {Ë}{{\"E}}1 {Ï}{{\"I}}1 {Ö}{{\"O}}1 {Ü}{{\"U}}1
  {â}{{\^a}}1 {ê}{{\^e}}1 {î}{{\^i}}1 {ô}{{\^o}}1 {û}{{\^u}}1
  {Â}{{\^A}}1 {Ê}{{\^E}}1 {Î}{{\^I}}1 {Ô}{{\^O}}1 {Û}{{\^U}}1
  {œ}{{\oe}}1 {Œ}{{\OE}}1 {æ}{{\ae}}1 {Æ}{{\AE}}1 {ß}{{\ss}}1
  {ű}{{\H{u}}}1 {Ű}{{\H{U}}}1 {ő}{{\H{o}}}1 {Ő}{{\H{O}}}1
  {ç}{{\c c}}1 {Ç}{{\c C}}1 {ø}{{\o}}1 {å}{{\r a}}1 {Å}{{\r A}}1
  {€}{{\euro}}1 {£}{{\pounds}}1 {«}{{\guillemotleft}}1
  {»}{{\guillemotright}}1 {ñ}{{\~n}}1 {Ñ}{{\~N}}1 {¿}{{?`}}1
}

%pr{\'e}sentation des compteurs de section, ...
\makeatletter
\renewcommand{\thesection}{\Roman{section}.}
\renewcommand{\thesubsection}{\arabic{subsection}.}
\renewcommand{\thesubsubsection}{\arabic{subsubsection}.}
\renewcommand{\labelenumii}{\theenumii.}
\makeatother


\newtheorem*{thm}{Théorème}
\newtheorem{thmn}{Théorème}
\newtheorem*{prop}{Proposition}
\newtheorem{propn}{Proposition}
\newtheorem*{pa}{Présentation axiomatique}
\newtheorem*{propdef}{Proposition - Définition}
\newtheorem*{lem}{Lemme}
\newtheorem{lemn}{Lemme}

\theoremstyle{definition}
\newtheorem*{defi}{Définition}
\newtheorem*{nota}{Notation}
\newtheorem*{exple}{Exemple}
\newtheorem*{exples}{Exemples}


\newenvironment{demo}{\renewcommand{\proofname}{Preuve}\begin{proof}}{\end{proof}}
%\renewcommand{\proofname}{Preuve} doit etre après le begin{document} pour fonctionner

\theoremstyle{remark}
\newtheorem*{rem}{Remarque}
\newtheorem*{rems}{Remarques}

%\usepackage{maths}
%\newcommand{\dbf}{\leftrightarrows}

%En tete et pied de page
\lhead{Informatique}
%\chead{Introduction aux systèmes informatiques}
\rhead{MPSI B Hoche}
\lfoot{\tiny{Cette création est mise à disposition selon le Contrat\\ Paternité-Partage des Conditions Initiales à l'Identique 2.0 France\\ disponible en ligne http://creativecommons.org/licenses/by-sa/2.0/fr/  
} 
\rfoot{\tiny{Rémy Nicolai \jobname \; \today } }
}
\makeindex

%En tete et pied de page
\lhead{Cours IPT}
\chead{cours 1: Introduction à la syntaxe de Python le 12/09/16}

\begin{document}
L'objet de ce document est de donner les premiers éléments de syntaxe permettant de former du code Python correct mais aussi les concepts et le vocabulaire permettant de comprendre des \href{http://docs.python.org/3.3/reference/}{documentations plus complètes} et d'aller y chercher des informations plus précises.
\subsection{Expressions}
Une expression est une phrase que l'interpréteur peut évaluer à un objet. La liste suivante présente les principaux types d'éléments constituants une expression.
\begin{description}
 \item[nom] ou variable: des mots formés avec les lettres de a à z en minuscule ou majuscule (Python est sensible à la casse c'est à dire que \verb|Cassy| et \verb|cassy| sont deux noms différents), le caractère underscore \verb|_| et les chiffres de \verb|0| à \verb|9| (sauf pour le premier caractère). Si un mot est réservé par Python pour un usage prédéfini, il ne peut pas être un nom.
 \item[littéral] : un littéral désigne un objet fixé et pré-défini. La syntaxe dépend du type d'objet représenté, retenir seulement les cas suivants.\newline
Les délimiteurs "espace" pour les entiers (int) avec éventuellement \verb|-|: par exemple \verb| 12 | désigne le nombre $12$. Il faut seulement le séparer d'un autre mot par des espaces.\newline
Les délimiteurs \verb|" "| ou \verb|' '| pour les chaînes de caractères: \verb|"coucou"| est une chaîne de caractères.\newline
Le \verb|.| pour les objets du type float: \verb|1.1| est un objet du type float, \verb|1.0| est un objet du type float alors que \verb|1| est un objet du type int et que \verb|"1"| est un objet du type str (chaîne de caractère).\newline
Le \verb|j| pour un type "complex" : \verb|1.0j| représente l'imaginaire pur $i$. Noter que la partie imaginaire est de type float. Pour former des complexes avec une partie réelle non nulle, utiliser \verb|+| et un littéral du type float.

\item[opérateurs] Un opérateur est un mot (contenant en général seulement un ou deux caractères) placé entre deux expressions (les opérandes). La phrase ainsi formée est elle même une expression. Il existe une hiérarchie entre les opérateurs qui élimine les ambiguités lorsque plus de deux opérateurs interviennent. Les parenthèses \verb|()| sont utilisées lorsque cette hierarchie n'est pas très claire. Pour qu'une telle phrase soit correcte, les opérandes doivent représenter des objets dont le type est en accord avec l'opérateur. Le résultat de l'évaluation est un objet dont le type dépend de l'opérateur.  
\end{description}
Présentons les opérateurs à retenir en les classant suivant le type de l'objet produit par l'évaluation de l'expression.
\begin{description}
 \item[opérateurs arithmétiques] Le résultat est d'un type représentant un nombre (en convenant que num designe l'un ou l'autre des types \verb|int| ou \verb|float|). Pour les opérateurs présentés dans le tableau suivant, il est conseillé de se limiter à des expressions dans les lesquelles les opérandes sont de type indiqué, le résultat de l'opération est alors de même type.
 {%
\begin{center}
\begin{tabular}{cccccc}
addition & multiplication & puissance & reste & quotient & division \\
\verb|+| & \verb|*| & \verb|**| & \verb|%| & \verb|//|& \verb|/|    \\
num      & num      & num       & int      & int      & float    \\
 \end{tabular}
 \end{center}
}%

\item[opérateurs booléens] Les opérandes sont de type booléens, l'expression formée s'évalue en booléen. 
\begin{center}
\begin{tabular}{ccc}
\verb|and| & \verb|or| & \verb|not| \\
 \end{tabular}
\end{center}
On peut remarquer que \verb|not| n'est pas tout à fait à sa place car il s'agit d'un opérateur \emph{unaire} qui se place à gauche d'une expression booléenne.

\item[opérateurs sur les chaînes de caractères] Le seul à retenir pour le moment est l'opérateur de \emph{concaténation} qui se note \verb|+|.\newline
Il colle bout à bout deux chaînes :
\begin{verbatim}
 "tagada"+"tsoin tsoin"
\end{verbatim}
s'évalue à \verb|"tagadatsoin tsoin"|.

\item[opérateurs de comparaison] Une condition est une expression qui s'évalue en une valeur booléenne. Les opérateurs de comparaison forment des conditions. Les caractères utilisés pour les opérateurs présentés dans le tableau suivant sont assez clairs quant à leur signification. 
{%
\begin{center}
\begin{tabular}{cccccc}
\verb|==|  & \verb|!=| & \verb|>|  & \verb|<|  & \verb|>=|  &  \verb|<=|
 \end{tabular}
 \end{center}
}
sauf peut être \verb|!=| qui signifie "est différent de". Le détail de ce que fait l'opérateur == mérite d'être approfondi suivant les types des objets comparés.

\item[opérateurs d'affectation] Ils sont de nature très différentes des précédents. Ce qui est à gauche d'un tel opérateur doit être un nom, ce qui est à droite est une expression. L'exécution référence par le nom à gauche un objet évalué à partir de l'expression de droite.\newline
Le principal opérateur de ce type est \verb|=| que l'on peut appeler assignation simple. Le nom à gauche référence l'objet évalué à droite.\newline
Il existe aussi des \emph{assignations augmentées} par des opérateurs
\begin{center}
\begin{tabular}{cccccc}
\verb|+=|  & \verb|*=| & \verb|/=|  & \verb|%=|
 \end{tabular}
 \end{center}
Par exemple \verb|a += 1| revient exactement au même que \verb|a = a + 1|. Les autres opérateurs sont sur le même principe.
\end{description}
Le mélange de divers types d'objets dans des expressions soulève de délicates questions qui méritent un examen détaillé de la documentation officielle du langage.\newline
Une particularité de Python est qu'il permet des assignations multiples.\newline
Après \verb|a , b = c , d|, les noms \verb|a| et \verb|c| désignent le même objet (celui que désignait \verb|c| avant l'exécution) et de même pour \verb|b| et \verb|d|. Cela permet en particulier de permuter les noms de deux objets.\newline
Supposons que \verb|a| désigne l'objet1 et \verb|b| désigne l'objet2, et exécutons \verb|a , b = b , a|. Les noms sont alors permutés : \verb|a| désigne l'objet2 et \verb|b| désigne l'objet1 sans que les objets soient modifiés.

\subsection{Instructions}
Du point de vue de la syntaxe, une instruction\footnote{Le terme utilisé pour "instruction" dans la littérature informatique en anglais est \emph{statement} (ce qui affaiblit fortement l'aspect impératif).} est une \emph{ligne de code}.\newline
L'\emph{indentation}, c'est à dire les caractères "espaces" présents au début d'une ligne joue un rôle très important dans la syntaxe Python.\newline
Il existe des moyens de faire continuer le texte d'une instruction sur plusieurs lignes en utilisant des caractères spéciaux voir la documentation sur "logical line", "physical line", "line joining".\newline
Un bloc d'instructions est une séquence d'intructions qui se suivent.
\subsubsection{Commentaires}
Un commentaire est un texte qui n'est pas interprété par Python. Lorsque l'interpréteur rencontre le caractère \verb|#|, il ignore le texte qui le suit jusqu'à la fin de la ligne. Pour insérer une ligne de commentaire, il suffit donc de la faire commencer par \verb|#|.\newline
Si on veut insérer un commentaire sur plusieurs lignes sans les faire débuter par dièse, on peut utiliser 
\begin{verbatim}
 """
ligne 1 de commentaire
ligne 2
...
"""
\end{verbatim}
sauf dans certaines circonstances, le texte entre les délimiteurs \verb|"""| sera ignoré par l'interpréteur. Ceci n'est toutefois pas un véritable commentaire (voir la documentation). 

\subsubsection{Instruction simple}
Dans un premier temps, les instructions simples seront des assignations, d'autres sortes d'instructions seront rencontrées ensuite.
\subsubsection{Instructions conditionnelles}
Une instruction conditionnelle ou un bloc conditionnel d'instructions ne s'exécute que si une certaine condition est évaluée à vrai. La syntaxe est de la forme suivante
\begin{verbatim}
if condition :
  instruction1
  instruction2
  ...
  instructionN
instructiontoujoursinterpretee
\end{verbatim}
Noter le \verb|:| dans la première ligne et l'indentation qui \emph{caractérise} les instructions du bloc. Utilisez deux ou quatre espaces de préférence à une tabulation pour l'indentation.\newline
Il existe des variantes avec des \verb|else :| ou des \verb| elif condition1 :| qui ne sont pas présentées ici et dont la syntaxe est assez naturelle, voir éventuellement la documentation.

\subsubsection{Boucles conditionnelles}
Une boucle conditionnelle est un bloc d'instructions qui sera exécuté tout entier tant qu'une certaine condition est évaluée à vrai. La syntaxe est la suivante
\begin{verbatim}
instructiondavant
while condition :
  instruction1
  instruction2
  ...
  instructionN
instructiondapres
\end{verbatim}
Bien noter que le bloc peut n'être jamais exécuté car l'évaluation de la condition se fait avant la première exécution. Il est aussi possible que se produise une \emph{boucle infinie} lorsque les exécutions du blocs ne changent pas l'évaluation de la condition.\newline
La syntaxe des boucles inconditionnelles( boucles "for") sera présentée plus loin car, en Python, ces boucles sont intimement reliées à certains types particuliers d'objets.
\end{document}
