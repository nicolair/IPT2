%!  pour pdfLatex
\documentclass{beamer}

%\usepackage[pdftex]{graphicx,color}
%\usepackage[pdftex,colorlinks={true},urlcolor={blue},pdfauthor={remy Nicolai}]{hyperref}

\usepackage[utf8]{inputenc}
\usepackage[T1]{fontenc}
\usepackage{lmodern}
\usepackage[frenchb]{babel}

\usetheme{Warsaw}

%\usepackage{fancyhdr}
%\pagestyle{fancy}

%\usepackage{floatflt}
\usepackage{maths}

\usepackage{parcolumns}
\setlength{\parindent}{0pt}

%\usepackage{caption}
%\usepackage{subcaption}

\usepackage[french,ruled,vlined]{algorithm2e}
\SetKwComment{Comment}{\#}{}
\SetKwFor{Tq}{tant que}{}{}
\SetKwFor{Pour}{pour}{}{}
\DontPrintSemicolon
\SetAlgoLined

\usepackage{listings}
\lstset{language=Python,frame=single}

%pr{\'e}sentation des compteurs de section, ...
\makeatletter
\renewcommand{\thesection}{\Roman{section}.}
\renewcommand{\thesubsection}{\arabic{subsection}.}
\renewcommand{\thesubsubsection}{\arabic{subsubsection}.}
%\renewcommand{\labelenumii}{\theenumii.}
\makeatother


\newtheorem*{thm}{Théorème}
\newtheorem{thmn}{Théorème}
\newtheorem*{prop}{Proposition}
\newtheorem{propn}{Proposition}
\newtheorem*{pa}{Présentation axiomatique}
\newtheorem*{propdef}{Proposition - Définition}
\newtheorem*{lem}{Lemme}
\newtheorem{lemn}{Lemme}

\theoremstyle{definition}
\newtheorem*{defi}{Définition}
\newtheorem*{nota}{Notation}
\newtheorem*{exple}{Exemple}
\newtheorem*{exples}{Exemples}


\newenvironment{demo}{\renewcommand{\proofname}{Preuve}\begin{proof}}{\end{proof}}
%\renewcommand{\proofname}{Preuve} doit etre après le begin{document} pour fonctionner

\theoremstyle{remark}
\newtheorem*{rem}{Remarque}
\newtheorem*{rems}{Remarques}

%\usepackage{maths}
%\newcommand{\dbf}{\leftrightarrows}

%En tete et pied de page
%\lhead{Informatique}
%\chead{Introduction aux systèmes informatiques}
%\rhead{MPSI B Hoche}
%\lfoot{\tiny{Cette création est mise à disposition selon le Contrat\\ Paternité-Partage des Conditions Initiales à l'Identique 2.0 France\\ disponible en ligne http://creativecommons.org/licenses/by-sa/2.0/fr/
%} }
%\rfoot{\tiny{Rémy Nicolai \jobname}}

\nonstopmode

\begin{document}
\begin{frame}
  \frametitle{Problèmes usuels}
  \begin{itemize}
    \item Recherches linéaires.
    \begin{itemize}
      \item premier indice d'une valeur
      \item plus grande valeur
      \item indice de la plus grande valeur
    \end{itemize}
    \item Dichotomie
    \begin{itemize}
      \item encadrer un nombre par deux valeurs consécutives d'une liste croissante
      \item équation $f(x)=0$ pour $f$ croissante
    \end{itemize}
    \item Motif dans une chaîne de caractères
    \begin{itemize}
      \item méthode naïve
      \item amélioration pour un motif sans répétition
    \end{itemize}

  \end{itemize}

\end{frame}

\begin{frame}
  \frametitle{Comparaison de problèmes}
  \begin{center}
Soient deux problèmes $\mathcal{P}_1$ et $\mathcal{P}_2$.

\bigskip
Si une solution de $\mathcal{P}_2$ permet d'obtenir une solution de $\mathcal{P}_1$, alors :

\bigskip
 $\mathcal{P}_2$ est \emph{plus difficile} que $\mathcal{P}_1$. 
     
  \end{center}

Pour résoudre $\mathcal{P}_1$,  vouloir résoudre $\mathcal{P}_2$ est une MAUVAISE ID\'EE.
\end{frame}

\begin{frame}
  \frametitle{Premier indice d'une valeur dans une liste}
\begin{algorithm}[H]
  \Donnees{\;
     $ L \leftarrow$ une liste de $n$ objets indexée à partir de 0\;
     $O \leftarrow$ un objet}
  \Comment{initialisation}
  $i\leftarrow$ 0 \;
  \Tq{ i < n \rm et \it L[i] != O}{
      incrémenter $i$\;
  }
  \Comment{après la boucle, $i$ est inférieur ou égal à $n$}
  \Comment{Si $i<n$, il désigne l'indice cherché.}
  \Comment{Si $i=n$, l'objet ne figure pas dans la liste}
  \caption{}
\end{algorithm}
\end{frame}

\begin{frame}
\frametitle{Recherche de la plus grande valeur d'une liste}
\begin{algorithm}[H]
  \Donnees{\;
     $ L \leftarrow$ une liste (indexée de 0 à $n$) d'objets comparables}
  \Comment{initialisation}
  $vmax\leftarrow L[0]$  \;
  $i \leftarrow 1$\;
  \Tq{ i < n}{
      \Si{vmax < L[i]}{vmax = L[i]}
      incrémenter $i$\;
  }
  \Comment{$vmax$ désigne la plus grande valeur}
  \caption{}
\end{algorithm}  
\end{frame}

\begin{frame}
  \frametitle{Recherche de l'indice de la plus grande valeur}
  Comment modifier l'algorithme précédent ?

\begin{algorithm}[H]
\only<2->{
\Donnees{\;
     $ L \leftarrow$ une liste (indexée de 0 à $n-1$) d'objets comparables}
  \Comment{initialisation}
  $vmax\leftarrow L[0]$  \;
  \only<3->{$imax \leftarrow 0$ \;}
  $i \leftarrow 1$\;
  \Tq{ i < n}{
      \Si{vmax < L[i]}{vmax = L[i] \; \only<3->{$imax \leftarrow i$ \;}}
      incrémenter $i$\;
  }
  \Comment{$vmax$ désigne la plus grande valeur\;}
  \only<3->{\Comment{$imax$ désigne un indice de la plus grande valeur\;}}
  \caption{}
}
\end{algorithm}  
\end{frame}

\begin{frame}
  \frametitle{Encadrement par dichotomie}
\begin{algorithm}[H]
  \Donnees{\;
    $L \leftarrow$ une liste (indexée à partir de 0) croissante (pas forcément strictement) de $n$ nombres\;
    $v \leftarrow$ un nombre vérifiant $L[0]\leq v < L[n-1]$\;
  }
  \Comment{initialisation}
  $i\leftarrow 0$\;
  $j\leftarrow n-1$\;
  \Comment{$k$ désignera un entier strictement entre $i$ et $j$}
  \Tq{ j-i > 1}{
    $k\leftarrow$ partie entière de $\frac{i+j}{2}$ \;
    \eSi{L[k] $\leq$ v}{
      $i\leftarrow k$ \;
    }{
      $j\leftarrow k$
    }
  }
  \Comment{$i$ désigne l'indice cherché.}
  \caption{}
\end{algorithm}
\end{frame}

\begin{frame}
  \frametitle{Recherche d'un mot : problème}
On cherche les occurences d'un mot $M$ de longueur $m>0$ dans un texte $T$ de longueur $t$. Les chaînes de caractères sont indexées à partir de 0.
\bigskip

Pour chaque indice $i$ entre $0$ et $t-m$, on note $T_i$ le mot dont les caractères sont $T[i] T[i+1] \cdots T[i+m-1]$.
\bigskip

L'algorithme naïf de recherche de motif consiste simplement à comparer $M$ à chaque $T_i$.
\bigskip

En Python $T_i$ peut s'obtenir avec la syntaxe \texttt{T[i:i+m]}
\end{frame}

\begin{frame}
\frametitle{Recherche d'un mot : algorithme naïf}
\begin{algorithm}[H]
  \Donnees{\;
    $M \leftarrow$ une chaîne de $m$ caractères (mot)\;
    $T \leftarrow$ une chaîne de $t$ caractères (texte)\;
  }
  \Comment{initialisation}
  $O$ désigne la liste qui recevra les indices d'occurence du mot\;
  $i\leftarrow 0$ un indice pour le texte \;
  \Tq{i $\leq$ t - m}{
    \If{ M = $T_i$}{
      placer $i$ à la fin de la liste $O$\;
    }
    incrémenter $i$
  }
  \caption{}
\end{algorithm}
\end{frame}

\begin{frame}
  \frametitle{Principe d'une amélioration}
\begin{center}
  \item Utiliser le travail fourni en comparant $M$ à $T_i$ pour incrémenter $i$ plus efficacement afin d'éviter des comparaisons dont on peut savoir à l'avance qu'elles échoueront. 
\end{center}

Exemple :

Implémenter une fonction $NbCarComm(M,T_i)$ qui renvoie le nombre de caractères (entre 0 et $m$) que les deux mots ont en commun (au début des mots).

Par exemple, si $T_i$ désigne \texttt{'tstt-tsst'} et $M$ désigne \texttt{'tsoin'}, la fonction renverra $2$ car seuls les deux premiers caractères \texttt{'t'} et \texttt{'s'} sont communs.

Les mots sont égaux si et seulement si la fonction renvoie $m$.
\end{frame}

\begin{frame}
  \frametitle{Cas particulier d'un mot sans répétition}
Si
\begin{itemize}
  \item les $p$ premiers caractères de $M$ et $T_i$ sont égaux
  \item $p<m$ c'est à dire $M\neq T_i$
  \item les caractères de $M$ sont deux à deux distincts
\end{itemize}
alors:
\begin{itemize}
  \item le premier caractère des mots $T_{i+1}, T_{i+2},\cdots T_{i+p}$ est différent du premier caractère de $M$ (puisqu'il s'agit d'un des caractères suivants de $M$.)
  \item les mots $T_{i+1}, T_{i+2},\cdots T_{i+p-1}$ sont distincts de $M$
\end{itemize}
On peut incrémenter $i$ de $\max(1,p)$ (au lieu de 1)
\end{frame}

\begin{frame}
  \frametitle{Fonction \texttt{NbCarComm}}
\begin{algorithm}[H]
  \Donnees{\;
    $M \leftarrow$ une chaîne de $m$ caractères\;
    $T \leftarrow$ une chaîne de $m$ caractères\;
  }
  \Comment{initialisation}
  $i\leftarrow 0$ (nombre de premiers indices avec les mêmes valeurs)\;
  \Tq{ i < m et M[i] = T[i]}{
    incrémenter $i$\;
  }
  renvoyer $i$\;
  \caption{}
\end{algorithm}
\end{frame}

\begin{frame}
  \frametitle{Recherche d'un motif sans répétition}
\begin{algorithm}[H]
  \Donnees{\;
    $M \leftarrow$ une chaîne de $m$ caractères (mot) sans répétition\;
    $T \leftarrow$ une chaîne de $t$ caractères (texte)\;
  }
  \Comment{initialisation}
  $O$ désigne la liste qui recevra les indices d'occurence du mot\;
  $i\leftarrow 0$ un indice pour le texte \;
  $NbCarComm \leftarrow$ une fonction comme décrite plus haut\;
  \Tq{i $\leq$ t - m}{
    $p \leftarrow NbCarComm(M , T_i)$\; 
    \If{ p = m}{
      \Comment{dans ce cas $M=T_i$}
      placer $i$ à la fin de la liste $O$\;
    }
    \Comment{incrémenter $i$}
    $i \leftarrow i + \max(1,p)$\;
  }
  \caption{}
\end{algorithm}
\end{frame}


\end{document}
