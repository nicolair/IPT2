\section*{Partie I. Mod\`ele \`a compartiments}
\begin{enumerate}
 \item On peut prendre $X = (S,I,R,D) \in \R^4$ et la fonction
 \[
  f:
  \left\lbrace 
  \begin{aligned}
   \R^4 &\rightarrow \R^4 \\
   (S,I,R,D) &\mapsto (-rIS, rIS - (a+b)I, aI, bI)
  \end{aligned}
\right. .
 \]

 \item On suppose que le paramètre \texttt{X} est passé comme un tuple et pas comme un tableau. Les premières lignes complétées sont:
\begin{lstlisting}
def f(X):
	""" Fonction definissant l'equ. differentielle """
	global r,a,b
	S, I, R, D = X
	return np.array([-r*S*I, r*S*I - (a+b)*I, a*I, b*I])
\end{lstlisting}
Si $X$ est passé comme une liste, il faut utiliser \texttt{S = X[0]}, \texttt{I = X[1]} ...

 \item On peut présumer que l'erreur est plus importante pour $N = 7$ que pour $N = 250$. Le temps de calcul le plus long est certainement pour $N = 250$.
 
 \item Code complété
\begin{lstlisting}
def f(X,Itau):
	"""
	Fonction def l'equ. diff.
	Itau est la valeur 
	   de I(t-p*dt)	
	"""
	global r,a,b
	S, I, R, D = X
	return np.array([-r*S*Itau, r*S*Itau - (a+b)*I, a*I, b*I])
\end{lstlisting}
\begin{lstlisting}
# Methode d'Euler
for i in range(N):
	t = t + dt
	# calcul de Itau
	Itau = 0.05
	if t > p * dt:
	  Itau = XX[i - p][1]
	X = X + dt * f(X, Itau)
	tt.append(t)
	XX.append(X)
\end{lstlisting}

 \item Il suffit de remplacer la ligne \texttt{Itau = XX[i-p][1]} par :
\begin{lstlisting}
 Itau = dt * sum(XX[i-j][1] * h(j * dt) for j in range(p) )
\end{lstlisting}



\section*{Partie II. Mod\'elisation dans des grilles}
 \item L'instruction renvoie une liste de $n$ listes ne contenant que des $0$.
 
 \item On initialise par le code suivant
\begin{lstlisting}
import random as rd

def init(n):
    G = [n*[0] for i in range(n)]
    i = rd.randrange(n)
    j = rd.randrange(n)
    G[i][j] = 1
    return G 
\end{lstlisting}

 \item Code pour compter les cases des différents états
 \begin{lstlisting}
def compte(G):
    cpte = 4*[0]
    for L in G:
        for e in L:
            cpte[e] += 1
    return cpte
 \end{lstlisting}

 \item La fonction \texttt{voisins(i,j,n} renvoie une liste de liste de 2 entiers. Cette liste représente les voisins en tenant compte des effets de bords.\newline
 La fonction \texttt{est\_exposee(G,i,j)} renvoie une valeur booléenne? \texttt{True} si la case est voisine d'une case infectée.
 
 \item Dans la fonction \texttt{voisins}, il manque le test pour décider si la case doit être insérée parmi les voisins c'est à dire qu'elle est bien une case de la grille.\newline Soit (à l'indentation près):
\begin{lstlisting}
if u >= 0 and u < n and v>= 0 and v < n and [a,b] != [0,0]:
    V.append([u,v])
\end{lstlisting}
Une case est exposée si une de ses voisines est contaminée c'est à dire que la valeur de $G$ associée est $1$. On examine si le produit des valeurs de G -1 pour les cases voisines est nul.
\begin{lstlisting}
def est_exposee(G,i,j):
    n = len(G)
    V = voisins(i,j,n)
    est_exp = 1
    for v in V:
        est_exp *= G[v[0]][v[1]] - 1
    return est_exp == 0
\end{lstlisting}

 \item Implémentation de la fonction \texttt{suivant(G,p1,p2)}.
\begin{lstlisting}
def suivant(G,p1,p2):
    n = len(G)
    GG = [[g for g in L] for L in G]
    for i in range(n):
        for j in range(n):
            if GG[i][j] == 1:
                if bernoulli(p1):
                    GG[i][j] = 3
                else:
                    GG[i][j] = 2
            elif GG[i][j] == 0:
                if est_exposee(G,i,j):
                    if bernoulli(p2):
                        GG[i][j] = 1
    return GG
\end{lstlisting}

 \item Implémentation de la fonction \texttt{simulation(n,p1,p2)}.
\begin{lstlisting}
def simulation(n,p1,p2):
    G = init(n)
    cpt = 0
    #l'epidemie continue tant que quelqu'un est infecte
    while compte(G)[1] > 0:
        G = suivant(G,p1,p2)
        cpt += 1
    print(cpt)
    N = n*n   #nb total de cases
    return [c/N for c in compte(G)]
\end{lstlisting}
L'énoncé ne le demandait pas mais on fait afficher un compteur de la durée de l'épidémie.

 \item \`A la fin $x_1=0$ car c'est la condition de sortie de la boucle. Comme les $x_i$ sont des proportions:
\[
 x_1 + x_2 + x_3 + x_4 = 1.
\]
Les cases atteintes par l'épidémie sont les décédées ou les guéries c'est à dire
\[
x_{\text{atteintes}} = x_2 + x_3.
\]

 \item Il s'agit quasiment d'une question de cours. On doit implémenter la méthode de dichotomie. On cherche i tel que $Lxa[i]<= 0.5$ et $Lxa[i+1] > 0.5$. Au cours de la boucle, $i$ et $j$ verifieront toujours
\begin{center}
 $i$ et $j$ entiers et $i < j$ et $Lxa[i] <= 0.5$ et $Lxa[j] > 0.5$.
\end{center}
La boucle s'arrête quand $j-i < 2$.
\begin{lstlisting}
def seuil(Lp2,Lxa):
   i = 0
   j = len(Lxa) - 1
   while j - i >= 2:
       k = (i+j)//2
       if Lxa[k] <= 0.5:
           i = k
       else:
           j = k
   return Lp2[i]
 \end{lstlisting}
 
 \item On ne peut pas supprimer le test de la ligne 8 car on initialise avec une case infectée qui ne doit pas être vaccinée. De plus, il ne faut pas revacciner une case qui a été vaccinée.
 
 \item L'appel \texttt{init\_vac(5, 0.2)} renvoie une grille $5 \times 5$ dont exactement $5$ cases sont à 2 (rétablie = vaccinée), une est à 1 (infectée) et toutes les autres à 0 (saine).

\end{enumerate}