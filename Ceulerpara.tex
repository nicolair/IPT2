\begin{enumerate}
  \item Pour le schéma d'Euler usuel, les relations permettant de calculer les $v_i$ sont
\begin{displaymath}
\forall i\in \llbracket 0, n-1 \rrbracket,\; v_{i+1} = v_i + (g - Kv_i^2)h  
\end{displaymath}

  \item Une implémentation de la fonction \texttt{vpara} est donnée au dessous. 
\lstinputlisting[firstline=10, lastline=19]{Ceulerpara.py}

  \item Le code proposé permet d'afficher le graphe des vitesses calculées par la méthode d'Euler en fonction du temps pour des vitesses initiales de $0.$, $10.$, $20.$, $30.$ (dans les unités usuelles). Sur le graphique fourni, on constate que ces vitesses semblent converger vers une vitesse limite qui est indépendante de la vitesse initiale.
  
  \item On peut décomposer en éléments simples:
\begin{displaymath}
  \frac{1}{g - Kv^2} = \frac{\frac{1}{2\sqrt{g}}}{\sqrt{g} - \sqrt{K}v} + \frac{\frac{1}{2\sqrt{g}}}{\sqrt{g} + \sqrt{K}v}
\end{displaymath}
On en déduit qu'une primitive de $t\mapsto \frac{1}{g - Kv^2}$ est
\begin{displaymath}
  \frac{1}{2\sqrt{Kg}}\left( -\ln|\sqrt{g} - \sqrt{K}v| + \ln|\sqrt{g} + \sqrt{K}v|\right) 
  = \frac{1}{2\sqrt{Kg}} \ln \left| \frac{\sqrt{g} + \sqrt{K}v}{\sqrt{g} - \sqrt{K}v}\right|
\end{displaymath}
En intégrant entre $t_0 = 0$ et $t$, on obtient
\begin{displaymath}
 \frac{1}{2\sqrt{Kg}}\ln \left| \frac{(\sqrt{g} + \sqrt{K}v)(\sqrt{g} - \sqrt{K}v_0)}{(\sqrt{g} - \sqrt{K}v)(\sqrt{g} + \sqrt{K}v_0)}\right| = t
\end{displaymath}
En posant 
\begin{displaymath}
  M = 2\sqrt{Kg},\hspace{1cm} A = \left|\frac{\sqrt{g} + \sqrt{K}v_0}{\sqrt{g} - \sqrt{K}v_0}\right|
\end{displaymath}
on en tire
\begin{displaymath}
\left|\frac{\sqrt{g} + \sqrt{K}v}{\sqrt{g} - \sqrt{K}v}\right|
= Ae^{Mt}
\end{displaymath}
Comme cette expression diverge vers $+\infty$ en $+\infty$, on en déduit que $\sqrt{g} - \sqrt{K}v$ converge vers $0$. Ce calcul prouve l'existence d'une vitesse limite égale à $\sqrt{\frac{g}{K}}$.\newline
Remarque.\newline
Si on \emph{admet} l'existence d'une vitesse limite et la convergence vers $0$ de l'accélération, on obtient facilement l'expression de la vitesse limite à partir de l'équation différentielle:
\begin{displaymath}
  v'(t) = g - Kv(t) \rightarrow 0 \Rightarrow v(t)\rightarrow \sqrt{\frac{g}{K}}
\end{displaymath}

\end{enumerate}
