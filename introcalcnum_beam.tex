%!  pour pdfLatex
\documentclass{beamer}

%\usepackage[pdftex]{graphicx,color}
%\usepackage[pdftex,colorlinks={true},urlcolor={blue},pdfauthor={remy Nicolai}]{hyperref}

\usepackage[utf8]{inputenc}
\usepackage[T1]{fontenc}
\usepackage{lmodern}
\usepackage[frenchb]{babel}

\usetheme{Warsaw}

%\usepackage{fancyhdr}
%\pagestyle{fancy}

%\usepackage{floatflt}
\usepackage{maths}

\usepackage{parcolumns}
\setlength{\parindent}{0pt}

%\usepackage{caption}
%\usepackage{subcaption}

\usepackage[french,ruled,vlined]{algorithm2e}
\SetKwComment{Comment}{\#}{}
\SetKwFor{Tq}{tant que}{}{}
\SetKwFor{Pour}{pour}{}{}
\DontPrintSemicolon
\SetAlgoLined

\usepackage{listings}
\lstset{language=Python,frame=single}

%pr{\'e}sentation des compteurs de section, ...
\makeatletter
\renewcommand{\thesection}{\Roman{section}.}
\renewcommand{\thesubsection}{\arabic{subsection}.}
\renewcommand{\thesubsubsection}{\arabic{subsubsection}.}
%\renewcommand{\labelenumii}{\theenumii.}
\makeatother


\newtheorem*{thm}{Théorème}
\newtheorem{thmn}{Théorème}
\newtheorem*{prop}{Proposition}
\newtheorem{propn}{Proposition}
\newtheorem*{pa}{Présentation axiomatique}
\newtheorem*{propdef}{Proposition - Définition}
\newtheorem*{lem}{Lemme}
\newtheorem{lemn}{Lemme}

\theoremstyle{definition}
\newtheorem*{defi}{Définition}
\newtheorem*{nota}{Notation}
\newtheorem*{exple}{Exemple}
\newtheorem*{exples}{Exemples}


\newenvironment{demo}{\renewcommand{\proofname}{Preuve}\begin{proof}}{\end{proof}}
%\renewcommand{\proofname}{Preuve} doit etre après le begin{document} pour fonctionner

\theoremstyle{remark}
\newtheorem*{rem}{Remarque}
\newtheorem*{rems}{Remarques}

%\usepackage{maths}
%\newcommand{\dbf}{\leftrightarrows}

%En tete et pied de page
%\lhead{Informatique}
%\chead{Introduction aux systèmes informatiques}
%\rhead{MPSI B Hoche}
%\lfoot{\tiny{Cette création est mise à disposition selon le Contrat\\ Paternité-Partage des Conditions Initiales à l'Identique 2.0 France\\ disponible en ligne http://creativecommons.org/licenses/by-sa/2.0/fr/
%} }
%\rfoot{\tiny{Rémy Nicolai \jobname}}

\nonstopmode
\lstset{language=Scilab,frame=single}
\begin{document}
\begin{frame}
  \frametitle{Scilab : présentation}
Langage de programmation interprété associé à une collection d'algorithmes \emph{numériques}

L'espace de travail est constitué de plusieurs fenêtres:
\begin{itemize}
  \item la console pour exécuter des commandes
  \item l'éditeur pour écrire des programmes
  \item les fenêtres graphiques pour afficher des graphiques
  \item le navigateur d'aide
\end{itemize}
\end{frame}

\begin{frame}[fragile]
  \frametitle{Scilab: console}
Exécution interactive de commandes 

Fenêtre en haut à droite : "Navigateur de variables" pour garder la maîtrise de ce que représentent les noms déjà définis.

Après l'invite \texttt{---->}, saisir une commande et appuyer sur la touche "Entrée" ou "Retour".
\begin{verbatim}
  --> 57/4
  -->1 + 2
\end{verbatim}
si ";" après une commande: rien ne s'affiche.

Pour interrompre un programme en cours d'exécution, passer par le menu "Contrôle" de la barre de menu.
\end{frame}

\begin{frame}
  \frametitle{Scilab: données}
\begin{itemize}
  \item Le point du \texttt{3.} dans le deuxième exemple signifie que ce qui est renvoyé est de type float.
  \item En Scilab, tous \emph{les calculs sont numériques} (norme IEEE 754 en double précision sur 64 bits).
  \item Les objets mathématiques sont hors de portée. Scilab ne peut en fournir que des valeurs numériques approchées.\newline
 \texttt{\%e } et \texttt{\%pi} désignent les valeurs approchées des nombres $e$ et $\pi$ et non ces nombres eux mêmes.
  \item En fait Scilab manipule essentiellement des \emph{matrices} de nombres flottants.
\end{itemize}
\end{frame}

\begin{frame}
  \frametitle{Scilab: navigateur d'aide}
Le \emph{navigateur d'aide} est le principal outil.

ouvrir le navigateur d'aide :
\begin{itemize}
  \item avec le "?" de la barre de menu puis "Aide de Scilab"
  \item en tapant simplement sur le "F1" du clavier
\end{itemize}
se déplacer dans le navigateur d'aide
\begin{itemize}
  \item dans le navigateur d'aide : table des matières ou ligne de recherche.
  \item dans la console taper \texttt{--> help mot} (avec guillemets)
\end{itemize}
\end{frame}

\begin{frame}
  \frametitle{Scilab: éditeur de scripts}
L'éditeur est appelé sciNotes.\newline
Fichiers de code: extension ".sce".\newline
Pour l'ouvrir: premiere icone de la barre d'outils ou  "Applications>SciNotes" de la barre de menus.\newline
Enregistrement par "Fichier>Enregistrer" de la barre de menus ou par l'icone de la barre d'outils. \newline
Exécution du code dans la console par "Exécuter>...fichier.." de la barre de menus. Possibilités d'afficher ou non les commandes.\newline
Les commentaires sont précédés par "//".\newline
En principe une seule commande par ligne, plusieurs en les séparant par des "," ou des ";".\newline
Espaces ignorés, casse (majuscule / minuscule) significative.
\end{frame}

\begin{frame}[fragile]
  \frametitle{Scilab: types de données}
\begin{itemize}
  \item "nombres réels" : float
  \item matrices de nombres
  \item booléens \texttt{\%t} ou \texttt{\%T}, \texttt{\%f} ou \texttt{\%F}. Les opérateurs 
  \begin{verbatim}
    &  |  ~  ==  ~=  <>  <  >  <=  >=
  \end{verbatim}
renvoient des valeurs booléennes. Noter que \verb|~a| est la négation de \texttt{a} et que l'on peut utiliser indifféremment \verb|~=| ou \texttt{<>}.
\item \texttt{\&}: forme infixe de la fonction préfixe \texttt{and}. Idem pour \texttt{|} et \texttt{or}.
\item Entiers : plusieurs sortes d'entiers en dehors des "entiers flottants".
\item Chaînes de caractères: délimiteurs \texttt{"} (guillemets).
\end{itemize}
\end{frame}

\begin{frame}
  \frametitle{Scilab: fonctions usuelles}
Syntaxe d'appel : nom de la fonction suivie des paramètres entre parenthèses.
\begin{itemize}
  \item opérations usuelles : "+", "*", "\textasciicircum" (puissance).
  \item \texttt{disp} affiche l'évaluation de son argument.
  \item \texttt{tic} , \texttt{toc} déclenche et arrête un chronomètre.
  \item \texttt{format} nombre de chiffres \emph{présentés} (le nombre des chiffres \emph{utilisés} non modifié)
  \item \texttt{x} désignant un nombre complexe: \texttt{sqrt(x)} (partie réelle $\geq 0$), \texttt{exp(x)}, \texttt{abs(x)}, \texttt{int(x)} (troncature), \texttt{floor(x)}, \texttt{ceil(x)}
  \item \texttt{x} désignant une matrice: \texttt{sum(x)}, \texttt{prod(x)}, \texttt{gsort(x)} (trier), \texttt{length(x)} , \texttt{size(x)}
\end{itemize}

\end{frame}

\begin{frame}[fragile]
  \frametitle{Scilab: matrices, construction}
\begin{itemize}
  \item Toutes les données sont des matrices.
  \item Les nombres (éventuellement complexes mais float) sont des matrices $1\times1$.
  \item Constructeur-délimiteur : les crochets "\texttt{[ ]}" avec des séparateurs 
  \begin{itemize}
    \item "," pour des matrices lignes
    \item ";" pour des matrices colonnes
  \end{itemize}
\begin{verbatim}
  disp([1,3,7]) ; disp([1;3;7]);
  truc = [1,2,[4;8],[77,55,88]];
\end{verbatim}

  \item L'opérateur "\texttt{:}" renvoie une matrice ligne de nombres (incrément 1 par défaut) . On peut introduire un deuxième ":". Le terme du milieu désigne l'incrément.
\begin{verbatim}
  disp(1.2:5.1); disp(1.2:0.5:5.1);
\end{verbatim}
\end{itemize}
\end{frame}

\begin{frame}[fragile]
  \frametitle{Scilab: matrices, itération}
L'itérateur \texttt{for} permet de parcourir une matrice ligne 
\begin{verbatim}
  // matrice est le nom d'une matrice
  for truc = matrice
    // truc désigne l'élément courant de la matrice 
    instructions;
  end
\end{verbatim}
Pour faire varier un compteur \texttt{i} en l'incrémentant de 1 entre deux entiers désignés par \texttt{a} et \texttt{b}, utiliser 
\begin{verbatim}
  for i = a:b instructions; end
\end{verbatim}
\end{frame}

\begin{frame}[fragile]
  \frametitle{Scilab: matrices, valeurs}
\begin{itemize}
  \item Parenthèses pour accéder aux valeurs et les modifier.
  \item Si \texttt{nomv} est le nom d'une variable, c'est le nom d'une matrice. Si \texttt{i} entier, \texttt{nomv(i)} évalué comme la ieme valeur de la matrice ou erreur d'index si \texttt{i} ne désigne pas un indice acceptable.
\begin{verbatim}
u = [1,2,3,4,7,8]; disp(u(4)); disp(u(9));  
u = [1;2;3;4;7;8]; disp(u(4)); disp(u(9)); 
u = [[1,2];[3,4];[7,8]]]; disp(u(1)); 
disp(u(2)); disp(u(3));disp(u(4)); disp(u(9));  
\end{verbatim}
  \item Si \texttt{nomv(i)} est à gauche d'un signe \texttt{=} et si l'expression à droite est évaluée à un type compatible avec \texttt{nomv}, la matrice est étendue par des valeurs nulles jusqu'à l'indice \texttt{i}.
\begin{verbatim}
  u = 4:10 ; u(15) = 17.5; disp(u)
\end{verbatim}
  \item Attention le premier indice d'une matrice est $1$ et non pas $0$.
\end{itemize}
\end{frame}

\begin{frame}[fragile]
  \frametitle{Scilab: noms}
\begin{itemize}
  \item Langage non typé.
  \item Un nom désigne une matrice ou une fonction.
  \item Syntaxe d'assignation d'un nom de matrice: nom = expression.
  \item Si le résultat d'un calcul n'est pas assigné à un nom, il l'est par défaut à \texttt{ans}. (answer)
  \item Syntaxe de définition d'une fonction. Le nom doit être placé à \emph{droite} de l'égalité:
\begin{verbatim}
function sortie = nom_de_fonction(entree)
  instructions assignant une valeur à sortie;
endfunction
\end{verbatim}
\texttt{sortie} et \texttt{entree} sont locaux au code entre \texttt{function} et \texttt{endfunction}. Ils peuvent être remplacés par n'importe quels autres mots (non utilisés).
\end{itemize}
\end{frame}

\begin{frame}[fragile]
  \frametitle{Scilab: contrôles}
\begin{itemize}
  \item Syntaxe du \texttt{if} (\texttt{cond} s'évalue en booléen)
\begin{verbatim}
  if ( cond ) then
    instructions
  else
    autres intructions
  end
\end{verbatim}
Le \texttt{else} et le segment de code qui suit sont facultatifs. On peut introduire des segments de code avec des \texttt{elseif}.

\item Le branchement \texttt{select} permet de combiner plusieurs \texttt{if} pour diriger vers plusieurs possibilités. Voir l'aide.

\item La syntaxe de la boucle \texttt{while} 
\begin{verbatim}
  while ( cond )
    instructions
  end
\end{verbatim}
\end{itemize}
\end{frame}

\begin{frame}
  \frametitle{Des suites définies par récurrence}
On considère des suites de nombres réels définies par une relation de récurrence d'ordre $2$ et deux valeurs initiales.
\begin{displaymath}
  u_1>0,\; u_2>0,\;
  \forall n>2:\;
  u_{n}= 111 - \frac{1130}{u_{n-1}} + \frac{3000}{u_{n-1}u_{n-2}}
\end{displaymath}
On se propose de les explorer numériquement avec Scilab.
\end{frame}

\begin{frame}
  \frametitle{Implémentation numérique Scilab 1}
On garde la matrice de toutes les valeurs de la suite
\lstinputlisting[language=Scilab, firstline=5, lastline=12]{introcalcnum.sce}
\end{frame}

\begin{frame}[fragile]
  \frametitle{Implémentation numérique Scilab 2}
On ne garde que les valeurs permettant le calcul.
\lstinputlisting[language=Scilab, firstline=14, lastline=20]{introcalcnum.sce}

Pour les conditions initiales proposées, il semble que la suite converge vers $100$ et pour $(\frac{11}{2},\frac{61}{11})$ ?
\end{frame}

\begin{frame}
  \frametitle{Implémentation numérique Python}
Comme la précédente, mais en Python
\lstinputlisting[language=Python, firstline=7, lastline=15]{introcalcnum.py}
\end{frame}

\begin{frame}
  \frametitle{Implémentation formelle Python}
Valeurs initiales dans $\Q$ $\Rightarrow$ toutes les valeurs de la suite dans $\Q$.\newline
Pour calculer dans $\Q$: bibliothèque de calcul formel sympy.
\lstinputlisting[language=Python, firstline=26, lastline=34]{introcalcnum.py}
\end{frame}

\begin{frame}[fragile]
  \frametitle{Maths avec Scilab et SymPy}
Si convergence, la limite est une racine du polynôme
\begin{displaymath}
  X^3 - 111X^2+1130X-3000
\end{displaymath}
\'Evaluation numérique des racines avec Scilab
\begin{verbatim}
P = poly([-3000,1130,-111,1],"X","coeff")
disp(P)
racines = roots(P)  
\end{verbatim}
Il semble que les racines soient $5$, $6$ et $100$. Preuve avec Sympy
\begin{verbatim}
X = sympy.symbols('X')
P = (X-5)*(X-6)*(X-100)
P = sympy.expand(P)
print(P)
\end{verbatim}
\end{frame}

\begin{frame}
  \frametitle{Interprétation géométrique}
$\mathcal{P} = \R^2$ privé des deux axes de coordonnées, $F$ de $\mathcal{P}$ dans $\R^2$ 
\begin{displaymath}
 F:\; (x,y)\mapsto (y, 111 - \frac{1130}{y} + \frac{3000}{xy})
\end{displaymath}
Les suites de nombres sont associées à des suites de points:
\begin{displaymath}
  P_1 = (u_1,u_2),\; \forall n\geq 2, P_n = (u_n,u_{n+1}) = F(P_{n-1})
\end{displaymath}
\begin{itemize}
  \item Points fixes pour $F$: $A=(5,5)$, $B=(6,6)$, $C=(100,100)$.
  \item Numériquement $C$ stable et $A$, $B$ instables.
  \item Mathématiquement incohérent.
\end{itemize}
\end{frame}

\begin{frame}
  \frametitle{Une explication}
\begin{itemize}
  \item Il existe une courbe $\Gamma$ (contenant $A$ et $B$ mais pas $C$) et globalement stable: $M\in \Gamma \Rightarrow f(M)\in \Gamma$.
  \item Point initial $\in\Gamma$ $\Rightarrow$ suite sur $\Gamma$ et converge vers $B$.
  \item Point initial $\notin \Gamma$ $\Rightarrow$ suite de points converge vers $C$.
  \item \'Equation de la courbe à coefficients entiers, calcul formel exact avec un point initial $\in \Gamma$, la convergence mathématique se réalise.
  \item Calcul numérique. Erreurs d'arrondi $\Rightarrow$ les points sortent de $\Gamma$ et sont capturés par le bassin d'attraction de $C$.
\end{itemize}
\end{frame}

\begin{frame}
  \frametitle{Comment trouver la courbe stable?}
\begin{itemize}
  \item Avec SymPy. Mais calcul formel sorti du programme.
  \item Numériquement.
  \begin{itemize}
    \item On considére tous les points d'abscisse entre $1.1$ et $10$.
    \item On forme la suite dont les valeurs initiales sont les coordonnées du point.
    \item On trace le point seulement si le nombre d'itérations nécessaire pour être proche de $100$ est supérieur à une valeur donnée.
  \end{itemize}
\end{itemize}
\end{frame}

\begin{frame}
  \frametitle{Implémentation Scilab: bonindice}
\lstinputlisting[language=Scilab, firstline=23, lastline=33]{introcalcnum.sce}
\end{frame}

\begin{frame}
  \frametitle{Implémentation Scilab: tracé}
\lstinputlisting[language=Scilab, firstline=35, lastline=50]{introcalcnum.sce}
\end{frame}

\end{document}

