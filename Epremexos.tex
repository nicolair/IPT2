\begin{enumerate}
 \item \'Ecrire du code calculant et affichant la factorielle d'un entier strictement positif.
 \item Soit $n$ un entier  supérieur ou égal à 2 désigné par une variable \verb|n|. \'Ecrire un code affichant \verb|premier| ou \verb|composé| suivant que $n$ est premier ou non. \'Ecrire un code calculant et affichant le nombre de diviseurs naturels de $n$.
 \item Implémenter le pseudo-code suivant
 \begin{verbatim}
n <-- un entier
code <-- une chaine de caractère vide
tant que n > 0
  si n est pair
    n <-- m tel que n = 2m
    concaténer "P" à droite de code
  si n est impair
    n <-- m tel que n = 2m + 1
    concaténer "I" à droite de code
afficher code
 \end{verbatim}
Modifier ce code pour qu'il calcule la chaine de caractère égale à la décomposition binaire de l'entier fourni. Elle commence par \texttt{'0b'}. Modifier encore pour fournir la décomposition en base 3. Elle commence par \texttt{'0t'}.

 \item Former une phrase qui s'évalue à VRAI après l'initialisation et après l'exécution de chaque bloc de la boucle du code suivant. \`A quoi sert ce code ? Comment l'améliorer ?
 \begin{verbatim}
n = 2554 ; p = 39
i = 0
while n > p :
  n -= p
  i += 1
print(n,i)
 \end{verbatim}
 
\item Dessiner avec \texttt{print}.\newline
\texttt{print} (mot-clé ou statement en 2.7, fonction en 3) ne fait qu'une chose: se placer au début d'une nouvelle ligne et afficher caractère par caractère la chaîne qu'on lui passe.
\begin{enumerate}
  \item On se donne des entiers \texttt{l}, \texttt{c}, \texttt{q}. En utilisant seulement les trois caractères \texttt{'-'} (tiret horizontal), \texttt{'|'} (tiret vertical) et \texttt{' '} (espace), écrire un code qui affiche un tableau vide 
  \begin{verbatim}
                -----------------            
                |   |   |   |   |          Exemple avec
                |   |   |   |   |             l = 6
                |   |   |   |   |             c = 4
                |   |   |   |   |             q = 3   
                -----------------
  \end{verbatim}    
  formé de \texttt{p} lignes. Chacune (sauf les deux bords) contient \texttt{c} cellules constituées de \texttt{q} espaces.
  
  \item On souhaite afficher un cercle approximatif uniquement avec \texttt{print}.\newline
  Un entier naturel désigné par \texttt{r} est fourni. On utilisera aussi deux noms \texttt{x} et \texttt{y} qui désignent les coordonnées entières des points du plan d'impression muni d'un repère (défini en figure \ref{fig:Epremexos_1}) ainsi que \texttt{e} qui désigne un flottant caractérisant le degré d'imprécision que l'on se laisse.\newline
\begin{figure}[h]
  \centering
  \includegraphics{./Epremexos_1.pdf}
  % Epremexos_1.pdf: 0x0 pixel, 0dpi, 0.00x0.00 cm, bb=
  \caption{Repère dans le plan d'impression}
  \label{fig:Epremexos_1}
\end{figure}
Le cercle à tracer est de rayon $r$, les coordonnées de son centre sont $(r,r)$. Imprimer approximativement ce cercle à partir de la donnée d'un \texttt{r}. Vous serez amené à utiliser l'expression
\begin{center}
  \texttt{abs((x-r)**2 + (y-r)**2 -r**2) < e}
\end{center}
et à expérimenter sur les valeurs de \texttt{e} et sur les caractères à imprimer pour obtenir un résultat acceptable.
\end{enumerate}

\item La suite de Syracuse est définie par son premier terme $u_0$ (entier naturel non nul) et la relation de récurrence
\begin{displaymath}
 u_{n+1} = 
 \left\lbrace 
 \begin{aligned}
  \frac{u_n}{2} &\text{ si } u_n \text{ est pair}\\
  3u_n+1 &\text{ sinon }
 \end{aligned}
 \right.
\end{displaymath}
\begin{enumerate}
 \item Former un programme qui affiche les 20 premiers termes de la suite pour $u_0=15$.
 \item Modifier le programme précédent pour qu'il affiche seulement le plus petit $n$ tel que $u_n=1$.
 \item On admet que, pour tout $x$ entre $1$ et $20$, la suite de Syracuse initialisée par $u_0=x$ prend la valeur $1$ pour un indice plus petit que $20$. On note $i(x)$ le plus petit de ces indices pour un $x$ fixé. Calculer 
\begin{displaymath}
  \max\left\lbrace i(x), x\in\llbracket 0, n \rrbracket\right\rbrace 
\end{displaymath}
\end{enumerate}

\end{enumerate}
