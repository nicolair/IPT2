%!  pour pdfLatex
\documentclass{beamer}

%\usepackage[pdftex]{graphicx,color}
%\usepackage[pdftex,colorlinks={true},urlcolor={blue},pdfauthor={remy Nicolai}]{hyperref}

\usepackage[utf8]{inputenc}
\usepackage[T1]{fontenc}
\usepackage{lmodern}
\usepackage[frenchb]{babel}

\usetheme{Warsaw}

%\usepackage{fancyhdr}
%\pagestyle{fancy}

%\usepackage{floatflt}
\usepackage{maths}

\usepackage{parcolumns}
\setlength{\parindent}{0pt}

%\usepackage{caption}
%\usepackage{subcaption}

\usepackage[french,ruled,vlined]{algorithm2e}
\SetKwComment{Comment}{\#}{}
\SetKwFor{Tq}{tant que}{}{}
\SetKwFor{Pour}{pour}{}{}
\DontPrintSemicolon
\SetAlgoLined

\usepackage{listings}
\lstset{language=Python,frame=single}

%pr{\'e}sentation des compteurs de section, ...
\makeatletter
\renewcommand{\thesection}{\Roman{section}.}
\renewcommand{\thesubsection}{\arabic{subsection}.}
\renewcommand{\thesubsubsection}{\arabic{subsubsection}.}
%\renewcommand{\labelenumii}{\theenumii.}
\makeatother


\newtheorem*{thm}{Théorème}
\newtheorem{thmn}{Théorème}
\newtheorem*{prop}{Proposition}
\newtheorem{propn}{Proposition}
\newtheorem*{pa}{Présentation axiomatique}
\newtheorem*{propdef}{Proposition - Définition}
\newtheorem*{lem}{Lemme}
\newtheorem{lemn}{Lemme}

\theoremstyle{definition}
\newtheorem*{defi}{Définition}
\newtheorem*{nota}{Notation}
\newtheorem*{exple}{Exemple}
\newtheorem*{exples}{Exemples}


\newenvironment{demo}{\renewcommand{\proofname}{Preuve}\begin{proof}}{\end{proof}}
%\renewcommand{\proofname}{Preuve} doit etre après le begin{document} pour fonctionner

\theoremstyle{remark}
\newtheorem*{rem}{Remarque}
\newtheorem*{rems}{Remarques}

%\usepackage{maths}
%\newcommand{\dbf}{\leftrightarrows}

%En tete et pied de page
%\lhead{Informatique}
%\chead{Introduction aux systèmes informatiques}
%\rhead{MPSI B Hoche}
%\lfoot{\tiny{Cette création est mise à disposition selon le Contrat\\ Paternité-Partage des Conditions Initiales à l'Identique 2.0 France\\ disponible en ligne http://creativecommons.org/licenses/by-sa/2.0/fr/
%} }
%\rfoot{\tiny{Rémy Nicolai \jobname}}

\nonstopmode
\lstset{language=Python,frame=single}
\begin{document}
\begin{frame}
  \frametitle{Résolution numérique d'équations: principes généraux}
Toute équation à une inconnue réelle peut se mettre sous la forme
\begin{displaymath}
  f(x) = 0
\end{displaymath}
où $f$ est une fonction définie dans une partie $I$ de $\R$. Deux étapes :
\begin{itemize}
  \item Séparation des solutions.\newline
Trouver des intervalles $J$ dans $I$. Chaque $J$ contient au moins une solution et $f|_J$ a de \og bonnes\fg  propriétés.
  \item \'Evaluation numérique. \newline
  Algorithme, avec des données spécifiques à chaque intervalle $J$, renvoyant une valeur numérique approchant une solution dans $J$ avec une erreur plus petite qu'une valeur prescrite.
\end{itemize} 
\end{frame}

\begin{frame}
  \frametitle{Méthode dichotomique: propriétés à vérifier}
\begin{itemize}
  \item $J$ est un segment $[a,b]$.
  \item $f$ continue sur $J$ et $f(a)f(b)\leq 0$. 
\end{itemize}
Conséquences
\begin{itemize}
  \item TVI: il existe une solution dans $J$.
  \item $f$ strictement monotone: unicité dans $J$. 
\end{itemize}
\end{frame}

\begin{frame}
  \frametitle{Méthode dichotomique: preuve de l'algorithme}
Les valeurs renvoyées sont des binaires normalisés: majorant de l'erreur en $\frac{1}{2^p}$.

$k$ compteur de boucle
\begin{center}
\renewcommand{\arraystretch}{1.5}
\begin{tabular}{|c|c|c|}\hline
Fonction de terminaison                     & Invariant      & Sortie de boucle \\ \hline
$p-k + \lceil \frac{\ln(b-a)}{\ln 2}\rceil$ & $f(a)f(b)\leq 0$ & $ b - a \leq e$ \\ \hline
\end{tabular}
\end{center}
$\Rightarrow$ il existe toujours un zéro de la fonction dans l'intervalle $[a,b]$

$\Rightarrow$ après la sortie de boucle tout nombre dans le segment est une valeur approchée à $e$ près d'une solution.

$\Rightarrow$ on renvoie une extrémité ou on peut chercher à faire mieux.
\end{frame}

\begin{frame}
  \frametitle{Méthode dichotomique: pseudo-code}
\begin{algorithm}[H]
  \Donnees{\;
    $a$ et $b$ deux flottants définissant l'intervalle\;
    $f$ fonction renvoyant un flottant pour tout flottant entre $a$ et $b$\;
    $e$ flottant : l'erreur doit être inférieure à $e$\;
  }
  
  $va\leftarrow f(a)$   $vb\leftarrow f(b)$\;
  
  \Tq{ $b-a > e$}{
    $m\leftarrow  \frac{a+b}{2}$ \;
    $vm \leftarrow f(m)$\;
    \eSi{ $vm\, va \leq 0$}{
      $b \leftarrow m$       $vb \leftarrow vm$ \;
    }{
      $a \leftarrow m$       $va \leftarrow vm$ \;
    }
  }
  Renvoyer $a$\;
  \caption{Méthode dichotomique}
  \label{resolnumequ_1}
\end{algorithm}
\end{frame}

\begin{frame}
  \frametitle{Méthode dichotomique: implémentation Python (fonction) }
\lstinputlisting[firstline=11, lastline=21]{resolnumequ.py}
\end{frame}

\begin{frame}
  \frametitle{Méthode dichotomique: implémentation Python (appel) }
\lstinputlisting[firstline=22, lastline=26]{resolnumequ.py}
Opérateur \texttt{lambda} pour passer des fonctions \emph{anonymes}.
\lstinputlisting[firstline=27, lastline=28]{resolnumequ.py}

\end{frame}

\begin{frame}
  \frametitle{Méthode dichotomique: exercice (début)}
Modifier le code Python du paragraphe précédent pour renvoyer
\begin{enumerate}
  \item le milieu du segment
  \item la meilleure des deux extrémités (celle qui a la plus petite valeur de la fonction)
  \item une moyenne des extrémités pondérée à l'aide des valeurs de la fonction de manière à privilégier l'extrémité dont la valeur est la plus petite.
\end{enumerate}
\end{frame}

\begin{frame}[fragile]
  \frametitle{Méthode dichotomique: exercice (fin)}

Pour comparer les valeurs approchées de $\sqrt{2}$, tracer sur un même graphique les courbes des erreurs pour chaque méthode en fonction du nombre d'itérations.

On prendra pour valeur \og exacte\fg la valeur renvoyée par la fonction \texttt{sqrt} de la bibiothèque \texttt{math}.
\end{frame}

\begin{frame}
  \frametitle{Méthode de Newton: propriétés à vérifier}
\begin{itemize}
  \item $J$ est un segment $[a,b]$
  \item classe $\mathcal{C}^2$
  \item strictement croissante, convexe $f''>0$
  \item $f(a)<0$, $f(b)>0$. 
\end{itemize}
On peut être amené à remplacer $f$ par $-f$ pour que la condition de croissance soit vérifiée. 
\end{frame}

\begin{frame}
  \frametitle{Méthode de Newton: formalisme mathématique}
\begin{itemize}
  \item Il existe dans $[a,b]$ une unique solution notée $c$.
  \item Remplacer l'extrémité droite de l'intervalle par l'intersection de la tangente en ce point avec l'axe des abscisses.
  \item Traduction par une suite définie par récurrence
\begin{displaymath}
  b_0 = b,\hspace{0.5cm}\forall n\in\N,\;b_{n+1} = b_n -\frac{f(b_n)}{f'(b_n)}
\end{displaymath}
  \item Suite strictement décroissante qui converge vers $c$.
  \item Convergence quadratique (majoration de l'erreur)
\begin{displaymath}
  0< b_n -c \leq \frac{M_2}{2m_1}(b_{n-1}-b_n)^2,\hspace{0.5cm} 0< b_{n+1} -c \leq \frac{M_2}{2m_1}(c-b_n)^2
\end{displaymath}
avec $m_1 = \min_{[a,b]} f'>0$ et $M_2 = \max_{[a,b]}\left|f''\right|$.
\end{itemize}
\end{frame}

\begin{frame}
  \frametitle{Méthode de Newton et point fixe}
Remplacer un problème de recherche de zéro par un problème de recherche de \emph{point fixe}.
\begin{displaymath}
  \Phi(x) = x -\frac{f(x)}{f'(x)}, \hspace{1cm} f(x) = 0 \Leftrightarrow \Phi(x) = x
\end{displaymath}
$c$ un zéro de $f$ implique $c$ point fixe de $\Phi$ \og super-attractif\fg c'est à dire que $\Phi'(c)=0$.

Ceci explique la convergence très rapide (quadratique).

Le nombre de décimales exactes double à chaque étape.
\end{frame}

\begin{frame}
  \frametitle{Méthode de Newton: implémentation Python}
\lstinputlisting[firstline=84, lastline=97]{resolnumequ.py}
\end{frame}


\begin{frame}
  \frametitle{Méthode de Newton: exercice dans le champ complexe}
\begin{itemize}
  \item La méthode de Newton fonctionne encore dans le champ complexe.
  \item Exemple: fonction $z\mapsto z^3 -1$.
  \item Convergence sauf pour un ensemble dénombrable de valeurs initiales.
  \item Si convergence, la limite est $1$ ou $j$ ou $j^2$.
\end{itemize}

Dessiner les bassins d'attraction de ces racines en coloriant chaque point suivant la limite de la suite de Newton formée avec cette condition initiale.
\end{frame}

\end{document}
