L'objet de cet exercice est de faire comprendre que lorsque l'on considère des suites de la forme
\begin{displaymath}
  \left( \sum_{k=1}^n a_k \right)_{n\in \N^*} 
\end{displaymath}
où $\left( a_n\right)_{n\in \N^*}$ est une suite de nombres réels fixée, l'ordre dans lequel on somme les $a_k$ peut jouer un rôle absolument capital (en particulier lorsque les $a_k$ ne sont pas du même signe).\newline
On considère pour cela la suite définie par
\begin{displaymath}
  \forall k\in \N^*:\;a_k = \frac{(-1)^{k+1}}{k}
  =
  \left\lbrace 
  \begin{aligned}
    &\frac{1}{k} &\text{ si $k$ est impair}\\
    &-\frac{1}{k} &\text{ si $k$ est pair}\\
  \end{aligned}
\right. 
\end{displaymath}
On admettra (ce qui a été démontré en cours de maths) que
\begin{displaymath}
    \left( \sum_{k=1}^n \frac{1}{k} \right)_{n\in \N^*}\rightarrow +\infty,\hspace{0.5cm}
    \left( \sum_{k=1}^n \frac{1}{2k} \right)_{n\in \N^*}\rightarrow +\infty,\hspace{0.5cm}
    \left( \sum_{k=1}^n \frac{1}{2k+1} \right)_{n\in \N^*}\rightarrow +\infty
\end{displaymath}
On considère une fonction bijective croissante $\varphi$ de $\N^*$ dans $\N^*$ et on forme la suite
\begin{displaymath}
  S_{\varphi} = \left( \sum_{k=1}^n a_{\varphi(k)} \right)_{n\in \N^*}
\end{displaymath}
\begin{enumerate}
  \item Que fait le code suivant ? Pourquoi avoir ajouté le branchement \texttt{if} à la fin ?
\begin{verbatim}
n = 507 # un entier >= 2
nini = 1 ; pepe = 2 ; S = 0
while nini<= n and pepe <= n:
    S += 1/float(nini) -1/float(pepe) 
    nini += 2
    pepe += 2
if nini < n:
    S += 1/nini
\end{verbatim}
  \item Le code suivant calcule certains termes d'une somme $S_{\varphi}$.
\begin{verbatim}
n = 507 # un entier >= 2
p = 5 ; i = 4
nini = 1 ; pepe = 2 ; S = 0
cptE = 0
while cptE < n:
    cpt = 0
    while cpt < i:
      S += 1/float(nini) 
      nini += 2
      cpt += 1
    cpt = 0
    while cpt < p:
      S += -1/float(pepe)
      pepe += 2
      cpt += 1
    cptE += 1
\end{verbatim}
Expliquer comment se construit la bijection $\varphi$ et préciser les indices des termes calculés. Comparer la valeur de \texttt{S} à $\ln(1+\frac{i}{p})$ pour de grandes valeurs de $n$.\newline
Modifier le code précédent pour qu'il renvoie, dans une liste nommée \texttt{fifi} les premières valeurs $\varphi(1),\varphi(2),\cdots$ de la fonction $\varphi$.

\item Soit $x$ un réel quelconque et $\varepsilon$ un réel strictement positif petit. Il existe une bijection croissante $\varphi$ telle que la suite $S_{\varphi}$ converge vers $x$.\newline
On ne demande pas de le prouver mais de construire ses premiers termes. Modifier le code de la question précédente pour renvoyer les premières valeurs de $\varphi$ permettant de calculer une valeur de $S_{\varphi}$ dans l'intervalle $]x-\varepsilon, x]$.\newline
La convergence est très lente. Il est inutile de prendre des valeurs de $\varepsilon$ trop petites, le processus numériques ne sera plus pertinent à cause des erreurs d'arrondi et des boucles infinies se formeront alors que l'argumentation mathématique est valable. Tester votre programme avec $x=0.4$ et $\varepsilon = 10^{-3}$.
\end{enumerate}
