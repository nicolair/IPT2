%<dscrpt>Syntaxe Python.</dscrpt>
\subsection{Chaines}
\begin{enumerate}
  \item 
Qu'affichent les scripts suivants ?
\begin{verbatim}
#script 1
resultat = ""
for c in "Bonsoir":
    resultat = resultat + c
print(resultat)

#script 2
resultat = ""
for c in "Bonsoir":
    resultat = resultat + c
    print(resultat)
    
#script 3
resultat = ""
for c in "Bonsoir":
    resultat =  c + resultat
print(resultat)
\end{verbatim}
\item Parmi les scripts suivants, lesquels permettent de déterminer si une chaine de caractère n'est formée que par les caractères "A", "C", "G", "T"?
\begin{verbatim}
#script 1
def fonction_q2(chaine):
    resultat = True
    for c in chaine :
        if c == "A" or c == "C" or c == "G" or c == "T":
            resultat = False
    return resultat

#script 2
def fonction_q2(chaine):
    resultat = False
    for c in chaine :
        if c == "A" or c == "C" or c == "G" or c == "T":
            resultat = True
    return resultat

#script 3
def fonction_q2(chaine):
    c_valids = "ACGT"
    resultat = True
    for c in chaine :
        if c not in c_valids:
            resultat = False
    return resultat

#script 4
def fonction_q2(chaine):
    c_valids = "ACGT"
    resultat = True
    for c in chaine :
        temp = False
        for cv in c_valids:
            if c == cv:
              temp = True
        if not temp:
            resultat = False
    return resultat
\end{verbatim}
On distingue trois types de scripts incorrects, ceux qui produisent des \emph{faux positifs} peuvent indiquer qu'une chaine n'est constituée que de "A", "C", "G", "T" alors que ce n'est pas vrai, ceux qui produisent des \emph{faux négatifs} peuvent indiquer qu'une chaine n'est pas constituée de "A", "C", "G", "T" alors qu'elle l'est et ceux qui peuvent faire les deux erreurs.\newline
Indiquer les types d'erreur que peuvent faire les scripts incorrects.

\item On veut disposer d'une fonction \verb|fonction_q3| qui renvoie l'index de la première occurrence d'un caractère dans une chaine s'il y est présent et $-1$ sinon. Le script suivant fait-il cela? S'il ne le fait pas, comment le modifier?
\begin{verbatim}
def fonction_q3 (carac, chaine):
    i = -1
    k = 0
    while i == -1 and k <len(chaine):
        if carac == chaine[k]:
            i = i + 1
            k = k + 1
    return i
\end{verbatim}
\end{enumerate}

\subsection{\'Evolution de caractères et liens entre espèces.}
On s'intéresse aux liens de parentés de $m$ espèces à partir de l'évolution de $n$ caractères.\newline
Deux espèces distinctes ont au moins un caractère différent.\newline
Chaque caractère possède deux états: un état \emph{originel} (noté $0$) et un état \emph{dérivé} (noté 1). Il existe une espèce particulière (ancêtre commun) pour laquelle tous les caractères sont dans l'état 0.\newline
Voici un exemple avec 7 espèces A, B, C, D, E, F, G et 6 caractères $c_0$, $c_1$, $c_2$, $c_3$, $c_4$, $c_5$. Le tableau donne pour chaque espèce l'état de chaque caractère.
\begin{center}
\begin{tabular}{lllllll}
 & $c_0$ & $c_1$ & $c_2$ & $c_3$ & $c_4$ & $c_5$\\ \hline
A & 0 & 1 & 0 & 0 & 0 & 1\\ \hline
B & 0 & 0 & 0 & 1 & 0 & 1\\ \hline
C & 1 & 0 & 0 & 0 & 0 & 0\\ \hline
D & 0 & 0 & 0 & 0 & 0 & 1\\ \hline
E & 0 & 0 & 0 & 0 & 0 & 0\\ \hline
F & 0 & 0 & 1 & 0 & 0 & 1\\ \hline
G & 0 & 0 & 0 & 1 & 1 & 1 \\ \hline
\end{tabular}
\end{center}
Par exemple, pour l'espèce A, les caractères $c_1$ et $c_5$ sont à l'état dérivé, les autres sont à l'état originel. L'espèce E est l'ancêtre commun.\newline
Chaque espèce est représentée par une liste (Python), l'état du caractère $c_i$ est représenté par la valeur $0$ ou $1$ de la valeur d'index $i$ dela liste. L'ensemble des espèces est lui même stocké (par ordre alphabétique des noms) dans une liste désignée par \verb|d_phyl| (données phylogéniques).
\begin{enumerate}
  \item En considérant uniquement les trois premières espèces (A,B,C) et les trois premiers caractères ($c_0$, $c_1$, $c_2$) de l'exemple ci dessus, indiquer quelle instruction d'affectation en donne une représentation exacte.
\begin{verbatim}
#instruction 1
donnees_phylo = [[0,1,0],[0,0,0],[1,0,0]] 
#instruction 2
donnees_phylo = [[0,0,1],[1,0,0],[0,0,0]] 
\end{verbatim}
\item Que désigne \verb|donnees_phylo| après l'exécution du code suivant
\begin{verbatim}
def add_caractere(d_phyl,nouv_carac):
    if len(d_phyl) == len(nouv_carac):
        for i in range(len(d_phyl)):
            d_phyl[i] = d_phyl[i] + [nouv_carac[i]]

donnees_phylo = [[0,0,0,0],[0,0,0,0],[0,0,1,0]]
add_caractere(donnees_phylo,[1,1,0])
\end{verbatim}
\item Proposer une fonction pour ajouter une nouvelle espèce à la liste \verb|d_phyl| en complétant le code suivant
\begin{verbatim}
def add_espece(d_phyl,nouv_espece):
    if len(
\end{verbatim}
\item On considère deux espèces distinctes $a$ et $b$.
\begin{itemize}
  \item L'espèce $b$ est dite \emph{descendante} de l'espèce $a$ si et seulement si chaque caractère présent à l'état originel dans l'espèce $b$ est aussi dans l'état originel pour l'espèce $a$. On dit aussi que l'espèce $a$ est ancêtre de l'espèce $b$.
  \item L'espèce $b$ est dite descendante immédiate de l'espèce $a$ si et seulement si $b$ est descendante de $a$ après la mutation d'un seul caractère. On dit aussi que $a$ est ancêtre immédiat de $b$.
\end{itemize}
On suppose que chaque espèce sauf l'ancêtre commun possède un seul ancêtre immédiat.\newline
Dans l'exemple, G est descendante de D car les caractères à l'état originel de l'espèce G ($c_0$, $c_1$, $c_2$) sont aussi à l'état originel pour D. \newline
Dans le but de tester si une espèce est ancêtre d'une autre, on propose les deux fonctions suivantes
\begin{verbatim}
def est_ancetre1(esp_1, esp_2):
    """
    renvoie True si esp_1 est ancetre de esp_2
    false sinon
    """
    resultat = True
    if len(esp_1) == len(esp_2):
        for i in range(len(esp_1)):
            if esp_2[i] == 0 and esp_1[i] == 1:
                resultat = False
    print(i)
    return resultat

def est_ancetre2(esp_1, esp_2):
    """
    renvoie True si esp_1 est ancetre de esp_2
    false sinon
    """
    resultat = True
    if len(esp_1) == len(esp_2):
        i = 0
        while i < len(esp_1) and resultat:
            if esp_2[i] == 0 and esp_1[i] == 1:
                resultat = False
            i += 1
    print(i)
    return resultat
    
e1 = [0,0,0,1,0,1]
e2 = [0,0,1,0,0,1]
print(est_ancetre1(e1,e2))
print(est_ancetre2(e1,e2))
\end{verbatim}
Quel est l'affichage produit par l'exécution de ce code?
\item Compléter le code suivant en utilisant \verb|est_ancetre2|(dans lequel ou aura supprimé l'instruction d'affichage).
\begin{verbatim}
def nombre_descendants(d_phyl,esp):
    """
    renvoie le nombre nb de descendants de espece
    compte tenu des données stockées dans d_phyl
    """
    
    return nb
\end{verbatim}

\end{enumerate}
