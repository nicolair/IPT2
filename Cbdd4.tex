\begin{enumerate}
 \item On veut extraire les identifiants des patients pour lesquels une hernie discale a été diagnostiquée.
\begin{verbatim}
SELECT DISTINCT `idpatient` 
  FROM `EXAMEN` 
    WHERE `etat` = 'H'  
\end{verbatim} 
Le \texttt{DISTINCT} est utile car le même patient peut avoir subi plusieurs examens.

\item On veut extraire les noms et prénoms des patients pour lesquels une spondylolisthésis a été diagnostiquée.\newline
On modifie la requête précédente en joignant la table administrative pour récupérer les noms et prénoms.
\begin{verbatim}
SELECT DISTINCT 'PATIENT'.`nom`, 'PATIENT'.`prenom` 
  FROM `EXAMEN` JOIN 'PATIENT' 
    ON `EXAMEN`.`idpatient` = `PATIENT`.`id`
  WHERE `etat` = 'S'  
\end{verbatim} 

 \item On veut extraire les identifiants des patients pour lesquels une hernie discale et une spondylolisthésis ont été diagnostiquées.\newline
 Cette fois, il faut joindre la table des examens avec elle même. On utilise des alias \texttt{E1} et \texttt{E2}. Chaque ligne représente un couple d'examens subis par le même patient. Il faut bien remarquer que si $e_i$ et $e_j$ désignent deux examens subis par un patient, les deux couples $(e_i,e_j)$ et $(e_j,e_i)$ figurent dans la jointure.
\begin{verbatim}
SELECT DISTINCT E1.`idpatient` 
  FROM `EXAMEN` E1 JOIN `EXAMEN` E2 
    ON E1.`idpatient` = E2.`idpatient`
  WHERE E1.`etat` = 'H' AND  E2.`etat` = 'S' 
\end{verbatim} 
 
 \item On veut extraire les identifiants des patients qui ont été (à un moment) guéris d'une hernie discale. Ils ont pu rechuter ensuite.\newline
 On s'inspire de la requête précédente car il s'agit d'abord d'extraire les identifiants des patients ayant subis des examens conduisant aux états \texttt{H} et \texttt{N} (au lieu de \texttt{H} et \texttt{S}).  On filtre ensuite à l'aide de la date des examens. 
\begin{verbatim}
SELECT DISTINCT E1.`idpatient` 
  FROM `EXAMEN` E1 JOIN `EXAMEN` E2 
    ON E1.`idpatient` = E2.`idpatient`
  WHERE E1.`etat` = 'H' AND  E2.`etat` = 'N' 
               AND E1.`date` < E2.`date` 
\end{verbatim} 
Les patients sélectionnés ne sont pas forcément guéris car ils ont pu rechuter après un examen normal.
\end{enumerate}
