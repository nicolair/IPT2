\begin{enumerate}
  \item Pour chacune des trois boucles $d-g$ est une fonction de terminaison. Elle reste à valeurs entières à cause des conditions $g\leq d$. Elle diminue strictement dans les deux boucles internes par incrémentation de $g$ ou décrémentation de $d$. En ce qui concerne la boucle externe, si $d-g$ n'a pas diminué lors des boucles internes $g$ et $d$ n'ont pas changé de valeur donc $g$ reste inférieur à $d$ et $d-g$ est diminué de 2 lors du contrôle \texttt{if}.
  \item Les propriétés 
\begin{displaymath}
  \left( k < g \Rightarrow L[k] \leq v\right)  \text{ et } \left( d < k \Rightarrow L[k] > v\right) 
\end{displaymath}
sont des invariants de la boucle externe. Elles sont donc encore valables à la fin. Comme $d = g - 1$ à la sortie de la boucle externe, on en déduit que $L$ vérifie
\[
 \forall k \in \llbracket 0, l-1 \rrbracket, \;
 L[k]
 \left\lbrace 
 \begin{aligned}
  \leq v &\text{ si } k < g \\
  > v &\text{ si } g \leq k
 \end{aligned}
\right. .
\]

  \item Implémentation en Python de l'algorithme de séparation.
\lstinputlisting[firstline=8, lastline=20]{Csep.py}
Comme une liste est un objet modifiable, elle est modifiée à lors de l'appel à l'intérieur de la procédure. 
\end{enumerate}

