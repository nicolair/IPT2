Ce texte présente une équation aux dérivées partielles et montre les problèmes de convergence attachés au choix d'un schéma particulier de discrétisation des dérivées. L'équation considérée est
\begin{equation}
 \frac{\partial u}{\partial t} + \frac{\partial u}{\partial x} = 0 \label{EqTransport}
\end{equation}
où $u$ est une fonction définie dans une partie $A$ de $\R^2$. La première composante d'un couple $(t,x)\in A$ représente un instant et la deuxième un point.\newline
Pour un point fixé représenté par $x$, notons 
\begin{itemize}
 \item $u_x$ la fonction définie dans une partie $I$ (intervalle de temps) de $\R$ qui à un instant $t$ associe $u((t,x))$,  
 \item $u_t$ la fonction définie dans une partie $J$ (intervalle d'espace) de $\R$ qui à un point $x$ associe $u((t,x))$,  
 \item $f$ la fonction définie dans $I$ qui à un instant $t$ associe $\frac{\partial u}{\partial x}((t,x))$.
\end{itemize}
On ajoute une \og condition aux limites\fg~ imposant, pour un instant fixé $t_0=0$ que $u_{t_0}$ est une fonction $v_0$ donnée à l'avance. L'équation s'écrit alors
\[
 u_x'(t) = -f(t) \hspace{1cm} u_x(t_0) = v(x) \Leftrightarrow u_{t_0}=v.
\]
Pour chaque $x$, il s'agit donc bien d'un calcul de primitive d'une fonction de $t$.\newline
On admet qu'il existe une unique solution de l'équation (\ref{EqTransport}) avec la condition aux limites $v$.\newline
Pour résoudre numériquement cette équation, on discrétise le temps et l'espace
\begin{itemize}
 \item $t_0, t_1, t_2, \cdots $: pas constant $t_{k+1} - t_k = \delta_t$ pour $k\in \llbracket 0, n_t \rrbracket$,
 \item $x_0, x_1, x_2, \cdots$:  pas constant $x_{l+1} - x_l = \delta_x$ pour $l\in \llbracket 0, n_x \rrbracket$
\end{itemize}
et on note $u_{k,l}$ au lieu de $u(t_k,x_l)$ et $v_l = v(x_l)$ pour $l\in \llbracket 0, n_x\rrbracket$ (fonction de la condition aux limites). On remplace alors la dérivée par un schéma numérique.\newline
Pour la dérivée par rapport au temps, on utilise
\[
 \frac{\partial u}{\partial t}(t_k,x_l) \simeq \frac{u(t_{k+1},x_l) - u(t_{k},x_l)}{\delta_t} = \frac{u_{k+1,l} - u_{k,l}}{\delta_t}.
\]
Pour la dérivée par rapport à l'espace, on envisage plusieurs schémas numériques dans le tableau suivant
\begin{center}
\renewcommand{\arraystretch}{2.4}
\begin{tabular}{|c|c|c|c|}\hline
schémas                                         & décentré à droite                                    & centré                                                  & décentré à gauche\\ \hline
$\dfrac{\partial u}{\partial x}(t_k,x_l) \simeq$ & $\dfrac{u_{k,l+1} - u_{k,l}}{\delta_x}$ & $\dfrac{u_{k,l+1} - u_{k,l-1}}{2\delta_x}$ & $\dfrac{u_{k,l} - u_{k,l-1}}{\delta_x}$ \\  \hline
\end{tabular}
\end{center}
Dans toute la suite, on suppose que pour tous les instants $t$, la fonction $u_t$ est périodique de période $1$. c'est donc le cas en particulier pour la fonction $v$ de la condition aux limites.
\begin{figure}[h]
 \centering
 \includegraphics[width=10cm]{./Eequtransp_1_fig.pdf}
 \caption{Schéma décentré à droite}
 \label{fig: Eequtransp_1}
\end{figure}

\begin{figure}[h]
 \centering
 \includegraphics[width=10cm]{./Eequtransp_2_fig.pdf}
 \caption{Schéma centré}
 \label{fig: Eequtransp_2}
\end{figure}

\begin{figure}[h]
 \centering
 \includegraphics[width=10cm]{./Eequtransp_3_fig.pdf}
 \caption{Schéma décentré à gauche}
 \label{fig: Eequtransp_3}
\end{figure}

\begin{enumerate}
 \item Pour chacun des trois schémas, former les relations permettant de résoudre numériquement l'équation (1) c'est à dire de calculer les $u_{k,l}$ avec $k\in \llbracket 0, n_t \rrbracket$ et $l\in \llbracket 0, n_x \rrbracket$. 
 
 \item Pour un temps $t>0$, on veut évaluer numériquement la fonction $u_t$ d'une variable d'espace (on sait qu'elle est périodique de période $1$). On discrétise l'intervalle d'espace $[0,1]$ avec un pas $\delta_x = \frac{1}{n_x}$ pour $n_x$ entier et l'intervalle de temps $[0,t]$ avec un pas $\delta_t = \frac{t}{n_t}$ pour $n_t$ entier.\newline
 La condition initiale $v$ est donnée par une liste de $n_x$ valeurs $[v_0,v_1,\cdots ]$.
 \begin{enumerate}
  \item Former une fonction Python \texttt{transportDrte(t,nt,nx,v)} qui prend pour paramètres 
  \begin{itemize}
    \item un flottant \texttt{t} représentant un instant,
    \item des entiers \texttt{nt} et \texttt{nx} permettant de définir les pas en temps et en espace,
    \item une fonction \texttt{v} représentant la condition aux limites.
  \end{itemize}
Elle renvoie la liste des \texttt{nx}$+1$ flottants représentants les valeurs de la fonction $u_t$ aux points discrétisés calculés avec le schéma décentré à droite.

  \item Former une fonction Python \texttt{transportCtre(t,nt,nx,v)} analogue à la précédente mais pour le schéma centré.
  
  \item Former une fonction Python \texttt{transportGche(t,nt,nx,v)} analogue à la précédente mais pour le schéma décentré à gauche.
 \end{enumerate}
 
\item Pour évaluer la pertinence des méthodes proposées, on se place dans un cas pour lequel on connait l'expression d'une solution. Notons $S$ l'application définie dans $\R$ par:
\[
 \forall w\in \R, \; S(w) = \sin(2\pi\,w).
\]
\begin{enumerate}
 \item Montrer que la fonction $u$ définie dans $\R^2$ par:
\[
 \forall (t,x)\in \R^2, \; u((t,x)) = \sin(2\pi(x - t))
\]
est une solution de \ref{EqTransport} avec la condition aux limites $v = S$ pour $t_0=0$.
 \item Les figures \ref{fig: Eequtransp_1}, \ref{fig: Eequtransp_2}, \ref{fig: Eequtransp_3} présentent les graphes des évaluations des fonctions $u_t$ pour les trois schémas et diverses valeurs de $t$ et de $n_t$ (le nombre $n_x$ de points discrétisant l'espace est toujours $100$).\newline
 En examinant ces figures, que pouvez vous dire de la convergence des schémas lorsque le pas de temps $\delta_t = \frac{t}{n_t}$ devient petit?
\end{enumerate}

\item On revient aux notations de la question 1 et on suppose $\delta_t < \delta_x$. On pose
\[
 \forall k \in \llbracket 0, n_t\rrbracket, \; M_k = \max_{l \in \llbracket 0, n_x\rrbracket} \left| u_{k,l}\right|.
\]
Montrer les majorations suivantes pour les trois schémas
\begin{center}
\renewcommand{\arraystretch}{2.}
\begin{tabular}{|c|c|c|c|} \hline
Schéma        & décentré droite & centré & décentré gauche\\ \hline
$M_{k+1}\leq$ & $\left( 1+2\dfrac{\delta_t}{\delta_x}\right)M_k$  & $\left( 1 + \dfrac{\delta_t}{\delta_x}\right)M_k$ & $M_k$ \\ \hline
\end{tabular}
\end{center}

\end{enumerate}

