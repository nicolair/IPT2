%<dscrpt>Introduction à l'algorithmique</dscrpt>
% <rel_id_elt_parent>6469</rel_id_elt_parent> id de ``Algorithmique''
% <rel_id_type_rel>14</rel_id_type_rel> id de ``contient sana ordre l'élément''
%!  pour pdfLatex
\documentclass[a4paper]{article}
%\usepackage[hmargin={1.5cm,1.5cm},vmargin={2.4cm,2.4cm},headheight=13.1pt]{geometry}
\usepackage[a4paper,landscape,twocolumn,
            hmargin=1.8cm,vmargin=2.2cm,headheight=13.1pt]{geometry}

\usepackage[pdftex]{graphicx,color}
\usepackage[pdftex,colorlinks={true},urlcolor={blue},pdfauthor={remy Nicolai}]{hyperref}

\usepackage[T1]{fontenc}
\usepackage[utf8]{inputenc}

\usepackage{lmodern}
\usepackage[frenchb]{babel}

\usepackage{fancyhdr}
\pagestyle{fancy}

\usepackage{floatflt}
\usepackage{maths}

\usepackage{parcolumns}
\setlength{\parindent}{0pt}

\usepackage{caption}
\usepackage{subcaption}

\usepackage{makeidx}

\usepackage[french,ruled,vlined]{algorithm2e}
\SetKwComment{Comment}{\#}{}
\SetKwFor{Tq}{tant que}{}{}
\SetKwFor{Pour}{pour}{}{}
\DontPrintSemicolon
\SetAlgoLined

\usepackage{listings}
\lstset{language=Python,frame=single}
\lstset{literate=
  {á}{{\'a}}1 {é}{{\'e}}1 {í}{{\'i}}1 {ó}{{\'o}}1 {ú}{{\'u}}1
  {Á}{{\'A}}1 {É}{{\'E}}1 {Í}{{\'I}}1 {Ó}{{\'O}}1 {Ú}{{\'U}}1
  {à}{{\`a}}1 {è}{{\`e}}1 {ì}{{\`i}}1 {ò}{{\`o}}1 {ù}{{\`u}}1
  {À}{{\`A}}1 {È}{{\'E}}1 {Ì}{{\`I}}1 {Ò}{{\`O}}1 {Ù}{{\`U}}1
  {ä}{{\"a}}1 {ë}{{\"e}}1 {ï}{{\"i}}1 {ö}{{\"o}}1 {ü}{{\"u}}1
  {Ä}{{\"A}}1 {Ë}{{\"E}}1 {Ï}{{\"I}}1 {Ö}{{\"O}}1 {Ü}{{\"U}}1
  {â}{{\^a}}1 {ê}{{\^e}}1 {î}{{\^i}}1 {ô}{{\^o}}1 {û}{{\^u}}1
  {Â}{{\^A}}1 {Ê}{{\^E}}1 {Î}{{\^I}}1 {Ô}{{\^O}}1 {Û}{{\^U}}1
  {œ}{{\oe}}1 {Œ}{{\OE}}1 {æ}{{\ae}}1 {Æ}{{\AE}}1 {ß}{{\ss}}1
  {ű}{{\H{u}}}1 {Ű}{{\H{U}}}1 {ő}{{\H{o}}}1 {Ő}{{\H{O}}}1
  {ç}{{\c c}}1 {Ç}{{\c C}}1 {ø}{{\o}}1 {å}{{\r a}}1 {Å}{{\r A}}1
  {€}{{\euro}}1 {£}{{\pounds}}1 {«}{{\guillemotleft}}1
  {»}{{\guillemotright}}1 {ñ}{{\~n}}1 {Ñ}{{\~N}}1 {¿}{{?`}}1
}

%pr{\'e}sentation des compteurs de section, ...
\makeatletter
\renewcommand{\thesection}{\Roman{section}.}
\renewcommand{\thesubsection}{\arabic{subsection}.}
\renewcommand{\thesubsubsection}{\arabic{subsubsection}.}
\renewcommand{\labelenumii}{\theenumii.}
\makeatother


\newtheorem*{thm}{Théorème}
\newtheorem{thmn}{Théorème}
\newtheorem*{prop}{Proposition}
\newtheorem{propn}{Proposition}
\newtheorem*{pa}{Présentation axiomatique}
\newtheorem*{propdef}{Proposition - Définition}
\newtheorem*{lem}{Lemme}
\newtheorem{lemn}{Lemme}

\theoremstyle{definition}
\newtheorem*{defi}{Définition}
\newtheorem*{nota}{Notation}
\newtheorem*{exple}{Exemple}
\newtheorem*{exples}{Exemples}


\newenvironment{demo}{\renewcommand{\proofname}{Preuve}\begin{proof}}{\end{proof}}
%\renewcommand{\proofname}{Preuve} doit etre après le begin{document} pour fonctionner

\theoremstyle{remark}
\newtheorem*{rem}{Remarque}
\newtheorem*{rems}{Remarques}

%\usepackage{maths}
%\newcommand{\dbf}{\leftrightarrows}

%En tete et pied de page
\lhead{Informatique}
%\chead{Introduction aux systèmes informatiques}
\rhead{MPSI B Hoche}
\lfoot{\tiny{Cette création est mise à disposition selon le Contrat\\ Paternité-Partage des Conditions Initiales à l'Identique 2.0 France\\ disponible en ligne http://creativecommons.org/licenses/by-sa/2.0/fr/  
} 
\rfoot{\tiny{Rémy Nicolai \jobname \; \today } }
}
\makeindex


%En tete et pied de page
\lhead{Cours IPT}
\chead{Cours 2: Introduction à la numération le 26/09/16}

\begin{document}
\begin{prop}
 Soit $b$ un entier naturel supérieur ou égal à $2$. Pour tout entier naturel $x$ entre $0$ et $b^n-1$, il existe un unique $n$-uplet
\begin{displaymath}
 (a_0,a_1,\cdots,a_{n-1})\in \{0,1,\cdots,b-1\}^n
\end{displaymath}
tel que 
\begin{displaymath}
 x = a_0 + a_1 b +\cdots +a_{n-1}b^{n-1}
\end{displaymath}
\end{prop}
Cette proposition traduit l'existence et l'unicité de la décomposition d'un entier dans une base arbitraire. On utilise en particulier les bases $b=2$ (binaire), $b=10$ (décimale), $b=16$ (héxadécimal), $b=20$ \footnote{voir le système de numération maya. Cette base semble aussi avoir été utilisée par les Gaulois, le 80 quatre-vingt en serait un lointain vestige (ref wikipédia) }, $b=60$ (sexagésimale) \footnote{utilisé par les mésopotamiens voir en particulier les tablettes cuneiformes de Plimpton}
\begin{demo}
 Pour démontrer cette proposition, on va remarquer qu'elle est équivalente à la bijectivité d'une certaine application entre deux ensembles finis ayant le même nombre d'éléments. Pour une telle application, l'injectivité entraîne la surjectivité donc la bijectivité.\newline
La démonstration de l'injectivité est \emph{constructive}. Si un entier est décomposé alors chaque $a_i$ se calcule algorithmiquement en fonction de $x$ et de $b$. Ceci assure l'unicité de la décomposition donc l'injectivité de la fonction.

Considérons la fonction
\begin{displaymath}
 \Phi\left\lbrace 
\begin{aligned}
\{0,1,\cdots,b-1\}^n \rightarrow& \{0,1,\cdots,b^n-1\}\\
(a_0,a_1,\cdots,a_{n-1}) \rightarrow& a_0 + a_1 b +\cdots +a_{n-1}b^{n-1} 
\end{aligned}
\right.
\end{displaymath}
En fait, il faut commencer par montrer que
\begin{displaymath}
 a_0 + a_1 b +\cdots +a_{n-1}b^{n-1}\in \{0,1,\cdots,b^n-1\}
\end{displaymath}
Ceci résulte de l'encadrement
\begin{multline*}
 0\leq a_0 + a_1 b +\cdots +a_{n-1}b^{n-1} 
\leq (b-1) + (b-1) b +\cdots +(b-1)b^{n-1}\\
\leq (b-1)(1+b+\cdots +b^{n-1})=b^n-1
\end{multline*}
On démontre exactement de la même manière que, pour des $m\leq n$:
\begin{displaymath}
 a_0 + a_1 b +\cdots +a_{m-1}b^{m-1}\in \{0,1,\cdots,b^m-1\}
\end{displaymath}
Ceci servira plus loin pour justifier un des deux algorithmes proposés.\newline
La proposition est exactement équivalente à la bijectivité de la fonction $\Phi$. Les ensembles de départ et d'arrivée ont le même nombre d'éléments à savoir $b^n$.\newline
Si la fonction $\Phi$ est injective, les images sont deux à deux distinctes. Il y a donc autant d'images distinctes que d'éléments dans l'ensemble de départ. Mais alors tous les $b^n$ éléments de l'ensemble d'arrivée sont des images puisque cet ensemble ne contient que $b^n$ éléments.\newline
Démontrons maintenant l'injectivité\footnote{On peut présenter cette démonstration comme une analyse-synthèse. L'analyse correspond à l'injectivité ou à l'unicité, son argumentation est algorithmique. La synthèse correspond à la surjectivité ou à l'existence, son argumentation repose sur la théorie des ensembles}. On suppose
\begin{displaymath}
 x=a_0 + a_1 b +\cdots +a_{n-1}b^{n-1}
\end{displaymath}
On peut adopter un algorithme "glouton" en cherchant d'abord les "plus gros morceaux" c'est à dire le nombre $a_{n-1}$ de $b^{n-1}$ contenus dans $x$. Comme 
\begin{displaymath}
 0\leq a_0 + a_1 b +\cdots +a_{n-2}b^{n-2}\leq b^{n-1}-1
\end{displaymath}
ce nombre est le reste de la division de $x$ par $b^{n-1}$ et $a_{n-1}$ en est le quotient. Ceci assure l'unicité de $a_{n-1}$ et on peut poursuivre le raisonnement en divisant le reste précédent par $b^{n-2}$.\newline
On peut aussi procéder en partant du bas. Dans la division par $b$ de
\begin{align*}
 x= a_0 + a_1 b +\cdots +a_{m-1}b^{m-1} = a_0 +(a_1+a_2b+\cdots+a_{n-1}b^{n-2})b
\end{align*}
le reste est $a_0$ et le quotient est $a_1+a_2b+\cdots+a_{n-1}b^{n-2}$. Ceci assure l'unicité du $a_0$ et le raisonnement se poursuit en divisant par $b$ le quotient précédent.
\end{demo}
L'expression en pseudo-code de cet algorithme est la suivant
\begin{verbatim}
b <-- la base
x <-- un nombre entier >0
# recherche du plus petit n tel que x<b^n
n <-- 0
p<-- 1
tant que p <= x
  n<-- n+1
  p<-- p*b
#formation du développement
resultat <-- liste vide
p<-- p/b
tant que x>0
  resultat.append(quotient de la div de x par p)
  x<-- reste de la division de x par p
  p<- p/b
print(resultat)
\end{verbatim}
La syntaxe des structures de contrôle sera présentée en détail en TP mais cet exemple (algorithme de numération : les grands d'abord) est une bonne introduction.
\begin{verbatim}
b = 10
x = 259
n = 0
p = 1
#recherche du plus petit n tel que x<b^n
while p <= x:
    n += 1
    p *= b
print(n)
#formation du développement
resultat = []
p = p//b
while x>0:
    resultat.append(x // p)
    x = x % p
    p = p / b
print(resultat) 
\end{verbatim}
On peut aussi implémenter avec les petits d'abords
  \begin{verbatim}
b = 10 ; x = 16**8 ; print(x)
resultat = []
while x > 0 :
    resultat.append(x % b)
    x = x // b
print(resultat)  
\end{verbatim}
Le problème est alors que les décimales sont rangées à l'envers dans la liste. 

\printindex
\end{document}
 