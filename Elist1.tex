L'objet de cet exercice est de former un tableau de fréquences.
\begin{enumerate}
  \item On se donne un entier désigné par \texttt{n} (prendre 100 par exemple). Former un tableau \texttt{notes} indexé de 0 à \texttt{n -1} dont les valeurs sont des notes aléatoires entre 0 et 20 (avec une décimale) mais les notes $0$ et $20$ \emph{doivent figurer} dans la liste. Vous utiliserez l'appel \texttt{randint(0,200)} de la fonction \texttt{randint} dans la bibiothèque \texttt{random} qui renvoie un nombre aléatoire entre 0 et 200.
  \item Former un tableau \texttt{freq} indexé de $0$ à $19$ tel que \texttt{freq[i]} désigne le nombre de notes dans $[i,i+1[$ pour $i$ entre 0 et 19 et dans $[19,20]$ pour $i=19$.
  \item  Qu'observe-t-on si on prend de grandes valeurs de $n$? Que peut-on en déduire sur la fonction \texttt{randint} ?
\end{enumerate}
