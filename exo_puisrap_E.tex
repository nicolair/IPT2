\begin{algorithm}
  $n \leftarrow$ un naturel non nul $n_{ini}$\;
  $x \leftarrow $ un nombre $x_{ini}$\;
  $p \leftarrow 1$\;
  \Tq{ $n > 0$ }{
    \eSi{ $n$ est pair}{
      $x \leftarrow x * x$\;
      $n \leftarrow \frac{n}{2}$\;
    }{
      $p \leftarrow p *x$\;
      $n\leftarrow n - 1$\;
    }
  }
  \caption{calcul rapide d'une puissance}
  \label{exo_puisrap_E_1}
\end{algorithm}
Pour le pseudo-code présenté dans l'algorithme \ref{exo_puisrap_E_1}, montrer que $n$ est un variant et que 
\begin{displaymath}
  \Phi = \left( px^n == x_{ini}^{n_{ini}} \right) 
\end{displaymath}
est un invariant. Que désignent $n$ et $p$ après la sortie de la boucle? Combien de multiplications sont effectuées lors de l'exécution avec $n_{ini} = 79$?