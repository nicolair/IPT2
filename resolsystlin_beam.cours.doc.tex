%!  pour pdfLatex
\documentclass{beamer}

%\usepackage[pdftex]{graphicx,color}
%\usepackage[pdftex,colorlinks={true},urlcolor={blue},pdfauthor={remy Nicolai}]{hyperref}

\usepackage[utf8]{inputenc}
\usepackage[T1]{fontenc}
\usepackage{lmodern}
\usepackage[frenchb]{babel}

\usetheme{Warsaw}

%\usepackage{fancyhdr}
%\pagestyle{fancy}

%\usepackage{floatflt}
\usepackage{maths}

\usepackage{parcolumns}
\setlength{\parindent}{0pt}

%\usepackage{caption}
%\usepackage{subcaption}

\usepackage[french,ruled,vlined]{algorithm2e}
\SetKwComment{Comment}{\#}{}
\SetKwFor{Tq}{tant que}{}{}
\SetKwFor{Pour}{pour}{}{}
\DontPrintSemicolon
\SetAlgoLined

\usepackage{listings}
\lstset{language=Python,frame=single}

%pr{\'e}sentation des compteurs de section, ...
\makeatletter
\renewcommand{\thesection}{\Roman{section}.}
\renewcommand{\thesubsection}{\arabic{subsection}.}
\renewcommand{\thesubsubsection}{\arabic{subsubsection}.}
%\renewcommand{\labelenumii}{\theenumii.}
\makeatother


\newtheorem*{thm}{Théorème}
\newtheorem{thmn}{Théorème}
\newtheorem*{prop}{Proposition}
\newtheorem{propn}{Proposition}
\newtheorem*{pa}{Présentation axiomatique}
\newtheorem*{propdef}{Proposition - Définition}
\newtheorem*{lem}{Lemme}
\newtheorem{lemn}{Lemme}

\theoremstyle{definition}
\newtheorem*{defi}{Définition}
\newtheorem*{nota}{Notation}
\newtheorem*{exple}{Exemple}
\newtheorem*{exples}{Exemples}


\newenvironment{demo}{\renewcommand{\proofname}{Preuve}\begin{proof}}{\end{proof}}
%\renewcommand{\proofname}{Preuve} doit etre après le begin{document} pour fonctionner

\theoremstyle{remark}
\newtheorem*{rem}{Remarque}
\newtheorem*{rems}{Remarques}

%\usepackage{maths}
%\newcommand{\dbf}{\leftrightarrows}

%En tete et pied de page
%\lhead{Informatique}
%\chead{Introduction aux systèmes informatiques}
%\rhead{MPSI B Hoche}
%\lfoot{\tiny{Cette création est mise à disposition selon le Contrat\\ Paternité-Partage des Conditions Initiales à l'Identique 2.0 France\\ disponible en ligne http://creativecommons.org/licenses/by-sa/2.0/fr/
%} }
%\rfoot{\tiny{Rémy Nicolai \jobname}}

\nonstopmode
\lstset{language=Python,frame=single}
\begin{document}
\begin{frame}
\frametitle{Méthode du pivot de Gauss}
\begin{itemize}
  \item La méthode du pivot de Gauss permet de résoudre un système d'équations linéaires ($p$ équations, $q$ inconnues).\newline
Cas particulier: système de Cramer: $p=q$ et unique solution (qui est un $p$-uplet).
  \item Résoudre un système d'équations c'est trouver une description paramétrique de l'ensemble des solutions.
  \item Méthode
\begin{itemize}
  \item Transformer par \emph{opérations élémentaires} (sans changer l'ensemble des solutions) en un système triangulaire.
  \item Résoudre ce système triangulaire (phase de remontée).
\end{itemize}
\end{itemize}
\end{frame}


\begin{frame}
  \frametitle{Traduction matricielle}
Un système de $p$ équations à $q$ inconnues $x_1,\cdots x_q$
\begin{displaymath}
\left\lbrace 
\begin{aligned}
  a_{1 1}x_1 + a_{1 2}x_2 + \cdots + a_{1 q}x_q &= y_1\\
  \vdots\hspace{0.5cm} &= \; \vdots \\
  a_{p 1}x_1 + a_{p 2}x_2 + \cdots + a_{p q}x_q &= y_p
\end{aligned}
\right. 
\end{displaymath}
se traduit par une équation matricielle $A X = Y$ avec 
\begin{displaymath}
A = 
\begin{pmatrix}
  a_{1 1} & a_{1 2} & \cdots & a_{1 q}\\
  \vdots  &         &        & \vdots \\
  a_{p 1} & a_{p 2} & \cdots & a_{p q}
\end{pmatrix},
\;
X = 
\begin{pmatrix}
  x_1 \\ x_2 \\ \vdots \\ x_q
\end{pmatrix},
\;
Y = 
\begin{pmatrix}
  y_1 \\ y_2 \\ \vdots \\ y_p
\end{pmatrix}
\end{displaymath}
Système de Cramer $\Leftrightarrow$ $p=q$ et $A$ inversible.

\end{frame}
\begin{frame}
  \frametitle{Opérations élémentaires}
Il s'agit exclusivement des opérations suivantes (système de $p$ équations à $q$ inconnues).
\begin{itemize}
 \item Permuter deux équations.
 \item Pour $i$ et $i'$ entre $1$ et $p$ et $\lambda$ scalaire quelconque: ajouter l'équation $i'$ multipliée par $\lambda$ à l'équation $i$. Bien noter que seule l'équation $i$ est modifiée.
 \item Multiplier une équation par un scalaire $\lambda$ \emph{non nul}.
\end{itemize}
Le système obtenu par une transformation élémentaire a le même ensemble de solutions que le système de départ.\newline
Formulation algorithmique: l'ensemble des solutions est un invariant pour toute transformation élémentaire.
\end{frame}

\begin{frame}
  \frametitle{Méthode de Gauss: pseudo-code}
\begin{algorithm}[H]
  \Donnees{\;
    $A$ tableau à deux entrées entre $1$ et $p$: la matrice du système\;
    $Y$ tableau indexé de $1$ à $p$: le second membre\;
    On suppose que le système est de Cramer
  }
  \Comment{initialisation}
  $i\leftarrow 1$ : compteur\;
  \Tq{ $i \leq p$}{
    Chercher un indice-pivot $ip$ dans la fin de la colonne $i$\;
    Permuter les lignes $ip$ et $i$ de $A$ et $Y$\;
    Nettoyer les lignes $i+1$ à $p$ avec la ligne $i$\;
    Incrémenter $i$
  }
  \caption{Méthode de Gauss}
  \label{resolsystlin_1}
\end{algorithm}
\og Chercher un indice pivot\fg~ et \og Nettoyer les lignes\fg~ sont à préciser.  
\end{frame}

\begin{frame}
  \frametitle{Chercher un indice pivot}
Parmi les valeurs $a_{i i}, a_{i+1 i}, \cdots a_{p i}$, en existe-t-il une non nulle? 
\begin{itemize}
  \item Si oui, on en choisit une (on l'appelle le \emph{pivot}) et on renvoie son indice comme \emph{indice-pivot} $ip$.
  \item Si non, l'algorithme s'arrête.
\end{itemize}
Le fait de chercher un pivot uniquement dans le bas de colonne est caractéristique de la méthode dite du pivot \emph{partiel}.
\end{frame}

\begin{frame}
  \frametitle{Comment chercher?}
\begin{itemize}
  \item Tester si un flottant est non nul est une mauvaise pratique.
  \item Choisir l'indice de la plus grande valeur absolue des termes.
  \item Pour un système de Cramer, il existe un pivot non nul.
  \item Si on implémente un algorithme qui peut s'exécuter pour un système qui n'est pas de Cramer: l'arrêter lorsque la recherche d'un indice-pivot échoue c'est à dire lorsque
\begin{displaymath}
a_{i i} = a_{i+1 i} = \cdots = \cdots = a_{p i}=0  
\end{displaymath}
\end{itemize}
\end{frame}

\begin{frame}
  \frametitle{Nettoyer les lignes}
\begin{itemize}
  \item Faire apparaître (\emph{avec des opérations élémentaires}) des $0$ aux places $i+1$ à $p$ de la colonne $i$.
  \item En multipliant la ligne $i$ par un coefficient tel que, après soustraction à la ligne $j$, le terme d'indice $j,i$ soit nul.
  \item La ligne $i$ n'est pas modifiée.
\end{itemize}
\end{frame}

\begin{frame}
  \frametitle{Nettoyer les lignes: pseudo-code}
\begin{algorithm}[H]
  \Donnees{\;
    $A$ tableau à deux entrées entre $1$ et $p$: la matrice du système.\;
    $Y$ tableau indexé de $1$ à $p$: le second membre.\;
    $i$ compteur.
  }
  \Comment{initialisation}
  $j\leftarrow i+1$ : compteur\;
  $\lambda$ : flottant un coefficient multiplicatif\;
  \Tq{ $j \leq p$}{
    $\lambda \leftarrow \frac{a_{j i}}{a_{i i}}$\;
    $L_j(A)\leftarrow L_j(A)-\lambda L_i(A)$\;
    $y_j\leftarrow y_j-\lambda y_i$\;
    Incrémenter $j$\;
  }
  \caption{Nettoyer les lignes}
  \label{resolsystlin_2}
\end{algorithm}
\end{frame}

\begin{frame}
  \frametitle{Nettoyer les lignes: erreur usuelle}
\'Ecrire
\begin{algorithm}[H]
  \Pour{ $numcol$ entre $1$ et $p$}{
    $A[j, numcol] \leftarrow A[j, numcol] - \frac{A[j, i]}{A[i, i]}A[i, numcol]$\;
  }
\end{algorithm}
à la place de
\begin{align*}
    \lambda &\leftarrow \frac{a_{j i}}{a_{i i}}\\
    L_j(A)  &\leftarrow L_j(A)-\lambda L_i(A)  
\end{align*}
Lorsque $numcol$ prend la valeur $i$, la valeur de $A[j,i]$ est modifiée et le coefficient n'est plus le même pour la fin de la ligne.
\end{frame}

\begin{frame}
  \frametitle{Forme du sytème à l'issue du nettoyage}
Une propriété \emph{invariante} par la séquence de nettoyage:
\begin{displaymath}
  \mathcal{P}(i) = \left( \forall j \text{ tq } 0\leq j < i,\; \forall k\text{ tq } j<k\leq p: \; a_{k,j} = 0\right) 
\end{displaymath}
\begin{itemize}
  \item Pour $i=1$ la propriété est vraie car elle est vide.\newline
  \item Le nettoyage assure que cette propriété est vraie pour la valeur suivante de $i$.\newline
  \item \`A la fin, $i=p$. Le système est triangulaire supérieur avec des termes \emph{non nuls} sur la diagonale (les pivots).
\end{itemize}  
\end{frame}

\begin{frame}
  \frametitle{Phase de remontée}
\begin{algorithm}[H]
  \Donnees{\;
    $A$ tableau triangulaire supérieur à deux entrées entre $1$ et $p$\;
    $Y$ tableau indexé de $1$ à $p$: le second membre\;
    $i$ compteur
  }
  \Comment{initialisation}
  $i\leftarrow p$ : compteur\;
  $X$ : tableau indexé de $1$ à $p$ qui contiendra la solution.\;
  \Tq{ $i \geq 1$}{
    $X[i] \leftarrow \frac{Y[i] -\sum_{k=i+1}^{p}A[i][k]X[k]}{A[i][i]} $\;
    Décrémenter $i$\;
  }
  \caption{Phase de remontée}
  \label{resolsystlin_3}
\end{algorithm}
\end{frame}

\begin{frame}
  \frametitle{Complexité: produits matriciels}
\begin{itemize}
  \item Matrice colonne ($p$ lignes) par matrice ligne ($p$ colonnes) :\newline
  $p$ multiplications et $p-1$ additions. Complexité $O(p)$.
  \item Matrice carrée par matrice colonne: $p$ calculs du type précédent. Complexité en $O(p^2)$.
  \item Produit de deux matrices carrées: $p^2$ fois des produits ligne $\times$ colonne. Complexité en $O(p^3)$.
\end{itemize}
\end{frame}

\begin{frame}
  \frametitle{Complexité: système triangulaire équivalent}
\begin{itemize}
  \item Recherche d'un indice-pivot: $O(p)$.
  \item Permutation de deux lignes: $O(p)$.
  \item Nettoyage des lignes $i+1$ à $p$: $O((p-i)p)$
  \item Exécuter ces opérations pour $i$ de $1$ à $p$ 
\end{itemize}
Complexité en $O(p^3)$.
\end{frame}

\begin{frame}
  \frametitle{Complexité: phase de remontée}
\begin{itemize}
  \item Calcul de $x_p$: une addition et une division.
  \item Calcul de $x_{p-1}$: $1$ multiplication, $1$ addition, une division.
  \item Calcul de $x_{p-2}$: $2$ multiplications, $2$ additions, une division.
  \item Calcul de $x_{p-3}$: $3$ multiplications, $3$ additions, une division.
  \item $\vdots$
\end{itemize}
En sommant:  complexité en $O(p^2)$.

Pour une matrice triangulaire et une colonne données:
\begin{center}
complexité de la résolution = complexité d'une multiplication  
\end{center}
\end{frame}

\begin{frame}
  \frametitle{Complexité: méthode de Gauss}
En combinant les résultats précédents. 

Résolution d'un système de Cramer de $p$ équations par la méthode du pivot: complexité $O(p^3)$.
\end{frame}

\begin{frame}
\frametitle{Complexité: inversion d'une matrice}
L'inversion d'une matrice inversible à $p$ lignes revient à la résolution de $p$ systèmes de Cramer avec des second membres élémentaires
\begin{displaymath}
\begin{pmatrix}
0 \\ \vdots \\ 1 \\ \vdots \\0   
 \end{pmatrix}
\end{displaymath}
(avec le $1$ dans la ligne $k$) donnant la colonne $k$ de la matrice inverse.


 Complexité de l'inversion: en $O(p^4)$.
 
 Pour résoudre numériquement un système de Cramer, \emph{il ne faut pas commencer par inverser la matrice}.
\end{frame}

\begin{frame}
  \frametitle{Implémentation Python: types de données}
\begin{itemize}
  \item Choix: représenter les matrices par des listes de listes.
  \item Chaque ligne est représentée par une liste de $p$ objets et la matrice est la liste de ces listes.
  \item Attention, en Python, les listes sont indexées à partir de $0$.
  \item Choix discutable: certaines propriétés des listes ne sont pas utiles. Il serait plus efficace d'utiliser des types de données spécialement adaptés comme le type tableau (array) de la bibliothèque numpy.
\end{itemize} 
\end{frame}

\begin{frame}
  \frametitle{Implémentation Python: outils}
\lstinputlisting[firstline=16, lastline=24]{resolsystlin.py}
\end{frame}

\begin{frame}
  \frametitle{Implémentation Python: chercher l'indice d'un pivot}
\lstinputlisting[firstline=8, lastline=14]{resolsystlin.py}
\end{frame}

\begin{frame}
  \frametitle{Implémentation Python: nettoyage}
\lstinputlisting[firstline=30, lastline=36]{resolsystlin.py}
\end{frame}

\begin{frame}
  \frametitle{Implémentation Python: remontée}
\lstinputlisting[firstline=37, lastline=43]{resolsystlin.py}
\end{frame}

\begin{frame}
  \frametitle{Exemple avec des coniques}
En général, par cinq points distincts d'un plan passe une seule conique.

L'équation de cette conique s'obtient en résolvant un système de $5$  équations linéaires à $5$ inconnues.

On se propose, à titre d'exemple, d'expérimenter numériquement sur ce thème.

\end{frame}

\begin{frame}
  \frametitle{\'Equation d'une conique}
Dans un plan muni d'un repère $\mathcal{R}$, les fonctions coordonnées sont notées $x$ et $y$.\newline
Une conique affine $\mathcal{C}$ est un ensemble de points dont l'équation est de degré $2$.\newline
Il existe des réels $a$, $b$, $c$, $d$, $e$, $f$ tels que 
\begin{displaymath}
  M\in \mathcal{C} \Leftrightarrow
  ax(M)^2 +by(M)^2 + cx(M)y(M) + dx(M) + eY(M) +f = 0
\end{displaymath}
Si on multiplie par un réel non nul, on ne change pas l'ensemble des points $M$ vérifiant cette équation.\newline
$f=0$ si et seulement si l'origine $O$ du repère est dans $\mathcal{C}$.\newline
On ne considère que des coniques \emph{qui ne passent pas par} $O$. On peut diviser par $f$ ce qui revient à $f=1$.
\end{frame}

\begin{frame}
  \frametitle{Système d'équations}
On se donne des points $M_1, M_2, M_3, M_4, M_5$ et on forme le système
\begin{displaymath}
\left\lbrace
\begin{aligned}
  ax(M_1)^2 +by(M_1)^2 + cx(M_1)y(M_1) + dx(M_1) + eY(M_1) = -1 \\
  ax(M_2)^2 +by(M_2)^2 + cx(M_2)y(M_2) + dx(M_2) + eY(M_2) = -1 \\
  ax(M_3)^2 +by(M_3)^2 + cx(M_3)y(M_3) + dx(M_3) + eY(M_3) = -1 \\
  ax(M_4)^2 +by(M_4)^2 + cx(M_4)y(M_4) + dx(M_4) + eY(M_4) = -1 \\
  ax(M_5)^2 +by(M_5)^2 + cx(M_5)y(M_5) + dx(M_5) + eY(M_5) = -1  
\end{aligned}
\right. 
\end{displaymath}
 aux inconnues $a$, $b$, $c$, $d$, $e$.\newline
En général il est de Cramer. Dans ce cas, une seule conique passe par les $5$ points.\newline
Il existe des configurations dans lesquelles le système n'est pas de Cramer en particulier si plusieurs coniques passent par les points (points alignés par exemple).
\end{frame}

\begin{frame}[fragile]
  \frametitle{Implémentation Python}
\begin{verbatim}
############## génération aléatoire de 5 points
import random as rdm
M=[(rdm.random()*10,rdm.random()*10)\\
                           for i in range(5)]
############## formation de la matrice du système
A = [0]*5
for i in range(5):
    x,y = M[i]
    A[i] = [x**2 , y**2, x*y, x, y]
############## second membre
Y = [[-1.],[-1.],[-1.],[-1.],[-1.]]
############## résolution
E = resol(A,Y)
\end{verbatim}
\end{frame}

\begin{frame}[fragile]
  \frametitle{Vérification}
Pour vérifier, on forme la fonction qui permet d'écrire l'équation de la conique et on l'évalue aux points donnés.
\begin{verbatim}
def F(E,P):
    a,b,c,d,e = E
    x,y = P
    return a*x*x + b*y*y + c*x*y + d*x + e*y +1
for P in M:
    plt.plot(P[0],P[1],'bo')
    print(F(E,P))
\end{verbatim}
Les valeurs des $F(M_i)$ sont de l'ordre de $10^{-15}$ et résultent des erreurs d'arrondi.
\end{frame}

\begin{frame}
  \frametitle{Paramétrisation rationnelle}
Pour tracer d'autres points: \emph{paramétrisation rationnelle}.

$\mathcal{D}_t$ : droite passant par $M_0$ et de direction le vecteur de coordonnées $(1,t)$.

Les points de  $\mathcal{D}_t$ sont les $P_\lambda(t)$  de coordonnées $(x(M_0)+\lambda, y(M_0)+\lambda t)$ et:
\begin{displaymath}
P_\lambda(t) \in \mathcal{C} \Leftrightarrow F(P_\lambda(t)) = 0  
\end{displaymath}
Pour $t$ fixé: équation du second degré en $\lambda$ dont $0$ est solution. 

L'équation se ramène à une équation du premier degré.

Pour chaque $t$ réel, $\lambda_t$ la solution de l'équation, l'application $\Phi : t \mapsto P_{\lambda_t}(t)$ est une paramétrisation rationnelle.
\end{frame}

\begin{frame}[fragile]
  \frametitle{Paramétrisation: implémentation}
\begin{verbatim}
def phi(E,P,t):
    a,b,c,d,e = E
    x,y = P
    num = 2.*a*x + c*y + d +(2.*b*y + c*x + e)*t
    den = a + (b*t + c)*t
    l = - num/den
    return x + l , y+t*l
\end{verbatim}
\end{frame}

\begin{frame}[fragile]
  \frametitle{Tracé}
Calcul des $t$ associés aux $M_1,\cdots M_4$. Tracé entre les valeurs extrêmes des $t$.
\begin{verbatim}
import matplotlib.pyplot as plt
plt.clf()
for P in M:
    plt.plot(P[0],P[1],'bo')
    print(F(E,P))
t = [(M[i+1][1] - M[0][1])/(M[i+1][0]\\
     - M[0][0]) for i in range(4)]
tmin, tmax = min(t), max(t)
n,p = 30,5
pas = (tmax - tmin)/n
tt = tmin - p*pas
    P = phi(E,M[0],tt)
    plt.plot(P[0],P[1],'ro')
    tt += pas
\end{verbatim}
\end{frame}



\end{document}
