%!  pour pdfLatex
\documentclass[a4paper]{article}
%\usepackage[hmargin={1.5cm,1.5cm},vmargin={2.4cm,2.4cm},headheight=13.1pt]{geometry}
\usepackage[a4paper,landscape,twocolumn,
            hmargin=1.8cm,vmargin=2.2cm,headheight=13.1pt]{geometry}

\usepackage[pdftex]{graphicx,color}
\usepackage[pdftex,colorlinks={true},urlcolor={blue},pdfauthor={remy Nicolai}]{hyperref}

\usepackage[T1]{fontenc}
\usepackage[utf8]{inputenc}

\usepackage{lmodern}
\usepackage[frenchb]{babel}

\usepackage{fancyhdr}
\pagestyle{fancy}

\usepackage{floatflt}
\usepackage{maths}

\usepackage{parcolumns}
\setlength{\parindent}{0pt}

\usepackage{caption}
\usepackage{subcaption}

\usepackage{makeidx}

\usepackage[french,ruled,vlined]{algorithm2e}
\SetKwComment{Comment}{\#}{}
\SetKwFor{Tq}{tant que}{}{}
\SetKwFor{Pour}{pour}{}{}
\DontPrintSemicolon
\SetAlgoLined

\usepackage{listings}
\lstset{language=Python,frame=single}
\lstset{literate=
  {á}{{\'a}}1 {é}{{\'e}}1 {í}{{\'i}}1 {ó}{{\'o}}1 {ú}{{\'u}}1
  {Á}{{\'A}}1 {É}{{\'E}}1 {Í}{{\'I}}1 {Ó}{{\'O}}1 {Ú}{{\'U}}1
  {à}{{\`a}}1 {è}{{\`e}}1 {ì}{{\`i}}1 {ò}{{\`o}}1 {ù}{{\`u}}1
  {À}{{\`A}}1 {È}{{\'E}}1 {Ì}{{\`I}}1 {Ò}{{\`O}}1 {Ù}{{\`U}}1
  {ä}{{\"a}}1 {ë}{{\"e}}1 {ï}{{\"i}}1 {ö}{{\"o}}1 {ü}{{\"u}}1
  {Ä}{{\"A}}1 {Ë}{{\"E}}1 {Ï}{{\"I}}1 {Ö}{{\"O}}1 {Ü}{{\"U}}1
  {â}{{\^a}}1 {ê}{{\^e}}1 {î}{{\^i}}1 {ô}{{\^o}}1 {û}{{\^u}}1
  {Â}{{\^A}}1 {Ê}{{\^E}}1 {Î}{{\^I}}1 {Ô}{{\^O}}1 {Û}{{\^U}}1
  {œ}{{\oe}}1 {Œ}{{\OE}}1 {æ}{{\ae}}1 {Æ}{{\AE}}1 {ß}{{\ss}}1
  {ű}{{\H{u}}}1 {Ű}{{\H{U}}}1 {ő}{{\H{o}}}1 {Ő}{{\H{O}}}1
  {ç}{{\c c}}1 {Ç}{{\c C}}1 {ø}{{\o}}1 {å}{{\r a}}1 {Å}{{\r A}}1
  {€}{{\euro}}1 {£}{{\pounds}}1 {«}{{\guillemotleft}}1
  {»}{{\guillemotright}}1 {ñ}{{\~n}}1 {Ñ}{{\~N}}1 {¿}{{?`}}1
}

%pr{\'e}sentation des compteurs de section, ...
\makeatletter
\renewcommand{\thesection}{\Roman{section}.}
\renewcommand{\thesubsection}{\arabic{subsection}.}
\renewcommand{\thesubsubsection}{\arabic{subsubsection}.}
\renewcommand{\labelenumii}{\theenumii.}
\makeatother


\newtheorem*{thm}{Théorème}
\newtheorem{thmn}{Théorème}
\newtheorem*{prop}{Proposition}
\newtheorem{propn}{Proposition}
\newtheorem*{pa}{Présentation axiomatique}
\newtheorem*{propdef}{Proposition - Définition}
\newtheorem*{lem}{Lemme}
\newtheorem{lemn}{Lemme}

\theoremstyle{definition}
\newtheorem*{defi}{Définition}
\newtheorem*{nota}{Notation}
\newtheorem*{exple}{Exemple}
\newtheorem*{exples}{Exemples}


\newenvironment{demo}{\renewcommand{\proofname}{Preuve}\begin{proof}}{\end{proof}}
%\renewcommand{\proofname}{Preuve} doit etre après le begin{document} pour fonctionner

\theoremstyle{remark}
\newtheorem*{rem}{Remarque}
\newtheorem*{rems}{Remarques}

%\usepackage{maths}
%\newcommand{\dbf}{\leftrightarrows}

%En tete et pied de page
\lhead{Informatique}
%\chead{Introduction aux systèmes informatiques}
\rhead{MPSI B Hoche}
\lfoot{\tiny{Cette création est mise à disposition selon le Contrat\\ Paternité-Partage des Conditions Initiales à l'Identique 2.0 France\\ disponible en ligne http://creativecommons.org/licenses/by-sa/2.0/fr/  
} 
\rfoot{\tiny{Rémy Nicolai \jobname \; \today } }
}
\makeindex

\lhead{Cours IPT}
\chead{Représentation binaire d'un rationnel: résumé pratique}
\begin{document}
\section{Opérations}
Soit $x$ un nombre rationnel strictement positif. 
\begin{enumerate}
  \item Division euclidienne : partie entière, partie fractionnaire
\begin{displaymath}
  x = \frac{a}{q},\hspace{0.5cm} a = mq + r, \hspace{0.5cm} x = m + \frac{r}{q}
\end{displaymath}
  \item Algorithme \og les petits d'abord\fg~ : développement de la partie entière pour placer les chiffres à gauche de la virgule
  \begin{center}
    tant que >0 : noter le reste mod 2; remplacer par le quotient de la division par 2
  \end{center}

  \item Algorithme \og les grands d'abord\fg~ développement de la partie fractionnaire pour placer les chiffres à droite de la virgule.
  \begin{center}
    Pour obtenir le nombre de chiffres souhaités:\newline
    multiplier par 2 : si $\geq 1$ noter 1 sinon noter 0; remplacer par la partie fractionnaire
  \end{center}
  On constatera que la suite devient périodique à partir d'un certain rang.
  
  \item Décaler la virgule et trouver l'exposant pour la forme normalisée. La mantisse est un nombre dans $[1,2[$.
\end{enumerate}

Tout nombre décimal est rationnel. La méthode est absolument la même.
  \section{Exemple}
Exemple  : $x=\frac{65}{12}$.
\begin{enumerate}
  \item Division euclidienne $65 = 5 \times 12 +5$ 
\begin{displaymath}
  x = \underset{\text{partie entière}}{\underbrace{5}} + \underset{\text{partie fractionnaire}}{\underbrace{\frac{5}{12}}}
\end{displaymath}

  \item \og Les petits d'abord.\fg
\begin{displaymath}
\renewcommand{\arraystretch}{1.7}
\begin{array}{lccc}
5& = 2\times 2 + 1                           &:  & 1,0  \\
2& = 1\times 2 + 0                           &:  & 01,0  \\
1& = 0\times 2 + 1                           &:  & 101,0  \\
0& \text{STOP}                               & :  & 101,0
\end{array}
\end{displaymath}


  \item \og Les grands d'abord.\fg
\begin{displaymath}
\renewcommand{\arraystretch}{1.7}
\begin{array}{lccc}
\frac{5}{12}&                                &:  & 0,  \\
\frac{10}{12}&                               &:  & 0,0  \\
\frac{20}{12}&\rightarrow \frac{8}{12}       &:  & 0,01  \\
\frac{16}{12}&\rightarrow \frac{4}{12}       &:  & 0,011  \\
\frac{8}{12}&                                &:  & 0,0110  \\
\frac{16}{12}&\rightarrow \frac{4}{12}       &:  & 0,01101  \\
\frac{8}{12}&                                &:  & 0,011010  
\end{array}
\end{displaymath}
Le développement devient périodique : $0,01\,10\,10\,\cdots \,10$.

  \item Décalage. La mantisse doit appartenir à $[1,2[$.
\begin{displaymath}
  101,01\,10\,10\,\cdots \,10 \;\rightarrow \; \underset{\text{mantisse}}{\underbrace{1,0101\,10\,10\,\cdots \,10}}\;\times 2^2
\end{displaymath}
On rappelle que pour la mantisse, seuls les chiffres \emph{après la virgule} sont stockés (sur 52 bits pour la norme IEEE 754).
\end{enumerate}

\end{document}
