Une base de données médicale\footnote{D'après le sujet de CCINP PSI 2019.} contient des informations administratives sur des patients et des informations médicales résultant des examens médicaux subis par ces patients.\newline
La table \texttt{PATIENT} contient les attributs suivants:
\begin{itemize}
 \item \texttt{id} : identifiant unique d'un individu (entier), clé primaire;
 \item \texttt{nom} : nom du patient (chaîne de caractères);
 \item \texttt{prenom}: prénom du patient (chaîne de caractères);
 \item \texttt{adresse} : adresse du patient (chaîne de caractères);
 \item \texttt{email} : (chaîne de caractères);
 \item \texttt{naissance} : année de naissance (entier).
\end{itemize}
La table \texttt{EXAMEN} contient les attributs suivants:
\begin{itemize}
 \item \texttt{id} : identifiant unique de l'examen (entier), clé primaire;
 \item \texttt{date} : la date à laquelle l'examen a été réalisé (date);
 \item \texttt{data1} : donnée (flottant);
 \item \texttt{data2} : donnée (flottant);
 \item ... ;
 \item \texttt{idpatient} : identifiant du patient ayant subi l'examen représenté par l'attribut id de la table PATIENT (entier);
 \item \texttt{etat} : description codée de l'état du patient (une chaine de caractères parmi N H ou S).
\end{itemize}

Les attributs \texttt{data1}, \texttt{data2}, ... sont des mesures effectuées lors de l'examen. L'attribut \texttt{etat} a été renseigné à l'issue de l'examen. La signification des codes d'état est donnée dans le tableau suivant:
\begin{center}
\renewcommand{\arraystretch}{1.2}
\begin{tabular}{|l|l|l|} \hline
\texttt{N} & \texttt{H}     & \texttt{S} \\ \hline
Normal     & Hernie discale & Spondylolisthésis \\ \hline
\end{tabular}
\end{center}
On ne soulèvera pas de difficulté au sujet du type \og date\fg. On supposera en particulier que l'on peut comparer deux dates avec une simple inégalité
\begin{multline*}
 \left( \text{date de l'événement 1} \leq \text{date de l'événement 2}\right)\\
 \Leftrightarrow \text{ l'événement 1 s'est déroulé avant l'événement 2}.
\end{multline*}
\begin{enumerate}
 \item \'Ecrire une requête SQL permettant d'extraire les identifiants des patients pour lesquels une hernie discale a été diagnostiquée.
 \item \'Ecrire une requête SQL permettant d'extraire les noms et prénoms des patients pour lesquels une spondylolisthésis a été diagnostiquée.
 \item \'Ecrire une requête SQL permettant d'extraire les identifiants des patients pour lesquels une hernie discale et une spondylolisthésis ont été diagnostiquées.
 \item \'Ecrire une requête SQL permettant d'extraire les identifiants des patients pour lesquels une hernie discale a été diagnostiquée lors d'un examen mais qui ont subi plus tard un autre examen dignostiquant l'état \texttt{N}.
\end{enumerate}
