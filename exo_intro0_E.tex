\begin{enumerate}
  \item Expliquer ce que va faire et afficher le programme suivant:
\lstinputlisting[firstline=1, lastline=4]{exo_intro0_E.py}  
avant de le taper dans l'éditeur et de l'exécuter.

  \item \'Ecrire deux programmes dont l'un exécutera 100 fois un bloc de code contenant un test et l'autre exécutera 49 fois un bloc contenant une incrémentation plus pertinente pour les problèmes suivants.
\begin{enumerate}
  \item Afficher tous les nombres impairs entre $0$ et $100$ par ordre croissant.
  \item Afficher tous les nombres pairs entre $0$ et $100$ par ordre décroissant.
\end{enumerate}

  \item \'Ecrire deux programmes qui affichent les 20 premiers termes de la suite de Fibonacci définie par
\begin{displaymath}
  f_0 = f_1 = 1, \hspace{1cm}\forall n \in \N,\; f_{n+2} = f_{n+1} + f_n
\end{displaymath}
Le premier programme utilisera les assignations multiples et pas le second. Modifier les programmes pour qu'ils n'affichent que le 20ème terme (c'est à dire $f_{19}$)

  \item \'Ecrire un programme qui affiche les 10 premiers termes de la suite récurrente définie par
\begin{displaymath}
  u_0 = 0, \hspace{0.5cm} \forall n \in \N, \;u_{n+1} = 2 + 2u_n
\end{displaymath}

  \item \`A l'aide d'un programme utilisant 3 boucles "while" imbriquées, calculer le nombre de triplets $(x,y,z)$ d'entiers de $\llbracket 1, 10 \rrbracket$ tels que
  \begin{displaymath}
  1  \leq x < y < z \leq 10
  \end{displaymath}

\end{enumerate}
