\subsection*{I - Outils}
\begin{enumerate}
  \item Les indices en Python commencencent à $0$, on convient donc que la fonction \verb|indice(lili,obj)| doit renvoyer \verb|-1| lorsque \verb|obj| n'est pas une valeur de la liste \verb|lili|. On peut implémenter cette fonction avec le code suivant:
\begin{verbatim}
def indice(lili,obj):
    # n nb de valeurs de la liste
    n = len(lili)
    #indice dans la liste
    i = 0
    while i< n:
        if lili[i] == obj:
            return i
        else:
            i = i + 1
    return -1  
\end{verbatim}

  \item 
Dans l'algorithme \ref{Cfreqtexte_1} proposé, on utilise les variables suivantes:
\begin{itemize}
  \item $L$ une liste de $n>10$ valeurs comparables.
  \item $Imax\leftarrow$ une liste (pour les 10 indices de plus grande valeur).
  \item $Vmax\leftarrow$ une liste (les 10 plus grandes valeurs).
  \item $Imax[0]\leftarrow 0$ .
  \item $Vmax[0]\leftarrow L[0]$.
  \item $p\leftarrow 1$ nombre de valeurs dans ces listes au début.
  \item $i\leftarrow 1$ indice dans $L$.

\end{itemize}
\begin{algorithm}
  \Comment{traitement des premières valeurs}
  \Tq{ i < 10}{
    \eIf{L[i] $\leq$ Vmax[p-1]}{
      $Vmax[p]\leftarrow L[i]$\;
      $Imax[p]\leftarrow i$\;
    }{
      $j\leftarrow p-1$\;
      \Tq{j$\geq 0$ et L[i] > Vmax[j]}{
        $Vmax[j+1]\leftarrow Vmax[j]$\;
        $Imax[j+1]\leftarrow Imax[j]$\;
        $j\leftarrow j-1$\;
      }
      \Comment{ici $L[i]\leq Vmax[j]$}
      $Vmax[j+1]\leftarrow L[i]$\;
      $Imax[j+1]\leftarrow i$\;
    }
    $i\leftarrow i+1$\;
    $p\leftarrow p+1$\;
  }
  \Comment{traitement des autres valeurs}
  \Tq{i < n}{
    \If{L[i] > Vmax[p-1]}{
      $j\leftarrow p-1$\;
      \Tq{$j\geq 0$ et L[i] > Vmax[j]}{
        $Vmax[j]\leftarrow Vmax[j-1]$\;
        $Imax[j]\leftarrow Imax[j-1]$\;
        $j\leftarrow j-1$\;
      }
      \Comment{ici $L[i] \leq Vmax[j]$ }
      $Vmax[j+1]\leftarrow L[i]$\;
      $Imax[j+1]\leftarrow i$\;    
    }
    $i\leftarrow i+1$\;
  }
  \caption{Algorithme pour top10}
  \label{Cfreqtexte_1}
\end{algorithm}
Le code suivant est une implémentatio en Python de la fonction demandée
\begin{verbatim}
def top10(L):
    n = len(L)
    Imax = ["" for i in range(10)]
    Vmax = ["" for i in range(10)]
    Imax[0] = 0
    Vmax[0] = L[0]
    p = 1
    i = 1
    #traitement des premières valeurs
    while i < 10:
        if L[i] <= Vmax[p-1]:
            Vmax[p] = L[i]
            Imax[p] = i
        else:
            j = p-1
            #print( (j>=0) & (L[i] > Vmax[j]))
            while (j>=0) & (L[i] > Vmax[j]):
                Vmax[j+1] = Vmax[j]
                Imax[j+1] = Imax[j]
                j = j-1
            Vmax[j+1] = L[i]
            Imax[j+1] = i
        i += 1
        p += 1
    print(Imax)
    #traitement des autres valeurs
    while i < n:
        if L[i] > Vmax[p-1]:
            j = p-1
            while (j>=0) & (L[i] > Vmax[j]):
                Vmax[j] = Vmax[j-1]
                Imax[j] = Imax[j-1]
                j = j-1
            Vmax[j+1] = L[i]
            Imax[j+1] = i
        i += 1
    return Imax  
\end{verbatim}

\end{enumerate}

\subsection*{II - Fréquences de caractères}
\subsection*{III - Fréquences de mots}