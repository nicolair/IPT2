\subsection{Calcul de carrés}
\begin{enumerate}
  \item Dans une expression \texttt{ a // b}, les noms \texttt{a} et \texttt{b} doivent désigner des objets du type entier. L'expression s'évalue alors au quotient entier de la division de \texttt{a} par \texttt{b}. 
  \item Tous les nombres (y compris $a$ et $b$) représentés par des listes de longueur $l+1$ sont plus petits que
\begin{displaymath}
  1 + \frac{1}{2} + \frac{1}{2^2} + \cdots + \frac{1}{2^l} = \frac{1-\frac{1}{2^{l+1}}}{1-\frac{1}{2}}= 2 -\frac{1}{2^l} < 2
\end{displaymath}
Il est commode de considérer une troisième propriété: \og \texttt{t} désigne 0 ou 1 ou 2 ou 3\fg. Les trois propriétés sont des invariants car:
\begin{multline*}
\left. 
\begin{aligned}
a_i \in \{0,1\}\\ b_i \in \{0,1\} \\r_i \in\{0,1\}  
\end{aligned}
\right\rbrace 
\Rightarrow t_{i+1} \in \{0,1,2,3\}\\
\Rightarrow
\left\lbrace 
\begin{aligned}
&l_{i+1} \in\{0,1\}\text{ (reste de $t_{i+1}$ modulo 2)}\\
&r_{i+1} \in\{0,1\}\text{ (quotient de la div. de $t_{i+1}$ par 2)}  
\end{aligned}
\right. 
\end{multline*}
\`A la fin, la \og retenue\fg~ \texttt{ret} peut prendre la valeur 0 ou 1. La liste $S$ représente $a+b$ lorsque $a+b<2$ c'est à dire lorsque \texttt{ret} désigne $0$. Sinon $s+2 = a+b$.
  \item 
\begin{enumerate}
  \item Codage Python
\lstinputlisting[firstline=28, lastline=36]{Cbrigglog.py}
\item Désignons par $a_l$ la dernière valeur de la liste \texttt{A} avant le décalage.
\begin{displaymath}
  a' =
\left\lbrace 
\begin{aligned}
&\frac{a}{2}&\text{ si } a_l = 0 \\
&\frac{a}{2} - \frac{1}{2^{l+1}}&\text{ si } a_l = 1 
\end{aligned}
\right. 
\end{displaymath}
\end{enumerate}

\item
\begin{enumerate}
  \item Si on remplace l'instruction \og\texttt{B = [a for a in A]}\fg~ par \og\texttt{B = A}\fg, on change profondément l'algorithme. En effet les deux noms désignent alors la même liste et cette liste sera modifiée par l'appel \texttt{shift(B,d)}. En effet cette fonction modifie l'objet (modifiable) désignée par \texttt{B}. Cela va pertuber la boucle d'énumération des objets de \texttt{A} :\og \texttt{for a in A}\fg~ puisqu'ils sont modifiés à l'intérieur même de l`énumération.
  
  \item Le produit de deux nombres binaires de longueur $l$ n'est pas forcément un nombre binaire de longueur $l$. Par exemple, si $a$ et $b$ sont dans $\llbracket 0,l \rrbracket$ avec $a+b >l$, les nombres $2^{-a}$ et $2^{-b}$ sont binaires de longueur $l$ mais pas leur produit.
  
  \item L'appel de la fonction correspond à une suite de décalages et d'additions
\begin{align*}
\begin{array}{clllll}
                &0 & 0 & 0 & 0 & 0\\
\text{shift 0}: &1 & 1 & 0 & 1 & 0
\end{array}
\Rightarrow 
\left\lbrace
\begin{aligned}
&ret \rightarrow 0 \\ &C \rightarrow (1,1,0,1,0)  
\end{aligned}
\right. \\
\begin{array}{clllll}
                &1 & 1 & 0 & 1 & 0\\
\text{shift 1}: &0 & 1 & 1 & 0 & 1
\end{array}
\Rightarrow 
\left\lbrace
\begin{aligned}
&ret \rightarrow 1 \\ &C \rightarrow (0,0,1,1,1)  
\end{aligned}
\right. \\
\begin{array}{clllll}
                &0 & 0 & 1 & 1 & 1\\
\text{shift 2}: &0 & 0 & 0 & 1 & 1
\end{array}
\Rightarrow 
\left\lbrace
\begin{aligned}
&ret \rightarrow 1 \\ &C \rightarrow (0,1,0,1,0)  
\end{aligned}
\right. 
\end{align*}

  \item Cette fonction renvoie une valeur approchée par défaut du carré d'un nombre binaire de $[0,1[$. Un tel nombre est dans $[0,4[$, la retenue permet de savoir s'il est plus grand que $2$. Il est bien clair que le nombre renvoyé est plus petit que le carré. Le principal reproche à faire à cette fonction c'est que l'erreur n'est pas facile à majorer. En particulier, pour certains $x$, le nombre renvoyé \emph{ne sera pas} l'approximation binaire par défaut de $x^2$.
\end{enumerate}

\end{enumerate}

\subsection{Calculs de logarithmes particuliers}
\begin{enumerate}
  \item Les suites $\left( x_n\right)_{n\in \N}$, $\left( b_n\right)_{n\in \N}$ vérifient les relations de récurrence:
\begin{displaymath}
b_{k+1} =
\left\lbrace 
\begin{aligned}
  &0 &\text{ si } x_k^2 < 10 \\
  &1 &\text{ si } x_k^2 \geq 10
\end{aligned}
\right., \\
\hspace{1cm} x_{k+1} = \frac{x_k^2}{10^{b_{k+1}}}
\end{displaymath}

\item Le nom $x$ désigne au départ un nombre entre $1$ et $10$. Après chaque assignation $x \leftarrow x *x$, il désigne un nombre entre $0$ et $100$. On teste s'il est plus grand que 10 et on le divise par 10 si c'est le cas (ce qui redonne un nombre plus petit que $10$). On en déduit que \og $x\in [1,10[$\fg est un invariant de boucle.\newline
La liste $B$ est augmentée par la valeur de $b$ qui n'est assigné qu'à 0 ou 1. On en tire que \og$B$ est une liste de 0 et de 1\fg est aussi un invariant de boucle. 

\item On montre par récurrence la relation demandée en l'élevant au carré et en utilisant $x_{k+1}10^{b_{k+1}}=x_k^{2}$.
\begin{multline*}
  x_{k+1} = 
\left( 
\frac{x_{ini}^{2^k}}{10^{2^{k}\left(\frac{b_1}{2}+\frac{b_2}{2^2}+\cdots+ \frac{b_k}{2^k}\right) }}
\right)^2 \frac{1}{10^{b_{k+1}}}
=
\frac{x_{0}^{2^{k+1}}}{10^{2^{k+1}\left(\frac{b_1}{2}+\frac{b_2}{2^2}+\cdots+ \frac{b_k}{2^k}\right) +b_{k+1}}}\\
=
\frac{x_{0}^{2^{k+1}}}{10^{2^{k+1}\left(\frac{b_1}{2}+\frac{b_2}{2^2}+\cdots+ \frac{b_k}{2^k} + \frac{b_{k+1}}{2^{k+1}}\right) }}
\end{multline*}

\item Par définition, comme la liste $B$ commence par un $0$, le nombre binaire représenté est
\begin{displaymath}
  b = \frac{b_1}{2}+\frac{b_2}{2^2}+\cdots+ \frac{b_n}{2^n}
\end{displaymath}
La relation de la question 3 s'écrit alors
\begin{multline*}
x_n 10^{2^nb} = x_0^{2^{n}}
\Rightarrow
\ln x_n + 2^n\,b\,\ln 10 = 2^n \ln x_0
\Rightarrow
\frac{\ln x_0}{\ln 10} = b + 2^{-n}\frac{\ln x_n}{\ln 10}\\
\Rightarrow
b \leq \frac{\ln x_0}{\ln 10} < b + 2^{-n}
\end{multline*}
car on a montré que $0<x_n<10$. Cet algorithme permet donc de calculer le développement binaire du logarithme décimal d'un nombre entre $1$ et $10$.
\end{enumerate}
