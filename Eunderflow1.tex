Dans les deux premières questions, les tableaux placés à la fin qui donnent l'écriture décimale de quelques puissances de $2$ peuvent être utiles.
\begin{enumerate}
  \item Décomposer $\frac{136}{13}$ en base $2$.
  \item Préciser la forme normalisée (mantisse et exposant) des nombres décimaux
  \begin{displaymath}
    257 \hspace{0.5cm} \frac{1}{31} \hspace{0.5cm} 512.001953125
  \end{displaymath}

  \item On considère le code suivant.  
\lstinputlisting[firstline=1, lastline=10]{Eunderflow1.py}
Vous devez commenter les éléments de syntaxe que vous jugez intéressants, expliquer ce que désignent les noms \texttt{x}, \texttt{xx}, \texttt{y} mathématiquement (en  introduisant des suites) et informatiquement (types des objets), envisager l'exécution et prévoir ce qui se passe. 
\end{enumerate}
\begin{center}
  \renewcommand{\arraystretch}{1.5}
\begin{tabular}{|c|c|c|c|c|c|} \hline
  -10        & -9          & -8         & -7        & -6       & -5       \\ \hline
0.0009765625 & 0.001953125 & 0.00390625 & 0.0078125 & 0.015625 & 0.03125  \\  \hline
\end{tabular} 

\vspace{0.5cm}
\begin{tabular}{|c|c|c|c|c|c|c|c|c|c|c|c|c|c|} \hline
 -4    & -3    & -2   & -1  &  1 & 2 & 3 & 4  & 5  & 6  & 7   & 8   & 9   & 10  \\ \hline
0.0625 & 0.125 & 0.25 & 0.5 &  2 & 4 & 8 & 16 & 32 & 64 & 128 & 256 & 512 & 1024 \\  \hline
\end{tabular} 
\end{center}
