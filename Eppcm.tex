Soit $a \in \N^*$. Un nombre naturel $m$ est un \emph{multiple} de $a$ si et seulement si il existe $q\in \N$ tel que $m=qa$. On se limite dans ce texte aux entiers naturels sans considérer les multiples entiers relatifs négatifs.

\begin{enumerate}
  \item Expliquer ce que renvoie l'appel \texttt{continuer(lili)} où \texttt{lili} désigne une liste et prouver la correction de cette fonction.
\lstinputlisting[firstline=1, lastline=6]{Eppcm_1.py}

\begin{algorithm}
  \Donnees{\;
    $A \leftarrow$ liste des $a_0, \cdots , a_{p-1}$\;
  }
  \Comment{initialisation}
    $M \leftarrow$ liste des $a_0, \cdots , a_{p-1}$\;
    $imin$ désigne un entier naturel entre 0 et $p-1$\;
  \Tq{ continuer(M)}{
    $imin\leftarrow $ entier tel que $M[imin]=\min\left\lbrace M[0], \cdots, M[p]\right\rbrace$\;
    $M[imin] \leftarrow M[imin] + A[imin]$\;
  }
  \caption{Calcul d'un ppcm}
  \label{ppcm_1}
\end{algorithm}

  \item Soit $p$ un nombre naturel supérieur ou égal à $2$ et $a_0, \cdots , a_{p-1}$ des nombres naturels non nuls. On considère dans cette question l'algorithme \ref{ppcm_1} qui utilise la fonction \texttt{continuer} de la première question.
\begin{enumerate}
  \item Justifier mathématiquement l'existence d'un entier $\mu$ appelé \emph{plus petit commun multiple} (ppcm) de $a_0, \cdots , a_{p-1}$.
  
  \item Reformuler avec la plus petite et la plus grande valeur de $M$ la condition \og $continuer(M) == True$\fg~ d'exécution de la boucle. Montrer que la proposition 
   \begin{center}
     \og toutes les valeurs de $M$ sont inférieures ou égales à $\mu$\fg
   \end{center}
  est un invariant de la boucle.
  
  \item Préciser une fonction de terminaison de la boucle. Expliquez ce qui se passe après la sortie de la boucle et prouvez ce que vous affirmez.
\end{enumerate}

  \item 
\begin{enumerate}
  \item Coder une fonction dont l'appel \texttt{indiceMin(M)} renvoie l'entier tel que $M[imin]=\min\left\lbrace M[0], \cdots, M[p]\right\rbrace$ lorsque \texttt{M} est une liste de $p$ entiers naturels non nuls.
  \item Coder une fonction dont l'appel \texttt{ppcm(A)} renvoie le ppcm des valeurs de \texttt{A} lorsque \texttt{A} est une liste d'entiers naturels non nuls. 
\end{enumerate}

\end{enumerate}
