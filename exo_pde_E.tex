Soit $b\in \Z$ et $a\in \N^*$, la division euclidienne de $b$ par $a$ se traduit par:
\begin{displaymath}
  \exists q\in \Z \text{ et } r\in \llbracket 0, a \llbracket \text{ tq } b = qa +r
\end{displaymath}
On désigne par \emph{pseudo division euclidienne} (pde) de $b$ par $a$ la propriété :
\begin{displaymath}
  \exists q\in \Z \text{ et } r\in \llbracket 0, a \llbracket \text{ tq } b = qa - r
\end{displaymath}
On dira encore que $q$ est le quotient et $r$ le reste dans la pseudo division de $b$ par $a$.
\begin{enumerate}
  \item Exprimer avec les symboles $\lfloor\;\rfloor$ et $\lceil\;\rceil$ le quotient de $b$ par $a$ pour chaque division.
  \item 
\begin{algorithm}
  $b\leftarrow$ un naturel non nul\;
  $a\leftarrow a_{ini}$ un naturel non nul strictement plus petit que $b$\;
  $L\leftarrow$ une liste vide\;
  $s\leftarrow 0$\;
  $Q\leftarrow 1$\;
  \Tq{ a > 0}{
    $q\leftarrow$ quotient de la pde de $b$ par $a$\;
    $r\leftarrow$ reste de la pde de $b$ par $a$\;
    $a\leftarrow r$\;
    $Q\leftarrow Q * q$\;
    $s \leftarrow s + \frac{1}{Q}$\;
    placer $Q$ à la fin de $L$\;
  }
  \caption{Un développement}
  \label{exo_pde_E_1}
\end{algorithm}
Pour la boucle de l'algorithme \ref{exo_pde_E_1}, montrer que $a$ est un variant et 
\begin{displaymath}
  \Phi = \left( sb + \frac{a}{Q} == a_{ini}\right) 
\end{displaymath}
est un invariant. Que désigne $a$ après la sortie de la boucle? Quelle sorte de décomposition cet algorithme permet-il de calculer?
\end{enumerate}
