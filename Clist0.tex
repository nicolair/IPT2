\begin{enumerate}
  \item 
 
\begin{verbatim}
nmax = 4
CoeffBin = (nmax + 1)*[0]
#initialisée comme une liste de 0
for n in range(nmax + 1):
  CoeffBin[n] = (n+1)*[1]
  #sous-liste initialisée comme une liste de 1
  for p  in range(n-1):
    CoeffBin[n][p+1] = CoeffBin[n-1][p+1] + CoeffBin[n-1][p]  
print(CoeffBin)
>>> 
[[1], [1, 1], [1, 2, 1], [1, 3, 3, 1], [1, 4, 6, 4, 1]]\end{verbatim}

  \item Le coefficient du binôme $\binom{6}{4}=15$ calculé avec le quotient de produits. Un double slash est utilisé pour que le calcul renvoie un entier et non un flottant.

  \item Inversion de listes.
\begin{verbatim}lili = [0,1,0,0,0,1]
l = len(lili)
i = 0
j = l-1
while i < j:
    lili[i],lili[j] = lili[j],lili[i]
    i += 1
    j-= 1
print(lili)\end{verbatim}
Elle peut se faire en place (en modifiant l'objet) car une liste est modifiable.\newline
En revanche une chaîne de caractères n'est pas modifiable, il faut un nouvel objet.
\begin{verbatim}str = 'tagada'
rnv = ''
for chacha in str:
    rnv = chacha + rnv
print(rnv)\end{verbatim}

\end{enumerate}
