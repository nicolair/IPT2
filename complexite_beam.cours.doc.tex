%!  pour pdfLatex
\documentclass{beamer}

%\usepackage[pdftex]{graphicx,color}
%\usepackage[pdftex,colorlinks={true},urlcolor={blue},pdfauthor={remy Nicolai}]{hyperref}

\usepackage[utf8]{inputenc}
\usepackage[T1]{fontenc}
\usepackage{lmodern}
\usepackage[frenchb]{babel}

\usetheme{Warsaw}

%\usepackage{fancyhdr}
%\pagestyle{fancy}

%\usepackage{floatflt}
\usepackage{maths}

\usepackage{parcolumns}
\setlength{\parindent}{0pt}

%\usepackage{caption}
%\usepackage{subcaption}

\usepackage[french,ruled,vlined]{algorithm2e}
\SetKwComment{Comment}{\#}{}
\SetKwFor{Tq}{tant que}{}{}
\SetKwFor{Pour}{pour}{}{}
\DontPrintSemicolon
\SetAlgoLined

\usepackage{listings}
\lstset{language=Python,frame=single}

%pr{\'e}sentation des compteurs de section, ...
\makeatletter
\renewcommand{\thesection}{\Roman{section}.}
\renewcommand{\thesubsection}{\arabic{subsection}.}
\renewcommand{\thesubsubsection}{\arabic{subsubsection}.}
%\renewcommand{\labelenumii}{\theenumii.}
\makeatother


\newtheorem*{thm}{Théorème}
\newtheorem{thmn}{Théorème}
\newtheorem*{prop}{Proposition}
\newtheorem{propn}{Proposition}
\newtheorem*{pa}{Présentation axiomatique}
\newtheorem*{propdef}{Proposition - Définition}
\newtheorem*{lem}{Lemme}
\newtheorem{lemn}{Lemme}

\theoremstyle{definition}
\newtheorem*{defi}{Définition}
\newtheorem*{nota}{Notation}
\newtheorem*{exple}{Exemple}
\newtheorem*{exples}{Exemples}


\newenvironment{demo}{\renewcommand{\proofname}{Preuve}\begin{proof}}{\end{proof}}
%\renewcommand{\proofname}{Preuve} doit etre après le begin{document} pour fonctionner

\theoremstyle{remark}
\newtheorem*{rem}{Remarque}
\newtheorem*{rems}{Remarques}

%\usepackage{maths}
%\newcommand{\dbf}{\leftrightarrows}

%En tete et pied de page
%\lhead{Informatique}
%\chead{Introduction aux systèmes informatiques}
%\rhead{MPSI B Hoche}
%\lfoot{\tiny{Cette création est mise à disposition selon le Contrat\\ Paternité-Partage des Conditions Initiales à l'Identique 2.0 France\\ disponible en ligne http://creativecommons.org/licenses/by-sa/2.0/fr/
%} }
%\rfoot{\tiny{Rémy Nicolai \jobname}}

\nonstopmode

\begin{document}
\begin{frame}
  \frametitle{Types de complexité}
\'Evaluer une complexité c'est établir qu'une certaine fonction d'un entier naturel $n$ est \emph{dominée} (au sens mathématique) par une suite usuelle.
\begin{itemize}
  \item $O(1)$: complexité constante. La suite est majorée. 
  \item $O(\ln(n))$: complexité sous-linéaire ou logarithmique.
  \item $O(n)$: complexité linéaire.
  \item $O(n\ln(n))$: complexité quasi-linéaire.
  \item $O(n^2)$: complexité quadratique.
  \item $O(n^k)$: complexité polynomiale.
  \item $O(C^n)$: complexité exponentielle.
\end{itemize}
\end{frame}

\begin{frame}
  \frametitle{De quelles fonctions parle-t-on ?}
de fonctions renvoyant des nombres d'instructions selon la taille des données  
\end{frame}

\begin{frame}
  \frametitle{Nombres d'instructions}
ressources utilisées = instructions élémentaires exécutées.

lecture-écriture en mémoire, calcul avec des entiers ou avec des flottants.

on se limite à un type particulier d'instructions.

Dans le cas des calculs (entiers ou flottants), on parle de \emph{complexité en temps}.

Dans le cas des instructions de lecture-écriture, comme elles sont très rapides, on considère plutôt la taille de la mémoire nécessaire à ces instructions plutôt que leur nombre. On parle alors de \emph{complexité en espace}\index{complexité en espace}.
\end{frame}

\begin{frame}
  \frametitle{Taille des données}
Exécution à partir d'un jeu de données numériques (booléennes, entières, flottantes).

La donnée n'est pas un nombre en elle même.

On doit préciser un entier naturel: la \emph{taille} de la donnée.

Plusieurs données de même taille.

Deux données de même taille ne conduisent pas au même nombre d'instructions exécutées.
\begin{itemize}
  \item \emph{complexité dans le pire des cas}: majorer le nombre d'instructions pour \emph{toutes} les données de taille $n$.
  \item \emph{complexité moyenne} : l'ensemble des données de taille $n$ est probabilisé, le nombre d'instructions est une variable aléatoire, la complexité moyenne est l'espérance de cette variable.
\end{itemize}
\end{frame}

\begin{frame}
  \frametitle{Estimations en temps}
rapidité d'un processeur = nombre d'instructions qu'il peut exécuter par seconde.

L'unité est le \emph{flops} (floating operations per second).

Un ordinateur personnel $\simeq$ 200 gigaFLOPS = $200\times 10^{9}$ FLOPS.

Donnée de taille $10^{6}$.
\begin{center}
\renewcommand{\arraystretch}{1.5}
\begin{tabular}{|l|l|l|l|l|l|} \hline
log. & lin. & quasi-lin. & quad. & pol. (3) & pol. (4)\\ \hline
69 ns & 5 $\mu$s & 69 $\mu$s & 5 s & 57 jours & 158548 ans \\ \hline
\end{tabular}
\end{center}
\end{frame}

\begin{frame}
  \frametitle{Enchaînement}
Un algorithme s'exécute avec des données de taille $n$, il est constitué de deux algorithmes successifs:
\begin{itemize}
  \item un pré-traitement qui est de complexité constante
  \item le traitement proprement dit est de complexité linéaire  
\end{itemize}
L'algorithme complet est aussi de complexité linéaire.
\end{frame}

\begin{frame}
  \frametitle{Recherche dichotomique: complexité logarithmique}
\begin{algorithm}[H]
  \Donnees{\;
    $L \leftarrow$ liste croissante de $n$ objets comparables.\;
    $v \leftarrow$ un objet vérifiant $L[0]\leq v < L[n-1]$\;
  }
  $i\leftarrow 0$\;
  $j\leftarrow n-1$\;
  \Tq{ $j-i > 1$}{
    $k\leftarrow$ partie entière de $\frac{i+j}{2}$ \;
    \eSi{L[k] $\leq$ v}{
      $i\leftarrow k$ \;
    }{
      $j\leftarrow k$
    }
  }
  \Comment{$i$ désigne l'indice cherché.}
  \caption{Encadrement par dichotomie dans une liste}
  \label{complexite_1}
\end{algorithm}
\end{frame}

\begin{frame}
  \frametitle{Complexité optimale d'un tri}
Un \emph{algorithme de tri} forme une liste strictement croissante à partir de la donnée d'une liste de $n$ objets deux à deux distincts.

Soit $c(n)$ la complexité d'un tel algorithme considérée comme le nombre de comparaisons en fonction de la longueur $n$ de la liste.

Alors:

\begin{displaymath}
  n\ln n \in  O(c(n)) 
\end{displaymath}
Il ne peut donc exister un algorithme de tri dont la complexité dans le pire des cas soit meilleure que quasi linéaire.

Il existe effectivement des algorithmes de cette complexité. Elle est alors optimale.
\end{frame}

\begin{frame}
\frametitle{Complexité optimale d'un tri : démonstration}
Il existe $n!$ manières de permuter les valeurs et une seule conduit à une liste strictement croissante. Un algorithme de tri doit obtenir assez d'informations à partir des comparaisons qu'il effectue pour déterminer cette permutation.\newline
Si l'algorithme termine toujours après $c(n)$ comparaisons, il a pu distinguer $2^{c(n)}$ cas et ces cas doivent recouvrir toutes les permutations possibles soit
\begin{displaymath}
  n! \leq 2^{c(n)}\Rightarrow c(n)\geq \frac{1}{\ln 2} \ln(n!)
\end{displaymath}
Or $\ln(n!)\sim n\ln(n)$ donc $n\ln n \in O(c(n))$. 
\end{frame}

\begin{frame}
  \frametitle{Sommes, tranches et dominos}
La donnée est une liste $[a_0,a_1,\cdots,a_{n-1}]$ de $n$ nombres indexée à partir de $0$. 

Pour un entier $m$ tel que $0\leq m \leq n$, une \emph{tranche} de longueur $m$ est une liste de $m$ valeurs consécutives de la liste. 

Il existe $n-m$ tranches, on souhaite calculer les sommes de ces tranches et évaluer le nombre d'opérations (addition ou soustraction) à exécuter.
\end{frame}

\begin{frame}
  \frametitle{Sommes, tranches et dominos: algorithmes}
\begin{itemize}
  \item Algorithme naïf. Calcul direct des sommes.\newline Chacune est le résultat de $m-1$ additions,\newline $(n-m)(m-1)$ additions. 
  \item Algorithme amélioré. Prétraitement: calcul des sommes partielles
\begin{displaymath}
  A_i = a_0 + a_1 + \cdots + a_i
\end{displaymath}
On doit exécuter $n$ opérations car 
\begin{multline*}
A_{-1} = 0, A_0 = a_0, A_1 = A_0 + a_1,\\ A_2 = A_1 + a_2,\cdots A_n = A_{n-1} + a_n
\end{multline*}
Calcul de la somme d'une tranche, une seule soustraction
\begin{displaymath}
  a_k+\cdots+a_{k+m-1} = A_{k+m-1} - A_{k-1}
\end{displaymath}
Avec cet algorithme amélioré, seulement $2n-m$ opérations.
\end{itemize}
\end{frame}

\begin{frame}
  \frametitle{Tableau de décompte}
$m$ un entier naturel fixé. Les données sont des tableaux de nombres entre $0$ et $m$. On considère que la taille des données est la longueur $n$ des tableaux et que les instructions à compter sont les assignations et les additions.

Si $A$ est un tel tableau, on appelle \emph{tableau de décompte} de $A$, un tableau $C$ de longueur $m+1$ tel que: 
\begin{displaymath}
  \forall v\in \llbracket 0, m \rrbracket,\; C[v] = \text{nombre d'occurrences de $v$ dans $A$}
\end{displaymath}
Il est intéressant de remarquer que la somme des valeurs du tableau de décompte est la longueur $n$ du tableau donné $A$.
\end{frame}

\begin{frame}
  \frametitle{Tableau de décompte: implémentation Python}
\lstinputlisting[firstline=8, lastline=12]{complexite.py}
L'algorithme sous-jacent a exécuté $m+1+n$ instructions, il est linéaire en $n$.
\end{frame}

\begin{frame}
  \frametitle{Tableau de décompte: application au tri}
Plusieurs tableaux avec le même tableau de décompte. Un seul est croissant obtenu avec
\lstinputlisting[firstline=14, lastline=23]{complexite.py}
$2+2(m+1)+n$ instructions, complexité linéaire en $n$.

Algorithme linéaire de tri en contradiction avec la complexité optimale ?
\end{frame}

\begin{frame}
  \frametitle{\'Echanges de valeurs}
Deux listes $A$ et $B$ de $n$ nombres entiers naturels dans $\llbracket 0,m\rrbracket$.

Est-il est possible d'échanger deux valeurs pour que les listes aient la même somme? 

taille des données = longueur des listes, nombre d'additions.

$n^2$ couples de valeurs à échanger. 

Pour chaque couple: calcul des deux sommes et test d'égalité.

Chaque calcul de somme est de complexité linéaire

Pire des cas = pas d'échange satisfaisant : complexité en $O(n^3)$.
\end{frame}

\begin{frame}
  \frametitle{\'Echanges de valeurs: formulation}
$s_A$ et $s_B$ les deux sommes, échange possible si et seulement si 
\begin{multline*}
  \exists (i,j)\in \llbracket 0, n-1 \rrbracket \text{ tq } s_A + b_j -a_i = s_B + a_i - b_j\Leftrightarrow s_B - s_A = 2(b_j - a_i)\\
\Leftrightarrow b_j = a_i + \frac{s_B - s_A}{2}
\end{multline*}
$s_B - s_A$ doit être pair pour que l'échange soit possible.
\end{frame}

\begin{frame}
  \frametitle{\'Echanges de valeurs avec décompte préalable}
\begin{algorithm}[H]
  calcul de $S = s_B - s_A$\;
  \Si{$S$ impair}{
    renvoyer \texttt{false}\;
  }
  $S \leftarrow S/2$\;
  calcul de $C=$ tableau de décompte de $B$\;
  \PourCh{$i$ entre 0 et $n-1$}{
    $v \leftarrow A[i] + S$\;
    \Si{$C[v]>0$}{
      renvoyer \texttt{true}\;
    }
  }
  renvoyer \texttt{false}\;
  \caption{échange de valeurs}
  \label{complexite_3}
\end{algorithm}
Le prétraitement et le traitement sont linéaires, l'algorithme est de complexité $O(n)$.
\end{frame}

\begin{frame}
  \frametitle{Crible d'\'Eratosthène: liste des nombres premiers}
\lstinputlisting[firstline=25, lastline=36]{complexite.py}
\begin{itemize}
  \item Complexité est en $O(n\ln(\ln(n)))$: difficile.
  \item Complexité est $O(n\ln(n))$: trop grand mais facile.
  \item Pourquoi la boucle \texttt{while} interne démarre-t-elle à $k=p^2$?
\end{itemize}
\end{frame}

\begin{frame}
  \frametitle{Crible d'\'Eratosthène: tableau des plus petits diviseurs premiers}
Calcul d'un tableau $F$ qui ressemble au précédent. Au lieu de \texttt{False}, on place en $F[k]$ le plus petit diviseur premier de $k$.
\lstinputlisting[firstline=39, lastline=50]{complexite.py}
\end{frame}

\begin{frame}
  \frametitle{Crible d'\'Eratosthène: diviseurs premiers d'un nombre}
\`A l'aide de $F$ on forme la liste des facteurs premiers de $x\in  \llbracket 2,n\rrbracket$.
\begin{algorithm}[H]
  $facteursPrem \leftarrow$ liste vide \;
  \Tq{$F[x]>0$}{
    placer $F[x]$ à la fin de $facteursPrem$\;
    $x\leftarrow$ quotient de $x$ par $F[x]$
  }
  \Comment{Comme $F[x]=0$, maintenant $x$ est premier}
  placer $x$ à la fin de $facteursPrem$\;
  renvoyer $facteursPrem$ \;
  \caption{liste des facteurs premiers}
  \label{complexite_4}
\end{algorithm}
Le nombre $s$ de facteurs premiers de $x$ décroît strictement lors de la boucle, c'est une fonction de terminaison.\newline
\begin{displaymath}
  \text{nb prem.}\geq 2 \Rightarrow 2^s \leq x \Rightarrow s\leq \frac{\ln x}{\ln 2}
\end{displaymath}
La complexité de cette factorisation est donc en $O(\ln(n))$.
\end{frame}

\end{document}
