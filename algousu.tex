%!  pour pdfLatex
\documentclass[a4paper]{article}
%\usepackage[hmargin={1.5cm,1.5cm},vmargin={2.4cm,2.4cm},headheight=13.1pt]{geometry}
\usepackage[a4paper,landscape,twocolumn,
            hmargin=1.8cm,vmargin=2.2cm,headheight=13.1pt]{geometry}

\usepackage[pdftex]{graphicx,color}
\usepackage[pdftex,colorlinks={true},urlcolor={blue},pdfauthor={remy Nicolai}]{hyperref}

\usepackage[T1]{fontenc}
\usepackage[utf8]{inputenc}

\usepackage{lmodern}
\usepackage[frenchb]{babel}

\usepackage{fancyhdr}
\pagestyle{fancy}

\usepackage{floatflt}
\usepackage{maths}

\usepackage{parcolumns}
\setlength{\parindent}{0pt}

\usepackage{caption}
\usepackage{subcaption}

\usepackage{makeidx}

\usepackage[french,ruled,vlined]{algorithm2e}
\SetKwComment{Comment}{\#}{}
\SetKwFor{Tq}{tant que}{}{}
\SetKwFor{Pour}{pour}{}{}
\DontPrintSemicolon
\SetAlgoLined

\usepackage{listings}
\lstset{language=Python,frame=single}
\lstset{literate=
  {á}{{\'a}}1 {é}{{\'e}}1 {í}{{\'i}}1 {ó}{{\'o}}1 {ú}{{\'u}}1
  {Á}{{\'A}}1 {É}{{\'E}}1 {Í}{{\'I}}1 {Ó}{{\'O}}1 {Ú}{{\'U}}1
  {à}{{\`a}}1 {è}{{\`e}}1 {ì}{{\`i}}1 {ò}{{\`o}}1 {ù}{{\`u}}1
  {À}{{\`A}}1 {È}{{\'E}}1 {Ì}{{\`I}}1 {Ò}{{\`O}}1 {Ù}{{\`U}}1
  {ä}{{\"a}}1 {ë}{{\"e}}1 {ï}{{\"i}}1 {ö}{{\"o}}1 {ü}{{\"u}}1
  {Ä}{{\"A}}1 {Ë}{{\"E}}1 {Ï}{{\"I}}1 {Ö}{{\"O}}1 {Ü}{{\"U}}1
  {â}{{\^a}}1 {ê}{{\^e}}1 {î}{{\^i}}1 {ô}{{\^o}}1 {û}{{\^u}}1
  {Â}{{\^A}}1 {Ê}{{\^E}}1 {Î}{{\^I}}1 {Ô}{{\^O}}1 {Û}{{\^U}}1
  {œ}{{\oe}}1 {Œ}{{\OE}}1 {æ}{{\ae}}1 {Æ}{{\AE}}1 {ß}{{\ss}}1
  {ű}{{\H{u}}}1 {Ű}{{\H{U}}}1 {ő}{{\H{o}}}1 {Ő}{{\H{O}}}1
  {ç}{{\c c}}1 {Ç}{{\c C}}1 {ø}{{\o}}1 {å}{{\r a}}1 {Å}{{\r A}}1
  {€}{{\euro}}1 {£}{{\pounds}}1 {«}{{\guillemotleft}}1
  {»}{{\guillemotright}}1 {ñ}{{\~n}}1 {Ñ}{{\~N}}1 {¿}{{?`}}1
}

%pr{\'e}sentation des compteurs de section, ...
\makeatletter
\renewcommand{\thesection}{\Roman{section}.}
\renewcommand{\thesubsection}{\arabic{subsection}.}
\renewcommand{\thesubsubsection}{\arabic{subsubsection}.}
\renewcommand{\labelenumii}{\theenumii.}
\makeatother


\newtheorem*{thm}{Théorème}
\newtheorem{thmn}{Théorème}
\newtheorem*{prop}{Proposition}
\newtheorem{propn}{Proposition}
\newtheorem*{pa}{Présentation axiomatique}
\newtheorem*{propdef}{Proposition - Définition}
\newtheorem*{lem}{Lemme}
\newtheorem{lemn}{Lemme}

\theoremstyle{definition}
\newtheorem*{defi}{Définition}
\newtheorem*{nota}{Notation}
\newtheorem*{exple}{Exemple}
\newtheorem*{exples}{Exemples}


\newenvironment{demo}{\renewcommand{\proofname}{Preuve}\begin{proof}}{\end{proof}}
%\renewcommand{\proofname}{Preuve} doit etre après le begin{document} pour fonctionner

\theoremstyle{remark}
\newtheorem*{rem}{Remarque}
\newtheorem*{rems}{Remarques}

%\usepackage{maths}
%\newcommand{\dbf}{\leftrightarrows}

%En tete et pied de page
\lhead{Informatique}
%\chead{Introduction aux systèmes informatiques}
\rhead{MPSI B Hoche}
\lfoot{\tiny{Cette création est mise à disposition selon le Contrat\\ Paternité-Partage des Conditions Initiales à l'Identique 2.0 France\\ disponible en ligne http://creativecommons.org/licenses/by-sa/2.0/fr/  
} 
\rfoot{\tiny{Rémy Nicolai \jobname \; \today } }
}
\makeindex

%En tete et pied de page
\lhead{Cours IPT}
\chead{Cours 4 : Algorithmique - Algorithmes usuels le 7/11/16}
\begin{document}
On présente ici quelques problèmes simples et les algorithmes usuels qui les résolvent.\newline
Quand on cherche un algorithme répondant à un problème donné, on doit commencer par examiner si ce problème peut se ramener à un de ces problèmes (ou à une combinaison). Lorsque c'est le cas, il faut adapter ces algorithmes usuels.\newline
Il ne faut pas chercher à résoudre un problème plus difficile que celui qui vous est proposé. Comment préciser cette notion de difficulté?\newline
Imaginons deux problèmes $\mathcal{P}_1$ et $\mathcal{P}_2$. Si une solution de $\mathcal{P}_2$ vous permet d'obtenir facilement une solution de $\mathcal{P}_1$, alors on peut dire que $\mathcal{P}_2$ est \emph{plus difficile} que $\mathcal{P}_1$. Dans ces conditions, si $\mathcal{P}_1$ vous est proposé, commencer par vouloir résoudre $\mathcal{P}_2$ est une MAUVAISE ID\'EE qu'il faut en général rejeter.  Par exemple, si vous cherchez \emph{un} objet vérifiant certaines propriétés, vous ne devez pas chercher \emph{tous} les objets vérifiant cette propriété.
\section{Recherches dans une liste} 
\subsection{Problématique}
On se donne une liste d'objets (on la désigne par \texttt{L}) et un objet quelconque (désignons le par \verb|O|). L'objet \verb|O| est-il une valeur de \texttt{L}? Quel est le le premier indice pour lequel la valeur associée dans \texttt{L} est \verb|O|?\newline
Si les valeurs de \texttt{L} sont des nombres que l'on peut comparer, quelle est la plus grande valeur? Pour quels indices cette plus grande valeur est-elle atteinte?  Quelle est la valeur moyenne de la liste? Quelle est la variance de la liste?\newline
Si $L$ est une liste indexée de $0$ à $n-1$, les formules pour la moyenne et la variance sont les suivantes
\begin{displaymath}
  m = \frac{1}{n}\sum_{k=0}^{n-1}L[k], \hspace{0.5cm}
  v = \frac{1}{n}\sum_{k=0}^{n-1}(L[k]-m)^2
\end{displaymath}

\subsection{Pseudo-code}
L'algorithme \ref{algousu_1} présente la recherche d'une valeur dans une liste et l'algorithme \ref{algousu_2} la recherche de la plus grande valeur. 
\begin{algorithm}
  \Donnees{\;
     $ L \leftarrow$ une liste de $n$ objets indexée à partir de 0\;
     $O \leftarrow$ un objet}
  \Comment{initialisation}
  $i\leftarrow$ 0 \;
  \Tq{ i < n \rm et \it L[i] != O}{
      incrémenter $i$\;
  }
  \Comment{après la boucle, $i$ est inférieur ou égal à $n$}
  \Comment{Si $i<n$, il désigne l'indice cherché.}
  \Comment{Si $i=n$, l'objet ne figure pas dans la liste}
  \caption{Recherche dans une liste du premier indice associé à une valeur}
  \label{algousu_1}
\end{algorithm}
\begin{algorithm}
  \Donnees{\;
     $ L \leftarrow$ une liste (indexée de 0 à $n$) d'objets comparables}
  \Comment{initialisation}
  $i \leftarrow 0$\;
  $vmax\leftarrow L[i]$  \;
  \Comment{$vmax$ est la plus grande des valeurs de $L$ pour les indices $\leq i$}
  \Tq{ i < n-1}{
      incrémenter $i$\;
      \Si{vmax < L[i]}{vmax = L[i]}
      \Comment{$vmax$ est la plus grande des valeurs de $L$ pour les indices $\leq i$}
  }
  \Comment{$i$ désigne $n-1$ dernier indice}
  \Comment{$vmax$ désigne la plus grande valeur}
  \caption{Recherche dans une liste de la plus grande valeur}
  \label{algousu_2}
\end{algorithm}


\subsection{Exercices}
\begin{enumerate}
  \item Former un algorithme pour calculer la valeur moyenne ou la variance d'une liste de nombres.
  \item Former un algorithme pour calculer la liste de tous les indices correspondant à la plus grande valeur d'une liste donnée de nombres.
  \item Former un algorithme pour calculer la liste de tous les indices correspondant à la deuxième plus grande valeur d'une liste donnée de nombres.
\end{enumerate}

\section{Recherches par dichotomie}
\subsection{Problématique}
Les objets de la liste $L$ sont ici des nombres rangés par ordre croissant. Les recherches dans la liste sont facilitées.\newline
Soit $L$ une liste croissante (indexée à partir de 0) de $n$ nombres et $v$ un nombre vérifiant $L[0] \leq v < L[n-1]$. On cherche l'indice $i$ tel que $L[i]\leq v < L[i+1]$. On ne suppose pas que $L$ soit strictement croissante.\newline
Un autre problème est de chercher un encadrement $u\leq s < v$ d'une solution $s$ d'une équation $f(x)=0$ dans le cas où $f$ est une fonction strictement croissante sur un intervalle $[a,b]$ et vérifiant $f(a) < 0 < f(b)$. Il faut aussi se donner un nombre (petit) $\varepsilon$ caractérisant la précison demandée. On peut envisager deux possibilités pour le test de précison: soit $b-a < \varepsilon$ soit $|f(s)|< \varepsilon$.
\subsection{Pseudo-code}
L'algorithme \ref{algousu_3} permet d'encadrer un nombre donné par deux valeurs consécutives d'une suite croissante. Les commentaires justifient la correction de l'algorithme.
\begin{algorithm}
  \Donnees{\;
    $L \leftarrow$ une liste (indexée à partir de 0) croissante (pas forcément strictement) de $n$ nombres\;
    $v \leftarrow$ un nombre vérifiant $L[0]\leq v < L[n-1]$\;
  }
  \Comment{initialisation}
  $i\leftarrow 0; $ 
  $j\leftarrow n-1$\;
  \Comment{$L[i]\leq v < L[j]$ est vrai}
  \Comment{$k$ désignera un entier strictement entre $i$ et $j$}
  \Tq{ j-i > 1}{
    $k\leftarrow$ partie entière de $\frac{i+j}{2}$ \;
    \eSi{L[k] $\leq$ v}{
      $i\leftarrow k$ 
      \Comment{donc $L[i]\leq v < L[j]$ est vrai}
    }{
      $j\leftarrow k$
      \Comment{alors $L[i]\leq v < L[j]$ est vrai car $v < L[k]$}
    }
    \Comment{$L[i]\leq v < L[j]$ est vrai et $j-i$ a strictement décru}
  }
  \Comment{$i$ désigne l'indice cherché.}
  \caption{Encadrement par dichotomie dans une liste}
  \label{algousu_3}
\end{algorithm}

\subsection{Exercice}
Soit $f$ une fonction strictement croissante continue dans un intervalle $[a,b]$ avec $f(a)<0$ et $f(b)>0$, soit $\varepsilon>0$. On note $s$ l'unique réel entre $a$ et $b$ tel que $f(s)=0$. Former un algorithme calculant des réels $u$ et $v$ tels que $u\leq s < v$ avec $v-u < \varepsilon$.  

\section{Recherche d'un mot dans une chaîne}
\subsection{Problématique}
On dit aussi recherche de motif. On cherche les occurences d'un mot $M$ de longueur $m>0$ dans un texte $T$ de longueur $t$. Les chaînes de caractères sont indexées à partir de 0. Pour chaque indice $i$ entre $0$ et $t-m$, on note $T_i$ le mot dont les caractères sont $T[i] T[i+1] \cdots T[i+m-1]$. L'algorithme naïf de recherche de motif consiste simplement à comparer $M$ à chaque $T_i$. 
\subsection{Pseudo-code}
\begin{algorithm}
  \Donnees{\;
    $M \leftarrow$ une chaîne de $m$ caractères (mot)\;
    $T \leftarrow$ une chaîne de $t$ caractères (texte)\;
  }
  \Comment{initialisation}
  $O$ désigne la liste qui recevra les indices d'occurence du mot\;
  $i\leftarrow 0$ un indice pour le texte \;
  \Tq{i $\leq$ t - m}{
    \If{ M = $T_i$}{
      placer $i$ à la fin de la liste $O$\;
    }
    incrémenter $i$
  }
  \caption{Recherche de motif (algorithme naïf)}
  \label{algousu_4}
\end{algorithm}
Pour une implémentation en Python, il est utile de remarquer que $T_i$ s'obtient simplement avec la syntaxe \verb|T[i:i+m]|. Cette syntaxe est désignée par le terme \emph{slice} dans la documentation.

Il existe de nombreux algorithmes de recherche de motifs\footnote{par exemple algorithmes de Rabin-Karp, utilisant des automates finis ou de Knuth-Morris-Pratt (tirés de \emph{Introduction à l'algorithmique} Dunod)}. Une des idées pour améliorer l'algorithme \ref{algousu_4} (naïf) est d'utiliser le travail fourni en comparant $M$ à $T_i$ pour incrémenter $i$ plus efficacement afin d'éviter des comparaisons dont on peut savoir à l'avance qu'elles échoueront.\newline
Comparons $M$ et $T_i$ en appelant une fonction $NbCarComm(M,T_i)$ qui renvoie le nombre de caractères (entre 0 et $m$) que les deux mots ont en commun (au début des mots).\newline
Par exemple, si $T_i$ désigne \verb|tstt-tsst| et $M$ désigne \verb|tsoin|, la fonction renverra $2$ car seuls les deux premiers caractères \verb|t| et \verb|s| sont communs. Les mots sont donc égaux si et seulement si la fonction renvoie $m$.\newline
Plaçons nous maintenant dans le cas particulier où les caractères de $M$ sont deux à deux distincts.\newline
Si les $p$ premiers caractères de $M$ et $T_i$ sont égaux avec $p<m$, comme les caractères de $M$ sont deux à deux distincts, il est certain que le premier caractère des mots $T_{i+1}, T_{i+2},\cdots T_{i+p-1}$ est différent du premier caractère de $M$ puisqu'il s'agit d'un des caractères suivants de $M$.\newline
On peut donc modifier l'algorithme précédent en formant les algorithmes \ref{algousu_5} et \ref{algousu_6}.
\begin{algorithm}
  \Donnees{\;
    $M \leftarrow$ une chaîne de $m$ caractères\;
    $T \leftarrow$ une chaîne de $m$ caractères\;
  }
  \Comment{initialisation}
  $i\leftarrow 0$ (nombre de premiers indices avec les mêmes valeurs)\;
  \Tq{ i < m et M[i] = T[i]}{
    incrémenter $i$\;
  }
  renvoyer $i$\;
  \caption{Fonction NbCarComm}
  \label{algousu_5}
\end{algorithm}
\begin{algorithm}
  \Donnees{\;
    $M \leftarrow$ une chaîne de $m$ caractères (mot) sans répétition\;
    $T \leftarrow$ une chaîne de $t$ caractères (texte)\;
  }
  \Comment{initialisation}
  $O$ désigne la liste qui recevra les indices d'occurence du mot\;
  $i\leftarrow 0$ un indice pour le texte \;
  $NbCarComm \leftarrow$ une fonction comme décrite plus haut\;
  \Tq{i $\leq$ t - m}{
    $p \leftarrow NbCarComm(M , T_i)$\; 
    \If{ p = m}{
      \Comment{dans ce cas $M=T_i$}
      placer $i$ à la fin de la liste $O$\;
    }
    \Comment{incrémenter $i$}
    $i \leftarrow i + \max(1,p) $\;
  }
  \caption{Recherche d'un motif sans répétition}
  \label{algousu_6}
\end{algorithm}

\section{Autres exercices: fusion de listes croissantes}
\begin{enumerate}
  \item (dans une troisième liste) On se donne deux listes (nommées $L1$ et $L2$) croissantes de nombres. Former un algorithme calculant une troisième liste $L$ constituée des valeurs de $L1$ et $L2$ rangées par ordre croissant.
  
  \item (en place) On se donne une seule liste $L$ indexée de $id$ (i début) à $if$ (i final) et contenant deux sous-listes croissantes. C'est à dire qu'il existe un indice $im$ tel que $id \leq im < if$ avec $L$ soit croissante de $id$ à $im -1$ (si $id < im$) et de $im$ à $if$. Il s'agit cette fois de former un algorithme modifiant $L$ pour que ses valeurs soient rangées par ordre croissant de $id$ à $if$.
\end{enumerate}

\end{document}
