%!  pour pdfLatex
\documentclass{beamer}

%\usepackage[pdftex]{graphicx,color}
%\usepackage[pdftex,colorlinks={true},urlcolor={blue},pdfauthor={remy Nicolai}]{hyperref}

\usepackage[utf8]{inputenc}
\usepackage[T1]{fontenc}
\usepackage{lmodern}
\usepackage[frenchb]{babel}

\usetheme{Warsaw}

%\usepackage{fancyhdr}
%\pagestyle{fancy}

%\usepackage{floatflt}
\usepackage{maths}

\usepackage{parcolumns}
\setlength{\parindent}{0pt}

%\usepackage{caption}
%\usepackage{subcaption}

\usepackage[french,ruled,vlined]{algorithm2e}
\SetKwComment{Comment}{\#}{}
\SetKwFor{Tq}{tant que}{}{}
\SetKwFor{Pour}{pour}{}{}
\DontPrintSemicolon
\SetAlgoLined

\usepackage{listings}
\lstset{language=Python,frame=single}

%pr{\'e}sentation des compteurs de section, ...
\makeatletter
\renewcommand{\thesection}{\Roman{section}.}
\renewcommand{\thesubsection}{\arabic{subsection}.}
\renewcommand{\thesubsubsection}{\arabic{subsubsection}.}
%\renewcommand{\labelenumii}{\theenumii.}
\makeatother


\newtheorem*{thm}{Théorème}
\newtheorem{thmn}{Théorème}
\newtheorem*{prop}{Proposition}
\newtheorem{propn}{Proposition}
\newtheorem*{pa}{Présentation axiomatique}
\newtheorem*{propdef}{Proposition - Définition}
\newtheorem*{lem}{Lemme}
\newtheorem{lemn}{Lemme}

\theoremstyle{definition}
\newtheorem*{defi}{Définition}
\newtheorem*{nota}{Notation}
\newtheorem*{exple}{Exemple}
\newtheorem*{exples}{Exemples}


\newenvironment{demo}{\renewcommand{\proofname}{Preuve}\begin{proof}}{\end{proof}}
%\renewcommand{\proofname}{Preuve} doit etre après le begin{document} pour fonctionner

\theoremstyle{remark}
\newtheorem*{rem}{Remarque}
\newtheorem*{rems}{Remarques}

%\usepackage{maths}
%\newcommand{\dbf}{\leftrightarrows}

%En tete et pied de page
%\lhead{Informatique}
%\chead{Introduction aux systèmes informatiques}
%\rhead{MPSI B Hoche}
%\lfoot{\tiny{Cette création est mise à disposition selon le Contrat\\ Paternité-Partage des Conditions Initiales à l'Identique 2.0 France\\ disponible en ligne http://creativecommons.org/licenses/by-sa/2.0/fr/
%} }
%\rfoot{\tiny{Rémy Nicolai \jobname}}

\nonstopmode

\begin{document}

\begin{frame}
  \frametitle{Morceau de code}
  Que fait ce morceau de code ?
\lstinputlisting[firstline=1, lastline=8]{exporap.py}
\end{frame}

\begin{frame}
  \frametitle{Développement binaire d'un entier}
\begin{itemize}
  \item Il met en \oe{}uvre l'algorithme de calcul du développement binaire d'un nombre.
  \item Il affiche une chaine de caractère présentant la décomposition en base $2$ du nombre désigné par \texttt{nini}.
\end{itemize}
En exécutant le code, le mot \texttt{11001} s'affiche ce qui traduit
\begin{displaymath}
 25 =  1\times 2^4 + 1\times 2^3 + 0\times 2^2 + 0\times 2^1 + 1\times 2^0  
\end{displaymath}
\end{frame}

\begin{frame}
  \frametitle{Variante}
Que fait ce code ?
\lstinputlisting[firstline=11, lastline=17]{exporap.py}
\end{frame}

\begin{frame}
  \frametitle{Place des bits égaux à 1}
Il affiche seulement les exposants du développement pour lesquels le coefficient est $1$ : soit $0$, $3$, $4$.
\end{frame}

\begin{frame}
  \frametitle{Autre variante}
\lstinputlisting[firstline=20, lastline=27]{exporap.py}  
\begin{itemize}
  \item Que désigne le nom \texttt{p} lors du déroulement de l'exécution?
  \item Montrer que la valeur de \texttt{x} affichée à la fin est $7^{250}$.
  \item Modifier le code précédent en introduisant un compteur \texttt{c} permettant de connaître et d'afficher le nombre de multiplications réellement exécutées.
\end{itemize}
\end{frame}

\begin{frame}
  \frametitle{Puissances de $a$}
\begin{itemize}
  \item Le troisième bloc dérive toujours de l'algorithme de numération avec la variante du second bloc.
  \item Le nom \texttt{a} désigne toujours le même nombre (ici $7$).
  \item Le nom \texttt{p} est initialisé à $a$ . Il est à chaque fois élevé au carré. La suite de ses valeurs est donc
\begin{displaymath}
 a=a^{1}=a^{2^{0}}, a^2=a^{2^{1}}, a^4=a^{2^{2}}, \cdots , a^{2^{k}}
\end{displaymath}
car 
\begin{displaymath}
 a^{2^{k-1}}a^{2^{k-1}}= a^{2^{k-1} + 2^{k-1}}=a^{2^{k}}
\end{displaymath}
\end{itemize}
\texttt{p} désigne les puissances de $a$ avec des exposants qui sont les puissances de $2$.
\end{frame}

\begin{frame}
  \frametitle{Calcul de $7^{250}$}
Considérons la décomposition binaire de l'exposant $n$
\begin{eqnarray*}
 n &=& c_0 + c_1 2^1 + c_2 2^2+ \cdots +c_m 2^m \text{ avec les } c_k\in\{0,1\}\\
 a^n &=& a^{c_0}a^{c_1 2^1} \cdots a^{c_m 2^m} \\
 a^n &=& a^{2^{i_1}} a^{2^{i_2}} ...
\end{eqnarray*}
où les $i_1, i_2, \cdots$ sont les exposants pour lesquels les coefficients de la décomposition sont égaux à $1$.

Le code :\begin{itemize}
  \item calcule toutes les puissances d'exposant une puissance de $2$,
  \item les multiplie quand il faut à une variable \texttt{x} initialisée à $1$.
\end{itemize}
\`A la fin, le nom \texttt{x} désigne donc bien $a^n$.
\end{frame}

\begin{frame}
  \frametitle{Exponentiation rapide}
\lstinputlisting[firstline=31, lastline=38]{exporap.py}  
Il suffit de 15 multiplications pour calculer $7^{250}$.
\end{frame}

\end{document}