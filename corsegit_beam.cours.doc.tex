%!  pour pdfLatex
\documentclass{beamer}

%\usepackage[pdftex]{graphicx,color}
%\usepackage[pdftex,colorlinks={true},urlcolor={blue},pdfauthor={remy Nicolai}]{hyperref}

\usepackage[utf8]{inputenc}
\usepackage[T1]{fontenc}
\usepackage{lmodern}
\usepackage[frenchb]{babel}

\usetheme{Warsaw}

%\usepackage{fancyhdr}
%\pagestyle{fancy}

%\usepackage{floatflt}
\usepackage{maths}

\usepackage{parcolumns}
\setlength{\parindent}{0pt}

%\usepackage{caption}
%\usepackage{subcaption}

\usepackage[french,ruled,vlined]{algorithm2e}
\SetKwComment{Comment}{\#}{}
\SetKwFor{Tq}{tant que}{}{}
\SetKwFor{Pour}{pour}{}{}
\DontPrintSemicolon
\SetAlgoLined

\usepackage{listings}
\lstset{language=Python,frame=single}

%pr{\'e}sentation des compteurs de section, ...
\makeatletter
\renewcommand{\thesection}{\Roman{section}.}
\renewcommand{\thesubsection}{\arabic{subsection}.}
\renewcommand{\thesubsubsection}{\arabic{subsubsection}.}
%\renewcommand{\labelenumii}{\theenumii.}
\makeatother


\newtheorem*{thm}{Théorème}
\newtheorem{thmn}{Théorème}
\newtheorem*{prop}{Proposition}
\newtheorem{propn}{Proposition}
\newtheorem*{pa}{Présentation axiomatique}
\newtheorem*{propdef}{Proposition - Définition}
\newtheorem*{lem}{Lemme}
\newtheorem{lemn}{Lemme}

\theoremstyle{definition}
\newtheorem*{defi}{Définition}
\newtheorem*{nota}{Notation}
\newtheorem*{exple}{Exemple}
\newtheorem*{exples}{Exemples}


\newenvironment{demo}{\renewcommand{\proofname}{Preuve}\begin{proof}}{\end{proof}}
%\renewcommand{\proofname}{Preuve} doit etre après le begin{document} pour fonctionner

\theoremstyle{remark}
\newtheorem*{rem}{Remarque}
\newtheorem*{rems}{Remarques}

%\usepackage{maths}
%\newcommand{\dbf}{\leftrightarrows}

%En tete et pied de page
%\lhead{Informatique}
%\chead{Introduction aux systèmes informatiques}
%\rhead{MPSI B Hoche}
%\lfoot{\tiny{Cette création est mise à disposition selon le Contrat\\ Paternité-Partage des Conditions Initiales à l'Identique 2.0 France\\ disponible en ligne http://creativecommons.org/licenses/by-sa/2.0/fr/
%} }
%\rfoot{\tiny{Rémy Nicolai \jobname}}

\nonstopmode

\begin{document}

\begin{frame}
  \frametitle{Preuve d'un segment itératif}
Un \emph{segment itératif} est une boucle \og\texttt{tant que}\fg.

Le segment est\og correct\fg~lorsque :
\begin{itemize}
  \item il se termine
  \item il fait ce qu'il doit faire
\end{itemize}
Comment prouver qu'un segment est correct?
\end{frame}

\begin{frame}
  \frametitle{Modèle logique de correction}
\begin{algorithm}[H]
  \Tq{ Condition}{
      instruction 1\;
      $\vdots $\;
      instruction p\;
  }
  \Comment{Après la boucle une propriété $\mathcal{P}$ doit être vraie.}
  \caption{Un segment itératif}
  \label{corsegit_1}
\end{algorithm}
\begin{itemize}
  \item $Condition$ doit s'évaluer à \texttt{False} après un nombre fini d'exécutions de la boucle.
  \item Lorsque $Condition$ s'évalue à \texttt{False}, la propriété $\mathcal{P}$ doit s'évaluer à \texttt{True}.
\end{itemize}
\end{frame}

\begin{frame}
  \frametitle{Fonction de terminaison : définition}
\begin{defi}
Une \emph{fonction de terminaison}, on dit aussi \emph{variant de boucle} est une expression telle que :
\begin{itemize}                                                                                                                                                                                                                                               \item l'expression s'évalue comme un entier naturel
\item la valeur que prend l'expression \emph{diminue strictement} à chaque exécution de la séquence.                                                                                                                                                                                                                             \end{itemize}
\end{defi}
\end{frame}

\begin{frame}
  \frametitle{Fonction de terminaison : utilisation}
\begin{prop}
  Si un segment itératif admet une fonction de terminaison de valeur initiale $m$, il se termine en au plus $m$ itérations.
\end{prop}
\begin{itemize}
  \item Soit $A$ ensemble des valeurs prises par la fonction.
  \item $A$ est une partie de $\N$ majorée par $m$ donc finie; soit $n$ son nombre d'éléments.
  \item Après $n-1$ exécutions exactement, on sort de la boucle.
  \item $n\leq m+1 \Rightarrow n-1 \leq m$
\end{itemize}
\end{frame}

\begin{frame}
  \frametitle{Invariant de boucle : définition}
\begin{defi}
  Un \emph{invariant de boucle} est une expression vérifiant les propriétés suivantes:
\begin{itemize}
  \item à valeurs booléennes (on dit que c'est une propriété),
  \item sa valeur est le même après chaque itération
\end{itemize}
\end{defi}
En principe sa valeur intiale est \texttt{True}.
\end{frame}

\begin{frame}
  \frametitle{Invariant de boucle : utilisation}
\begin{prop}
S'il existe une fonction de terminaison $V$ et un invariant $\Phi$ tels que
\begin{displaymath}
  \left( \left( \text{ non } Condition\right) \text{ et } \Phi \right) \Rightarrow \mathcal{P}
\end{displaymath}
alors le segment itératif est correct.
\end{prop}
\end{frame}

\begin{frame}
  \frametitle{Exemple 1}
\begin{algorithm}[H]
  \Comment{Initialisation}
  $a$, $b$, $c$ désignent des entiers\;
  $a\longleftarrow$ un entier naturel\;
  $b\longleftarrow a$\;
  $c\longleftarrow 0$\;
  \Tq{ b > 0}{
      $c\longleftarrow c + a$\;
      $b\longleftarrow b - 1$\;
  }
  \Comment{fin du segment itératif}
  afficher $c$\;
  \caption{Calcul du carré d'un entier.}
  \label{corsegit_2}
\end{algorithm}
\end{frame}

\begin{frame}
  \frametitle{Exemple 1: preuves}
\begin{itemize}
  \item fonction de terminaison : $b$
  \item invariant : propriété $a^2 = c +b*a$
\end{itemize}
Preuve de l'invariance.\newline
Après $p$ séquences, $b$ désigne $b_p$ et $c$ désigne $c_p$.\newline
Après $p+1$ séquences : $b$ désigne $b_p-1$ et $c$ désigne $c_p+a$ donc $c+b*a$ désigne
  \begin{displaymath}
   c_p + a + (b_p-1)*a = c_p + b_p*a = a^2 
  \end{displaymath}
Preuve de l'algorithme.\newline
Comme il existe une fonction de terminaison, la boucle se termine avec $b$ désignant $0$ donc $c+b*a$ s'évaluant à $c$.
\`A cause de l'invariant $c$ désigne bien $a^2$.
\end{frame}

\begin{frame}
  \frametitle{Exemple 2.}
\begin{algorithm}[H]
  \Comment{Initialisation}
  $a$, $n$, $r$ désignent des entiers\;
  $a\longleftarrow$ un entier naturel non nul\;
  $n\longleftarrow$ un entier naturel non nul\;
  $r\longleftarrow 1$\;
  \Tq{ n > 0}{
      $r\longleftarrow r * a$\;
      $n\longleftarrow n - 1$\;
  }
  \Comment{fin du segment itératif}
  afficher $r$\;
  \caption{Calcul d'une puissance.}
\end{algorithm}
\begin{itemize}
  \item fonction de terminaison $n$
  \item invariant $\Phi = r * a^n$
\end{itemize}
\end{frame}

\begin{frame}
 \frametitle{Calcul rapide d'une puissnce}
On veut maintenant former un nouvel algorithme de calcul de puissance dans lequel cette fois les deux noms $a$ et $n$ vont désigner des nombres différents lors de l'évolution.\newline
On s'assure de la correction avec les mêmes variant et invariant.
\[
 V = n, \hspace{0.5cm} \Phi = \left( r * a^n == {a_{\text{ini}}}^{n_{\text{ini}}}\right)
\]
On discute selon la parité de $n$.
\[
 \text{Si } n = 2p + 1,\; a^n = a \,(a^2)^p ; \hspace{1cm} \text{Si } n = 2p,\; a^n = (a^2)^p.
\] 
\end{frame}

\begin{frame}
 \frametitle{Algorithme d'exponentiation rapide}
\begin{algorithm}[H]
  \Comment{Initialisation}
  $a$, $n$, $r$ désignent des entiers\;
  $a\longleftarrow$ un entier naturel non nul $a_{\text{ini}}$\;
  $n\longleftarrow$ un entier naturel non nul $n_{\text{ini}}$\;
  $r\longleftarrow 1$\;
  \Tq{ n > 0}{
      \eSi{$n$ pair}{
          $a\longleftarrow a * a$\;
          $n\longleftarrow n // 2$\;
      }{
          $r\longleftarrow r * a$\;
          $n\longleftarrow n -1$\;
      }
  }
  \Comment{fin du segment itératif}
  afficher $r$\;
\end{algorithm}
\end{frame}


\begin{frame}
  \frametitle{Exemple 3.}
\begin{algorithm}[H]
  l $\leftarrow$ longueur de L\;
  ligv $\leftarrow$ liste des entiers de 0 à 9\;
  actualiser  imin\;
  j $\leftarrow 10$\;
  \Tq{j < l}{
      \If{ L[ligv[imin]] < L[j] }{
          ligv[imin] $\leftarrow$ j\;
          actualiser imin\;
      }
      incrémenter \texttt{j}
  }
  \caption{liste de 10 indices des plus grandes valeurs}
\end{algorithm}
fonction de terminaison $V = l-j$ et invariant
\begin{displaymath}
  \Phi = 
\left( 
\left. 
\begin{aligned}
  &u \text{ et } v \leq j\\
  &u \text{ n'est pas une valeur de }ligv \\
  &v \text{ est une valeur de }ligv 
\end{aligned}
\right\rbrace 
\Rightarrow L[u] \leq L[v]
\right) 
\end{displaymath}
\end{frame}

\begin{frame}
  \frametitle{Exemple 4.}
\begin{algorithm}[H]
  \Comment{Initialisations}
  $a\longleftarrow$ un entier naturel non nul\;
  $b\longleftarrow$ un entier naturel non nul\;
  $u\longleftarrow a$\;
  $v\longleftarrow b$\;
  \Tq{ u != v}{
      \eIf{u > v}{
        $u\longleftarrow u - v$\;
      }{
        $v\longleftarrow v - u$\;
      }
  }
  afficher $u$\;
  \caption{Calcul d'un pgcd.}
\end{algorithm}
fonction de terminaison $V = u + v$ et invariant 
\begin{displaymath}
 \Phi = \left( \text{pgcd}(a,b) == \text{pgcd}(u,v)\right) 
\end{displaymath}
\end{frame}

\begin{frame}
  \frametitle{Exemple 5. Présentation d'un problème de \og complétion\fg~}
\begin{itemize}
  \item Une relation $\mathcal{R}$ est définie dans un ensemble fini $\Omega$ contenant $n$ éléments.
  \item Pour chaque élément $e\in \Omega$, on désigne par $V(e)$ l'ensemble des $f\in \Omega$ qui sont en relation avec $e$.
  \item Une partie $A$ de $\Omega$ est donnée.
\end{itemize}
 On cherche une partie $B$ de $\Omega$ contenant $A$ et vérifiant la propriété
\begin{displaymath}
  \mathcal{P}: \hspace{0.5cm} \forall b \in B, V(b) \subset B
\end{displaymath}
On pourrait prendre $B=\Omega$ mais on veut $B$ aussi petit que possible.
\end{frame}

\begin{frame}
  \frametitle{Exemple 5. Algorithme}
\begin{algorithm}[H]
  \Comment{Initialisation}
  $B \leftarrow A$ liste donnée d'éléments 2 à 2 distincts de $\Omega$\;
  $i \leftarrow 0$\;
  \Tq{ $i < $ longueur de $B$}{
     $e \leftarrow $ valeur d'indice $i$ de $B$\;
     \PourCh{$f \in \Omega$ en relation avec $e$}{
       \Si{$f$ n'est pas dans $B$}{
          placer $f$ à la fin de $B$\;
       }
     }
     incrémenter $i$\;
  }
  \caption{Complété d'une partie pour une relation}
\end{algorithm}
\end{frame}

\begin{frame}
  \frametitle{Exemple 5. Preuve}
\begin{itemize}
  \item deux boucles imbriquées. Terminaisons ?
  \begin{itemize}
    \item boucle interne: l'ensemble des éléments en relation avec un élément donné est fini.
    \item boucle externe: ???? car la liste $B$ est modifiée \emph{à l'intérieur de la boucle}.
  \end{itemize}
  \item fonction de terminaison: $n-i$. (décroissance facile)
  \item invariant1: \og les valeurs de $B$ sont deux à deux distinctes\fg~
  \item invariant2:
\begin{displaymath}
  \bigcup_{k< i}V(b_k) \subset \left\lbrace \text{ valeurs de }B \right\rbrace 
\end{displaymath}

\end{itemize}
L'invariant 1 prouve $n-i \geq 0$ donc c'est bien une fonction de terminaison. L'invariant 2 prouve la complétion.
\end{frame}


\end{document}
