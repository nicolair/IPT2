Le premier exercice est de maitriser l'arborescence de fichiers dans laquelle vous travaillez. Vous devez créer un dossier dans lequel vous allez placer les fichiers python, il comportera éventuellement des sous-dossiers. \newline
Si vous utilisez votre machine personnelle, vous le placez où vous voulez. Si vous utilisez les machines du lycée, trouvez ou créez dans \og Mes documents\fg~ un sous-dossier pour la classe nommé \og mpsiB\_18\fg~ dans lequel vous créerez un sous-dossier à votre nom.\newline
Le réseau entre les stations de travail des salles de TP est mal commode. Pour communiquer entre elles ou entre votre station de travail et une autre ou une machine perso externe, il est plus facile de passer par par une clé usb (mais attention aux virus) ou un hébergeur web du type Google Drive ou Dropbox.