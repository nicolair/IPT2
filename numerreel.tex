%<dscrpt>Introduction à l'algorithmique</dscrpt>
% <rel_id_elt_parent>6469</rel_id_elt_parent> id de ``Algorithmique''
% <rel_id_type_rel>14</rel_id_type_rel> id de ``contient sana ordre l'élément''
%!  pour pdfLatex
\documentclass[a4paper]{article}
%\usepackage[hmargin={1.5cm,1.5cm},vmargin={2.4cm,2.4cm},headheight=13.1pt]{geometry}
\usepackage[a4paper,landscape,twocolumn,
            hmargin=1.8cm,vmargin=2.2cm,headheight=13.1pt]{geometry}

\usepackage[pdftex]{graphicx,color}
\usepackage[pdftex,colorlinks={true},urlcolor={blue},pdfauthor={remy Nicolai}]{hyperref}

\usepackage[T1]{fontenc}
\usepackage[utf8]{inputenc}

\usepackage{lmodern}
\usepackage[frenchb]{babel}

\usepackage{fancyhdr}
\pagestyle{fancy}

\usepackage{floatflt}
\usepackage{maths}

\usepackage{parcolumns}
\setlength{\parindent}{0pt}

\usepackage{caption}
\usepackage{subcaption}

\usepackage{makeidx}

\usepackage[french,ruled,vlined]{algorithm2e}
\SetKwComment{Comment}{\#}{}
\SetKwFor{Tq}{tant que}{}{}
\SetKwFor{Pour}{pour}{}{}
\DontPrintSemicolon
\SetAlgoLined

\usepackage{listings}
\lstset{language=Python,frame=single}
\lstset{literate=
  {á}{{\'a}}1 {é}{{\'e}}1 {í}{{\'i}}1 {ó}{{\'o}}1 {ú}{{\'u}}1
  {Á}{{\'A}}1 {É}{{\'E}}1 {Í}{{\'I}}1 {Ó}{{\'O}}1 {Ú}{{\'U}}1
  {à}{{\`a}}1 {è}{{\`e}}1 {ì}{{\`i}}1 {ò}{{\`o}}1 {ù}{{\`u}}1
  {À}{{\`A}}1 {È}{{\'E}}1 {Ì}{{\`I}}1 {Ò}{{\`O}}1 {Ù}{{\`U}}1
  {ä}{{\"a}}1 {ë}{{\"e}}1 {ï}{{\"i}}1 {ö}{{\"o}}1 {ü}{{\"u}}1
  {Ä}{{\"A}}1 {Ë}{{\"E}}1 {Ï}{{\"I}}1 {Ö}{{\"O}}1 {Ü}{{\"U}}1
  {â}{{\^a}}1 {ê}{{\^e}}1 {î}{{\^i}}1 {ô}{{\^o}}1 {û}{{\^u}}1
  {Â}{{\^A}}1 {Ê}{{\^E}}1 {Î}{{\^I}}1 {Ô}{{\^O}}1 {Û}{{\^U}}1
  {œ}{{\oe}}1 {Œ}{{\OE}}1 {æ}{{\ae}}1 {Æ}{{\AE}}1 {ß}{{\ss}}1
  {ű}{{\H{u}}}1 {Ű}{{\H{U}}}1 {ő}{{\H{o}}}1 {Ő}{{\H{O}}}1
  {ç}{{\c c}}1 {Ç}{{\c C}}1 {ø}{{\o}}1 {å}{{\r a}}1 {Å}{{\r A}}1
  {€}{{\euro}}1 {£}{{\pounds}}1 {«}{{\guillemotleft}}1
  {»}{{\guillemotright}}1 {ñ}{{\~n}}1 {Ñ}{{\~N}}1 {¿}{{?`}}1
}

%pr{\'e}sentation des compteurs de section, ...
\makeatletter
\renewcommand{\thesection}{\Roman{section}.}
\renewcommand{\thesubsection}{\arabic{subsection}.}
\renewcommand{\thesubsubsection}{\arabic{subsubsection}.}
\renewcommand{\labelenumii}{\theenumii.}
\makeatother


\newtheorem*{thm}{Théorème}
\newtheorem{thmn}{Théorème}
\newtheorem*{prop}{Proposition}
\newtheorem{propn}{Proposition}
\newtheorem*{pa}{Présentation axiomatique}
\newtheorem*{propdef}{Proposition - Définition}
\newtheorem*{lem}{Lemme}
\newtheorem{lemn}{Lemme}

\theoremstyle{definition}
\newtheorem*{defi}{Définition}
\newtheorem*{nota}{Notation}
\newtheorem*{exple}{Exemple}
\newtheorem*{exples}{Exemples}


\newenvironment{demo}{\renewcommand{\proofname}{Preuve}\begin{proof}}{\end{proof}}
%\renewcommand{\proofname}{Preuve} doit etre après le begin{document} pour fonctionner

\theoremstyle{remark}
\newtheorem*{rem}{Remarque}
\newtheorem*{rems}{Remarques}

%\usepackage{maths}
%\newcommand{\dbf}{\leftrightarrows}

%En tete et pied de page
\lhead{Informatique}
%\chead{Introduction aux systèmes informatiques}
\rhead{MPSI B Hoche}
\lfoot{\tiny{Cette création est mise à disposition selon le Contrat\\ Paternité-Partage des Conditions Initiales à l'Identique 2.0 France\\ disponible en ligne http://creativecommons.org/licenses/by-sa/2.0/fr/  
} 
\rfoot{\tiny{Rémy Nicolai \jobname \; \today } }
}
\makeindex


%En tete et pied de page
\lhead{Cours IPT}
\chead{Cours 4. Développement d'un nombre réel dans une base le 7/11/16}

\begin{document}
\section{Partie entière et division}
On a déjà vu que tout nombre entier naturel se décompose de manière unique dans une base $b\geq2$ quelconque. Pratiquement, cette décomposition peut se faire en utilisant deux algorithmes: \og les petits d'abord \fg~ ou \og les grands d'abord \fg. On peut étendre ces principes des entiers aux réels.

D'après une propriété de $\R$, tout nombre réel se décompose de manière unique sous la forme
\begin{displaymath}
  \text{un nombre réel} = \text{un nombre entier} + \text{un nombre réel dans $[0,1[$}
\end{displaymath}
Notation
\begin{displaymath}
  x = \underset{\text{partie entière de } x\in \Z }{\underbrace{\lfloor x \rfloor}} + \underset{\text{partie décimale de } x\in [0,1[ }{\underbrace{\{ x \}}}
\end{displaymath}
Comme $\Q$ est une partie de $\R$, la partie entière rend compte de la division euclidienne arithmétique\newline
Dans la division de $m\in \Z$ par $n\in \N^*$ 
\begin{displaymath}
  m = \underset{\in \Z}{\underbrace{n}}q + \underset{\in \llbracket 0, n\llbracket}{\underbrace{r}}
  \Leftrightarrow \frac{m}{n} = \underset{\in \N}{\underbrace{q}} + \underset{\in [0,1[}{\underbrace{\frac{r}{n}}} 
\end{displaymath}
On en déduit que le quotient de la division de $m$ par $n$ est $\lfloor \frac{m}{n} \rfloor$.
\section{Développement pratique d'un réel}
\begin{algorithm}
  $y\longleftarrow \lfloor x \rfloor$\;
  \Tq{ $y > 0$}{
      enregistrer le reste de la division de $y$ par $b$\;
      $y\longleftarrow $ le quotient de la division de $y$ par $b$\;
  }
  \caption{\`A gauche: les petits d'abord.}
  \label{nbbin_1}
\end{algorithm}

\begin{algorithm}
  $y\longleftarrow \{ x \}$\;
  \Tq{ $y > 0$}{
      enregistrer $\lfloor by \rfloor$\;
      $y\longleftarrow \{ by\}$\;
  }
  \caption{\`A droite: les grands d'abord.}
  \label{nbbin_2}
\end{algorithm}
Soit $b$ une base fixée (entier naturel strictement plus grand que $1$, le développement d'un nombre réel strictement positif $x$  s'effectue de la manière suivante :
\begin{itemize}
  \item $x = \lfloor x \rfloor + \{x\}$
  \item à gauche : \og les petits d'abord \fg appliqué = $\lfloor x \rfloor$.
  \item à droite : \og les grands d'abord \fg appliqué = $\{ x \}$.
\end{itemize}

Dans l'algorithme \ref{nbbin_1}, on peut remarquer que, à chaque itération, le nombre entier naturel désigné par $y$ diminue strictement ce qui assure que la boucle se termine. En revanche dans l'algorithme \ref{nbbin_2}, $y$ désigne un nombre réel dans $[0,1[$ dont on ne sait pas grand-chose. Rien ne permet d'affirmer qu'il devienne nul et que l'on sorte de la boucle. 

Le développement en base $b$ d'un nombre réel $x>0$ conduit donc à une suite $\left( a_n\right)_{n\leq m_x}$ d'éléments de $\llbracket 0 , b \llbracket$ où $m_x$ est un entier relatif. Les coefficients d'indices positifs correspondent au développement de la partie entière.


Exemple: développement de $7.3$ en base $2$. on trouve $111.01001\,1001\,1001\,\cdots$.

\section{Questions mathématiques}
La partie gauche du développement précédent correspond au développement en base $b$ de la partie entière.

En adoptant une notation séquentielle, le développement pratique (partie droite) détaillé plus haut s'écrit
\begin{displaymath}
\left. 
\begin{aligned}
  a_{n+1} &= \lfloor by_n\rfloor \\ y_{n+1} &= \{ny_n\}
\end{aligned}
\right\rbrace
\Rightarrow by_n = a_{n+1} + y_{n+1}
\Rightarrow y_n = \frac{a_{n+1}}{b} + \frac{a_{n+1}}{b}
\end{displaymath}
Cela montre qu'il existe des nombres 
\begin{displaymath}
a_1, a_2, \cdots, a_n \in \llbracket 0, b\llbracket, \hspace{0.5cm} y_n \in [0,1[  
\end{displaymath}
tels que 
\begin{displaymath}
  x = \lfloor x \rfloor + a_1b^{-1} + a_2b^{-2} + \cdots + + a_nb^{-n} + y_nb^{-n}  
\end{displaymath}
Ce formalisme permet de poser des questions:
\begin{itemize}
  \item l'algorithme de la partie droite se termine-t-il?
  \item dans le cas où il ne se termine pas, la suite 
\begin{displaymath}
  \left( a_1b^{-1} + a_2b^{-2} + \cdots + + a_nb^{-n}\right)_{n\in \N^*}
\end{displaymath}
converge-t-elle ? vers $\{x\}$?
\end{itemize}
Pour un $x$ donné, l'algorithme se termine si et seulement si il existe un entier $n$ tel que $xb^n \in \Z$. Pour chaque base $b$, on définit un ensemble particulier formé par les nombres $x$ vérifiant cette propriété. On peut remarquer que tous ces nombres sont rationnels mais que tous les rationnels ne sont pas de cette forme. Lorsque $b=10$ cet ensemble est noté $\D$ (ensemble des nombres décimaux). Lorsque $b=2$, cet ensemble est noté $\B$ (ensemble des nombres binaires).
\begin{align*}
  &\text{nombres binaires}:& &x\in \B \Leftrightarrow \exists n \in\N \text{ tel que } x\,2^{n} \in \Z \\
  &\text{nombres décimaux}:& &x\in \D \Leftrightarrow \exists n \in\N \text{ tel que } x\,10^{n} \in \Z 
\end{align*}
Un nombre décimal est-il binaire? Non! Un nombre binaire est-il décimal? Oui!
Lorsque l'algorithme ne se termine pas, la suite converge vers $\{x\}$ car 
\begin{equation}
  \{x\} - \left( a_1b^{-1} + a_2b^{-2} + \cdots +  a_nb^{-n}\right)  = \frac{y_n}{b^n} \text{ avec } 0 < y_n < b \label{reste}
\end{equation}

Si un $y_n = 0$, on choisit de poser $a_{n+1} = a_{n+2} = \cdots = 0$ de manière à former une suite de $a_k$ dans tous les cas. Cette suite est le développement en base $b$ de $\{x\}$.

Pour chaque nombre réel dans $[0,1[$
\begin{itemize}
  \item L'algorithme associe une unique suite $\left( a_n\right)_{n\in \N}$ d'éléments de $\llbracket 0, b\llbracket$ appelé développement en base $b$.
  \item La suite  $\left( \left( a_1b^{-1} + a_2b^{-2} + \cdots + a_nb^{-n}\right)\right)_{n\in \N^*}$ converge vers le nombre donné.
  \item Deux nombres distincts ont des développements distincts
\end{itemize}
La démonstration du dernier point repose sur des majorations à partir de la relation ~\eqref{reste}.

Attention:
\begin{displaymath}
  \left( \sum_{k=0}^n b^{-k} \right)_{n\in \N} \rightarrow \frac{1}{1-\frac{1}{b}} = \frac{b}{b-1}
  \Rightarrow 
  \left( \sum_{k=0}^n (b-1)b^{-k} \right)_{n\in \N} \rightarrow  b
\end{displaymath}
donc $x = a_1b^{-1} + \cdots + a_pb^{-p}$ avec $a_p>0$ est aussi la limite de la suite
\begin{displaymath}
    \left( a_1b^{-1} + \cdots +(a_p-1)b^{-p} + (b-1)b^{-p-1}+ (b-1)b^{-p-2} + \cdots +(b-1)b^{-n}\right)_{n\in \N} 
\end{displaymath}
Dans le cas de la base 10, les nombres décimaux sont ceux pour lesquels il existe un développement stationnaire de valeur $0$. Un tel nombre pourrait aussi s'écrire avec un développement stationnaire de valeur $9$ par exemple 
\begin{displaymath}
 1.1199999999999999999999999\cdots = 1.1200000000\cdots = 1.12
\end{displaymath}
mais on évite de le faire.
\begin{prop}
  Un nombre réel est rationnel (c'est à dire dans $\Q$) si et seulement si son développement en base $b$ est périodique à partir d'un certain rang.
\end{prop}
\begin{demo}
On fournit seulement une ébauche de démonstration.
\begin{itemize}
  \item Si le développement est périodique de période $p$ à partir d'un certain rang, il est la somme à partir de ce rang de $p$ suites constantes. On peut calculer les limites avec des suites géométriques et on obtient un nombre rationnel.
  \item Supposons que le nombre à développer soit rationnel: par exemple $\frac{p}{q}$ avec $p<q$.  On prouve alors que chaque $y_n=\frac{r_n}{q}$ où $r_n$ est le reste de la division de $pb^n$ par $q$. Comme la suite des $r_n$ prend ses valeurs dans un ensemble fini de $q$ éléments, elle ne peut pas être injective et il existe un $p$ et $q>$ tels que $r_p = r_q$ donc $y_p = y_q$, le développement sera $q-p$-périodique à partir de ce rang.
\end{itemize}
\end{demo}

\printindex
\end{document}
 