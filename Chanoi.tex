\begin{enumerate}
  \item On se permet de ne pas écrire les \texttt{deplacer} mais seulement les suites de couples d'indices.\newline
Première liste:
\begin{multline*}
(1,2)\rightarrow (1,3) \rightarrow (2,3)  \rightarrow (1,2)  \rightarrow (3,2)  \rightarrow (3,1)  \rightarrow (2,1) \\
 \rightarrow (2,3)  \rightarrow (1,2)  \rightarrow (1,3)  \rightarrow (2,3)
\end{multline*}
Soit 11 déplacements, on sent bien que l'on peut faire plus rapidement.\newline
Deuxième liste:
\begin{displaymath}
(1,3)\rightarrow (1,2) \rightarrow (3,2)  \rightarrow (1,3)  \rightarrow (2,1)  \rightarrow (3,2)  \rightarrow (1,3) 
\end{displaymath}
Soit 7 déplacements seulement.

\item Si on peut passer du tas 1 au tas 3, on peut également passer du tas 1 au tas 2 et la séquence comprendra exactement le même nombre d'instructions. Précisément, si dans la séquence d'instructions pour passer de 1 à 2, on permute 2 et 3 dans tous les \texttt{deplacer()}, on obient une séquence d'instruction qui fait passer du tas 1 au tas 2.\newline
On organise alors les transferts de la manière suivante
\begin{multline*}
\begin{aligned}
\begin{pmatrix}
1 \\ 2 \\ \vdots \\ n  
\end{pmatrix}
& &
\begin{pmatrix}
 \\ 
\end{pmatrix}
& &
\begin{pmatrix}
 \\
\end{pmatrix}
\end{aligned}
  \longrightarrow 
\begin{aligned}
\begin{pmatrix}
 n  
\end{pmatrix}
& &
\begin{pmatrix}
1 \\ 2 \\ \vdots \\ n-1 
\end{pmatrix}
& &
\begin{pmatrix}
 \\
\end{pmatrix}
\end{aligned}
  \longrightarrow 
\begin{aligned}
\begin{pmatrix}
 \\  
\end{pmatrix}
& &
\begin{pmatrix}
1 \\ 2 \\ \vdots \\ n-1 
\end{pmatrix}
& &
\begin{pmatrix}
 n
\end{pmatrix}
\end{aligned}  \\
  \longrightarrow 
\begin{aligned}
\begin{pmatrix}
 \\  
\end{pmatrix}
& &
\begin{pmatrix}
 \\
\end{pmatrix}
& &
\begin{pmatrix}
1 \\ 2 \\ \vdots \\ n
\end{pmatrix}
\end{aligned}  
\end{multline*}
Dans le schéma précédent, les transferts du début et de la fin viennent de l'hypothèse de récurrence relative à un tas de hauteur $n-1$.\newline
Notons $h_n$ le nombre de déplacements nécessaires pour transférer un tas de hauteur $n$. Cette notation est justifiée par le fait que ce problème est celui des \emph{tours de Hanoï}. On a alors
\begin{displaymath}
  h_3 = 7 \text{ et } h_n=2h_{n-1}+1
\end{displaymath}
Pour exprimer $h_n$, on introduit un "point fixe" $c$ qui fait apparaitre une suite géométrique:
\begin{displaymath}
  \left. 
  \begin{aligned}
    h_n &=2h_{n-1}+1 \\ c &= 2c+1
  \end{aligned}
\right\rbrace 
\Rightarrow
h_n - c = 2(h_{n-1}-c) = 2 ^{n-3}(h_3-c)
\end{displaymath}
Avec $c=-1$ et $h_3=7$, on obtient finalement
\begin{displaymath}
  h_n = 2^n -1
\end{displaymath}

\end{enumerate}
