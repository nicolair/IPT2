\begin{algorithm}
  $i\leftarrow 0$\;
  $j\leftarrow n-1$\;
  \Tq{ $i \leq j$ }{
    \uSi{$L[i] + L[j] < x$}{
      $i \leftarrow i+1$ \;
    }
    \SinonSi{$L[i] + L[j] > x$}{
      $j \leftarrow j-1$ \;
    }
    \Sinon{
       sortir de la boucle et renvoyer vrai\;
    }
  }
   renvoyer faux\;
  \caption{$x$ est-il somme de deux valeurs de $L$?}
  \label{exo_som2_C_1}
\end{algorithm}

\begin{enumerate}
  \item L'algorithme \ref{exo_som2_C_1} est complété pour résoudre le problème. L'énoncé ne le précise pas, les deux indices des valeurs de $L$ dont la somme doit valoir $x$ peuvent être égaux. Si on souhaite qu'ils soient distincts, il faut remplacer l'inégalité stricte de la condition d'exécution de la boucle "tant que" par une inégalité large.
\item Implémentation de l'algorithme dans une fonction:
\begin{verbatim}
def EstSomme(L,x):
    n = len(L)
    i = 0
    j = n-1
    while i <= j:
        if L[i] + L[j] < x:
            i += 1
        elif L[i] + L[j] > x:
            j -= 1
        else:
            return True
    return False
    
L = [1,2,4,6,8,9,11,15,18,20,22]
print(EstSomme(L,21))
\end{verbatim}

\end{enumerate}
