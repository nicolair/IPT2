\subsection*{I - Outils}
\begin{enumerate}
  \item
\begin{enumerate}
  \item   Codage de la fonction \texttt{indice}. Elle renvoie par convention la valeur booléenne \texttt{False} lorsque l'objet n'est pas une valeur de la liste.
\begin{verbatim}
def indice(lili,v):
    l = len(lili)
    i = 0
    while i < l and lili[i] != v:
        i += 1
    if i < l:
        return i
    else:
        return False
\end{verbatim}

  \item Codage de \texttt{valeurMax}
\begin{verbatim}
def valeurMax(lili):
    vmax = lili[0]
    for v in lili:
      if vmax < v:
        vmax = v
    return vmax
\end{verbatim}
  
  
  \item Codage de \texttt{indicesValeurMax}
\begin{verbatim}
def indicesValeurMax(lili):
    vmax = lili[0]
    livm = [0]
    i = 1 
    l = len(lili)
    while i < l:
      if lili[i] > vmax:
        livm = [i]
        vmax = lili[i]
      elif lili[i] == vmax:
        livm.append(i)
      i += 1
    return livm
\end{verbatim}
  
\end{enumerate}
  

  \item 
\begin{enumerate}
  \item L'algorithme \ref{Cfreqtexte1_1} en pseudo-code est complété par l'instruction manquante
\begin{displaymath}
\texttt{ligv[imin]} \leftarrow \texttt{j}  
\end{displaymath}
\begin{algorithm}
  \Comment{initialisations}
  \texttt{l} $\leftarrow$ longueur de \texttt{L}\;
  \texttt{ligv} $\leftarrow$ liste des entiers de 0 à 9\;
  actualiser  \texttt{imin}\;
  \Comment{traitement des autres valeurs}
  \texttt{j} $\leftarrow 10$\;
  \Tq{j < l}{
      \If{ \texttt{L[ligv[imin]]} < \texttt{L[j]} }{
          \texttt{ligv[imin]} $\leftarrow$ \texttt{j}\;
          actualiser \texttt{imin}\;
      }
      incrémenter \texttt{j}
  }
  \caption{calcul d'une liste de 10 indices des plus grandes valeurs}
  \label{Cfreqtexte1_1}
\end{algorithm}
\item Codage de la fonction \texttt{actuImin}
\begin{verbatim}
def actuImin():
    k = 0
    i = 1
    for i in range(1,10):
        if L[ligv[i]] < L[ligv[k]]:
            k = i
    return k
\end{verbatim}
\item Codage de la fonction \texttt{top10}.
\begin{verbatim}
def top10(L):
    p = 10
    # bibliothèque
        # actuImin est locale à top10
    def actuImin():
        k = 0
        i = 1
        for i in range(1,p):
            if L[ligv[i]] < L[ligv[k]]:
                k = i
        return k
        
    # initialisations
    l = len(L)
    ligv = [ i for i in range(p)]
    imin = actuImin()
    
    #boucle principale
    for j in range(p,l):
        if L[ligv[imin]] < L[j]:
            ligv[imin] = j
            imin = actuImin()
    # renvoi et fin
    return ligv
\end{verbatim}
\end{enumerate}
\end{enumerate}

\subsection*{II - Fréquences de caractères}
\begin{enumerate}
  \item Le texte contient \texttt{len(CarNum)=len(NbOcCarNum)} caractères distincts.
  \item Pseudo-code pour le calcul des listes.
\begin{quote}
  Pour chaque caractère \texttt{c} du texte
  \begin{itemize}
    \item \texttt{didi} $\leftarrow$ \texttt{indice(CarNum,c)}
    \item si \texttt{didi} $\neq$ \texttt{False}\newline
    $\phantom{:}\hspace{1cm}$ incrémenter \texttt{NbOcCarNum[didi]}\newline
    sinon\newline
    $\phantom{:}\hspace{1cm}$ \texttt{CarNum.append(c)}\newline
    $\phantom{:}\hspace{1cm}$ \texttt{NbOcCarNum.append(1)}
  \end{itemize}
\end{quote}

  \item Codage Python de la fonction extrayant les 10 caractères les plus fréquents.
\begin{verbatim}
def freqC(texte):
    # calcul de CarNum et NbOcCarNum
    CarNum = []
    NbOcCarNum = []
    for c in texte :
        didi = indice(CarNum,c)
        if didi != False:
            NbOcCarNum[didi] += 1
        else:
            CarNum.append(c)
            NbOcCarNum.append(1)
            
    # calcul de la liste des indices
    # des 10 caractères les plus fréquents
    li10cpf = top10(NbOcCarNum)
    # affichage du résultat
    res = ""
    for i in li10cpf:
        res += str(CarNum[i]) + ": " + str(NbOcCarNum[i]) + ", "
    print(res)
\end{verbatim}

\end{enumerate}

\subsection*{III - Fréquences de mots}
\begin{enumerate}
  \item Pseudo code pour établir la liste (avec répétition) de tous les mots du texte.
\begin{quote}
\begin{itemize}
  \item \texttt{Mots} initialisé comme une liste vide.
  \item \texttt{mot} initialisé comme une chaîne de caractère vide.
  \item pour chaque caractère \texttt{c} de \texttt{texte}:
  \begin{itemize}
    \item si \texttt{c} est un espace\newline
    $\phantom{:}\hspace{1cm}$ placer \texttt{mot} à la fin de \texttt{Mots}\newline
    $\phantom{:}\hspace{1cm}$ assigner la chaîne vide à \texttt{mot}
    \item sinon\newline
    $\phantom{:}\hspace{1cm}$ concaténer \texttt{c} à la fin de \texttt{mot}
  \end{itemize}
\end{itemize}
\end{quote}
Les deux autres listes sont calculées en même temps par un algorithme analogue à celui de la question II.2
\begin{quote}
  \texttt{MotNum} et \texttt{NbOcMotNum} sont initialisées comme des listes vides.
  Pour chaque valeur \texttt{mot} de la liste \texttt{Mots}
  \begin{itemize}
    \item \texttt{didi} $\leftarrow$ \texttt{indice(MotNum,mot)}
    \item si \texttt{didi} $\neq$ \texttt{False}\newline
    $\phantom{:}\hspace{1cm}$ incrémenter \texttt{NbOcMotNum[didi]}\newline
    sinon\newline
    $\phantom{:}\hspace{1cm}$ \texttt{MotNum.append(mot)}\newline
    $\phantom{:}\hspace{1cm}$ \texttt{NbOcMotNum.append(1)}
  \end{itemize}
\end{quote}
\item Code de la fonction \texttt{freqM}.
\begin{verbatim}
def freqM(texte):
    # calcul de la liste Mots
    Mots = []
    mot = ""
    for c in texte :
        if c == " ":
            Mots.append(mot)
            mot = ""
        else:
            mot = mot + c

    # calcul de MotNum et NbOcMotNum
    MotNum = []
    NbOcMotNum = []
    for mot in Mots :
        didi = indice(MotNum,mot)

        if didi != False:
            NbOcMotNum[didi] += 1
        else:
            MotNum.append(mot)
            NbOcMotNum.append(1)
            
    # liste des indices des 10 mots les plus fréquents
    li10cpf = top10(NbOcMotNum)
    # affichage du résultat
    res = ""
    for i in li10cpf:
        res += MotNum[i] + ": " + str(NbOcMotNum[i]) + ", "
    print(res)
\end{verbatim}

\end{enumerate}
