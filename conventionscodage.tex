Pour que votre code écrit en Python soit plus facile à comprendre et à relire par vous ou par d'autres, il est utile d'adopter des conventions dans la présentation des fichiers considérées comme de bonnes pratiques par les développeurs. 

\begin{itemize}
  \item taille des indentations: 4 espaces,
  \item  taille maximale d'une ligne: 79 caractères,
  \item toujours placer un espace après une virgule, un point-virgule ou un deux-points (sauf pour la syntaxe des tranches),
  \item ne jamais placer d'espace avant une virgule, un point-virgule ou un deux-points,
  \item toujours placer un espace de chaque côté d'un opérateur,
  \item ne pas placer d'espace entre le nom d'une fonction et la première parenthèse délimitant sa liste d'arguments.
\end{itemize}
